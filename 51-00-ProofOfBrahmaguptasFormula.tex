\documentclass[12pt]{article}
\usepackage{pmmeta}
\pmcanonicalname{ProofOfBrahmaguptasFormula}
\pmcreated{2013-03-22 13:09:14}
\pmmodified{2013-03-22 13:09:14}
\pmowner{giri}{919}
\pmmodifier{giri}{919}
\pmtitle{proof of Brahmagupta's formula}
\pmrecord{6}{33594}
\pmprivacy{1}
\pmauthor{giri}{919}
\pmtype{Proof}
\pmcomment{trigger rebuild}
\pmclassification{msc}{51-00}

\endmetadata

% this is the default PlanetMath preamble.  as your knowledge
% of TeX increases, you will probably want to edit this, but
% it should be fine as is for beginners.

% almost certainly you want these
\usepackage{amssymb}
\usepackage{amsmath}
\usepackage{amsfonts}

% used for TeXing text within eps files
%\usepackage{psfrag}
% need this for including graphics (\includegraphics)
\usepackage{graphicx}
% for neatly defining theorems and propositions
%\usepackage{amsthm}
% making logically defined graphics
%%%\usepackage{xypic}

% there are many more packages, add them here as you need them

% define commands here
\begin{document}
We shall prove that the area of a cyclic quadrilateral with sides $p,q,r,s$  is given by $$\sqrt{(T-p)(T-q)(T-r)(T-s)}$$where $T = \frac{p+q+r+s}{2}.$

\begin{center}
\includegraphics{quadcyclic.eps}
\end{center}

Area of the cyclic quadrilateral = Area of $\triangle ADB +$ Area of $\triangle BDC.$ $$ = \frac{1}{2}pq\sin A  + \frac{1}{2}rs\sin C$$
But since $ABCD$ is a cyclic quadrilateral, $\angle DAB = 180^\circ - \angle DCB.$
Hence $\sin A = \sin C.$ Therefore area now is
$$ Area = \frac{1}{2}pq\sin A + \frac{1}{2}rs\sin A$$
$$ (Area)^2 = \frac{1}{4}\sin^2 A (pq + rs)^2$$
$$ 4(Area)^2 = (1 - \cos^2 A)(pq + rs)^2$$
$$ 4(Area)^2 = (pq + rs)^2 - cos^2 A (pq + rs)^2$$
Applying cosines law for $\triangle ADB$ and $\triangle BDC$ and equating the expressions for side $DB,$ we have
$$p^2 + q^2 - 2pq\cos A = r^2 + s^2 - 2rs\cos C$$
Substituting $\cos C = -\cos A$ (since angles $A$ and $C$ are suppplementary) and rearranging, we have
$$2\cos A (pq + rs) = p^2 + q^2 - r^2 - s^2$$
substituting this in the equation for area,
$$ 4(Area)^2 = (pq + rs)^2 - \frac{1}{4}(p^2 + q^2 - r^2 - s^2)^2$$
$$ 16(Area)^2 = 4(pq + rs)^2 - (p^2 + q^2 - r^2 - s^2)^2$$
which is of the form $a^2-b^2$ and hence can be written in the form $(a+b)(a-b)$ as
$$(2(pq + rs) + p^2 + q^2 -r^2 - s^2)(2(pq + rs) - p^2 - q^2 + r^2 +s^2)$$
$$ = ( (p+q)^2 - (r-s)^2 )( (r+s)^2 - (p-q)^2 )        $$
$$ = (p+q+r-s)(p+q+s-r)(p+r+s-q)(q+r+s-p)$$
Introducing $T = \frac{p+q+r+s}{2},$
$$ 16(Area)^2 = 16(T-p)(T-q)(T-r)(T-s)$$
Taking square root, we get
$$ Area = \sqrt{(T-p)(T-q)(T-r)(T-s)}$$
%%%%%
%%%%%
\end{document}
