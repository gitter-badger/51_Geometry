\documentclass[12pt]{article}
\usepackage{pmmeta}
\pmcanonicalname{Kite}
\pmcreated{2013-03-22 15:49:22}
\pmmodified{2013-03-22 15:49:22}
\pmowner{yark}{2760}
\pmmodifier{yark}{2760}
\pmtitle{kite}
\pmrecord{9}{37789}
\pmprivacy{1}
\pmauthor{yark}{2760}
\pmtype{Definition}
\pmcomment{trigger rebuild}
\pmclassification{msc}{51-00}
\pmsynonym{deltoid}{Kite}
%\pmkeywords{quadrilateral}
%\pmkeywords{parallelogram}
%\pmkeywords{rhombus}
%\pmkeywords{square}
%\pmkeywords{geometry}
\pmrelated{Parallelogram}
\pmrelated{Quadrilateral}
\pmrelated{Rhombus}

\usepackage{graphicx}

\newcommand{\figuraex}[2]{\begin{center}\includegraphics[#2]{#1}\end{center}}

\begin{document}
A \emph{kite} or \emph{deltoid} is a quadrilateral with  two pairs of equal sides, each pair consisting of adjacent sides. Contrast with parallelograms, where the equal sides are opposite.
\figuraex{GeometricKite.eps}{scale=0.75}

The pairs of equal sides imply several properties:
\begin{itemize}
\item One diagonal divides the kite into two isosceles triangles, and the other divides the kite into two congruent triangles.
\item The angles  between the sides of unequal length are equal. In the picture, they are both equal to the sum of the blue angle with the red angle.
\item The diagonals are perpendicular. 
\item If $d_1$ and $d_2$ are the lengths of the diagonals, then the area is
\[A=\frac{d_1d_2}{2}\]
Alternatively, if $a$ and $b$ are the lengths of the sides, and $\theta$ the angle between unequal sides, then the area is
\[A={a b \sin\theta}.\]
\item A kite possesses an inscribed circle. That is, there exists a circle that is tangent (touches) the four sides.
\item Kites always possess at least one symmetry axis, being the diagonal that divides it into two congruent triangle.
\end{itemize}

When all the side lengths are the same, the kite becomes a rhombus, and when both diagonals have the same length, the kite becomes a square.
%%%%%
%%%%%
\end{document}
