\documentclass[12pt]{article}
\usepackage{pmmeta}
\pmcanonicalname{NURBSSurface}
\pmcreated{2013-03-22 17:23:51}
\pmmodified{2013-03-22 17:23:51}
\pmowner{stitch}{17269}
\pmmodifier{stitch}{17269}
\pmtitle{NURBS surface}
\pmrecord{6}{39766}
\pmprivacy{1}
\pmauthor{stitch}{17269}
\pmtype{Definition}
\pmcomment{trigger rebuild}
\pmclassification{msc}{51N05}
\pmsynonym{nonuniform rational B-spline surface}{NURBSSurface}
\pmrelated{NURBS}

% this is the default PlanetMath preamble.  as your knowledge
% of TeX increases, you will probably want to edit this, but
% it should be fine as is for beginners.

% almost certainly you want these
\usepackage{amssymb}
\usepackage{amsmath}
\usepackage{amsfonts}

% used for TeXing text within eps files
%\usepackage{psfrag}
% need this for including graphics (\includegraphics)
%\usepackage{graphicx}
% for neatly defining theorems and propositions
%\usepackage{amsthm}
% making logically defined graphics
%%%\usepackage{xypic}

% there are many more packages, add them here as you need them

% define commands here

\begin{document}
\section{Introduction}
A \emph{NURBS surface}, which is an acronym for \emph{Non-Uniform Rational B-Spline surface}, is a generalization of both \PMlinkname{B\'ezier}{BezierCurve} and B-splines surfaces. NURBS are commonly used in computer graphics, computer-aided design (CAD), engineering (CAE), and manufacturing (CAM).

\section{Definition}
A NURBS surface is parametric surface defined by its \PMlinkescapetext{degree}, an array of $n+1$ rows and $m+1$ columns weighted control points and a knot vector in each direction. It is defined as

\[
c(u,v) = \frac { \sum_{i=0}^{n}\sum_{j=0}^{m} N_{i,p}(u)N_{j,q}(v) w_{i,j} P_{i,j}} { \sum_{i=0}^{n}\sum_{j=0}^{m} N_{i,p}(u)N_{j,q}(v) w_{i,j}} \quad \quad 0 \leq u \leq 1, \quad 0 \leq v \leq 1
\]

where $u$ and $v$ are the parameters in each direction, $p$ is the \PMlinkescapetext{degree} in the $u$-direction, $q$ is the \PMlinkescapetext{degree} in the $v$-direction, $N_{i,p}$ and $N_{j,q}$ are the B-spline basis functions, $P_{i,j}$ are the control points and $w_{i,j}$ are the weights.
%%%%%
%%%%%
\end{document}
