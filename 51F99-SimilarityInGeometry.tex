\documentclass[12pt]{article}
\usepackage{pmmeta}
\pmcanonicalname{SimilarityInGeometry}
\pmcreated{2013-03-22 17:08:48}
\pmmodified{2013-03-22 17:08:48}
\pmowner{pahio}{2872}
\pmmodifier{pahio}{2872}
\pmtitle{similarity in geometry}
\pmrecord{19}{39455}
\pmprivacy{1}
\pmauthor{pahio}{2872}
\pmtype{Definition}
\pmcomment{trigger rebuild}
\pmclassification{msc}{51F99}
\pmclassification{msc}{51M05}
\pmclassification{msc}{51-00}
\pmsynonym{similarity}{SimilarityInGeometry}
\pmsynonym{similitude}{SimilarityInGeometry}
\pmrelated{Homothety}
\pmrelated{ProportionEquation}
\pmrelated{HarmonicMeanInTrapezoid}
\pmdefines{similar}
\pmdefines{ratio of similarity}
\pmdefines{similitude ratio}
\pmdefines{line ratio}

\endmetadata

% this is the default PlanetMath preamble.  as your knowledge
% of TeX increases, you will probably want to edit this, but
% it should be fine as is for beginners.

% almost certainly you want these
\usepackage{amssymb}
\usepackage{amsmath}
\usepackage{amsfonts}

% used for TeXing text within eps files
%\usepackage{psfrag}
% need this for including graphics (\includegraphics)
%\usepackage{graphicx}
% for neatly defining theorems and propositions
 \usepackage{amsthm}
% making logically defined graphics
%%%\usepackage{xypic}

% there are many more packages, add them here as you need them

% define commands here

\theoremstyle{definition}
\newtheorem*{thmplain}{Theorem}

\begin{document}
Two figures $K$ and $K'$ in a Euclidean plane or \PMlinkname{space}{EuclideanVectorSpace} are\, {\em similar}\, iff there exists a bijection $f$ from the set of points of $K$ onto the set of points of $K'$ such that, for any $P,Q \in K$, the ratio
$$\frac{P'Q'}{PQ}$$
of the lengths of the line segments $P'Q'$ and $PQ$ is always the same number $k$, where $P'=f(P)$ and $Q'=f(Q)$.

The number $k$ is called the {\em ratio of similarity} or the {\em line ratio} of the figure $K'$ with respect to the figure $K$ (N.B. the \PMlinkescapetext{order} in which the figures are mentioned!).\, The similarity of $K$ and $K'$ is often denoted by
$$K' \sim K\;\;\;(\mbox{or}\;\;K \sim K').$$\\

\textbf{Examples}
\begin{itemize}
\item All squares are similar.
\item All cubes are similar.
\item All circles are similar.
\item All parabolas are similar.
\item All sectors of circle with equal central angle are similar.
\item All spheres are similar.
\item All equilateral triangles are similar.
\end{itemize}

\textbf{Nonexamples}
\begin{itemize}
\item Not all rectangles are similar.
\item Not all rhombi are similar.
\item Not all rectangular prisms are similar.
\item Not all ellipses are similar.
\item Not all ellipsoids are similar.
\item Not all triangles are similar.
\end{itemize}

\textbf{Properties}
\begin{itemize}
\item The corresponding angles (consisting of corresponding points) of two similar figures are equal.
\item The lengths of any corresponding arcs of two similar figures are proportional in the ratio $k$.
\item The areas of two similar regions are proportional in the ratio $k^2$ when $k$ is the line ratio of the regions.
\item The volumes of two similar solids are proportional in the ratio $k^3$ when $k$ is the line ratio of the solids.
\end{itemize}

\textbf{Remarks}
\begin{itemize}
\item In any Euclidean space $E$, the \PMlinkname{relation}{Relation} of similarity (denoted $\sim$) on the set of figures in $E$ is an equivalence relation.
\item If one pair of corresponding line segments in the similar figures $K$ and $K'$ are equal, then all pairs of corresponding line segments are equal, i.e. the figures have also equal \PMlinkescapetext{sizes}: They are \PMlinkname{congruent}{Congruence} ($K' \cong K$).
\end{itemize}
%%%%%
%%%%%
\end{document}
