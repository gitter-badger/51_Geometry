\documentclass[12pt]{article}
\usepackage{pmmeta}
\pmcanonicalname{SomeTheoremsOnTheAxiomsOfOrder}
\pmcreated{2013-03-22 17:18:47}
\pmmodified{2013-03-22 17:18:47}
\pmowner{Mathprof}{13753}
\pmmodifier{Mathprof}{13753}
\pmtitle{some theorems on the axioms of order}
\pmrecord{6}{39662}
\pmprivacy{1}
\pmauthor{Mathprof}{13753}
\pmtype{Theorem}
\pmcomment{trigger rebuild}
\pmclassification{msc}{51G05}
\pmrelated{BetweennessRelation}

% this is the default PlanetMath preamble.  as your knowledge
% of TeX increases, you will probably want to edit this, but
% it should be fine as is for beginners.

% almost certainly you want these
\usepackage{amssymb}
\usepackage{amsmath}
\usepackage{amsfonts}

% used for TeXing text within eps files
%\usepackage{psfrag}
% need this for including graphics (\includegraphics)
%\usepackage{graphicx}
% for neatly defining theorems and propositions
\usepackage{amsthm}
% making logically defined graphics
%%%\usepackage{xypic}

% there are many more packages, add them here as you need them

% define commands here
\newtheorem{thm}{Theorem}

\begin{document}
Let $B$ be a betweenness relation on a set $A$.
\begin{thm}
\item If $(a,b,c)\in B$ and $(a,c,d)\in B$, then $(a,b,d)\in B$.
\end{thm}
\begin{thm}
For each pair of elements $p,q\in A$,
we can define five sets:
\begin{enumerate}
\item $B_{*pq}:=\lbrace r\in A\mid (r,p,q)\in B\rbrace$,
\item $B_{p*q}:=\lbrace r\in A\mid (p,r,q)\in B\rbrace$, 
\item $B_{pq*}:=\lbrace r\in A\mid (p,q,r)\in B\rbrace$,
\item $B_{pq}:=B_{p*q}\cup\lbrace q\rbrace\cup B_{pq*}$, and
\item $B(p,q):=B_{*pq}\cup\lbrace p\rbrace\cup B_{pq}$.
\end{enumerate}
Then
\begin{itemize}
\item[(1)] $B_{*pq}=B_{qp*}.$ 
\item[(2)] $B_{p*q}=B_{q*p}.$  
\item[(3)] The intersection of any pair of the first three sets contains at most one element, either $p$ or $q$.
\item[(4)] Each of the sets can be partially ordered.  
\item[(5)] The partial order on $B_{pq}$ and $B(p,q)$ extends that of the subsets.
\end{itemize}
\end{thm}



%%%%%
%%%%%
\end{document}
