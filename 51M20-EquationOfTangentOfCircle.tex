\documentclass[12pt]{article}
\usepackage{pmmeta}
\pmcanonicalname{EquationOfTangentOfCircle}
\pmcreated{2013-03-22 18:32:16}
\pmmodified{2013-03-22 18:32:16}
\pmowner{pahio}{2872}
\pmmodifier{pahio}{2872}
\pmtitle{equation of tangent of circle}
\pmrecord{10}{41254}
\pmprivacy{1}
\pmauthor{pahio}{2872}
\pmtype{Derivation}
\pmcomment{trigger rebuild}
\pmclassification{msc}{51M20}
\pmclassification{msc}{51M04}
\pmrelated{Polarising}

% this is the default PlanetMath preamble.  as your knowledge
% of TeX increases, you will probably want to edit this, but
% it should be fine as is for beginners.

% almost certainly you want these
\usepackage{amssymb}
\usepackage{amsmath}
\usepackage{amsfonts}

% used for TeXing text within eps files
%\usepackage{psfrag}
% need this for including graphics (\includegraphics)
%\usepackage{graphicx}
% for neatly defining theorems and propositions
 \usepackage{amsthm}
% making logically defined graphics
%%%\usepackage{xypic}
\usepackage{pstricks}
\usepackage{pst-plot}

% there are many more packages, add them here as you need them

% define commands here

\theoremstyle{definition}
\newtheorem*{thmplain}{Theorem}

\begin{document}
We derive the equation of tangent line for a circle with radius $r$.\, For simplicity, we chose for the origin the centre of the circle, when the points \,$(x,\,y)$\, of the circle satisfy the equation
\begin{align}
x^2+y^2 = r^2.
\end{align}

Let the point of tangency be\, $(x_0,\,y_0)$.\, Then the slope of radius with end point \,$(x_0,\,y_0)$\, is 
$\frac{y_0}{x_0}$, whence, according to the \PMlinkname{parent entry}{TangentOfCircle}, its opposite inverse 
$-\frac{x_0}{y_0}$ is the slope of the tangent, being perpendicular to the radius.\, Thus the equation of the tangent is written as
$$y-y_0 \;=\; -\frac{x_0}{y_0}(x-x_0).$$
Removing the denominator and the parentheses we obtain from this first\, $x_0x+y_0y = x_0^2+y_0^2$,\, and then
\begin{align}
x_0x+y_0y \;=\; r^2
\end{align}
since\, $(x_0,\,y_0)$\, satisfies (1).\\


\textbf{Remark.}\, In the equation (2) of the tangent, $x_0$, $y_0$ are the coordinates of the point of tangency and 
$x,\,y$ the coordinates of an arbitrary point of the tangent line.\, But one can of course swap those meanings; then  we interprete (2) such that $x_0$, $y_0$ are the coordinates of some \PMlinkescapetext{fixed} point $P$ outside the circle (1) and $x,\,y$ the coordinates of the point of tangency of either of the tangents which may be drawn from $P$ to the circle.\, If (2) now is again interpreted as an equation of a line (its \PMlinkname{degree}{AlgebraicEquation} is 1!), this line must pass through both the mentioned points of tangency $A$ and $B$ (they satisfy the equation!); in a \PMlinkescapetext{word}, (2) is now the equation of the tangent chord $AB$ of\, $P = (x_0,\,y_0)$.\, See also \PMlinkexternal{polar}{http://mathworld.wolfram.com/Polar.html}.

\begin{center}
\begin{pspicture}(-5.5,-3)(5.5,3)
\rput(-2.85,-0.1){$O$}
\rput[linecolor=blue](+2.85,-0.1){$P$}
\psdot(-2.6,0)
\psdot[linecolor=blue](+2.6,0)
\psline(-2.6,0)(2.6,0)
\pscircle[linecolor=blue](-2.6,0){2}

\psdots(-1.8308,1.8462)(-1.8308,-1.8462)
\psline[linestyle=dashed](-2.6,0)(-1.8308,+1.8462)
\psline[linestyle=dashed](-2.6,0)(-1.8308,-1.8462)
\psline[linecolor=blue](-1.8308,+1.8462)(-1.8308,-1.8462)
\rput(-1.8308,+2.2){$A$}
\rput(-1.8308,-2.2){$B$}
\psline[linecolor=blue](2.6,0)(-3,+2.3335)
\psline[linecolor=blue](2.6,0)(-3,-2.3335)
\rput(-5.5,-3){.}
\rput(5.5,3){.}
\end{pspicture}
\end{center}


%%%%%
%%%%%
\end{document}
