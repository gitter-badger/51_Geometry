\documentclass[12pt]{article}
\usepackage{pmmeta}
\pmcanonicalname{InversionOfPlane}
\pmcreated{2015-06-14 18:40:35}
\pmmodified{2015-06-14 18:40:35}
\pmowner{pahio}{2872}
\pmmodifier{pahio}{2872}
\pmtitle{inversion of plane}
\pmrecord{24}{39542}
\pmprivacy{1}
\pmauthor{pahio}{2872}
\pmtype{Topic}
\pmcomment{trigger rebuild}
\pmclassification{msc}{51K99}
\pmclassification{msc}{53A30}
\pmclassification{msc}{30E20}
\pmsynonym{mirroring in circle}{InversionOfPlane}
\pmsynonym{circle inversion}{InversionOfPlane}
\pmrelated{MobiusTransformation}
\pmrelated{PowerOfPoint}
\pmdefines{inverse point}
\pmdefines{inversion}
\pmdefines{inversion formulae}
\pmdefines{involutory}

% this is the default PlanetMath preamble.  as your knowledge
% of TeX increases, you will probably want to edit this, but
% it should be fine as is for beginners.

% almost certainly you want these
\usepackage{amssymb}
\usepackage{amsmath}
\usepackage{amsfonts}

% used for TeXing text within eps files
%\usepackage{psfrag}
% need this for including graphics (\includegraphics)
%\usepackage{graphicx}
% for neatly defining theorems and propositions
 \usepackage{amsthm}
% making logically defined graphics
%%%\usepackage{xypic}

% there are many more packages, add them here as you need them
\usepackage{pstricks}
% define commands here

\theoremstyle{definition}
\newtheorem*{thmplain}{Theorem}

\begin{document}
\PMlinkescapeword{base} \PMlinkescapeword{fixed}

Let $c$ be a fixed circle in the Euclidean plane with center $O$ and radius $r$.  Set for any point $P \neq O$ of the plane a corresponding point $P'$, called the \emph{inverse point} of $P$ with respect to $c$, on the closed ray from $O$ through $P$ such that the product
$$P'O \cdot PO$$
has the \PMlinkescapetext{constant} value $r^2$.  This mapping \;$P \mapsto P'$\; of the plane interchanges the inside and outside of the base circle $c$.  The point $O'$ is the ``infinitely distant point'' of the plane.

The following is an illustration of how to obtain $P'$ for a given circle $c$ and point $P$ outside of $c$.  The restricted tangent from $P$ to $c$ is drawn in blue, the line segment that determines $P'$ (perpendicular to $\overline{OP}$, having an endpoint on $\overline{OP}$, and having its other endpoint at the point of tangency $T$ of the circle and the tangent line) is drawn in red, and the radius $\overline{OT}$ is drawn in green.

\begin{center}
\begin{pspicture}(-2,-2)(4,2)
\psline[linecolor=blue](1,1.732)(4,0)
\psline[linecolor=red](1,1.732)(1,0)
\psline[linecolor=green](0,0)(1,1.732)
\psline(0,0)(4,0)
\pscircle(0,0){2}
\psdots(0,0)(1,0)(1,1.732)(4,0)
\rput[a](0,-0.3){$O$}
\rput[a](4,-0.3){$P$}
\rput[l](1.12,1.83){$T$}
\rput[a](1.0,-0.3){$P'$}
\rput(0.38,0.84){$r$}
\rput(-1.55,1.6){$c$}
\rput[b](0,-2){.}
\rput[l](-2,0){.}
\rput[a](0,2){.}
\end{pspicture}
\end{center}
The picture justifies the correctness of $P'$, since the triangles 
$\triangle OPT$ and $\triangle OTP'$ are similar, implying the 
proportion \,$PO:TO = TO:P'O$\, whence\, $P'O \cdot PO = (TO)^2 = r^2$.\, Note that this same \PMlinkescapetext{argument} holds if $P$ and $P'$ were swapped in the picture.

\textbf{Inversion formulae.}\, If $O$ is chosen as the origin of 
$\mathbb{R}^2$ and\, $P = (x,\,y)$\, and\, $P' = (x',\,y')$,\, 
then
$$x' = \frac{rx}{x^2+y^2}, \qquad y' = \frac{ry}{x^2+y^2}; 
\qquad x = \frac{rx'}{x'^{\,2}+y'^{\,2}}, \qquad y = 
\frac{ry'}{x'^{\,2}+y'^{\,2}}.$$


\textbf{Note.}\, Determining inverse points can also be done in the 
complex plane.\, Moreover, the mapping $P \mapsto P'$ is always a 
M\"{o}bius transformation.\, For example, if\, 
$c = \{z\in\mathbb{Z}\,\vdots\;\, |z|=1 \}$,\, \PMlinkname{i.e.}{Ie}\, 
$O=0$\, and\, $r=1$, then the mapping\, $P \mapsto P'$\, is given by $f\colon \mathbb{C} \cup \{ \infty \} \to \mathbb{C} \cup \{\infty\}$\, defined by\, $\displaystyle f(z)=\frac{1}{z}$.\\

\textbf{Properties of inversion}
\begin{itemize}
\item The inversion is {\em involutory}, i.e. if\, $P\mapsto P'$,\, then\, $P'\mapsto P$.
\item The inversion is inversely conformal, i.e. the intersection angle of two curves is preserved (check the \PMlinkname{Cauchy--Riemann equations}{CauchyRiemannEquations}!).
\item A line through the center $O$ is mapped onto itself.
\item Any other line is mapped onto a circle that passes through the center $O$.
\item Any circle through the center $O$ is mapped onto a line; if the circle intersects the base circle $c$, then the line passes through both intersection points.
\item Any other circle is mapped onto its homothetic circle with $O$ as the homothety center.
\end{itemize}

\begin{thebibliography}{9}
\bibitem{NP}{\sc E. J. Nystr\"om}: {\em Korkeamman geometrian alkeet sovellutuksineen}.\, Kustannusosakeyhti\"o Otava, Helsinki (1948).
\end{thebibliography}

%%%%%
%%%%%
\end{document}
