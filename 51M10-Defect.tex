\documentclass[12pt]{article}
\usepackage{pmmeta}
\pmcanonicalname{Defect}
\pmcreated{2013-03-22 16:05:22}
\pmmodified{2013-03-22 16:05:22}
\pmowner{Wkbj79}{1863}
\pmmodifier{Wkbj79}{1863}
\pmtitle{defect}
\pmrecord{10}{38150}
\pmprivacy{1}
\pmauthor{Wkbj79}{1863}
\pmtype{Definition}
\pmcomment{trigger rebuild}
\pmclassification{msc}{51M10}
\pmclassification{msc}{51-00}
\pmrelated{AreaOfASphericalTriangle}

\endmetadata

% this is the default PlanetMath preamble.  as your knowledge
% of TeX increases, you will probably want to edit this, but
% it should be fine as is for beginners.

% almost certainly you want these
\usepackage{amssymb}
\usepackage{amsmath}
\usepackage{amsfonts}

% used for TeXing text within eps files
%\usepackage{psfrag}
% need this for including graphics (\includegraphics)
%\usepackage{graphicx}
% for neatly defining theorems and propositions
%\usepackage{amsthm}
% making logically defined graphics
%%%\usepackage{xypic}

% there are many more packages, add them here as you need them

% define commands here

\begin{document}
Consider a triangle $\triangle ABC$ in either \PMlinkname{hyperbolic or spherical geometry}{NonEuclideanGeometry} in which its angle sum in radians is $\Sigma$.

In hyperbolic geometry, the \emph{defect} of $\triangle ABC$ is $\delta(\triangle ABC)=\pi-\Sigma$.

In spherical geometry, the \emph{defect} of $\triangle ABC$ is $\delta(\triangle ABC)=\Sigma-\pi$.

Note that, in both hyperbolic and spherical geometry, the area of a \PMlinkescapetext{triangle} is equal to its defect.
%%%%%
%%%%%
\end{document}
