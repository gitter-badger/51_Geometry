\documentclass[12pt]{article}
\usepackage{pmmeta}
\pmcanonicalname{PolarTangentialAngle}
\pmcreated{2013-03-22 19:02:32}
\pmmodified{2013-03-22 19:02:32}
\pmowner{pahio}{2872}
\pmmodifier{pahio}{2872}
\pmtitle{polar tangential angle}
\pmrecord{8}{41919}
\pmprivacy{1}
\pmauthor{pahio}{2872}
\pmtype{Derivation}
\pmcomment{trigger rebuild}
\pmclassification{msc}{51-01}
\pmclassification{msc}{53A04}
\pmrelated{LogarithmicSpiral}

\endmetadata

% this is the default PlanetMath preamble.  as your knowledge
% of TeX increases, you will probably want to edit this, but
% it should be fine as is for beginners.

% almost certainly you want these
\usepackage{amssymb}
\usepackage{amsmath}
\usepackage{amsfonts}

% used for TeXing text within eps files
%\usepackage{psfrag}
% need this for including graphics (\includegraphics)
%\usepackage{graphicx}
% for neatly defining theorems and propositions
 \usepackage{amsthm}
% making logically defined graphics
%%%\usepackage{xypic}
\usepackage{pstricks}
\usepackage{pst-plot}

% there are many more packages, add them here as you need them

% define commands here

\theoremstyle{definition}
\newtheorem*{thmplain}{Theorem}

\begin{document}
The angle, under which a polar curve is \PMlinkescapetext{cut} by a line through the origin, is called the \emph{polar tangential angle} belonging to the intersection point on the curve.\\

Given a polar curve
\begin{align}
r = r(\varphi)
\end{align}
in polar coordinates $r,\,\varphi$,\, we derive an expression for the tangent of the polar tangential angle $\psi$, using the classical differential geometric method.

\begin{center}
\begin{pspicture}(-5.5,-1.5)(5.5,5)
\psline{->}(0,0)(4,0)
\rput(3.9,-0.2){$x$}
\psdot(0,0)
\rput(0,-0.22){$O$}
\psplot[linecolor=blue]{0.6}{2.5}{3 x x x mul sub add}
\psline(0,0)(1.1,2.89)
\psline(0,0)(2.73,2.93)
\psarc(0,0){3.09}{47}{69}
\psarc(0,0){0.4}{0}{47}
\psarc(0,0){0.7}{47}{69}
\rput(0.55,0.2){$\varphi$}
\rput(0.6,1.0){$d\varphi$}
\rput(2.1,-0.22){$A$}
\rput(2.1,2){$dr$}
\rput(1.43,1.8){$P$}
\rput(1.23,3.1){$P'$}
\rput(2.05,2.55){$Q$}
\rput(2.5,-0.9){$r = r(\varphi)$}
\end{pspicture}
\end{center}

The point $P$ of the curve given by (1) corresponds to the polar angle \,$\varphi = \angle POA$\, and the polar radius $r = OP$.\, The ``near'' point $P'$ corresponds to the polar angle \,$\varphi\!+\!d\varphi = \angle P'OA$\, and the polar radius \,$r\!+\!dr = OP'$.\, In the diagram, $P'Q$ is the arc of the circle with $O$ as centre and $OP'$ as radius.\, Thus, in the triangle-like figure $PP'Q$ we have
\begin{align}
\frac{P'Q}{PQ} \;=\; \frac{(r\!+\!dr)d\varphi}{dr} \;=\; \frac{r\!+\!dr}{\frac{dr}{d\varphi}}.
\end{align}
This figure can be regarded as an infinitesimal right triangle with the catheti $P'Q$ and $PQ$.\, Accordingly, their ratio (2) is the tangent of the acute angle $P$ of the triangle.\, Because the addend $dr$ in the last numerator in negligible compared with the addend $r$, it can be omitted.\, Hence we get the tangent 
$$\tan\psi \;=\; \frac{r}{\frac{dr}{d\varphi}}$$
of the polar tangential angle, i.e.
\begin{align}
\tan\psi \;=\; \frac{r(\varphi)}{r'(\varphi)}.
\end{align}

%%%%%
%%%%%
\end{document}
