\documentclass[12pt]{article}
\usepackage{pmmeta}
\pmcanonicalname{Octant}
\pmcreated{2013-03-22 18:51:53}
\pmmodified{2013-03-22 18:51:53}
\pmowner{pahio}{2872}
\pmmodifier{pahio}{2872}
\pmtitle{octant}
\pmrecord{5}{41686}
\pmprivacy{1}
\pmauthor{pahio}{2872}
\pmtype{Definition}
\pmcomment{trigger rebuild}
\pmclassification{msc}{51N20}
\pmrelated{Quadrant}
\pmdefines{first octant}

\endmetadata

% this is the default PlanetMath preamble.  as your knowledge
% of TeX increases, you will probably want to edit this, but
% it should be fine as is for beginners.

% almost certainly you want these
\usepackage{amssymb}
\usepackage{amsmath}
\usepackage{amsfonts}

% used for TeXing text within eps files
%\usepackage{psfrag}
% need this for including graphics (\includegraphics)
%\usepackage{graphicx}
% for neatly defining theorems and propositions
 \usepackage{amsthm}
% making logically defined graphics
%%%\usepackage{xypic}

% there are many more packages, add them here as you need them

% define commands here

\theoremstyle{definition}
\newtheorem*{thmplain}{Theorem}

\begin{document}
The coordinate planes ($yz$-plane, $zx$-plane, $xy$-plane) \PMlinkescapetext{divide} the space $\mathbb{R}^3$ into eight trihedral angles, which are called the \emph{octants} of the space.\, The magnitude of each octant as a solid angle is $\frac{\pi}{2}$.\, 

Any coordinate of a point moving in a certain octant preserves its \PMlinkname{sign}{SignumFunction}.\, One calls the octant, where all three coordinates are nonnegative, \emph{the first octant}.\, The other octants have no agreed numbering; see still \PMlinkexternal{Wikipedia}{http://de.wikipedia.org/wiki/Oktant_(Geometrie)}.
%%%%%
%%%%%
\end{document}
