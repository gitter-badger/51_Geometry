\documentclass[12pt]{article}
\usepackage{pmmeta}
\pmcanonicalname{EquiangularPolygon}
\pmcreated{2013-03-22 17:12:43}
\pmmodified{2013-03-22 17:12:43}
\pmowner{Wkbj79}{1863}
\pmmodifier{Wkbj79}{1863}
\pmtitle{equiangular polygon}
\pmrecord{10}{39535}
\pmprivacy{1}
\pmauthor{Wkbj79}{1863}
\pmtype{Definition}
\pmcomment{trigger rebuild}
\pmclassification{msc}{51-00}
\pmsynonym{equiangular}{EquiangularPolygon}
\pmrelated{BasicPolygon}

\endmetadata

\usepackage{amssymb}
\usepackage{amsmath}
\usepackage{amsfonts}
\usepackage{pstricks}
\usepackage{psfrag}
\usepackage{graphicx}
\usepackage{amsthm}
%%\usepackage{xypic}

\begin{document}
A polygon is \emph{equiangular} if all of its interior angles are congruent.

Common examples of equiangular polygons are rectangles and regular polygons such as equilateral triangles and squares.

Let $T$ be a triangle in Euclidean geometry, hyperbolic geometry, or spherical geometry.  Then the following are equivalent:

\begin{itemize}
\item $T$ is equilateral;
\item $T$ is equiangular;
\item $T$ is regular.
\end{itemize}

If $T$ is allowed to be a polygon that has more than three sides, then the above statement is no longer true in any of the indicated geometries.

Below are some pictures of equiangular polygons drawn in Euclidean geometry that are not equilateral.

\begin{center}
\begin{pspicture}(0,0)(14,5)
\pspolygon(0,0)(5,0)(5,3)(0,3)
\pspolygon(8,4.527)(6,3.0743)(7,0)(9,0)(10,3.0743)
\pspolygon(11,1.732)(12,0)(13,0)(14,1.732)(13,3.464)(12,3.464)
\end{pspicture}
\end{center}
%%%%%
%%%%%
\end{document}
