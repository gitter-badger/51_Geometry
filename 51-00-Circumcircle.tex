\documentclass[12pt]{article}
\usepackage{pmmeta}
\pmcanonicalname{Circumcircle}
\pmcreated{2013-03-22 15:00:32}
\pmmodified{2013-03-22 15:00:32}
\pmowner{yark}{2760}
\pmmodifier{yark}{2760}
\pmtitle{circumcircle}
\pmrecord{8}{36714}
\pmprivacy{1}
\pmauthor{yark}{2760}
\pmtype{Definition}
\pmcomment{trigger rebuild}
\pmclassification{msc}{51-00}
%\pmkeywords{circle}
%\pmkeywords{triangle}
%\pmkeywords{center}
%\pmkeywords{radius}
\pmrelated{Triangle}
\pmrelated{CyclicQuadrilateral}
\pmrelated{SimsonsLine}
\pmdefines{circumcenter}
\pmdefines{circumcentre}
\pmdefines{circumradius}

\usepackage{graphicx}
%%%\usepackage{xypic} 
\usepackage{bbm}
\newcommand{\Z}{\mathbbmss{Z}}
\newcommand{\C}{\mathbbmss{C}}
\newcommand{\R}{\mathbbmss{R}}
\newcommand{\Q}{\mathbbmss{Q}}
\newcommand{\mathbb}[1]{\mathbbmss{#1}}
\newcommand{\figura}[1]{\begin{center}\includegraphics{#1}\end{center}}
\newcommand{\figuraex}[2]{\begin{center}\includegraphics[#2]{#1}\end{center}}
\newtheorem{dfn}{Definition}
\begin{document}
For any triangle $ABC$ there is always
a circle passing through its three vertices.
\begin{center}
\includegraphics{circumcircle}
\end{center}

Such circle is called a \emph{circumcircle}.
Its radius is the \emph{circumradius},
and its center is the \emph{circumcenter}.
The circumcenter lies at the intersection
of the perpendicular bisectors of the sides of the triangle.

{\small
Since the perpendicular bisector of a segment
is the locus of points at the same distance from the segment endpoints,
the points on the perpendicular bisector of $AB$ are equidistant to $A$ and $B$.
The points in the perpendicular bisector of $BC$ are equidistant to $B$ and $C$,
and thus the intersection point $O$ is at the same distance from $A,B$ and $C$.}
\bigskip

In a more general setting, if $P$ is any polygon,
its circumcircle would be a circle passing through all vertices,
and circumradius and circumcenter are defined similarly.
However, unlike triangles, circumcircles need not to exist for any polygon.
For instance, a non-rectangular parallelogram has no circumcircle,
for no circle passes through the four vertices.
A quadrilateral that does possess a circumcircle
is called a cyclic quadrilateral.
%%%%%
%%%%%
\end{document}
