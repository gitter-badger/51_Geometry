\documentclass[12pt]{article}
\usepackage{pmmeta}
\pmcanonicalname{ProofOfPythagoreanTheorem1}
\pmcreated{2013-03-22 12:48:39}
\pmmodified{2013-03-22 12:48:39}
\pmowner{drini}{3}
\pmmodifier{drini}{3}
\pmtitle{proof of Pythagorean theorem}
\pmrecord{8}{33130}
\pmprivacy{1}
\pmauthor{drini}{3}
\pmtype{Proof}
\pmcomment{trigger rebuild}
\pmclassification{msc}{51-00}

\endmetadata

\usepackage{graphicx}
%%%\usepackage{xypic} 
\usepackage{bbm}
\newcommand{\Z}{\mathbbmss{Z}}
\newcommand{\C}{\mathbbmss{C}}
\newcommand{\R}{\mathbbmss{R}}
\newcommand{\Q}{\mathbbmss{Q}}
\newcommand{\mathbb}[1]{\mathbbmss{#1}}
\newcommand{\figura}[1]{\begin{center}\includegraphics{#1}\end{center}}
\newcommand{\figuraex}[2]{\begin{center}\includegraphics[#2]{#1}\end{center}}
\begin{document}
Let $ABC$ be a right triangle with hypotenuse $BC$. Draw the height $AT$.
\figura{pythaproof}

Using the right angles $\angle BAC$ and $\angle ATB$ and the fact that the sum of angles on any triangle is $180^\circ$, it can be shown that
\begin{eqnarray*}
\angle BAT &=& \angle ACT\\
\angle TAC &=& \angle CBA
\end{eqnarray*}
and therefore we have the following triangle similarities:
$$\triangle ABC \sim \triangle TBA \sim \triangle TAC.$$

From those similarities, we have
$\frac{AB}{BC}=\frac{TB}{BA}$ and thus $AB^2 = BC\cdot TB$. Also
$\frac{AC}{BC}=\frac{TC}{AC}$ and thus $AC^2= BC \cdot TC$. We have then
$$AB^2 + AC^2 = BC(BT+TC) = BC\cdot BC = BC^2$$
which concludes the proof.
%%%%%
%%%%%
\end{document}
