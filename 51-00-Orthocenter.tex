\documentclass[12pt]{article}
\usepackage{pmmeta}
\pmcanonicalname{Orthocenter}
\pmcreated{2013-03-22 11:55:41}
\pmmodified{2013-03-22 11:55:41}
\pmowner{Mathprof}{13753}
\pmmodifier{Mathprof}{13753}
\pmtitle{orthocenter}
\pmrecord{11}{30646}
\pmprivacy{1}
\pmauthor{Mathprof}{13753}
\pmtype{Definition}
\pmcomment{trigger rebuild}
\pmclassification{msc}{51-00}
\pmrelated{HeightOfATriangle}
\pmrelated{Median}
\pmrelated{Triangle}
\pmrelated{EulerLine}
\pmrelated{OrthicTriangle}
\pmrelated{CEvasTheorem}
\pmrelated{CevasTheorem}
\pmrelated{CenterOfATriangle}
\pmrelated{Incenter}
\pmrelated{TrigonometricVersionOfCevasTheorem}
\pmrelated{Centroid}
\pmdefines{orthocentric tetrad}

\endmetadata

\usepackage{amssymb}
\usepackage{amsmath}
\usepackage{amsfonts}
\usepackage{graphicx}
%%%%\usepackage{xypic}
\begin{document}
The \emph{orthocenter} of a triangle is the point of intersection of its three heights.

\begin{center}\includegraphics{ortho}
\end{center}

In the figure, $H$ is the orthocenter of $ABC$. 

The orthocenter $H$ lies inside, on a vertex or outside the triangle depending on the triangle being acute, right or obtuse respectively. Orthocenter is one of the most important triangle centers and it is very related with the orthic triangle (formed by the three height's foots). It lies on the Euler line and the four quadrilaterals $FHDB, CHEC, AFHE$ are cyclic.

In fact,

\begin{itemize}
\item $A$ is the orthocenter of $B, C, H$;
\item $B$ is the orthocenter of $A, C, H$;
\item $C$ is the orthocenter of $A, B, H$.
\end{itemize}


The four points $A, B, C$, and $H$ form what is  called
 an \emph{orthocentric tetrad}.
%%%%%
%%%%%
%%%%%
%%%%%
\end{document}
