\documentclass[12pt]{article}
\usepackage{pmmeta}
\pmcanonicalname{Symmedian}
\pmcreated{2013-03-22 12:10:55}
\pmmodified{2013-03-22 12:10:55}
\pmowner{drini}{3}
\pmmodifier{drini}{3}
\pmtitle{symmedian}
\pmrecord{6}{31445}
\pmprivacy{1}
\pmauthor{drini}{3}
\pmtype{Definition}
\pmcomment{trigger rebuild}
\pmclassification{msc}{51M99}
\pmrelated{Triangle}
\pmrelated{LemoinePoint}
\pmrelated{GergonnePoint}
\pmrelated{Isogonal}
\pmrelated{IsogonalConjugate}
\pmrelated{FundamentalTheoremOnIsogonalLines}
\pmrelated{LemoineCircle}

\usepackage{graphicx}
%%%%\usepackage{xypic} 
\usepackage{bbm}
\newcommand{\Z}{\mathbbmss{Z}}
\newcommand{\C}{\mathbbmss{C}}
\newcommand{\R}{\mathbbmss{R}}
\newcommand{\Q}{\mathbbmss{Q}}
\newcommand{\mathbb}[1]{\mathbbmss{#1}}
\newcommand{\figura}[1]{\begin{center}\includegraphics{#1}\end{center}}
\newcommand{\figuraex}[2]{\begin{center}\includegraphics[#2]{#1}\end{center}}
\begin{document}
On any triangle, the three lines obtained by reflecting the medians in the (internal) angle bisectors are called the \emph{symmedians} of the triangle.

\figura{symmed}
In the picture, $BX$ is angle bisector and $BM$ a median. The reflection of $BM$ on $BX$ is $BN$, a symmedian.

It can be stated as symmedians are isogonal conjugates of medians.
%%%%%
%%%%%
%%%%%
\end{document}
