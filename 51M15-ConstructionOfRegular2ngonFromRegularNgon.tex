\documentclass[12pt]{article}
\usepackage{pmmeta}
\pmcanonicalname{ConstructionOfRegular2ngonFromRegularNgon}
\pmcreated{2013-03-22 17:19:32}
\pmmodified{2013-03-22 17:19:32}
\pmowner{Wkbj79}{1863}
\pmmodifier{Wkbj79}{1863}
\pmtitle{construction of regular $2n$-gon from regular $n$-gon}
\pmrecord{19}{39677}
\pmprivacy{1}
\pmauthor{Wkbj79}{1863}
\pmtype{Algorithm}
\pmcomment{trigger rebuild}
\pmclassification{msc}{51M15}
\pmclassification{msc}{51-00}

\endmetadata

\usepackage{amssymb}
\usepackage{amsmath}
\usepackage{amsfonts}
\usepackage{pstricks}
\usepackage{psfrag}
\usepackage{graphicx}
\usepackage{amsthm}
%%\usepackage{xypic}

\begin{document}
\PMlinkescapeword{center}
\PMlinkescapeword{regular}

Given a \PMlinkname{regular $n$-gon}{RegularPolygon}, one can construct a regular $2n$-gon using compass and straightedge.  This procedure will be demonstrated by starting with a regular pentagon; the procedure will thus produce a regular decagon.

The procedure is as follows:

\begin{enumerate}

\item Bisect two of the interior angles of the regular polygon.  These angle bisectors will intersect at the \PMlinkname{center}{Center9} of the regular polygon.

\begin{center}
\begin{pspicture}(-4,-3)(4,4)
\pspolygon(0,3.404)(-3.238,1.052)(-2,-2.754)(2,-2.754)(3.238,1.052)
\psarc[linecolor=blue](0,3.404){0.7}{195}{345}
\psarc[linecolor=blue](-0.56636,2.9926){0.7}{315}{340}
\psarc[linecolor=blue](0.56636,2.9926){0.7}{200}{225}
\psline[linecolor=blue]{->}(0,3.404)(0,-1)
\psarc[linecolor=blue](-3.236,1.052){0.7}{270}{420}
\psarc[linecolor=blue](-2.67,1.46434){0.7}{273}{298}
\psarc[linecolor=blue](-3.0213,0.386){0.7}{28}{53}
\psline[linecolor=blue]{->}(-3.236,1.052)(1,-0.3251)
\psdots(0,3.404)(-3.238,1.052)(-2,-2.754)(2,-2.754)(3.238,1.052)(0,0)
\end{pspicture}
\end{center}

\item Connect each vertex of the regular polygon to the center.

\begin{center}
\begin{pspicture}(-4,-3)(4,4)
\pspolygon(0,3.404)(-3.238,1.052)(-2,-2.754)(2,-2.754)(3.238,1.052)
\psarc(0,3.404){0.7}{195}{345}
\psarc(-0.56636,2.9926){0.7}{315}{340}
\psarc(0.56636,2.9926){0.7}{200}{225}
\psline[linecolor=blue](0,3.404)(0,0)
\psarc(-3.236,1.052){0.7}{270}{420}
\psarc(-2.67,1.46434){0.7}{273}{298}
\psarc(-3.0213,0.386){0.7}{28}{53}
\psline[linecolor=blue](-3.236,1.052)(0,0)
\psline[linecolor=blue](-2,-2.754)(0,0)
\psline[linecolor=blue](2,-2.754)(0,0)
\psline[linecolor=blue](3.238,1.052)(0,0)
\psdots(0,3.404)(-3.238,1.052)(-2,-2.754)(2,-2.754)(3.238,1.052)(0,0)
\end{pspicture}
\end{center}

\item Construct the circumscribed circle of the regular polygon.

\begin{center}
\begin{pspicture}(-4,-4)(4,4)
\pspolygon(0,3.404)(-3.238,1.052)(-2,-2.754)(2,-2.754)(3.238,1.052)
\psarc(0,3.404){0.7}{195}{345}
\psarc(-0.56636,2.9926){0.7}{315}{340}
\psarc(0.56636,2.9926){0.7}{200}{225}
\psline(0,3.404)(0,0)
\psarc(-3.236,1.052){0.7}{270}{420}
\psarc(-2.67,1.46434){0.7}{273}{298}
\psarc(-3.0213,0.386){0.7}{28}{53}
\psline(-3.236,1.052)(0,0)
\psline(-2,-2.754)(0,0)
\psline(2,-2.754)(0,0)
\psline(3.238,1.052)(0,0)
\pscircle[linecolor=blue](0,0){3.404}
\psdots(0,3.404)(-3.238,1.052)(-2,-2.754)(2,-2.754)(3.238,1.052)(0,0)
\end{pspicture}
\end{center}

\item Bisect each of the central angles of the circle to obtain the points where the angle bisectors intersect the circle.

\begin{center}
\begin{pspicture}(-4,-4)(4,4)
\pspolygon(0,3.404)(-3.238,1.052)(-2,-2.754)(2,-2.754)(3.238,1.052)
\psarc(0,3.404){0.7}{195}{345}
\psarc(-0.56636,2.9926){0.7}{315}{340}
\psarc(0.56636,2.9926){0.7}{200}{225}
\psline(0,3.404)(0,0)
\psarc(-3.236,1.052){0.7}{270}{420}
\psarc(-2.67,1.46434){0.7}{273}{298}
\psarc(-3.0213,0.386){0.7}{28}{53}
\psline(-3.236,1.052)(0,0)
\psline(-2,-2.754)(0,0)
\psline(2,-2.754)(0,0)
\psline(3.238,1.052)(0,0)
\pscircle(0,0){3.404}
\pscircle[linecolor=blue](0,0){0.7}
\psarc[linecolor=blue](0,0.7){0.7}{135}{195}
\psarc[linecolor=blue](-0.6657,0.2164){0.7}{60}{120}
\psline[linecolor=blue](0,0)(-2,2.754)
\psarc[linecolor=blue](-0.6657,0.2164){0.7}{210}{270}
\psarc[linecolor=blue](-0.41133,-0.5664){0.7}{140}{200}
\psline[linecolor=blue](0,0)(-3.238,-1.052)
\psarc[linecolor=blue](-0.41133,-0.5664){0.7}{270}{330}
\psarc[linecolor=blue](0.41133,-0.5664){0.7}{210}{270}
\psline[linecolor=blue](0,0)(0,-3.404)
\psarc[linecolor=blue](0.41133,-0.5664){0.7}{-20}{40}
\psarc[linecolor=blue](0.6657,0.2164){0.7}{270}{330}
\psline[linecolor=blue](0,0)(3.238,-1.052)
\psarc[linecolor=blue](0.6657,0.2164){0.7}{60}{120}
\psarc[linecolor=blue](0,0.7){0.7}{-15}{45}
\psline[linecolor=blue](0,0)(2,2.754)
\psdots(0,3.404)(-3.238,1.052)(-2,-2.754)(2,-2.754)(3.238,1.052)(0,0)(0,-3.404)(-3.238,-1.052)(-2,2.754)(2,2.754)(3.238,-1.052)
\end{pspicture}
\end{center}

\item Connect the dots to form the regular $2n$-gon.  In the picture below, all drawn figures except for the original polygon, the circle, and the formed polygon are drawn in cyan to emphasize the three figures that are not dashed.

\begin{center}
\begin{pspicture}(-4,-4)(4,4)
\psarc[linecolor=cyan](0,3.404){0.7}{195}{345}
\psarc[linecolor=cyan](-0.56636,2.9926){0.7}{315}{340}
\psarc[linecolor=cyan](0.56636,2.9926){0.7}{200}{225}
\psline[linecolor=cyan](0,3.404)(0,0)
\psarc[linecolor=cyan](-3.236,1.052){0.7}{270}{420}
\psarc[linecolor=cyan](-2.67,1.46434){0.7}{273}{298}
\psarc[linecolor=cyan](-3.0213,0.386){0.7}{28}{53}
\psline[linecolor=cyan](-3.236,1.052)(0,0)
\psline[linecolor=cyan](-2,-2.754)(0,0)
\psline[linecolor=cyan](2,-2.754)(0,0)
\psline[linecolor=cyan](3.238,1.052)(0,0)
\pscircle[linecolor=cyan](0,0){0.7}
\psarc[linecolor=cyan](0,0.7){0.7}{135}{195}
\psarc[linecolor=cyan](-0.6657,0.2164){0.7}{60}{120}
\psline[linecolor=cyan](0,0)(-2,2.754)
\psarc[linecolor=cyan](-0.6657,0.2164){0.7}{210}{270}
\psarc[linecolor=cyan](-0.41133,-0.5664){0.7}{140}{200}
\psline[linecolor=cyan](0,0)(-3.238,-1.052)
\psarc[linecolor=cyan](-0.41133,-0.5664){0.7}{270}{330}
\psarc[linecolor=cyan](0.41133,-0.5664){0.7}{210}{270}
\psline[linecolor=cyan](0,0)(0,-3.404)
\psarc[linecolor=cyan](0.41133,-0.5664){0.7}{-20}{40}
\psarc[linecolor=cyan](0.6657,0.2164){0.7}{270}{330}
\psline[linecolor=cyan](0,0)(3.238,-1.052)
\psarc[linecolor=cyan](0.6657,0.2164){0.7}{60}{120}
\psarc[linecolor=cyan](0,0.7){0.7}{-15}{45}
\psline[linecolor=cyan](0,0)(2,2.754)
\pspolygon(0,3.404)(-3.238,1.052)(-2,-2.754)(2,-2.754)(3.238,1.052)
\pscircle(0,0){3.404}
\pspolygon[linecolor=blue](0,3.404)(-2,2.754)(-3.238,1.052)(-3.238,-1.052)(-2,-2.754)(0,-3.404)(2,-2.754)(3.238,-1.052)(3.238,1.052)(2,2.754)
\psdots(0,3.404)(-3.238,1.052)(-2,-2.754)(2,-2.754)(3.238,1.052)(0,0)(0,-3.404)(-3.238,-1.052)(-2,2.754)(2,2.754)(3.238,-1.052)
\end{pspicture}
\end{center}

\end{enumerate}

This construction is justified because the triangles formed by the drawn radii of the circle and the drawn (blue) polygon are congruent by SAS (note that all of the central angles have \PMlinkname{measure}{AngleMeasure} $\frac{360^{\circ}}{2n}$), giving that all of the sides and all of the interior angles of the drawn polygon are congruent.

If you are interested in seeing the rules for compass and straightedge constructions, click on the \PMlinkescapetext{link} provided.
%%%%%
%%%%%
\end{document}
