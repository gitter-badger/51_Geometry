\documentclass[12pt]{article}
\usepackage{pmmeta}
\pmcanonicalname{EulerLine}
\pmcreated{2013-03-22 11:44:25}
\pmmodified{2013-03-22 11:44:25}
\pmowner{drini}{3}
\pmmodifier{drini}{3}
\pmtitle{Euler line}
\pmrecord{18}{30155}
\pmprivacy{1}
\pmauthor{drini}{3}
\pmtype{Theorem}
\pmcomment{trigger rebuild}
\pmclassification{msc}{51-00}
\pmclassification{msc}{58A05}
\pmclassification{msc}{83E15}
\pmclassification{msc}{83D05}
\pmclassification{msc}{83E05}
\pmclassification{msc}{83E50}
\pmclassification{msc}{81-00}
\pmclassification{msc}{83-00}
\pmclassification{msc}{82-00}
\pmrelated{Triangle}
\pmrelated{Orthocenter}
\pmrelated{Centroid}
\pmrelated{Collinear}
\pmrelated{Midpoint}
\pmrelated{OrthicTriangle}
\pmrelated{CenterOfATriangle}
\pmrelated{EulerLineProof}

\usepackage{amssymb}
\usepackage{amsmath}
\usepackage{amsfonts}
\usepackage{graphicx}
%%%%%%%\usepackage{xypic}
\begin{document}
In any triangle, the orthocenter $H$, the centroid $G$ and the circumcenter $O$ are collinear, and $OG/GH=1/2$. The line passing by these points is known as the \emph{Euler line} of the triangle. 
\medskip

This line also passes by the center of the nine-point circle (or Feuerbach circle) $N$, and $N$ is the midpoint of $OH$.\smallskip

\begin{center}
\includegraphics{eulin}
\end{center}
%%%%%
%%%%%
%%%%%
%%%%%
%%%%%
%%%%%
%%%%%
\end{document}
