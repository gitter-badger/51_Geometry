\documentclass[12pt]{article}
\usepackage{pmmeta}
\pmcanonicalname{Bspline}
\pmcreated{2013-03-22 17:21:40}
\pmmodified{2013-03-22 17:21:40}
\pmowner{CWoo}{3771}
\pmmodifier{CWoo}{3771}
\pmtitle{B-spline}
\pmrecord{28}{39721}
\pmprivacy{1}
\pmauthor{CWoo}{3771}
\pmtype{Definition}
\pmcomment{trigger rebuild}
\pmclassification{msc}{51N05}
\pmclassification{msc}{65D10}
\pmclassification{msc}{65D07}
\pmsynonym{basic spline}{Bspline}
\pmrelated{BezierCurve}
\pmrelated{Nurbs}
\pmdefines{knot sequence}
\pmdefines{knot vector}
\pmdefines{B-spline basis function}
\pmdefines{B-spline function}
\pmdefines{control points}
\pmdefines{B-spline curve}

% this is the default PlanetMath preamble.  as your knowledge
% of TeX increases, you will probably want to edit this, but
% it should be fine as is for beginners.

% almost certainly you want these
\usepackage{amssymb}
\usepackage{amsmath}
\usepackage{amsfonts}

% used for TeXing text within eps files
%\usepackage{psfrag}
% need this for including graphics (\includegraphics)
%\usepackage{graphicx}
% for neatly defining theorems and propositions
\usepackage{amsthm}
% making logically defined graphics
%%\usepackage{xypic}
\usepackage{pstricks}
\usepackage{pst-plot}

% there are many more packages, add them here as you need them

% define commands here

\begin{document}
\subsubsection*{B-spline basis functions}

Let $\boldsymbol{s}:=\lbrace s_i\mid i\in \mathbb{Z}\rbrace$ be a sequence of nondecreasing real numbers ($s_i\le s_{i+1}$).  For each $i$, we recursively define a set of real-valued functions $N_{i,j}$ (for $j=0,1,\ldots$) as follows:
\begin{eqnarray*}
N_{i,0}(t)&:=&\left\{
\begin{array}{ll}
1 & \mbox{ if } s_i\leq t<s_{i+1}\\
0 & \mbox{ otherwise }
\end{array}\right.\\
\\
N_{i,j+1}(t)&:=&\alpha_{i,j+1}(t) N_{i,j}(t) + [1-\alpha_{i+1,j+1}(t)] N_{i+1,j}(t)
\end{eqnarray*}
where
\begin{eqnarray*}
\alpha_{i,j}(t) &:=& \left\{
\begin{array}{ll}
\displaystyle{\frac{t - s_i}{s_{i+j} - s_i}} & \mbox{ if } s_{i+j}\ne s_i \\
0 & \mbox{ otherwise. }
\end{array}\right.
\end{eqnarray*}
\textbf{Definitions}.  Using the notations above:
\begin{enumerate}
\item
the sequence $\boldsymbol{s}$ is known as a \emph{knot sequence}, and the individual term in the sequence is a \emph{knot};  
\item the functions $N_{i,j}$ are called the \emph{$i$-th B-spline basis functions of order $j$}, and the recurrence relation is called the \emph{de Boor recurrence relation}, after its discoverer Carl de Boor;
\item 
given any non-negative integer $j$, the vector space $V_j(\boldsymbol{s})$ over $\mathbb{R}$, generated by the set of all B-spline basis functions of order $j$ is called the \emph{B-spline of order $j$}.  In other words, the B-spline $V_j(\boldsymbol{s}):=\operatorname{span}\lbrace N_{i,j}(t)\mid i=0,1,\ldots \rbrace$ over $\mathbb{R}$.  
\item 
Any element of $V_j(\boldsymbol{s})$ is a \emph{B-spline function} of order $j$.
\end{enumerate}

The $0$th order B-spline basis functions are nothing more than characteristic functions on half-open intervals (or the zero function if $s_i=s_{i+1}$).  When $j=1,2,3$, the B-spline basis functions are said to be \emph{linear}, \emph{quadratic}, or \emph{cubic}.

The calculations of the higher order B-spline basis functions can be easily understood by use of triangular difference tables.  For example, to calculate $N_{53}$, one would use
$$
\xymatrix @R=1pt @C=1.5cm {
s_5\! & \! N_{5,0} \ar@{->}[rd] & & &\\
 & & \! N_{5,1} \ar@{->}[rd] & &\\
s_6\! & \! N_{6,0} \ar@{->}[rd] \ar@{->}[ru] & & N_{5,2} \ar@{->}[rd] &\\
 & & \! N_{6,1} \ar@{->}[rd] \ar@{->}[ru] & & N_{5,3}\\
s_7\! & \! N_{7,0} \ar@{->}[rd] \ar@{->}[ru] & & N_{6,2} \ar@{->}[ru] &\\
 & & \! N_{7,1} \ar@{->}[ru] & &\\
s_8\! & \! N_{8,0} \ar@{->}[ru] & & &
}
$$
\textbf{Remarks}.
\begin{itemize}
\item Each B-spline basis function $N_{i,j}$ is completely defined by the finite set $\lbrace s_i,\ldots, s_{i+j+1}\rbrace$ of knots.  Furthermore, it is 
\begin{enumerate}
\item non-zero in the open interval $(s_i,s_{i+j+1})$,
\item $N_{i,j}$ restricted to each subinterval $[s_k,s_{k+1})$ is a polynomial function of degree $j$ for $k=i,\ldots, i+j$, and 
\item identically $0$ outside the interval $[s_i,s_{i+j+1})$.
\end{enumerate}
\item In $V_j(\boldsymbol{s})$, the non-zero B-spline basis functions of order $j$ are linearly independent.  In other words, the set of all non-zero $N_{i,j}$ forms a basis for $V_j(\boldsymbol{s})$ and hence the name B-spline \emph{basis} functions.
\item Given a knot sequence $\boldsymbol{s}$, a knot $s_i$ is said to be \emph{knotted} if $s_i=s_{i+1}$.  It is a double knot if $s_{i-1}<s_i=s_{i+1}<s_{i+2}$.  Triple knots and more generally $n$-multi knots are defined analogously.  A knot sequence is knotted if it contains a knotted knot.
\item If a knot sequence is not knotted, then it can be shown that the B-spline basis function $N_{i,j}$ is $C^{j-1}$ but not of $C^j$.  For example, $N_{i,1}$ is continuous but not differentiable.
\item If a knot sequence $\boldsymbol{s}$ is finite and not knotted, then $V_j(\boldsymbol{s})$ is finite dimensional.  If $\boldsymbol{s}$ has $n$ knots, then $V_j(\boldsymbol{s})$ has dimension $n-j$ for $j<n$, and $0$ otherwise.
\end{itemize}

\subsubsection*{B-spline curves}

In most applications of B-splines, finite knot sequences are considered.  A finite knot sequence $\lbrace s_0,\ldots, s_n\rbrace$ is also known as a \emph{knot vector}.

Let $\boldsymbol{s}=\lbrace s_0,\ldots, s_n\rbrace$ be a knot vector.  Let $P_0,\ldots, P_m$ be $m+1$ points in $\mathbb{R}^k$ (usually $k=2$ or $3$).  Then we may form a linear combination 
$$P(t)=\sum_{i=0}^m P_i N_{i,j}(t),$$ where $P_i N_{i,j}(t)$ is the scalar multiplication of the scalar $N_{i,j}(t)$ by the vector $P_i$.  This is possible only when there are at least as many $N_{i,j}$ (there are $n+1-j$ of them) as there are $P_i$ (there are $m$ of them).  In other words, $m\le n+1-j$.  Set $p=n+1-m$.  Then the function defined by 
$$P(t)=\sum_{i=0}^m P_i N_{i,p}(t)$$ is called a \emph{B-spline curve} with \emph{control points} $P_1,\ldots, P_m$.  Note that $P(t)$ is completely determined by the control points and the knot vector $\boldsymbol{s}$, and each coordinate of $P(t)$ is a B-spline function in $V_p(\boldsymbol{s})$.

More to come...

% \textbf{Examples}.  Let $\boldsymbol{s}=\lbrace 0,1,2,3,4,5\rbrace$.\subsubsection{B-spline basis functions}
%%%%%
%%%%%
\end{document}
