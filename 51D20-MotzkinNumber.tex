\documentclass[12pt]{article}
\usepackage{pmmeta}
\pmcanonicalname{MotzkinNumber}
\pmcreated{2013-03-22 16:19:36}
\pmmodified{2013-03-22 16:19:36}
\pmowner{PrimeFan}{13766}
\pmmodifier{PrimeFan}{13766}
\pmtitle{Motzkin number}
\pmrecord{4}{38455}
\pmprivacy{1}
\pmauthor{PrimeFan}{13766}
\pmtype{Definition}
\pmcomment{trigger rebuild}
\pmclassification{msc}{51D20}

\endmetadata

% this is the default PlanetMath preamble.  as your knowledge
% of TeX increases, you will probably want to edit this, but
% it should be fine as is for beginners.

% almost certainly you want these
\usepackage{amssymb}
\usepackage{amsmath}
\usepackage{amsfonts}

% used for TeXing text within eps files
%\usepackage{psfrag}
% need this for including graphics (\includegraphics)
%\usepackage{graphicx}
% for neatly defining theorems and propositions
%\usepackage{amsthm}
% making logically defined graphics
%%%\usepackage{xypic}

% there are many more packages, add them here as you need them

% define commands here

\begin{document}
A {\em Motzkin number} for a given number $n$ is the number of different ways of drawing non-intersecting chords on a circle between $n$ points. The Motzkin numbers have very diverse applications in geometry, combinatorics and number theory. The first few Motzkin numbers are 1, 1, 2, 4, 9, 21, 51, 127, 323, 835, (sequence A001006 in the OEIS).

The Motzkin number for $n$ is also the number of positive integer sequences $n - 1$ long in which the opening and ending elements are either 1 or 2, and the difference between any two consecutive elements is -1, 0 or 1.

Also on the upper right quadrant of a grid, the Motzkin number for $n$ gives the number of routes from coordinate (0, 0) to coordinate ($n$, 0) if one is allowed to move only to the right (either up, down or straight) at each step but forbidden from dipping below the $y = 0$ axis.

All together, there are at least fourteen different manifestations of Motzkin numbers in different branches of mathematics, as enumerated by Donaghey and Shapiro in their 1977 survey of Motzkin numbers.
%%%%%
%%%%%
\end{document}
