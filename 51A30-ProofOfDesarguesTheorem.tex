\documentclass[12pt]{article}
\usepackage{pmmeta}
\pmcanonicalname{ProofOfDesarguesTheorem}
\pmcreated{2013-03-22 13:47:51}
\pmmodified{2013-03-22 13:47:51}
\pmowner{drini}{3}
\pmmodifier{drini}{3}
\pmtitle{proof of Desargues' theorem}
\pmrecord{5}{34514}
\pmprivacy{1}
\pmauthor{drini}{3}
\pmtype{Proof}
\pmcomment{trigger rebuild}
\pmclassification{msc}{51A30}

\endmetadata

\usepackage{amssymb}
\usepackage{amsmath}
\usepackage{amsfonts}
\begin{document}
The claim is that if triangles $ABC$ and $XYZ$ are perspective from
a point $P$, then they are perspective from a line (meaning that
the three points
$$AB\cdot XY\qquad BC\cdot YZ\qquad CA\cdot ZX$$
are collinear) and conversely.

Since no three of $A,B,C,P$ are collinear, we can lay down
homogeneous coordinates such that
$$A=(1,0,0)\qquad B=(0,1,0)\qquad C=(0,0,1)\qquad P=(1,1,1)$$
By hypothesis, there are scalars $p,q,r$ such that
$$X=(1,p,p)\qquad Y=(q,1,q)\qquad Z=(r,r,1)$$
The equation for a line through $(x_1,y_1,z_1)$ and
$(x_2,y_2,z_2)$ is
$$(y_1z_2-z_1y_2)x+(z_1x_2-x_1z_2)y+(x_1y_2-y_1x_2)z=0\;,$$
giving us equations for six lines:
\begin{eqnarray*}
AB&:&z=0 \\ BC&:&x=0 \\ CA&:&y=0 \\
XY&:&(pq-p)x+(pq-q)y+(1-pq)z=0 \\
YZ&:&(1-qr)x+(qr-q)y+(qr-r)z=0 \\
ZX&:&(rp-p)x+(1-rp)y+(rp-r)z=0
\end{eqnarray*}
whence
\begin{eqnarray*}
AB\cdot XY&=&(pq-q,-pq+p,0)\\
BC\cdot YZ&=&(0,qr-r,-qr+q)\\
CA\cdot ZX&=&(-rp+r,0,rp-p).
\end{eqnarray*}
As claimed, these three points are collinear, since the
determinant
$$ \left| \begin{array}{ccc}
pq-q & -pq+p & 0 \\
0 & qr-r & -qr+q \\
-rp+r & 0 & rp-p
\end{array}\right| $$
is zero. (More precisely, all three points are on the line
$$p(q-1)(r-1)x+(p-1)q(r-1)y+(p-1)(q-1)rz=0\;.)$$

Since the hypotheses are self-dual, the converse is true also, by
the principle of duality.
%%%%%
%%%%%
\end{document}
