\documentclass[12pt]{article}
\usepackage{pmmeta}
\pmcanonicalname{MedianOfTrapezoid}
\pmcreated{2013-03-22 17:46:44}
\pmmodified{2013-03-22 17:46:44}
\pmowner{pahio}{2872}
\pmmodifier{pahio}{2872}
\pmtitle{median of trapezoid}
\pmrecord{6}{40237}
\pmprivacy{1}
\pmauthor{pahio}{2872}
\pmtype{Theorem}
\pmcomment{trigger rebuild}
\pmclassification{msc}{51M04}
\pmclassification{msc}{51M25}
\pmrelated{MutualPositionsOfVectors}
\pmrelated{MidSegmentTheorem}
\pmrelated{TriangleMidSegmentTheorem}
\pmrelated{HarmonicMeanInTrapezoid}

\endmetadata

% this is the default PlanetMath preamble.  as your knowledge
% of TeX increases, you will probably want to edit this, but
% it should be fine as is for beginners.

% almost certainly you want these
\usepackage{amssymb}
\usepackage{amsmath}
\usepackage{amsfonts}

% used for TeXing text within eps files
%\usepackage{psfrag}
% need this for including graphics (\includegraphics)
%\usepackage{graphicx}
% for neatly defining theorems and propositions
 \usepackage{amsthm}
% making logically defined graphics
%%%\usepackage{xypic}

% there are many more packages, add them here as you need them

\usepackage{pstricks}

% define commands here

\theoremstyle{definition}
\newtheorem*{thmplain}{Theorem}

\begin{document}
\PMlinkescapeword{legs}
The segment connecting the midpoints of the \PMlinkname{legs}{Trapezoid} of a trapezoid, i.e. the median of the trapezoid, is parallel to the bases and its length equals the arithmetic mean of the legs.\\

{\em Proof.}\, Let $AB$ and $CD$ be the bases of a trapezoid $ABCD$ and $E$ the midpoint of the leg $AD$ and $F$ the midpoint of the leg $BC$.\, Then the median $EF$ may be determined as vector as follows:
\begin{align*}
\overrightarrow{EF} &= \overrightarrow{ED}+\overrightarrow{DC}+\overrightarrow{CF}\\ 
& = \frac{1}{2}\overrightarrow{AD}+\overrightarrow{DC}+\frac{1}{2}\overrightarrow{CB}\\ 
& = \frac{1}{2}(\overrightarrow{AD}+\overrightarrow{DC}+\overrightarrow{CB})+\frac{1}{2}\overrightarrow{DC}\\
& = \frac{1}{2}\overrightarrow{AB}+\frac{1}{2}\overrightarrow{DC}\\
& = \frac{1}{2}(\overrightarrow{AB}+\overrightarrow{DC})
\end{align*}
The last expression tells that\; 
$\overrightarrow{EF} \parallel \overrightarrow{AB}+\overrightarrow{DC} \parallel \overrightarrow{AB}$\; and\; 
$\displaystyle|\overrightarrow{EF}| = \frac{|\overrightarrow{AB}\!+\!\overrightarrow{DC}|}{2} = 
\frac{|\overrightarrow{AB}|\!+\!|\overrightarrow{DC}|}{2}$.\; Q.E.D.

\begin{center}
\begin{pspicture}(-1,-1)(7,4)
\psline[linecolor=blue](0,0)(6,0)
\psline(6,0)(5,4)
\psline[linecolor=blue](2,4)(5,4)
\psline(2,4)(0,0)
\psline[linecolor=red](1,2)(5.5,2)
\psdot[linecolor=red](1,2)
\psdot[linecolor=red](5.5,2)
\rput[a](0,-0.3){$A$}
\rput[a](6,-0.3){$B$}
\rput[a](5,4.3){$C$}
\rput[a](2,4.3){$D$}
\rput[a](0.74,2){$E$}
\rput[a](5.7,2){$F$}
\rput[a](0.75,1){$p$}
\rput[a](1.74,3){$p$}
\rput[a](5.55,1){$q$}
\rput[a](5.05,3){$q$}
\end{pspicture}
\end{center}


%%%%%
%%%%%
\end{document}
