\documentclass[12pt]{article}
\usepackage{pmmeta}
\pmcanonicalname{Inscribe}
\pmcreated{2013-03-22 16:24:35}
\pmmodified{2013-03-22 16:24:35}
\pmowner{PrimeFan}{13766}
\pmmodifier{PrimeFan}{13766}
\pmtitle{inscribe}
\pmrecord{5}{38558}
\pmprivacy{1}
\pmauthor{PrimeFan}{13766}
\pmtype{Definition}
\pmcomment{trigger rebuild}
\pmclassification{msc}{51D20}
\pmsynonym{inscribed}{Inscribe}
\pmsynonym{inscription}{Inscribe}
\pmrelated{RegularPolygonAndCircles}
\pmrelated{Circumscribe}

\endmetadata

% this is the default PlanetMath preamble.  as your knowledge
% of TeX increases, you will probably want to edit this, but
% it should be fine as is for beginners.

% almost certainly you want these
\usepackage{amssymb}
\usepackage{amsmath}
\usepackage{amsfonts}

% used for TeXing text within eps files
%\usepackage{psfrag}

% need this for including graphics (\includegraphics)
\usepackage{graphicx}

% for neatly defining theorems and propositions
%\usepackage{amsthm}
% making logically defined graphics
%%%\usepackage{xypic}

% there are many more packages, add them here as you need them

% define commands here

\begin{document}
To \emph{inscribe} a polygon (such as a triangle, square, pentagon, etc.) or circle is to enclose it in another polygon or circle so that the vertices of the polygon are on the circumference of the circle or sides of the outer polygon, or so that the circumference of the inner circle touches the sides of the enclosing polygon. In the specific case of a polygon inside of a circle, this is called to circumscribe.

Now, a couple of examples: First, a circle inscribed in a pentagon.

\begin{center}
\includegraphics{CircleInPentagon}
\end{center}

Next, another example of a pentagon, but inscribed with a triangle this time.

\begin{center}
\includegraphics{TriangleInPentagon}
\end{center}

%%%%%
%%%%%
\end{document}
