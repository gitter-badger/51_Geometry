\documentclass[12pt]{article}
\usepackage{pmmeta}
\pmcanonicalname{SphericalCoordinates}
\pmcreated{2013-05-08 10:16:57}
\pmmodified{2013-05-08 10:16:57}
\pmowner{yark}{2760}
\pmmodifier{unlord}{1}
\pmtitle{spherical coordinates}
\pmrecord{17}{34491}
\pmprivacy{1}
\pmauthor{yark}{1}
\pmtype{Definition}
\pmcomment{try to get sidebars to show up}
\pmclassification{msc}{51M05}
%\pmkeywords{sphere}
\pmrelated{Sphere}
\pmrelated{CylindricalCoordinates}
\pmdefines{hyperspherical coordinates}
\pmdefines{azimuthal angle}
\pmdefines{polar angle}

\endmetadata

\usepackage{amssymb}
\usepackage{amsmath}
\usepackage{amsfonts}
\usepackage{amsthm}

\newcommand{\mc}{\mathcal}
\newcommand{\mb}{\mathbb}
\newcommand{\mf}{\mathfrak}
\newcommand{\ol}{\overline}
\newcommand{\ra}{\rightarrow}
\newcommand{\la}{\leftarrow}
\newcommand{\La}{\Leftarrow}
\newcommand{\Ra}{\Rightarrow}
\newcommand{\nor}{\vartriangleleft}
\newcommand{\Gal}{\text{Gal}}
\newcommand{\GL}{\text{GL}}
\newcommand{\Z}{\mb{Z}}
\newcommand{\R}{\mb{R}}
\newcommand{\Q}{\mb{Q}}
\newcommand{\C}{\mb{C}}
\newcommand{\<}{\langle}
\renewcommand{\>}{\rangle}
\begin{document}
\PMlinkescapeword{bottom}
\PMlinkescapeword{polar}
\PMlinkescapeword{top}

\emph{Spherical coordinates} are a system of coordinates for $\R^3$,
or more generally $\R^n$.
One coordinate is the distance from the origin,
which can be thought of as
the radius of the sphere centred at the origin on which the point lies.
The other coordinates are angles that specify the position of the point on this sphere.

In $\R^3$ the coordinates are given by
\begin{align*}
\left(
\begin{array}{c}
x\\
y\\
z
\end{array}\right) &=
\left(
\begin{array}{c}
r\sin\phi\cos\theta\\
r\sin\phi\sin\theta\\
r\cos\phi
\end{array}
\right),
\end{align*}
where $r$ is the distance from the origin,
$\theta$ is the \emph{azimuthal angle} defined for $\theta\in[0,2\pi)$,
and $\phi\in[0,\pi]$ is the \emph{polar angle}.
Note that $\phi=0$ corresponds to the top of the sphere and $\phi=\pi$ corresponds to the bottom of the sphere.
There is a clash between the mathematicians' and the physicists' definition of spherical coordinates, interchanging both the direction of $\phi$ and the choice of names for the two angles (physicists often use $\theta$ as the azimuthal angle and $\phi$ as the polar one).

Spherical coordinates are a generalization of polar coordinates,
and can be further generalized to $\R^n$,
with $n-2$ polar angles $\phi_1,\ldots,\phi_{n-2}$ and one azimuthal angle $\theta$:
\begin{align*}
\left(
\begin{array}{c}
x_1\\
x_2\\
\vdots\\
x_k\\
\vdots\\
x_{n-1}\\
x_n
\end{array}\right)&=
\left(
\begin{array}{c}
r\cos\phi_1\\
r\sin\phi_1\cos\phi_2\\
\vdots\\
r\left(\prod_{i=1}^{k-1}\sin\phi_i\right)\cos\phi_k\\
\vdots\\
r\sin\phi_1\sin\phi_2\cdots\cos\theta\\
r\sin\phi_1\sin\phi_2\cdots\sin\phi_{n-2}\sin\theta.
\end{array}
\right).
\end{align*}

These are sometimes called \emph{hyperspherical coordinates} if $n>3$.
\end{document}
