\documentclass[12pt]{article}
\usepackage{pmmeta}
\pmcanonicalname{UnitHyperbola}
\pmcreated{2015-02-04 11:10:22}
\pmmodified{2015-02-04 11:10:22}
\pmowner{pahio}{2872}
\pmmodifier{pahio}{2872}
\pmtitle{unit hyperbola}
\pmrecord{24}{35996}
\pmprivacy{1}
\pmauthor{pahio}{2872}
\pmtype{Definition}
\pmcomment{trigger rebuild}
\pmclassification{msc}{51N20}
\pmrelated{HyperbolicFunctions}
\pmrelated{AreaFunctions}
\pmrelated{ConjugateHyperbola}

% this is the default PlanetMath preamble.  as your knowledge
% of TeX increases, you will probably want to edit this, but
% it should be fine as is for beginners.

% almost certainly you want these
\usepackage{amssymb}
\usepackage{amsmath}
\usepackage{amsfonts}

% used for TeXing text within eps files
%\usepackage{psfrag}
% need this for including graphics (\includegraphics)
\usepackage{graphicx}
% for neatly defining theorems and propositions
%\usepackage{amsthm}
% making logically defined graphics
%%%\usepackage{xypic}

% there are many more packages, add them here as you need them

% define commands here
\begin{document}
The {\em unit hyperbola} (cf. the unit circle) is the special case
            $$x^2-y^2 = 1$$
of the hyperbola
         $$\frac{x^2}{a^2}-\frac{y^2}{b^2} = 1$$
where both the \PMlinkescapetext{{\em transverse semiaxis}} $a$ and the \PMlinkescapetext{{\em conjugate semiaxis}} $b$ have \PMlinkescapetext{length} equal to 1.\, The unit hyperbola is {\em rectangular}, i.e. its asymptotes ($y = \pm x$) are at right angles to each other.
\begin{center}
\includegraphics{unithyperola}
\end{center}

The unit hyperbola has the well-known parametric \PMlinkescapetext{representation}
      $$x = \pm\cosh{t},  \quad y = \sinh{t},$$
and also a trigonometric \PMlinkescapetext{representation}
      $$x = \sec{t},  \quad y = \tan{t}.$$
The former yields the rational \PMlinkescapetext{representation}
      $$x = \frac{u^2+1}{2u},  \quad y = \frac{u^2-1}{2u}$$
when one substitutes \,$e^t = u$, and the latter, via the substitution \,$\tan\frac{t}{2} = u$, the rational \PMlinkescapetext{representation}
      $$x = \frac{1+u^2}{1-u^2},  \quad y = \frac{2u}{1-u^2}$$
(which does not give the left apex of the hyperbola).
%%%%%
%%%%%
\end{document}
