\documentclass[12pt]{article}
\usepackage{pmmeta}
\pmcanonicalname{TriangleMidsegmentTheorem}
\pmcreated{2013-03-22 17:46:35}
\pmmodified{2013-03-22 17:46:35}
\pmowner{pahio}{2872}
\pmmodifier{pahio}{2872}
\pmtitle{triangle mid-segment theorem}
\pmrecord{12}{40234}
\pmprivacy{1}
\pmauthor{pahio}{2872}
\pmtype{Theorem}
\pmcomment{trigger rebuild}
\pmclassification{msc}{51M04}
\pmclassification{msc}{51M25}
\pmsynonym{mid-segment theorem}{TriangleMidsegmentTheorem}
\pmrelated{MutualPositionsOfVectors}
\pmrelated{ParallelogramTheorems}
\pmrelated{MedianOfTrapezoid}
\pmrelated{CommonPointOfTriangleMedians}
\pmrelated{Grafix}
\pmrelated{SimonStevin}
\pmrelated{InterceptTheorem}

% this is the default PlanetMath preamble.  as your knowledge
% of TeX increases, you will probably want to edit this, but
% it should be fine as is for beginners.

% almost certainly you want these
\usepackage{amssymb}
\usepackage{amsmath}
\usepackage{amsfonts}

% used for TeXing text within eps files
%\usepackage{psfrag}
% need this for including graphics (\includegraphics)
%\usepackage{graphicx}
% for neatly defining theorems and propositions
 \usepackage{amsthm}
% making logically defined graphics
%%%\usepackage{xypic}

% there are many more packages, add them here as you need them

\usepackage{pstricks}

% define commands here

\theoremstyle{definition}
\newtheorem*{thmplain}{Theorem}

\begin{document}
\textbf{Theorem.}\, The segment connecting the midpoints of any two sides of a triangle is parallel to the third side and is half as long.

\begin{center}
\begin{pspicture}(-1,-0.5)(5,5.5)
\psline(0,0)(4,0)
\psline[arrows=->,arrowsize=5pt,linecolor=blue](3,5)(4,0)
\rput[a](-0.3,-0.1){$A$}
\rput[a](4.2,-0.1){$B$}
\rput[a](2.9,5.24){$C$}
\rput[a](1.3,2.7){$A'$}
\rput[a](3.8,2.69){$B'$}
\psdot[linecolor=blue](1.5,2.5)
\psdot[linecolor=blue](3.5,2.5)
\psline[arrows=->,arrowsize=5pt,linecolor=red](1.5,2.5)(3.5,2.5)
\psline[arrows=->,arrowsize=5pt,linecolor=blue](0,0)(3,5)
\end{pspicture}
\end{center}

{\em Proof.}\, In the triangle $ABC$, let $A'$ be the midpoint of $AC$ and $B'$ the midpoint of $BC$.\, Using the side-vectors $\overrightarrow{AC}$ and $\overrightarrow{CB}$ as a \PMlinkname{basis}{Basis} of the plane, we calculate the mid-segment $A'B'$ as a vector:
$$\overrightarrow{A'B'} \,=\, \overrightarrow{A'C}+\overrightarrow{CB'} \,=\, 
\frac{1}{2}\overrightarrow{AC}+\frac{1}{2}\overrightarrow{CB} \,=\, \frac{1}{2}(\overrightarrow{AC}+\overrightarrow{CB}) 
\,=\, \frac{1}{2}\overrightarrow{AB}$$
The last expression indicates that the segment $A'B'$ is such as asserted.\\

\textbf{Corollary (Varignon's theorem).}\, If one connects the midpoints of the \PMlinkescapetext{adjacent sides} of a quadrilateral, one obtains a parallelogram.

%%%%%
%%%%%
\end{document}
