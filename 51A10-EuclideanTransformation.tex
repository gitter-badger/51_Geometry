\documentclass[12pt]{article}
\usepackage{pmmeta}
\pmcanonicalname{EuclideanTransformation}
\pmcreated{2013-03-22 15:59:46}
\pmmodified{2013-03-22 15:59:46}
\pmowner{CWoo}{3771}
\pmmodifier{CWoo}{3771}
\pmtitle{Euclidean transformation}
\pmrecord{17}{38021}
\pmprivacy{1}
\pmauthor{CWoo}{3771}
\pmtype{Definition}
\pmcomment{trigger rebuild}
\pmclassification{msc}{51A10}
\pmclassification{msc}{15A04}
\pmclassification{msc}{51A15}
\pmsynonym{rigid motion}{EuclideanTransformation}
\pmdefines{translation}
\pmdefines{translate}
\pmdefines{rotation}
\pmdefines{rotate}
\pmdefines{reflection}
\pmdefines{reflect}
\pmdefines{reflexion}
\pmdefines{glide reflection}
\pmdefines{angle of rotation}

\usepackage{amssymb,amscd}
\usepackage{amsmath}
\usepackage{amsfonts}

% used for TeXing text within eps files
%\usepackage{psfrag}
% need this for including graphics (\includegraphics)
%\usepackage{graphicx}
% for neatly defining theorems and propositions
%\usepackage{amsthm}
% making logically defined graphics
%%%\usepackage{xypic}

% define commands here

\begin{document}
\PMlinkescapeword{addition}
\PMlinkescapeword{conjugate}
\PMlinkescapeword{expressible}
\PMlinkescapeword{term}
\PMlinkescapeword{types}

Let $V$ and $W$ be Euclidean vector spaces.  A \emph{Euclidean
transformation} is an affine transformation $E:V\to W$, given by
$$E(v)=L(v)+w$$ such that $L$ is an \PMlinkname{orthogonal linear
transformation}{OrthogonalTransformation}.

As an affine transformation, all affine properties, such as
incidence and parallelism are preserved by $E$.  In addition, since
$E(u-v)=L(u-v)$ and $L$ is an \PMlinkescapetext{orthogonal linear transformation}, $E$
preserves lengths of line segments and \PMlinkname{angles between two line
segments}{AngleBetweenTwoLines}. Because of this, a Euclidean transformation is also called
a \emph{rigid motion}, which is a popular term used in mechanics.

\subsubsection*{Types of Euclidean transformations} There are three main
types of Euclidean transformations:

\begin{enumerate}
\item
\textbf{translation}. If $L=I$, then $E$ is just a translation.  Any
Euclidean transformation can be decomposed into a product of an
orthogonal transformation $L(v)$, followed by a
translation $T(v)=v+w$.
\item
\textbf{rotation}.  If $w=0$, then $E$ is just an orthogonal transformation.  If $\operatorname{det}(E)=1$, then $E$ is called a \emph{rotation}.  The
orientation of any basis (of $V$) is preserved under a rotation.  In the
case where $V$ is two-dimensional, a rotation is conjugate to a matrix of the form
\begin{eqnarray}
\begin{pmatrix}
\cos \theta & -\sin \theta \\
\sin \theta & \cos \theta
\end{pmatrix},
\end{eqnarray}
where $\theta\in \mathbb{R}$. Via this particular (unconjugated) map, any vector emanating from the origin is rotated
in the counterclockwise direction by an angle of $\theta$ to another vector emanating from the origin.  Thus, if $E$ is conjugate to the matrix given above, then $\theta$ is the \emph{angle of rotation} for $E$.
\item
\textbf{reflection}. If $w=0$ but $\operatorname{det}(E)=-1$ instead, then $E$ is a called
\emph{reflection}.  Again, in the two-dimensional case, a reflection is
\PMlinkescapetext{conjugate} to a matrix of the form
\begin{eqnarray}
\begin{pmatrix}
\cos \theta & \sin \theta \\
\sin \theta & -\cos \theta
\end{pmatrix},
\end{eqnarray}
where $\theta\in \mathbb{R}$.  Any vector is reflected by this particular (unconjugated) map to another
by a ``mirror'', a line of the form $y=x\tan(\frac{\theta}{2})$.
\end{enumerate}

\textbf{Remarks}.
\begin{itemize}
\item
In general, an orthogonal transformation can be represented by a matrix of the
form
$$
\begin{pmatrix}
A_1 & O & \cdots & O \\
O & A_2 & \cdots & O \\
\vdots & \vdots & \ddots & \vdots \\
O & O & \cdots & A_n
\end{pmatrix},
$$
where each $A_i$ is either $\pm 1$ or a rotation matrix (1) (or reflection
matrix (2)) given above.  When its determinant is -1 (a reflection), it has an invariant subspace of $V$ of codimension 1.  One can think of this hyperplane as the mirror.
\item  Another common rigid motion is the \emph{glide reflection}.  It is a Euclidean transformation that is expressible as a product of a reflection, followed by a translation.
\end{itemize}

%%%%%
%%%%%
\end{document}
