\documentclass[12pt]{article}
\usepackage{pmmeta}
\pmcanonicalname{ConvertingBetweenThePoincareDiscModelAndTheUpperHalfPlaneModel}
\pmcreated{2013-03-22 17:07:43}
\pmmodified{2013-03-22 17:07:43}
\pmowner{Wkbj79}{1863}
\pmmodifier{Wkbj79}{1863}
\pmtitle{converting between the Poincar\'{e} disc model and the upper half plane model}
\pmrecord{7}{39432}
\pmprivacy{1}
\pmauthor{Wkbj79}{1863}
\pmtype{Topic}
\pmcomment{trigger rebuild}
\pmclassification{msc}{51M10}
\pmclassification{msc}{51-00}
\pmrelated{PoincareDiscModel}
\pmrelated{UpperHalfPlaneModel}
\pmrelated{UnitDiskUpperHalfPlaneConformalEquivalenceTheorem}
\pmrelated{PoincareUpperHalfPlaneModel}
\pmrelated{UpperHalfPlane}

\usepackage{amssymb}
\usepackage{amsmath}
\usepackage{amsfonts}

\usepackage{psfrag}
\usepackage{graphicx}
\usepackage{amsthm}
%%\usepackage{xypic}

\begin{document}
If both the Poincar\'e disc model and the upper half plane model are considered as subsets of $\mathbb{C}$ rather than as subsets of $\mathbb{R}^2$ (that is, the Poincar\'e disc model is $\{ z \in \mathbb{C} : |z|<1\}$ and the upper half plane model is $\{ z \in \mathbb{C} : \operatorname{Im}(z)>0\}$), then one can use M\"obius transformations to convert between the two models.  The entry unit disk upper half plane conformal equivalence theorem yields that $f \colon \mathbb{C} \cup \{ \infty \} \to \mathbb{C} \cup \{ \infty \}$ defined by $\displaystyle f(z)=\frac{z-i}{z+i}$ maps the upper half plane model to the Poincar\'e disc model, and thus its inverse, $f^{-1} \colon \mathbb{C} \cup \{ \infty \} \to \mathbb{C} \cup \{ \infty \}$ defined by $\displaystyle f^{-1}(z)=\frac{-iz-i}{z-1}$, maps the Poincar\'e disc model to the upper half plane model.

Note that the M\"obius transformation $f^{-1}$ gives another justification of including $\infty$ in the boundary of the upper half plane model (see the entry on parallel lines in hyperbolic geometry for more details): $1$ (or the ordered pair $(1,0)$) is on the boundary of the Poincar\'e disc model and $f^{-1}(1)=\infty$.

Note also that lines in the Poincar\'e disc model passing through $1$ (or the ordered pair $(1,0)$) are in one-to-one correspondence with the lines that are vertical rays in the upper half plane model.
%%%%%
%%%%%
\end{document}
