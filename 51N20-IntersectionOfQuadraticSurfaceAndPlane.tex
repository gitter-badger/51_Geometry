\documentclass[12pt]{article}
\usepackage{pmmeta}
\pmcanonicalname{IntersectionOfQuadraticSurfaceAndPlane}
\pmcreated{2013-03-22 18:31:38}
\pmmodified{2013-03-22 18:31:38}
\pmowner{pahio}{2872}
\pmmodifier{pahio}{2872}
\pmtitle{intersection of quadratic surface and plane}
\pmrecord{7}{41231}
\pmprivacy{1}
\pmauthor{pahio}{2872}
\pmtype{Result}
\pmcomment{trigger rebuild}
\pmclassification{msc}{51N20}
\pmrelated{QuadraticSurfaces}
\pmrelated{QuadraticCurves}
\pmrelated{Conic}
\pmrelated{EquationOfPlane}
\pmrelated{ConicSection}

\endmetadata

% this is the default PlanetMath preamble.  as your knowledge
% of TeX increases, you will probably want to edit this, but
% it should be fine as is for beginners.

% almost certainly you want these
\usepackage{amssymb}
\usepackage{amsmath}
\usepackage{amsfonts}

% used for TeXing text within eps files
%\usepackage{psfrag}
% need this for including graphics (\includegraphics)
%\usepackage{graphicx}
% for neatly defining theorems and propositions
 \usepackage{amsthm}
% making logically defined graphics
%%%\usepackage{xypic}

% there are many more packages, add them here as you need them

% define commands here

\theoremstyle{definition}
\newtheorem*{thmplain}{Theorem}

\begin{document}
The \PMlinkname{intersection of a sphere with a plane}{IntersectionOfSphereAndPlane} is a circle, similarly the intersection of any surface of revolution formed by the revolution of an ellipse or a hyperbola about its axis with a plane perpendicular to the axis of revolution is a circle of latitude.\\

We can get as intersection curves of other quadratic surfaces and a plane also other quadratic curves (conics).\, If for example the ellipsoid
\begin{align}
\frac{x^2}{a^2}+\frac{y^2}{b^2}+\frac{z^2}{c^2} \;=\; 1
\end{align}
is cut with the plane\, $z = 0$ (i.e. the $xy$-plane), we substitute\, $z = 0$\, to the equation of the ellipsoid, and thus the intersection curve satisfies the equation 
$$\frac{x^2}{a^2}+\frac{y^2}{b^2} \;=\; 1,$$
which \PMlinkescapetext{represents} an ellipse.\, Actually, all plane intersections of the ellipsoid are ellipses, which may be in special cases circles.\\

As another exaple of quadratic surface we take the hyperbolic paraboloid
\begin{align}
\frac{x^2}{a^2}-\frac{y^2}{b^2} \;=\; 2z.
\end{align}
Cutting it e.g. with the plane\, $y = b$,\, which is parallel to the $zx$-plane, the substitution yields the equation
$$2z \;=\; \frac{x^2}{a^2}-1$$
meaning that the intersection curve in the plane\, $y = b$\, has the \PMlinkname{projection}{ProjectionOfPoint} parabola in the $zx$-plane with such an equation, and accordingly is such a parabola.

If we cut the surface (2) with the plane\, $z = \frac{1}{2}$, the result is the hyperbola
having the projection
$$\frac{x^2}{a^2}-\frac{y^2}{b^2} \;=\; 1$$
in the $xy$-plane.\, But cutting with\, $z = 0$\, gives\, $\frac{x^2}{a^2}-\frac{y^2}{b^2} \;=\; 0$, i.e. the pair of 
lines \,$y = \pm\frac{b}{a}x$\, which is a degenerate conic.\\

Let us then consider the general equation 
\begin{align}
Ax^2+By^2+Cz^2+2A'yz+2B'zx+2C'xy+2A''x+2B''y+2C''z+D \;=\; 0
\end{align}
of quadratic surface and an arbitrary plane
\begin{align}
ax\!+\!by\!+\!cz\!+\!d \;=\; 0
\end{align}
where at least one of the coefficients $a$, $b$, $c$ is distinct from zero.\, Their intersection equation is obtained, supposing that e.g.\, $c \neq 0$, by substituting the solved form
$$z \;=\; -\frac{ax\!+\!by\!+\!d}{c}$$
of (4) to the equation (3).\, We then apparently have the equation of the form
$$\alpha x^2+\beta y^2+2\gamma xy +2\delta x+2\varepsilon y+\zeta \;=\; 0,$$
which \PMlinkescapetext{represents} a \PMlinkname{quadratic curve}{QuadraticCurves} or some of the degenerated cases of them.





%%%%%
%%%%%
\end{document}
