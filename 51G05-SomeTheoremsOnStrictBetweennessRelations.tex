\documentclass[12pt]{article}
\usepackage{pmmeta}
\pmcanonicalname{SomeTheoremsOnStrictBetweennessRelations}
\pmcreated{2013-03-22 17:18:59}
\pmmodified{2013-03-22 17:18:59}
\pmowner{Mathprof}{13753}
\pmmodifier{Mathprof}{13753}
\pmtitle{some theorems on strict betweenness relations}
\pmrecord{6}{39666}
\pmprivacy{1}
\pmauthor{Mathprof}{13753}
\pmtype{Theorem}
\pmcomment{trigger rebuild}
\pmclassification{msc}{51G05}
\pmrelated{StrictBetweennessRelation}

% this is the default PlanetMath preamble.  as your knowledge
% of TeX increases, you will probably want to edit this, but
% it should be fine as is for beginners.

% almost certainly you want these
\usepackage{amssymb}
\usepackage{amsmath}
\usepackage{amsfonts}

% used for TeXing text within eps files
%\usepackage{psfrag}
% need this for including graphics (\includegraphics)
%\usepackage{graphicx}
% for neatly defining theorems and propositions
\usepackage{amsthm}
% making logically defined graphics
%%%\usepackage{xypic}

% there are many more packages, add them here as you need them

% define commands here
\newtheorem{thm}{Theorem}
\begin{document}
Let $B$ be a strict betweenness relation. In the following the sets $B_{*pq}, B_{p*q}, B_{pq*}, B_{pq}, B(p,q)$
are defined in the  entry about   some theorems on the axioms of order.

\begin{thm}
Three elements are
in a strict betweenness relation only if they are pairwise distinct.
\end{thm}
\begin{thm}
If $B$ is strict, then $B_{*pq}$, $B_{p*q}$ and $B_{pq*}$ are pairwise disjoint.  
Furthermore, if $p=q$ then all three sets are empty.
\end{thm}
\begin{thm} 
If $B$ is strict, then $B_{pq}\cap B_{qp}=B_{p*q}$ and $B_{pq}\cup B_{qp}=B(p,q)$.
\end{thm}
\begin{thm} 
If $B$ is strict, then for any $p,q\in A$, $p\ne q$, $B_{*pq}$, $B_{p*q}$ and $B_{pq*}$ are infinite.
\end{thm}
%%%%%
%%%%%
\end{document}
