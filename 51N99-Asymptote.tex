\documentclass[12pt]{article}
\usepackage{pmmeta}
\pmcanonicalname{Asymptote}
\pmcreated{2013-03-22 14:32:59}
\pmmodified{2013-03-22 14:32:59}
\pmowner{pahio}{2872}
\pmmodifier{pahio}{2872}
\pmtitle{asymptote}
\pmrecord{12}{36100}
\pmprivacy{1}
\pmauthor{pahio}{2872}
\pmtype{Definition}
\pmcomment{trigger rebuild}
\pmclassification{msc}{51N99}
\pmrelated{SincFunction}
\pmrelated{FamousCurvesInThePlane}

\endmetadata

\usepackage{amssymb}
\usepackage{amsmath}
\usepackage{amsfonts}

\usepackage{graphicx}
\begin{document}
If a plane curve $\gamma$ has a \PMlinkescapetext{branch} continuing infinitely far from the origin $O$, then $\gamma$ may have an {\em asymptote}: \,The direct line $l$ is an asymptote of $\gamma$, if 
  $$\lim_{d(P, \,O) \to \infty}d(P, \,l) = 0,$$
where $d(P, \,O)$ means the \PMlinkescapetext{distance} of the point $P$ of the \PMlinkescapetext{branch} from the origin and 
$d(P, \,l)$ the \PMlinkescapetext{distance} of $P$ from the line $l$.

\textbf{Examples}:\, The hyperbola \, $\frac{x^2}{a^2}-\frac{y^2}{b^2} = 1$ \,has the asymptotes \,$y = \pm\frac{b}{a}x$;\, the curve \, $y = \frac{\sin x}{x}$\, the asymptote\, $y = 0$.
\begin{center}
\includegraphics{asympt1}
\end{center}
%%%%%
%%%%%
\end{document}
