\documentclass[12pt]{article}
\usepackage{pmmeta}
\pmcanonicalname{ArclengthAsFilteredLimit}
\pmcreated{2013-03-22 15:49:34}
\pmmodified{2013-03-22 15:49:34}
\pmowner{rspuzio}{6075}
\pmmodifier{rspuzio}{6075}
\pmtitle{arclength as filtered limit}
\pmrecord{13}{37793}
\pmprivacy{1}
\pmauthor{rspuzio}{6075}
\pmtype{Result}
\pmcomment{trigger rebuild}
\pmclassification{msc}{51N05}

\endmetadata

% this is the default PlanetMath preamble.  as your knowledge
% of TeX increases, you will probably want to edit this, but
% it should be fine as is for beginners.

% almost certainly you want these
\usepackage{amssymb}
\usepackage{amsmath}
\usepackage{amsfonts}

% used for TeXing text within eps files
%\usepackage{psfrag}
% need this for including graphics (\includegraphics)
%\usepackage{graphicx}
% for neatly defining theorems and propositions
%\usepackage{amsthm}
% making logically defined graphics
%%%\usepackage{xypic}

% there are many more packages, add them here as you need them

% define commands here
\begin{document}
\PMlinkescapeword{length}

The \PMlinkname{length}{Rectifiable} of a rectifiable curve may be phrased as a filtered limit.
To do this, we will define a filter of partitions of an interval
$[a,b]$.  Let ${\bf P}$ be the set of all ordered tuplets of distinct
elements of $[a,b]$ whose entries are increasing: 
 \[{\bf P} = \{ (t_1, \ldots t_n) \mid ( a \le t_1 < t_2 < \cdots < t_n \le b) \wedge
 (n \in \mathbb{Z}) \wedge (n > 0) \}\]
 We shall refer to elements of ${\bf P}$ as partitions of the interval
$[a,b]$.  We shall say that $(t_1, \ldots ,t_n)$ is a refinement of a
partition $(s_1, \ldots, s_m)$ if $\{t_1, \ldots ,t_n\} \supset \{s_1,
\ldots, s_m\}$.  Let ${\bf F} \subset \mathcal{P} ({\bf P})$ be the
set of all subsets of ${\bf P}$ such that, if a certain partition
belongs to ${\bf F}$ then so do all refinements of that partition.

Let us see that ${\bf F}$ is a filter basis.  Suppose that $A$ and $B$
are elements of ${\bf F}$.  If a partition belongs to both $A$ and $B$
then every one of its refinements will also belong to both $A$ and
$B$, hence $A \cap B \in {\bf F}$.

Next, note that, if a partition of $B$ is a refinement of a partition
of $A$ then, by the triangle inequality, the length of $\Pi (B)$ is
greater than the length of $\Pi (A)$.  By definition, for every
$\epsilon > 0$, we can pick a partition $A$ such that the length of
$\Pi(A)$ differs from the length of the curve by at most $\epsilon$.
Since the length of $\Pi(B)$ for any partition $B$ refining $A$ lies
between the length of $\Pi(A)$ and the length of the curve, we see
that the length of $\Pi(B)$ will also differ by at most $\epsilon$, so
the length of the curve is the limit of the length of polygonal lines
according to the filter generated by ${\bf F}$.
%%%%%
%%%%%
\end{document}
