\documentclass[12pt]{article}
\usepackage{pmmeta}
\pmcanonicalname{Parallelotope}
\pmcreated{2013-03-22 17:37:35}
\pmmodified{2013-03-22 17:37:35}
\pmowner{slider142}{78}
\pmmodifier{slider142}{78}
\pmtitle{parallelotope}
\pmrecord{13}{40047}
\pmprivacy{1}
\pmauthor{slider142}{78}
\pmtype{Definition}
\pmcomment{trigger rebuild}
\pmclassification{msc}{51M20}
\pmrelated{parallelogram}
\pmrelated{parallelepiped}
\pmrelated{Parallelogram}

\endmetadata

% this is the default PlanetMath preamble.  as your knowledge
% of TeX increases, you will probably want to edit this, but
% it should be fine as is for beginners.

% almost certainly you want these
\usepackage{amssymb}
\usepackage{amsmath}
\usepackage{amsfonts}

% used for TeXing text within eps files
%\usepackage{psfrag}
% need this for including graphics (\includegraphics)
\usepackage{graphicx}
% for neatly defining theorems and propositions
%\usepackage{amsthm}
% making logically defined graphics
%%%\usepackage{xypic} 

% there are many more packages, add them here as you need them

% define commands here

\begin{document}
\PMlinkescapeword{geometric series}
\PMlinkescapeword{covering}

A parallelotope is the covering term used to denote any element of the series of geometric objects generalizing the parallelogram:

\begin{center}
\includegraphics{parallelotope}\\
\PMlinktofile{View SVG}{parallelotope.svg}
\end{center}

The hypercube is formed by moving the parallelepiped in a fourth direction, thus creating a 4-dimensional object bounded by 8 parallelepipeds (2 for each axis of freedom). The volume of an n-parallelotope can be calculated by taking the product of its lengths in each dimension or by taking the absolute value of the determinant of the vectors spanning it.
%%%%%
%%%%%
\end{document}
