\documentclass[12pt]{article}
\usepackage{pmmeta}
\pmcanonicalname{AreaOfAQuadrilateral}
\pmcreated{2013-03-22 16:58:22}
\pmmodified{2013-03-22 16:58:22}
\pmowner{Mathprof}{13753}
\pmmodifier{Mathprof}{13753}
\pmtitle{area of a quadrilateral}
\pmrecord{7}{39248}
\pmprivacy{1}
\pmauthor{Mathprof}{13753}
\pmtype{Theorem}
\pmcomment{trigger rebuild}
\pmclassification{msc}{51N20}

% this is the default PlanetMath preamble.  as your knowledge
% of TeX increases, you will probably want to edit this, but
% it should be fine as is for beginners.

% almost certainly you want these
\usepackage{amssymb}
\usepackage{amsmath}
\usepackage{amsfonts}

% used for TeXing text within eps files
%\usepackage{psfrag}
% need this for including graphics (\includegraphics)
%\usepackage{graphicx}
% for neatly defining theorems and propositions
%\usepackage{amsthm}
% making logically defined graphics
%%%\usepackage{xypic}

% there are many more packages, add them here as you need them

% define commands here

\begin{document}
Let $a,b,c,d$ be the lengths of the sides of a quadrilateral and $K$ be its area.
Let $s$ be the semiperimeter. 
Then
$$
K^2 = (s-a)(s-b)(s-c)(s-d) - abcd \cos^2 \left(\frac{\theta+\phi}{2}\right )
$$
where $\theta$ and $\phi$ are \PMlinkescapetext{opposite angles} of the quadrilateral.
Letting $d \to 0$ we obtain Heron's formula for the area of a triangle.

\begin{thebibliography}{99}
\bibitem{CAB}
C.A. Bretschneider,  \emph{Untersuchung der trigonometrischen Relationen des geradlinigen 
Viereckes}. Archiv der Math. 2, (1842), 225-261.
\bibitem{FS}
F. Strehlke,  \emph{Zwei neue S\"atze vom ebenen und shp\"arischen Viereck und 
Umkehrung des Ptolemaischen Lehrsatzes.} Archiv der Math. 2, (1842) 323-326.
\end{thebibliography}
%%%%%
%%%%%
\end{document}
