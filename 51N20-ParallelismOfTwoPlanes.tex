\documentclass[12pt]{article}
\usepackage{pmmeta}
\pmcanonicalname{ParallelismOfTwoPlanes}
\pmcreated{2013-03-22 18:48:10}
\pmmodified{2013-03-22 18:48:10}
\pmowner{pahio}{2872}
\pmmodifier{pahio}{2872}
\pmtitle{parallelism of two planes}
\pmrecord{15}{41603}
\pmprivacy{1}
\pmauthor{pahio}{2872}
\pmtype{Topic}
\pmcomment{trigger rebuild}
\pmclassification{msc}{51N20}
\pmclassification{msc}{51M04}
\pmclassification{msc}{51A05}
\pmsynonym{parallelism of planes}{ParallelismOfTwoPlanes}
\pmsynonym{parallel planes}{ParallelismOfTwoPlanes}
%\pmkeywords{parallel plane}
\pmrelated{PlaneNormal}
\pmrelated{ParallelAndPerpendicularPlanes}
\pmrelated{ParallelityOfLineAndPlane}
\pmrelated{ExampleOfUsingLagrangeMultipliers}
\pmrelated{NormalOfPlane}
\pmdefines{parallel}
\pmdefines{parallelism}

\endmetadata

% this is the default PlanetMath preamble.  as your knowledge
% of TeX increases, you will probably want to edit this, but
% it should be fine as is for beginners.

% almost certainly you want these
\usepackage{amssymb}
\usepackage{amsmath}
\usepackage{amsfonts}

% used for TeXing text within eps files
%\usepackage{psfrag}
% need this for including graphics (\includegraphics)
%\usepackage{graphicx}
% for neatly defining theorems and propositions
 \usepackage{amsthm}
% making logically defined graphics
%%%\usepackage{xypic}

% there are many more packages, add them here as you need them

% define commands here

\theoremstyle{definition}
\newtheorem*{thmplain}{Theorem}

\begin{document}
Two planes $\pi$ and $\varrho$ in the 3-dimensional Euclidean space are {\em parallel}\, iff they either have no common points or coincide, i.e. iff
\begin{align}
\pi\cap\varrho \;=\; \varnothing \quad \mbox{or} \quad \pi\cap\varrho\;=\; \pi.
\end{align}
An \PMlinkname{equivalent}{Equivalent3} condition of the parallelism is that the normal vectors of $\pi$ and $\varrho$ are parallel.\\
The parallelism of planes is an equivalence relation in any set of planes of the space.\\

If the planes have the equations
\begin{align}
A_1x\!+\!B_1y\!+\!C_1z\!+\!D_1 \;=\; 0 \quad \mbox{and} \quad A_2x\!+\!B_2y\!+\!C_2z\!+\!D_2 \;=\; 0,
\end{align}
the parallelism means the \PMlinkname{proportionality}{Variation} of the coefficients of the variables:\, there exists a \PMlinkescapetext{constant} $k$ such that
\begin{align}
A_1 \;=\; kA_2, \quad B_1 \;=\; kB_2, \quad C_1 \;=\; kC_2.
\end{align}
In this case, if also\, $D_1 \,=\, kD_2$,\, then the planes coincide.

Using vectors, the condition (3) may be written
\begin{align}
       \left(\!\begin{array}{c}A_1\\ B_1\\ C_1\end{array}\!\right) 
\;=\; k\left(\!\begin{array}{c}A_2\\ B_2\\ C_2\end{array}\!\right)
\end{align}
which equation utters the \PMlinkname{parallelism}{MutualPositionsOfVectors} of the normal vectors.\\


\textbf{Remark.}\, The shortest distance of the parallel planes 
$$Ax\!+\!By\!+\!Cz\!+\!D \;=\; 0 \quad \mbox{and} \quad Ax\!+\!By\!+\!Cz\!+\!E \;=\; 0$$
is obtained from the \PMlinkescapetext{formula}
\begin{align}
d \;=\; \frac{|D\!-\!E|}{\sqrt{A^2\!+\!B^2\!+\!C^2}},
\end{align}
as is easily shown by using \PMlinkname{Lagrange multipliers}{LagrangeMultiplierMethod} (see \PMlinkid{this entry}{11604}).

%%%%%
%%%%%
\end{document}
