\documentclass[12pt]{article}
\usepackage{pmmeta}
\pmcanonicalname{VolumeOfSolidOfRevolution}
\pmcreated{2013-03-22 17:20:12}
\pmmodified{2013-03-22 17:20:12}
\pmowner{pahio}{2872}
\pmmodifier{pahio}{2872}
\pmtitle{volume of solid of revolution}
\pmrecord{11}{39691}
\pmprivacy{1}
\pmauthor{pahio}{2872}
\pmtype{Topic}
\pmcomment{trigger rebuild}
\pmclassification{msc}{51M25}
%\pmkeywords{volume of solid}
\pmrelated{PappussTheoremForSurfacesOfRevolution}
\pmrelated{SurfaceOfRevolution}
\pmrelated{VolumeAsIntegral}

\endmetadata

% this is the default PlanetMath preamble.  as your knowledge
% of TeX increases, you will probably want to edit this, but
% it should be fine as is for beginners.

% almost certainly you want these
\usepackage{amssymb}
\usepackage{amsmath}
\usepackage{amsfonts}

% used for TeXing text within eps files
%\usepackage{psfrag}
% need this for including graphics (\includegraphics)
%\usepackage{graphicx}
% for neatly defining theorems and propositions
 \usepackage{amsthm}
% making logically defined graphics
%%%\usepackage{xypic}

% there are many more packages, add them here as you need them

% define commands here

\theoremstyle{definition}
\newtheorem*{thmplain}{Theorem}

\begin{document}
\PMlinkescapeword{generates} \PMlinkescapeword{generated} \PMlinkescapeword{formula}


Let us consider a solid of revolution, which is generated when a planar domain $D$ rotates about a line of the same plane.  We chose this line for the $x$-axis, and for simplicity we assume that the boundaries of $D$ are the mentioned axis, two ordinates \, $x = a$,\, $x = b\,(> a)$, and a continuous curve \, $y = f(x)$.

Between the bounds $a$ anb $b$ we fit a sequence of points\, $x_1,\,x_2,\,\ldots,\,x_{n-1}$\, and draw through these the ordinates which divide the domain $D$ in $n$ parts.  Moreover we form for every part the (maximal) inscribed and the (minimal) circumscribed rectangle.  In the revolution of $D$, each rectangle generates a circular cylinder.  The considered solid of revolution is part of the volume $V_>$ of the union of the cyliders generated by the circumscribed rectangles and at the same time contains the volume $V_<$ of the union of the cylinders generated by the inscribed rectangles.

Now it is apparent that
$$V_> = \pi[M_1^2(x_1-a)+M_2^2(x_2-x_1)+\ldots+M_n^2(b-x_{n-1})],$$
$$V_< = \pi[m_1^2(x_1-a)+m_2^2(x_2-x_1)+\ldots+m_n^2(b-x_{n-1})],$$
where\, $M_1,\,M_2,\,\ldots,\,M_n$\, are the greatest and\, $m_1,\,m_2,\,\ldots,\,m_n$\, the least values of the continuous function $f$ on the \PMlinkname{intervals}{Interval} \, $[a,\,x_1]$,\, $[x_1,\,x_2]$,\,\ldots,\,$[x_{n-1},\,b]$.  The volume $V$ of the solid of revolution thus satisfies
                   $$V_< \le V \le V_>,$$
and this is true for any \PMlinkescapetext{division}\, $x_1 < x_2 < \ldots < x_{n-1}$\, of the interval\, $[a,\,b]$.  
The theory of the Riemann integral guarantees that there exists only one real number $V$ having this property and that it is also the definition of the integral $\displaystyle\int_a^b\!\pi[f(x)]^2\,dx.$  Therefore the volume of the given solid of revolution can be obtained from
                $$V = \pi\int_a^b[f(x)]^2\,dx.$$



\begin{thebibliography}{8}
\bibitem{Lj}{\sc E. Lindel\"of}: {\em Johdatus korkeampaan analyysiin}. Nelj\"as painos.\, Werner S\"oderstr\"om Osakeyhti\"o, Porvoo ja Helsinki (1956).
\end{thebibliography} 

%%%%%
%%%%%
\end{document}
