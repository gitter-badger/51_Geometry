\documentclass[12pt]{article}
\usepackage{pmmeta}
\pmcanonicalname{PropertiesOfParallelCurves}
\pmcreated{2013-03-22 17:14:30}
\pmmodified{2013-03-22 17:14:30}
\pmowner{pahio}{2872}
\pmmodifier{pahio}{2872}
\pmtitle{properties of parallel curves}
\pmrecord{6}{39571}
\pmprivacy{1}
\pmauthor{pahio}{2872}
\pmtype{Topic}
\pmcomment{trigger rebuild}
\pmclassification{msc}{51N05}

% this is the default PlanetMath preamble.  as your knowledge
% of TeX increases, you will probably want to edit this, but
% it should be fine as is for beginners.

% almost certainly you want these
\usepackage{amssymb}
\usepackage{amsmath}
\usepackage{amsfonts}

% used for TeXing text within eps files
%\usepackage{psfrag}
% need this for including graphics (\includegraphics)
%\usepackage{graphicx}
% for neatly defining theorems and propositions
 \usepackage{amsthm}
% making logically defined graphics
%%%\usepackage{xypic}

% there are many more packages, add them here as you need them

% define commands here

\theoremstyle{definition}
\newtheorem*{thmplain}{Theorem}

\begin{document}
\begin{itemize}

\item Two plane curves are parallel curves of each other, if every normal of one curve is also a normal of the other curve (then one may show that the distance of the corresponding points of the curves is a \PMlinkescapetext{constant}).

\item Two curves are parallel curves of each other, if they are the loci of the end points of a line segment which moves perpendicularly to its own direction.

\item Every regular curve having a continuous curvature has an infinite family of parallel curves.

\item The parallelism of curves is an equivalence relation.

\item The two parallel curves $\gamma_{\pm a}$ on both sides of a curve $\gamma$ at the distance $a$ form the envelope of the family of circles with center on $\gamma$ and radius $a$.

\item If $\gamma$ is a \PMlinkescapetext{regular} and closed curve with perimeter $p$, then the perimeter of $\gamma_{\pm a}$ is equal to\, $p\pm 2\pi a$\, and the area between $\gamma$ and the parallel curve is equal to\,  
$pa\pm\pi a^2$ (one must also assume that the parallel curve don't intersect the evolute of $\gamma$).

\end{itemize}
%%%%%
%%%%%
\end{document}
