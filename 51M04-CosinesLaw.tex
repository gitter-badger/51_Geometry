\documentclass[12pt]{article}
\usepackage{pmmeta}
\pmcanonicalname{CosinesLaw}
\pmcreated{2013-03-22 11:42:35}
\pmmodified{2013-03-22 11:42:35}
\pmowner{drini}{3}
\pmmodifier{drini}{3}
\pmtitle{cosines law}
\pmrecord{26}{30027}
\pmprivacy{1}
\pmauthor{drini}{3}
\pmtype{Theorem}
\pmcomment{trigger rebuild}
\pmclassification{msc}{51M04}
\pmclassification{msc}{05A15}
\pmclassification{msc}{05A10}
\pmclassification{msc}{05A16}
\pmsynonym{law of cosines}{CosinesLaw}
%\pmkeywords{Cosine}
%\pmkeywords{Sine}
%\pmkeywords{Trigonometry}
%\pmkeywords{Triangle}
\pmrelated{SinesLaw}
\pmrelated{PythagorasTheorem}
\pmrelated{DerivationOfCosinesLaw}
\pmrelated{CosinesLaw}
\pmrelated{StewartsTheorem}
\pmrelated{AlternativeProofOfTheSinesLaw}
\pmrelated{SinesLawProof}
\pmrelated{ProofOfStewartsTheorem}

\endmetadata

\usepackage{amssymb}
\usepackage{amsmath}
\usepackage{amsfonts}
\usepackage{graphicx}
%%%%%%%%%%%%%%%\usepackage{xypic}
\begin{document}
\textbf{Cosines Law.}\, Let $a$, $b$, $c$ be the sides of a triangle and $A$ the angle opposite to $a$. Then 
$$a^2 = b^2+c^2-2bc\cos A.$$

\begin{center}
\includegraphics{coslaw}
\end{center}

\textbf{Remark.}\, Cosines law is the generalised form of Pythagorean theorem, which latter concerns only the right triangles.
%%%%%
%%%%%
%%%%%
%%%%%
%%%%%
%%%%%
%%%%%
%%%%%
%%%%%
%%%%%
%%%%%
%%%%%
%%%%%
%%%%%
%%%%%
\end{document}
