\documentclass[12pt]{article}
\usepackage{pmmeta}
\pmcanonicalname{AnalyticGeometry}
\pmcreated{2013-03-22 17:36:23}
\pmmodified{2013-03-22 17:36:23}
\pmowner{pahio}{2872}
\pmmodifier{pahio}{2872}
\pmtitle{analytic geometry}
\pmrecord{13}{40022}
\pmprivacy{1}
\pmauthor{pahio}{2872}
\pmtype{Topic}
\pmcomment{trigger rebuild}
\pmclassification{msc}{51N20}
\pmclassification{msc}{01A45}
\pmrelated{CartesianCoordinates}
\pmrelated{EuclideanGeometryOfThePlane}
\pmrelated{EuclideanGeometryOfTheSpace}
\pmrelated{LineInThePlane}
\pmrelated{LineInSpace}
\pmrelated{EquationOfPlane}
\pmrelated{Ellipse}
\pmrelated{Hyperbola}
\pmrelated{Parabola}
\pmrelated{QuadraticSurfaces}
\pmrelated{CissoidOfDiocles}
\pmrelated{GeneratricesOfOneSheetedHyperboloid}
\pmrelated{ConicSection}

% this is the default PlanetMath preamble.  as your knowledge
% of TeX increases, you will probably want to edit this, but
% it should be fine as is for beginners.

% almost certainly you want these
\usepackage{amssymb}
\usepackage{amsmath}
\usepackage{amsfonts}

% used for TeXing text within eps files
%\usepackage{psfrag}
% need this for including graphics (\includegraphics)
%\usepackage{graphicx}
% for neatly defining theorems and propositions
 \usepackage{amsthm}
% making logically defined graphics
%%%\usepackage{xypic}

% there are many more packages, add them here as you need them

% define commands here

\theoremstyle{definition}
\newtheorem*{thmplain}{Theorem}

\begin{document}
\emph{Analytic geometry} is the \PMlinkescapetext{branch} of geometry that uses mathematical analysis and \PMlinkescapetext{algebraic} calculations for investigating geometric problems.  Many such problems can be put into the form of equations, and, by analyzing these equations, one may obtain solutions which can be interpreted geometrically.  

The fundamental idea behind analytic geometry is that the position of any point on a plane can be given by an ordered pair of real numbers and any point in space by an ordered triple of real numbers; for this purpose, one has to have a coordinate system which determines the values of the numbers which serve as the coordinates of the points.  
One of the coordinate system often used in mathematics and physics is the Cartesian system which employs three orthogonal axes. The correspondence of the points on a plane and the ordered pairs (and similarly the points in space and the ordered triples) is a bijection.  A locus condition for a line or a curve on the plane as well as for a line, a curve, or a surface in space may then be expressed as an equation or a system of equations.

For example, if we want to study the line which passes through the points corresponding to the 
\PMlinkname{ordered pairs}{CartesianProduct} \,$(5,\,0)$\, and\, $(0,\,8)$,\, one can infer that all points\, $(x,\,y)$\, of this line satisfy the equation
                       $$\frac{x}{5}+\frac{y}{8} = 1.$$

The Cartesian coordinate system was introduced and applied by the French mathematician and philosopher Ren\'e Descartes in 1637 in his work {\em G\'eom\'etrie}.  Thus, Descartes is considered to be the founder of analytic geometry.
%%%%%
%%%%%
\end{document}
