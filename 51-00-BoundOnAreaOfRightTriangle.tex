\documentclass[12pt]{article}
\usepackage{pmmeta}
\pmcanonicalname{BoundOnAreaOfRightTriangle}
\pmcreated{2013-03-22 16:30:41}
\pmmodified{2013-03-22 16:30:41}
\pmowner{rspuzio}{6075}
\pmmodifier{rspuzio}{6075}
\pmtitle{bound on area of right triangle}
\pmrecord{17}{38688}
\pmprivacy{1}
\pmauthor{rspuzio}{6075}
\pmtype{Theorem}
\pmcomment{trigger rebuild}
\pmclassification{msc}{51-00}

% this is the default PlanetMath preamble.  as your knowledge
% of TeX increases, you will probably want to edit this, but
% it should be fine as is for beginners.

% almost certainly you want these
\usepackage{amssymb}
\usepackage{amsmath}
\usepackage{amsfonts}

% used for TeXing text within eps files
%\usepackage{psfrag}
% need this for including graphics (\includegraphics)
%\usepackage{graphicx}
% for neatly defining theorems and propositions
\usepackage{amsthm}
% making logically defined graphics
%%%\usepackage{xypic}

% there are many more packages, add them here as you need them

% define commands here

\newtheorem{theorem}{Theorem}
\begin{document}
We may bound the area of a right triangle in terms of its perimeter.
The derivation of this bound is a good exercise in constrained 
optimization using Lagrange multipliers.

\begin{theorem}
If a right triangle has perimeter $P$, then its area is bounded as
\[
A \le \frac{3 - 2 \sqrt{2}}{4} P^2
\]
with equality when one has an isosceles right triangle.
\end{theorem}

\begin{proof}
Suppose a triangle has legs of length $x$ and $y$.  Then its hypotenuse
has length $\sqrt{x^2 + y^2}$, so the perimeter is given as
 \[ P = x + y + \sqrt{x^2 + y^2} .\]
The area, of course, is
 \[ A = \frac{1}{2} x y .\]

We want to maximize $A$ subject to the constraint that $P$ be constant.
This means that the gradient of $A$ will be proportional to the gradient
of $P$.  That is to say, for some constant $\lambda$, we will have
\begin{eqnarray*}
\frac{\partial A}{\partial x} &=& \lambda \frac{\partial P}{\partial x} \\
\frac{\partial A}{\partial y} &=& \lambda \frac{\partial P}{\partial y}
\end{eqnarray*}
Together with the constraint, these form a system of three equations
for the three quantities $x$, $y$, and $\lambda$.  Writing them out
explicitly,
\begin{eqnarray*}
\frac{1}{2} y &=& \lambda \left( 1 + \frac{x}{\sqrt{x^2 + y^2}} \right) \\
\frac{1}{2} x &=& \lambda \left( 1 + \frac{y}{\sqrt{x^2 + y^2}} \right) \\
P &=& x + y + \sqrt{x^2 + y^2}
\end{eqnarray*} 
Not that we cannot have $\lambda = 0$ because that would mean that all
sides of our triangle would have zero length.  Hence, we may eliminate 
$\lambda$ between the first two equations to obtain
\[
x \left( 1 + \frac{x}{\sqrt{x^2 + y^2}} \right) =
y \left( 1 + \frac{y}{\sqrt{x^2 + y^2}} \right),
\]
which may be manipulated to yield
\[
(x - y) \left( 1 + \frac{x+y}{\sqrt{x^2 + y^2}} \right) = 0.
\]
We have two case to consider --- either the first factor or the second
factor may equal zero.  If the second factor equals zero,
\[
1 + \frac{x+y}{\sqrt{x^2 + y^2}} = 0,
\]
move the ``1'' to the other side of the equation and cross-multiply
to obtain
\[
x + y = - \sqrt{x^2 + y^2}.
\]
Since we want $x \ge 0$ and $y \ge 0$ but the right-hand side is
non-positive, the only option would be to have a trianagle of 
zero area.
%Squaring both sides and expanding,
%\[
%x^2 + 2 x y + y^2 = x^2 + y^2.
%\]
%Cancelling, we conclude that $x y = 0$ so either $x = 0$ or $y = 0$.
%Either way, one leg of the triangle would equal zero, so the triangle
%would have zero area.  This is the minimum, not the maximum, so we
%move on the next case,
The other possibility was to have the second factor equal zero,
which would give
\[
x - y = 0.
\]
In this case, $x$ equals $y$.  Imposing this condition on the constraint,
we see that
\[
P = (2 + \sqrt{2}) x,
\]
so we have the solution
\begin{eqnarray*}
x &=& \frac{P}{2 + \sqrt{2}} = \frac{2 - \sqrt{2}}{2} P \\
y &=& \frac{P}{2 + \sqrt{2}} = \frac{2 - \sqrt{2}}{2} P .
\end{eqnarray*}
\end{proof}
%%%%%
%%%%%
\end{document}
