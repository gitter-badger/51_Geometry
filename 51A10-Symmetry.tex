\documentclass[12pt]{article}
\usepackage{pmmeta}
\pmcanonicalname{Symmetry}
\pmcreated{2013-03-22 17:12:29}
\pmmodified{2013-03-22 17:12:29}
\pmowner{Wkbj79}{1863}
\pmmodifier{Wkbj79}{1863}
\pmtitle{symmetry}
\pmrecord{18}{39530}
\pmprivacy{1}
\pmauthor{Wkbj79}{1863}
\pmtype{Definition}
\pmcomment{trigger rebuild}
\pmclassification{msc}{51A10}
\pmclassification{msc}{15A04}
\pmclassification{msc}{51A15}
\pmrelated{DihedralGroup}
\pmrelated{DeterminingRotationsAndReflectionsInMathbbR2}
\pmdefines{symmetry about}
\pmdefines{symmetric}
\pmdefines{symmetric about}
\pmdefines{rotational symmetry}
\pmdefines{point symmetry}
\pmdefines{symmetry about a point}
\pmdefines{symmetric about a point}
\pmdefines{reflectional symmetry}
\pmdefines{line symmetry}
\pmdefines{symmetry about a line}
\pmdefines{symmetric about a line}

\endmetadata

\usepackage{pstricks}
\usepackage{psfrag}
\usepackage{amssymb}
\usepackage{amsmath}
\usepackage{amsfonts}
\usepackage{graphicx}
\usepackage{amsthm}
%%\usepackage{xypic}

\begin{document}
\PMlinkescapeword{center}
\PMlinkescapeword{conjugate}
\PMlinkescapeword{formula}
\PMlinkescapeword{terms}

Let $V$ be a Euclidean vector space, $F \subseteq V$, and $E \colon V \to V$ be a Euclidean transformation that is not the identity map.

The following terms are used to indicate that $E(F)=F$ if $E$ is a rotation:

\begin{itemize}
\item $F$ has \emph{rotational symmetry};
\item $F$ has \emph{point symmetry};
\item $F$ has \emph{symmetry about a point};
\item $F$ is \emph{symmetric about a point}.
\end{itemize}

If $V=\mathbb{R}^2$, then the last two terms may be used to indicate the specific case in which $E$ is conjugate to $\displaystyle \left( \begin{array}{rr}
-1 & 0 \\
0 & -1 \end{array} \right)$, \PMlinkname{i.e.}{Ie} the angle of rotation is $180^{\circ}$.

The following are classic examples of rotational symmetry in $\mathbb{R}^2$:

\begin{itemize}
\item Regular polygons: A regular $n$-gon is symmetric about its \PMlinkname{center}{Center9} with valid angles of rotation $\displaystyle \theta=\left( \frac{360k}{n} \right)^{\circ}$ for any positive integer $k<n$.
\item Circles: A circle is symmetric about its \PMlinkname{center}{Center8} with uncountably many valid angles of rotation.
\end{itemize}

As another example, let $\displaystyle F=\bigcup_{k=1}^4 P_k$, where each $P_k$ is defined thus:
\begin{eqnarray*}
\displaystyle P_1&=&\left\{ (x,y) : 0 \le x \le \frac{4}{1+\sqrt{3}} \text{ and } (2-\sqrt{3})x \le y \le x \right\},\\ \displaystyle P_2&=&\left\{ (x,y) : \frac{4}{1+\sqrt{3}} \le x \le 2 \text{ and } x \le y \le (2+\sqrt{3})x-4 \right\},\\
\displaystyle P_3&=&\left\{ (x,y) : 2 \le x \le \frac{4 \sqrt{3}}{1+\sqrt{3}} \text{ and } (-2+\sqrt{3})x+8-4\sqrt{3} \le y \le (-2-\sqrt{3})x+4+4\sqrt{3} \right\},\\
\displaystyle P_4&=&\left\{ (x,y) : \frac{4 \sqrt{3}}{1+\sqrt{3}} \le x \le 4 \text{ and } (-2+\sqrt{3})x+8-4\sqrt{3} \le y \le -x+4 \right\}. 
\end{eqnarray*}
Then $F$ has point symmetry with respect to the point $\displaystyle \left( 2, \frac{2}{\sqrt{3}} \right)$.  The valid angles of rotation for $F$ are $120^{\circ}$ and $240^{\circ}$.  The boundary of $F$ and the point $\displaystyle \left( 2, \frac{2}{\sqrt{3}} \right)$ are shown in the following picture.

\begin{center}
\begin{pspicture}(0,0)(4,3.5)
\pspolygon(0,0)(2,0.536)(4,0)(2.5359,1.4641)(2,3.4641)(1.4641,1.4641)
\psdot(2,1.1547)
\end{pspicture}
\end{center}

As a final example, the figure

\noindent $\{ (x,y) : -3 \le x \le -1 \text{ and } (x+1)^2+y^2 \le 4 \} \cup \big( [-1,1] \times [-2,2] \big) \cup \{ (x,y) : 1 \le x \le 3 \text{ and } (x-1)^2+y^2 \le 4 \}$ is symmetric about the origin.  The boundary of this figure and the point $(0,0)$ are shown in the following picture.

\begin{center}
\begin{pspicture}(-3,-2)(3,2)
\psarc(-1,0){2}{180}{270}
\psline(-1,-2)(1,-2)(1,0)(3,0)
\psarc(1,0){2}{0}{90}
\psline(1,2)(-1,2)(-1,0)(-3,0)
\psdot(0,0)
\end{pspicture}
\end{center}

If $E(F)=F$ and $E$ is a reflection, then $F$ has \emph{reflectional symmetry}.  In the special case that $V=\mathbb{R}^2$, the following terms are used:

\begin{itemize}
\item $F$ has \emph{line symmetry};
\item $F$ has \emph{symmetry about a line};
\item $F$ is \emph{symmetric about a line}.
\end{itemize}

The following are classic examples of line symmetry in $\mathbb{R}^2$:

\begin{itemize}
\item Regular polygons: There are $n$ lines of symmetry of a regular $n$-gon.  Each of these pass through its center and at least one of its vertices.
\item Circles: A circle is symmetric about any line passing through its center.
\end{itemize}

As another example, the isosceles trapezoid defined by $$T=\{ (x,y) : 0 \le x \le 6 \text{ and } 0 \le y \le \min\{x,2,-x+6\} \}$$ is symmetric about $x=3$.

\begin{center}
\begin{pspicture}(0,-1)(6,3)
\pspolygon(0,0)(6,0)(4,2)(2,2)
\psline[linecolor=cyan]{<->}(3,-0.5)(3,2.5)
\end{pspicture}
\end{center}

In the picture above, the boundary of $T$ is drawn in black, and the line $x=3$ is drawn in cyan.
%%%%%
%%%%%
\end{document}
