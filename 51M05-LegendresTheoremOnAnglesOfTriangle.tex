\documentclass[12pt]{article}
\usepackage{pmmeta}
\pmcanonicalname{LegendresTheoremOnAnglesOfTriangle}
\pmcreated{2013-05-11 13:55:53}
\pmmodified{2013-05-11 13:55:53}
\pmowner{pahio}{2872}
\pmmodifier{pahio}{2872}
\pmtitle{Legendre's theorem on angles of triangle}
\pmrecord{5}{87401}
\pmprivacy{1}
\pmauthor{pahio}{2872}
\pmtype{Theorem}
\pmclassification{msc}{51M05}

% this is the default PlanetMath preamble.  as your knowledge
% of TeX increases, you will probably want to edit this, but
% it should be fine as is for beginners.

% almost certainly you want these
\usepackage{amssymb}
\usepackage{amsmath}
\usepackage{amsfonts}

% need this for including graphics (\includegraphics)
\usepackage{graphicx}
% for neatly defining theorems and propositions
\usepackage{amsthm}

% making logically defined graphics
%\usepackage{xypic}
% used for TeXing text within eps files
%\usepackage{psfrag}

% there are many more packages, add them here as you need them

% define commands here

\begin{document}
Adrien-Marie Legendre has proved some theorems concerning the sum of the angles of triangle.\, Here we give one of them, being the inverse of the theorem in the entry ``sum of angles of triangle in Euclidean geometry''.


\textbf{Theorem.}\, If the sum of the interior angles of every triangle equals straight angle, then the parallel postulate is true, i.e., in the plane determined by a line and a point outwards it there is exactly one line through the point which does not intersect the line.

{\it Proof.}\, We consider a line $a$ and a point $B$ not belonging to $a$.\, Let $BA$ be the normal line of $a$
(with $A \in a$) and $b$ be the normal line of $BA$ through the point $B$.\, By the supposition of the theorem, $b$ does not intersect $a$.

We will show that in the plane determined by the line $a$ and the point $B$, there are through $B$ no other lines than $b$ not intersecting the line $a$.\, For this purpose, we choose through $B$ a line $b'$ which differs from $b$; let the line $b'$ form with $BA$ an acute angle $\beta$.

We determine on the line $a$ a point $A_1$ such that\, $AA_1 = AB$.\, By the supposition of the theorem, in the isosceles right triangle $BAA_1$ we have
$$\alpha_1 \;\,=:\;\, \angle AA_1B \;=\;\, \frac{\pi}{4} \;=\;\, \frac{\pi}{2^2}.$$

Next we determine on $a$ a second point $A_2$ such that\, 
$A_1A_2 = A_1B$.\, By the supposition of the theorem, in the isosceles triangle $BA_1A_2$ we have
$$\alpha_2 \;\,=:\;\, \angle AA_2B \;=\;\, \frac{\alpha_1}{2} \;=\;\, \frac{\pi}{2^3}.$$

We continue similarly by forming isosceles triangles using the points $A_3$, $A_3$, $\ldots$, $A_n$ of the line $a$ such that
$$A_2A_3 \,=\;\, BA_2,\;\; A_3A_4 \,=\;\, BA_3,\;\;\ldots,\;\; 
A_{n-1}A_n \,=\;\, BA_{n-1}.$$

Then the acute angles being formed beside the points are
$$\alpha_3 \;=\;\, \frac{\pi}{2^4},\;\; 
\alpha_4 \;=\;\, \frac{\pi}{2^5},\;\; \ldots, \;\;
\alpha_n \;=\;\, \frac{\pi}{2^{n+1}}.$$
They form a geometric sequence with the common ratio\, 
$r = \frac{1}{2}$.\, When $n$ is sufficiently great, the member $\alpha_n$ is less than any given positive angle.\, As we have so much triangles $BA_{n-1}A_n$ that\; 
$\alpha_n < \frac{\pi}{2}\!-\!\beta$,\, then
$$\angle ABA_n \;=\;\, 
\frac{\pi}{2}\!-\!\alpha_n \;>\;\, \beta.$$
Then the line $b'$ falls after penetrating $B$ into the inner territory of the triangle $ABA_n$.\, Thereafter it must leave from there and thus intersect the side $AA_n$ of this triangle.\, Accordingly, $b'$ intersects the line $a$.\, 

The above reasoning is possible for each line\, $b' \neq b$\, through $B$.\, Consequently, the parallel axiom is in force.

\begin{thebibliography}{8}
\bibitem{ariva}{\sc Karl Ariva}: {\it Lobatsevski geomeetria}.\, Kirjastus ``Valgus'', Tallinn (1992).
\end{thebibliography} 
\end{document}
