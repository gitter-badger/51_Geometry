\documentclass[12pt]{article}
\usepackage{pmmeta}
\pmcanonicalname{ProjectionOfRightAngle}
\pmcreated{2013-03-22 19:20:51}
\pmmodified{2013-03-22 19:20:51}
\pmowner{pahio}{2872}
\pmmodifier{pahio}{2872}
\pmtitle{projection of right angle}
\pmrecord{6}{42297}
\pmprivacy{1}
\pmauthor{pahio}{2872}
\pmtype{Theorem}
\pmcomment{trigger rebuild}
\pmclassification{msc}{51N99}
\pmclassification{msc}{51N20}
\pmrelated{AngleBetweenLineAndPlane}
\pmrelated{AngleOfView}
\pmrelated{AngleOfViewOfALineSegment}

\endmetadata

% this is the default PlanetMath preamble.  as your knowledge
% of TeX increases, you will probably want to edit this, but
% it should be fine as is for beginners.

% almost certainly you want these
\usepackage{amssymb}
\usepackage{amsmath}
\usepackage{amsfonts}

% used for TeXing text within eps files
%\usepackage{psfrag}
% need this for including graphics (\includegraphics)
%\usepackage{graphicx}
% for neatly defining theorems and propositions
 \usepackage{amsthm}
% making logically defined graphics
%%%\usepackage{xypic}

% there are many more packages, add them here as you need them

% define commands here

\theoremstyle{definition}
\newtheorem*{thmplain}{Theorem}

\begin{document}
\PMlinkescapeword{onto} \PMlinkescapeword{projection} \PMlinkescapeword{order}

\textbf{Theorem.}\, The \PMlinkname{projection}{ProjectionOfPoint} of a right angle in $\mathbb{R}^3$ onto a plane is a right angle if and only if at least one of its sides is parallel to the plane.\\

\emph{Proof.}\, Consider the projection of an angle $\alpha$ with \PMlinkname{vertex}{Angle} $P$ onto the plane $\pi$.\, Let $P'$ be the projection of $P$ onto $\pi$.\, If neither of the sides of $\alpha$ is parallel to $\pi$, then the lines of the sides intersect the plane in two distinct points $A$ and $B$.\, In order to that the angle of view of the segment $AB$ seen from the point $P$ would be a right angle, $P$ must be on a sphere with diameter $AB$ centered at a point $O$.\, In order to that the projection angle $AP'B$ would be a right angle, the point $P'$ must be on a circle of the plane $\pi$ having $AB$  as diameter.\, But $OP'$ is as the projection of the segment $OP$ shorter than $OP$.\, It follows that the angle $AP'B$ is obtuse and hence cannot be right.\\
On the other hand, it's not hard to see that the projection of a right angle is a right angle always when one or both of its sides are parallel to the projection plane.


\begin{thebibliography}{9}
\bibitem{NP}{\sc E. J. Nystr\"om}: {\em Korkeamman geometrian alkeet sovellutuksineen}.\, Kustannusosakeyhti\"o Otava, Helsinki (1948).
\end{thebibliography}

%%%%%
%%%%%
\end{document}
