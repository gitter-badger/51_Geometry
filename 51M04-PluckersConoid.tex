\documentclass[12pt]{article}
\usepackage{pmmeta}
\pmcanonicalname{PluckersConoid}
\pmcreated{2013-03-22 16:44:02}
\pmmodified{2013-03-22 16:44:02}
\pmowner{Mravinci}{12996}
\pmmodifier{Mravinci}{12996}
\pmtitle{Pl\"ucker's conoid}
\pmrecord{4}{38956}
\pmprivacy{1}
\pmauthor{Mravinci}{12996}
\pmtype{Definition}
\pmcomment{trigger rebuild}
\pmclassification{msc}{51M04}
\pmclassification{msc}{51M20}
\pmclassification{msc}{14J25}
\pmsynonym{Plucker's conoid}{PluckersConoid}
\pmsynonym{Pl\"ucker conoid}{PluckersConoid}
\pmsynonym{Plucker conoid}{PluckersConoid}
\pmsynonym{conical wedge}{PluckersConoid}
\pmsynonym{conocuneus of Wallis}{PluckersConoid}
\pmsynonym{Wallis conocuneus}{PluckersConoid}

\endmetadata

% this is the default PlanetMath preamble.  as your knowledge
% of TeX increases, you will probably want to edit this, but
% it should be fine as is for beginners.

% almost certainly you want these
\usepackage{amssymb}
\usepackage{amsmath}
\usepackage{amsfonts}

% used for TeXing text within eps files
%\usepackage{psfrag}
% need this for including graphics (\includegraphics)
%\usepackage{graphicx}
% for neatly defining theorems and propositions
%\usepackage{amsthm}
% making logically defined graphics
%%%\usepackage{xypic}

% there are many more packages, add them here as you need them

% define commands here

\begin{document}
\emph{Pl\"ucker's conoid} is a ruled surface that results from taking a straight line connected to an axis, rotating it about that axis and moving it straight up and down the axis to give the desired number of folds. Being an example of a right conoid, Pl\"ucker's conoid is sometimes called a conical wedge, or a conocuneus of Wallis or even a cylindroid.

The Cartesian equation for a conoid with two folds is $z = \frac{x^2 - y^2}{x^2 + y^2}$. This can be generalized to any desired number $n$ of folds as $x(r, \theta) = r \cos \theta$, $y(r, \theta) = r \sin \theta$ and $z(r, \theta) = c \sin (n\theta)$. Pl\"ucker's conoid has applications in mechanical drafting.

\begin{thebibliography}{1}
\bibitem{jp} J. Pl\"ucker, ``On a new geometry of space'', {\it Philosophical Transactions of the Royal Society of London} {\bf 155} (1965): 725 - 791
\bibitem{sr} S. P. Radzevich, ``A Possibility of Application of Pliicker's Conoid for Mathematical Modeling of Contact of Two Smooth Regular Surfaces in the First Order of Tangency'', {\it Mathematical and Computer Modelling} {\bf 42} (2005): 999 - 1022
\end{thebibliography}
%%%%%
%%%%%
\end{document}
