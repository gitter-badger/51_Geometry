\documentclass[12pt]{article}
\usepackage{pmmeta}
\pmcanonicalname{Antipodal}
\pmcreated{2013-03-22 13:57:45}
\pmmodified{2013-03-22 13:57:45}
\pmowner{mathcam}{2727}
\pmmodifier{mathcam}{2727}
\pmtitle{antipodal}
\pmrecord{6}{34731}
\pmprivacy{1}
\pmauthor{mathcam}{2727}
\pmtype{Definition}
\pmcomment{trigger rebuild}
\pmclassification{msc}{51M05}
\pmclassification{msc}{15-00}
\pmrelated{DiametralPoints}
\pmdefines{antipodal points}
\pmdefines{antipodal map}

% this is the default PlanetMath preamble.  as your knowledge
% of TeX increases, you will probably want to edit this, but
% it should be fine as is for beginners.

% almost certainly you want these
\usepackage{amssymb}
\usepackage{amsmath}
\usepackage{amsfonts}

% used for TeXing text within eps files
%\usepackage{psfrag}
% need this for including graphics (\includegraphics)
%\usepackage{graphicx}
% for neatly defining theorems and propositions
%\usepackage{amsthm}
% making logically defined graphics
%%%\usepackage{xypic}

% there are many more packages, add them here as you need them

% define commands here

\newcommand{\sR}[0]{\mathbb{R}}
\newcommand{\sC}[0]{\mathbb{C}}
\newcommand{\sN}[0]{\mathbb{N}}
\newcommand{\sZ}[0]{\mathbb{Z}}

% The below lines should work as the command
% \renewcommand{\bibname}{References}
% without creating havoc when rendering an entry in 
% the page-image mode.
\makeatletter
\@ifundefined{bibname}{}{\renewcommand{\bibname}{References}}
\makeatother
\begin{document}
{\bf Definition}
Suppose $x$ and $y$ are points on the 
\PMlinkname{$n$-sphere}{Sphere} $S^n$. 
If $x=-y$ then $x$ and 
$y$ are called {\bf antipodal points}. The
{\bf antipodal map} is the map $A: S^n\to S^n$ defined as
$A(x)=-x$.

\textbf{Properties}
\begin{enumerate}
\item
The antipodal map$A:S^n\to S^n$ is 
\PMlinkname{homotopic}{HomotopyOfMaps}
to the identity map if $n$ is odd \cite{guillemin}.
\item The \PMlinkname{degree}{DegreeMapOfSpheres} of the 
antipodal map is $(-1)^{n+1}$.
\end{enumerate}

\begin{thebibliography}{9}
 \bibitem{guillemin} V. Guillemin, A. Pollack,
\emph{Differential topology}, Prentice-Hall Inc., 1974.
 \end{thebibliography}
%%%%%
%%%%%
\end{document}
