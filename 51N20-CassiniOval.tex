\documentclass[12pt]{article}
\usepackage{pmmeta}
\pmcanonicalname{CassiniOval}
\pmcreated{2013-03-22 17:44:45}
\pmmodified{2013-03-22 17:44:45}
\pmowner{pahio}{2872}
\pmmodifier{pahio}{2872}
\pmtitle{Cassini oval}
\pmrecord{29}{40196}
\pmprivacy{1}
\pmauthor{pahio}{2872}
\pmtype{Definition}
\pmcomment{trigger rebuild}
\pmclassification{msc}{51N20}
\pmclassification{msc}{51-00}
\pmsynonym{cassinoid}{CassiniOval}
\pmsynonym{oval of Cassini}{CassiniOval}
\pmsynonym{Cassini curve}{CassiniOval}
\pmrelated{EuclideanDistance}
\pmrelated{PolarCurve}
\pmdefines{lemniscate of Bernoulli}

% this is the default PlanetMath preamble.  as your knowledge
% of TeX increases, you will probably want to edit this, but
% it should be fine as is for beginners.

% almost certainly you want these
\usepackage{amssymb}
\usepackage{amsmath}
\usepackage{amsfonts}

% used for TeXing text within eps files
%\usepackage{psfrag}
% need this for including graphics (\includegraphics)
%\usepackage{graphicx}
% for neatly defining theorems and propositions
 \usepackage{amsthm}
% making logically defined graphics
%%%\usepackage{xypic}
\usepackage{pstricks}
\usepackage{pst-plot}

% there are many more packages, add them here as you need them

% define commands here

\theoremstyle{definition}
\newtheorem*{thmplain}{Theorem}

\begin{document}
{\em Cassini oval}\; is the locus of the point $P$ in the plane having a constant product of the distances $PF_1$ and $PF_2$ measured from two \PMlinkescapetext{fixed points} $F_1$ and $F_2$ of the plane.

One obtains the simplest equation for the Cassini oval by choosing $F_1$ and $F_2$ on the other coordinate axis and equidistant ($= c > 0$) from the origin.\, Let\, $F_1 = (-c,\,0)$,\; $F_2 = (c,\,0)$\, and the locus condition
$$PF_1 \cdot PF_2 \;=\; a^2 \quad (a > 0).$$
This reads in the Cartesian coordinates
\begin{align}
\sqrt{(x+c)^2+y^2}\sqrt{(x-c)^2+y^2} \;=\; a^2,
\end{align}
which after squaring may be written
$$a^4 \;=\; (x^2\!+\!y^2\!+\!c^2\!+\!2cx)(x^2\!+\!y^2\!+\!c^2\!-\!2cx) \;\equiv\; (x^2\!+\!y^2\!+\!c^2)^2-(2cx)^2,$$
i.e.
\begin{align}
(x^2\!+\!y^2\!+\!c^2)^2-4c^2x^2 \;=\; a^4.
\end{align}
One sees that the curve is symmetric both in regard to $x$-axis and in regard to $y$-axis, whence it suffices to examine it in the first quadrant ($x \geqq 0$,\, $y \geqq 0$).\, If (2) is written as
\begin{align}
y^2 \;=\; \sqrt{a^4\!+\!4c^2x^2}-(x^2\!+\!c^2),
\end{align}
it appears that $y$ is real only for\, $\sqrt{a^4\!+\!4c^2x^2} \geqq x^2\!+\!c^2$,\, which condition can be simplified to
\begin{align}
|x^2\!-\!c^2| \;\leqq\; a^2.
\end{align}
In \PMlinkescapetext{order} to $y$ being real, (4) gives the three cases:\\
$1^{\underline{o}}$.\, $a < c$.\; We have\, $\sqrt{c^2\!-\!a^2} \leqq x \leqq \sqrt{c^2\!+\!a^2}$;\, thus the curve consists of two separate loops.\\
$2^{\underline{o}}$.\, $a = c$.\; Now\, $0 \leqq x \leqq c\sqrt{2}$;\, the two loops meet in the origin (the {\em lemniscate of Bernoulli}).\\
$3^{\underline{o}}$.\, $a > c$.\; Then\, $0 \leqq x \leqq \sqrt{c^2\!+\!a^2}$;\, there is one loop surrounding the origin.

\begin{center}
\begin{pspicture}(-3,-2.2)(3,2.5)
\psset{unit=2cm}
\psaxes[Dx=1,Dy=1]{->}(0,0)(-2.7,-1.5)(2.7,1.6)
\psplot[linecolor=blue]{-1.414}{1.414}{4 x mul x mul 1 add sqrt x x mul sub 1 sub sqrt}
\psplot[linecolor=blue]{-1.414}{1.414}{0 4 x mul x mul 1 add sqrt x x mul sub 1 sub sqrt sub}
\psplot[linecolor=blue]{-1.4865}{1.4865}{4 x mul x mul 1.4641 add sqrt x x mul sub 1 sub sqrt}
\psplot[linecolor=blue]{-1.4865}{1.4865}{0 4 x mul x mul 1.4641 add sqrt x x mul sub 1 sub sqrt sub}
\psplot[linecolor=blue]{-1.732}{1.732}{4 x mul x mul 4 add sqrt x x mul sub 1 sub sqrt}
\psplot[linecolor=blue]{-1.732}{1.732}{0 4 x mul x mul 4 add sqrt x x mul sub 1 sub sqrt sub}
\psplot[linecolor=blue]{-2}{2}{4 x mul x mul 9 add sqrt x x mul sub 1 sub sqrt}
\psplot[linecolor=blue]{-2}{2}{0 4 x mul x mul 9 add sqrt x x mul sub 1 sub sqrt sub}
\psplot[linecolor=blue]{-1.3063}{-0.5415}{4 x mul x mul 0.5 add sqrt x x mul sub 1 sub sqrt}
\psplot[linecolor=blue]{-1.3063}{-0.5415}{0 4 x mul x mul 0.5 add sqrt x x mul sub 1 sub sqrt sub}
\psplot[linecolor=blue]{0.5415}{1.3063}{4 x mul x mul 0.5 add sqrt x x mul sub 1 sub sqrt}
\psplot[linecolor=blue]{0.5415}{1.3063}{0 4 x mul x mul 0.5 add sqrt x x mul sub 1 sub sqrt sub}
\pscircle[linecolor=green](0,0){1}
\psdot[linecolor=red](-1,0)
\psdot[linecolor=red](+1,0)
\psdot[linecolor=blue](0,0)
\rput(-3,-2.2){.}
\rput(3,2.2){.}
\rput(0,-1.7){Cassinoids with\, $c = 1$ (blue)}
\end{pspicture}
\end{center}

When $a$ gets different values (the parametre $c$ being unchanged), (2) \PMlinkescapetext{represents} a family of curves.\, For any point $P$ of the plane (except\, $(\pm c,\,0)$), there is one representant of the family passing through $P$, corresponding the value\, $a = \sqrt{PF_1 \cdot PF_2}$.

As a matter of fact, the common name for all members of the family is {\em Cassini curve}, and only the special case, where\, $a = c\sqrt{2}$,\, is the {\em Cassini oval} proper; it and the other members with\, $a \geqq c\sqrt{2}$\, have the property of having only one highest point\, $(0,\,\sqrt{a^2\!-\!c^2})$.\, All other members (with\, 
$a < c\sqrt{2}$) have two distinct highest points.

The locus of the highest and lowest points of any member with\, $a \leqq c\sqrt{2}$\, is obtained by solving (2) with respect to $x^2$,
$$x^2 \;=\; c^2\!-\!y^2\pm\sqrt{a^4\!-\!4c^2y^2},$$
whence\, $y^2 \leqq \frac{a^4}{4c^2}$.\, When the radicand vanishes, $|y|$ gets its maximum value and then we have\, $x^2 = c^2\!-\!y^2$,\, which means a circle centered in the origin (green in the picture).\\

\textbf{Note 1.}\, Each Cassini oval is the intersection curve of a torus of revolution by a plane parallel to the axis of revolution.

\textbf{Note 2.}\, The astronomer Domenico Cassini found in 1680 the curve named after him; he thought that the \PMlinkescapetext{orbit} of Earth relative to the Sun was a cassinoid with the Sun in the other ``focus''.

\begin{thebibliography}{8}
\bibitem{FI}{\sc F. Iversen}: {\em Analyyttisen geometrian oppikirja}.  Second edition.\, Kustannusosakeyhti\"o Otava, Helsinki (1963).
\end{thebibliography} 




%%%%%
%%%%%
\end{document}
