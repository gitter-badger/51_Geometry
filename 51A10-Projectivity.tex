\documentclass[12pt]{article}
\usepackage{pmmeta}
\pmcanonicalname{Projectivity}
\pmcreated{2013-03-22 15:58:00}
\pmmodified{2013-03-22 15:58:00}
\pmowner{CWoo}{3771}
\pmmodifier{CWoo}{3771}
\pmtitle{projectivity}
\pmrecord{11}{37982}
\pmprivacy{1}
\pmauthor{CWoo}{3771}
\pmtype{Definition}
\pmcomment{trigger rebuild}
\pmclassification{msc}{51A10}
\pmclassification{msc}{51A05}
\pmrelated{PolaritiesAndForms}
\pmrelated{SesquilinearFormsOverGeneralFields}
\pmrelated{Perspectivity}
\pmrelated{ProjectiveSpace}
\pmrelated{LinearFunction}
\pmrelated{Collineation}
\pmdefines{projective transformation}
\pmdefines{projective property}

\endmetadata

\usepackage{amssymb,amscd}
\usepackage{amsmath}
\usepackage{amsfonts}

% used for TeXing text within eps files
%\usepackage{psfrag}
% need this for including graphics (\includegraphics)
%\usepackage{graphicx}
% for neatly defining theorems and propositions
%\usepackage{amsthm}
% making logically defined graphics
%%%\usepackage{xypic}

% define commands here

\begin{document}
Let $PG(V)$ and $PG(W)$ be projective geometries, with $V,W$ vector spaces over a field $K$.  A function $p$ from $PG(V)$ to $PG(W)$ is called a \emph{projective transformation}, or simply a \emph{projectivity} if
\begin{enumerate}
\item $p$ is a bijection, and
\item $p$ is order preserving.
\end{enumerate}

A \emph{projective property} is any geometric property, such as incidence, linearity, etc... that is preserved under a projectivity.

From the definition, we see that a projectivity $p$ carries 0 to 0, $V$ to $W$.  Furthermore, it carries points to points, lines to lines, planes to planes, etc.. In short, $p$ preserves linearity.  Because $p$ is a bijection, $p$ also preserves dimensions, that is $\dim(S)=\dim(p(S))$, for any subspace $S$ of $V$.  In particular, $\dim(V)=\dim(W)$.  Other properties preserved by $p$ are incidence: if $S\cap T\ne\varnothing$, then $p(S)\cap p(T)\neq \varnothing$; and \PMlinkname{cross ratios}{CrossRatio}.

Every bijective semilinear transformation defines a projectiviity.  To see this, let $f:V\to W$ be a semilinear transformation.  If $S$ is a subspace of $V$, then $f(S)$ is a subspace of $W$, as $x,y\in f(S)$, then $x+y=f(a)+f(b)=f(a+b)\in f(S)$, and $\alpha x={\beta}^{\theta}x={\beta}^{\theta}f(a)=f(\beta a)\in f(S)$, where $\theta$ is an automorphism of the common underlying field $K$.  Also, if $S$ is a subspace of a subspace $T$ of $V$, then $f(S)$ is a subspace of $f(T)$.  Now if we define $f^*:PG(V)\to PG(W)$ by $f^*(S)=f(S)$, it is easy to see that $f^*$ is a projectivity.

Conversely, if $V$ and $W$ are of finite dimension greater than $2$, then a projectivity $p:PG(V)\to PG(W)$ induces a semilinear transformation $\hat{p}:V\to W$.  This highly non-trivial fact is the (first) fundamental theorem of projective geometry.

If the semilinear transformation induced by the projectivity $p$ is in fact a linear transformation, then $p$ is a {\it collineation}: three distinct collinear points are mapped to three distinct collinear points.

\textbf{Remark}.  The definition given in this entry is a generalization of the definition typically given for a projective transformation.  In the more restictive definition, a projectivity $p$ is defined merely as a bijection between two projective spaces that is induced by a linear isomorphism.  More precisely, if $P(V)$ and $P(W)$ are projective spaces induced by the vector spaces $V$ and $W$, if $L:V\to W$ is a bijective linear transformation, then $p=P(L):P(V)\to P(W)$ defined by $$P(L)[v]=[Lv]$$ is the corresponding projective transformation.  $[v]$ is the homogeneous coordinate representation of $v$.  In this definition, a projectiity is always a collineation.  In the case where the vector spaces are finite dimensional with specified bases, $p$ is expressible in terms of an invertible matrix ($Lv=Av$ where $A$ is an invertible matrix).
%%%%%
%%%%%
\end{document}
