\documentclass[12pt]{article}
\usepackage{pmmeta}
\pmcanonicalname{RuledSurface}
\pmcreated{2016-03-03 17:28:55}
\pmmodified{2016-03-03 17:28:55}
\pmowner{pahio}{2872}
\pmmodifier{pahio}{2872}
\pmtitle{ruled surface}
\pmrecord{19}{37347}
\pmprivacy{1}
\pmauthor{pahio}{2872}
\pmtype{Topic}
\pmcomment{trigger rebuild}
\pmclassification{msc}{51M20}
\pmclassification{msc}{51M04}
%\pmkeywords{surface}
%\pmkeywords{ruled}
\pmrelated{EquationOfPlane}
\pmrelated{GraphOfEquationXyConstant}
\pmdefines{directrix}
\pmdefines{base curve}
\pmdefines{director curve}
\pmdefines{generatrix}
\pmdefines{generatrices}
\pmdefines{ruling}
\pmdefines{helicoid}

% this is the default PlanetMath preamble.  as your knowledge
% of TeX increases, you will probably want to edit this, but
% it should be fine as is for beginners.

% almost certainly you want these
\usepackage{amssymb}
\usepackage{amsmath}
\usepackage{amsfonts}

% used for TeXing text within eps files
%\usepackage{psfrag}
% need this for including graphics (\includegraphics)
\usepackage{graphicx}
% for neatly defining theorems and propositions
 \usepackage{amsthm}
% making logically defined graphics
%%%\usepackage{xypic}

% there are many more packages, add them here as you need them

% define commands here

\theoremstyle{definition}
\newtheorem*{thmplain}{Theorem}
\begin{document}
A straight line $g$ moving continuously in space sweeps a {\em ruled surface}.\,  Formally:\, A surface $S$ in $\mathbb{R}^3$ is a ruled surface if it is connected and if for any point $p$ of $S$, there is a line $g$ such that\, $p\in g\subset S$. 

 Such a surface may be formed by using two auxiliary curves given e.g. in the parametric forms
$$\vec{r} \;=\; \vec{a}(t), \qquad \vec{r} \;=\; \vec{b}(t).$$
 Using two parameters $s$ and $t$ we express the 
 \PMlinkname{position vector}{PositionVector} of  an arbitrary point of the ruled surface as
$$\vec{r} \;=\; \vec{a}(t)+ s\,\vec{b}(t).$$
Here\, $\vec{r} = \vec{a}(t)$\, is a curve on the ruled surface and is called \PMlinkescapetext{{\em directrix}} or the 
\PMlinkescapetext{{\em base curve}} of the surface, while\, $\vec{r} = \vec{b}(t)$\, is the {\em director curve} of the surface.\, Every position of $g$ is a {\em generatrix} or {\em ruling} of the ruled surface.

\textbf{Examples}

1.\, Choosing the $z$-axis ($\vec{r} = ct\vec{k}$,\, $c \neq 0$) as the \PMlinkescapetext{directrix} and the unit circle ($\vec{r} = \vec{i}\cos{t}+\vec{j}\sin{t}$) 
as the director curve we get the {\em helicoid} (``screw surface''; cf. the circular helix)
$$\vec{r} \;=\; ct\vec{k}+ s\,(\vec{i}\cos{t}+\vec{j}\sin{t}) \;=\;
   \left(\!\begin{array}{c}s\,\cos{t}\\ s\,\sin{t}\\ ct\end{array}\!\right)\!.$$

%\begin{figure}[!htb]
%\begin{center}
%\includegraphics{helicoid.eps}
%\end{center}
%\caption{The helicoid as a ruled surface}
%\end{figure}

2.\, The equation\,
$$z \;=\; xy$$
presents a hyperbolic paraboloid (if we \PMlinkname{rotate the coordinate system}{RotationMatrix} 45 \PMlinkescapetext{degrees} about the $z$-axis using the formulae\, $x = (x'-y')/\sqrt{2}$,\, 
$y = (x'+y')/\sqrt{2}$,\, the equation gets the form\, $x'^2-y'^2 = 2z$).\, Since the position vector of any point of the surface may be written using the parameters $s$ and $t$ as
$$\vec{r} \;=\; \left(\!\begin{array}{c}0\\ t\\ 0\end{array}\!\right)\!
+s\left(\!\begin{array}{c}1\\ 0\\ t\end{array}\!\right)\!,$$
we see that it's a question of a ruled surface with rectilinear directrix and director curve.

3.\, Other ruled surfaces are for example all cylindrical 
surfaces (plane included), conical surfaces, 
\PMlinkname{one-sheeted hyperboloid}{QuadraticSurfaces}.
%%%%%
%%%%%
\end{document}
