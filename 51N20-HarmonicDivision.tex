\documentclass[12pt]{article}
\usepackage{pmmeta}
\pmcanonicalname{HarmonicDivision}
\pmcreated{2013-03-22 17:34:29}
\pmmodified{2013-03-22 17:34:29}
\pmowner{pahio}{2872}
\pmmodifier{pahio}{2872}
\pmtitle{harmonic division}
\pmrecord{7}{39986}
\pmprivacy{1}
\pmauthor{pahio}{2872}
\pmtype{Definition}
\pmcomment{trigger rebuild}
\pmclassification{msc}{51N20}
\pmclassification{msc}{51M04}
%\pmkeywords{division ratio}
\pmrelated{BisectorsTheorem}
\pmrelated{ApolloniusCircle}
\pmdefines{harmonically}
\pmdefines{divide harmonically}

\endmetadata

% this is the default PlanetMath preamble.  as your knowledge
% of TeX increases, you will probably want to edit this, but
% it should be fine as is for beginners.

% almost certainly you want these
\usepackage{amssymb}
\usepackage{amsmath}
\usepackage{amsfonts}

% used for TeXing text within eps files
%\usepackage{psfrag}
% need this for including graphics (\includegraphics)
%\usepackage{graphicx}
% for neatly defining theorems and propositions
 \usepackage{amsthm}
% making logically defined graphics
%%%\usepackage{xypic}

% there are many more packages, add them here as you need them

% define commands here

\theoremstyle{definition}
\newtheorem*{thmplain}{Theorem}

\begin{document}
\PMlinkescapeword{divides} \PMlinkescapeword{divide}
\begin{itemize}
\item If the point $X$ is on the line segment $AB$ and\, $XA\!:\!XB = p\!:\!q$,\, then $X$ divides $AB$ {\em internally} in the ratio $p\!:\!q$.
\item If the point $Y$ is on the extension of line segment $AB$ and\, $YA\!:\!YB = p\!:\!q$,\, then $Y$ divides $AB$ {\em externally} in the ratio $p\!:\!q$.
\item If $p\!:\!q$ is the same in both cases, then the points $X$ and $Y$ divide $AB$ {\em harmonically} in the ratio $p\!:\!q$.
\end{itemize}

\textbf{Theorem 1.}  The bisectors of an angle of a triangle and its linear pair divide the opposite side of the triangle harmonically in the ratio of the adjacent sides.

\textbf{Theorem 2.}  If the points $X$ and $Y$ divide the line segment $AB$ harmonically in the ratio $p\!:\!q$, then the circle with diameter the segment $XY$ (the so-called Apollonius' circle) is the locus of such points whose distances from $A$ and $B$ have the ratio $p\!:\!q$.

The latter theorem may be proved by using analytic geometry.

%%%%%
%%%%%
\end{document}
