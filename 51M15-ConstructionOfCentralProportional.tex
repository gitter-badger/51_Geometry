\documentclass[12pt]{article}
\usepackage{pmmeta}
\pmcanonicalname{ConstructionOfCentralProportional}
\pmcreated{2013-03-22 17:34:14}
\pmmodified{2013-03-22 17:34:14}
\pmowner{pahio}{2872}
\pmmodifier{pahio}{2872}
\pmtitle{construction of central proportional}
\pmrecord{14}{39981}
\pmprivacy{1}
\pmauthor{pahio}{2872}
\pmtype{Algorithm}
\pmcomment{trigger rebuild}
\pmclassification{msc}{51M15}
\pmrelated{GoldenRatio}
\pmrelated{CompassAndStraightedgeConstructionOfGeometricMean}
\pmrelated{ConstructionOfFourthProportional}
\pmdefines{cathetus}
\pmdefines{catheti}

% this is the default PlanetMath preamble.  as your knowledge
% of TeX increases, you will probably want to edit this, but
% it should be fine as is for beginners.

% almost certainly you want these
\usepackage{amssymb}
\usepackage{amsmath}
\usepackage{amsfonts}

% used for TeXing text within eps files
%\usepackage{psfrag}
% need this for including graphics (\includegraphics)
%\usepackage{graphicx}
% for neatly defining theorems and propositions
 \usepackage{amsthm}
% making logically defined graphics
%%%\usepackage{xypic}

% there are many more packages, add them here as you need them

\usepackage{pstricks}

% define commands here

\theoremstyle{definition}
\newtheorem*{thmplain}{Theorem}

\begin{document}
\PMlinkescapeword{solution}
\textbf{Task.}  Given two line segments $p$ and $q$.  Using compass and straightedge, construct the central proportional (the geometric mean) of the line segments.

{\em Solution.}  Set the line segments\, $AD = p$\, and\, $DB = q$\, on a line so that $D$ is between $A$ and $B$.  Draw a half-circle with diameter $AB$ (for finding the centre, see the entry midpoint).  Let $C$ be the point where the normal line of $AB$ passing through $D$ intersects the arc of the half-circle.  The line segment $CD$ is the required central proportional.  Below is a picture that illustrates this solution:

\begin{center}
\begin{pspicture}(-3,-1)(3,3)
\rput[r](3,0){.}
\rput[a](0,2.5){.}
\psline(-3,0)(3,0)
\psarc(0,0){2.5}{0}{180}
\psline[linecolor=blue](-0.5,0)(-0.5,2.45)
\psline[linestyle=dashed](-2.5,0)(-0.5,2.45)
\psline[linestyle=dashed](2.5,0)(-0.5,2.45)
\psdots(-2.5,0)(-0.5,0)(2.5,0)(-0.5,2.45)
\rput[a](-2.5,-0.3){$A$}
\rput[a](-0.5,-0.3){$D$}
\rput[a](2.5,-0.3){$B$}
\rput[b](-0.5,2.63){$C$}
\rput[b](-1.4,0.11){$p$}
\rput[b](0.8,0.11){$q$}
\end{pspicture}
\end{center}

(For more details on the procedure to create this picture, see compass and straightedge construction of geometric mean.)

{\em Proof.}  By Thales' theorem, the triangle $ABC$ is a right triangle.  Its height $CD$ \PMlinkescapetext{divides} this triangle into two smaller right triangles which have equal angles with the triangle $ABC$ and thus are \PMlinkname{similar}{SimilarityInGeometry}.  Accordingly, we can write the proportion equation concerning the catheti of the smaller triangles
$$p:CD\, = \,CD:q.$$
The equation shows that $CD$ is the central proportional of $p$ and $q$.

\textbf{Note.} The word {\em catheti} (in sing. {\em cathetus}) \PMlinkescapetext{means} the two shorter sides of a right triangle.
%%%%%
%%%%%
\end{document}
