\documentclass[12pt]{article}
\usepackage{pmmeta}
\pmcanonicalname{FiniteProjectivePlane}
\pmcreated{2013-03-22 15:11:15}
\pmmodified{2013-03-22 15:11:15}
\pmowner{Mathprof}{13753}
\pmmodifier{Mathprof}{13753}
\pmtitle{finite projective plane}
\pmrecord{11}{36943}
\pmprivacy{1}
\pmauthor{Mathprof}{13753}
\pmtype{Topic}
\pmcomment{trigger rebuild}
\pmclassification{msc}{51E15}
\pmclassification{msc}{51A35}
\pmclassification{msc}{05B25}
%\pmkeywords{incidence}
%\pmkeywords{design}
%\pmkeywords{graph}
%\pmkeywords{geometry}
\pmrelated{ProjectivePlane2}
\pmrelated{IncidenceStructures}
\pmrelated{Geometry}
\pmrelated{DeBruijnErdHosTheorem}
\pmdefines{prime power conjecture}
\pmdefines{planar ternary ring}
\pmdefines{PTR}

\usepackage{amssymb}
% \usepackage{amsmath}
% \usepackage{amsfonts}

% used for TeXing text within eps files
%\usepackage{psfrag}
% need this for including graphics (\includegraphics)
%\usepackage{graphicx}

% for neatly defining theorems and propositions
%\usepackage{amsthm}
% making logically defined graphics
%%%\usepackage{xypic}

% there are many more packages, add them here as you need them

% define commands here %%%%%%%%%%%%%%%%%%%%%%%%%%%%%%%%%%
% portions from
% makra.sty 1989-2005 by Marijke van Gans %
                                          %          ^ ^
\catcode`\@=11                            %          o o
                                          %         ->*<-
                                          %           ~
%%%% CHARS %%%%%%%%%%%%%%%%%%%%%%%%%%%%%%%%%%%%%%%%%%%%%%

                        %    code    char  frees  for

\let\Para\S             %    \Para     ÃÃÃÃÃÃÃÃÃÃÃÃç   \S \scriptstyle
\let\Pilcrow\P          %    \Pilcrow  ÃÃÃÃÃÃÃÃÃÃÃÃö   \P
\mathchardef\pilcrow="227B

\mathchardef\lt="313C   %    \lt       <   <     bra
\mathchardef\gt="313E   %    \gt       >   >     ket

\let\bs\backslash       %    \bs       \
\let\us\_               %    \us       _     \_  ...

%%%% DIACRITICS %%%%%%%%%%%%%%%%%%%%%%%%%%%%%%%%%%%%%%%%%

%let\udot\d             % under-dot (text mode), frees \d
\let\odot\.             % over-dot (text mode),  frees \.
%let\hacek\v            % hacek (text mode),     frees \v
%let\makron\=           % makron (text mode),    frees \=
%let\tilda\~            % tilde (text mode),     frees \~
\let\uml\"              % umlaut (text mode),    frees \"

%def\ij/{i{\kern-.07em}j}
\def\trema#1{\discretionary{-}{#1}{\uml #1}}

%%%% amssymb %%%%%%%%%%%%%%%%%%%%%%%%%%%%%%%%%%%%%%%%%%%%

\let\le\leqslant
\let\ge\geqslant
%let\prece\preceqslant
%let\succe\succeqslant

%%%% USEFUL MISC %%%%%%%%%%%%%%%%%%%%%%%%%%%%%%%%%%%%%%%%

%def\C++{C$^{_{++}}$}

%let\writelog\wlog
%def\wl@g/{{\sc wlog}}
%def\wlog{\@ifnextchar/{\wl@g}{\writelog}}

%def\org#1{\lower1.2pt\hbox{#1}} 
% chem struct formulae: \bs, --- /  \org{C} etc. 

%%%% USEFUL INTERNAL LaTeX STUFF %%%%%%%%%%%%%%%%%%%%%%%%

%let\Ifnextchar=\@ifnextchar
%let\Ifstar=\@ifstar
%def\currsize{\@currsize}

%%%% KERNING, SPACING, BREAKING %%%%%%%%%%%%%%%%%%%%%%%%%

\def\comma{,\,\allowbreak}

%def\qqquad{\hskip3em\relax}
%def\qqqquad{\hskip4em\relax}
%def\qqqqquad{\hskip5em\relax}
%def\qqqqqquad{\hskip6em\relax}
%def\qqqqqqquad{\hskip7em\relax}
%def\qqqqqqqquad{\hskip8em\relax}

%%%% LAYOUT %%%%%%%%%%%%%%%%%%%%%%%%%%%%%%%%%%%%%%%%%%%%%

%%%% COUNTERS %%%%%%%%%%%%%%%%%%%%%%%%%%%%%%%%%%%%%%%%%%%

%let\addtoreset\@addtoreset
%{A}{B} adds A to list of counters reset to 0
% when B is \refstepcounter'ed (see latex.tex)
%
%def\numbernext#1#2{\setcounter{#1}{#2}\addtocounter{#1}{\m@ne}}

%%%% EQUATIONS %%%%%%%%%%%%%%%%%%%%%%%%%%%%%%%%%%%%%%%%%%

%%%% LEMMATA %%%%%%%%%%%%%%%%%%%%%%%%%%%%%%%%%%%%%%%%%%%%

%%%% DISPLAY %%%%%%%%%%%%%%%%%%%%%%%%%%%%%%%%%%%%%%%%%%%%

%%%% MATH LAYOUT %%%%%%%%%%%%%%%%%%%%%%%%%%%%%%%%%%%%%%%%

\let\D\displaystyle
\let\T\textstyle
\let\S\scriptstyle
\let\SS\scriptscriptstyle

% array:
%def\<#1:{\begin{array}{#1}}
%def\>{\end{array}}

% array using [ ] with rounded corners:
%def\[#1:{\left\lgroup\begin{array}{#1}} 
%def\]{\end{array}\right\rgroup}

% array using ( ):
%def\(#1:{\left(\begin{array}{#1}}
%def\){\end{array}\right)}

%def\hh{\noalign{\vskip\doublerulesep}}

%%%% MATH SYMBOLS %%%%%%%%%%%%%%%%%%%%%%%%%%%%%%%%%%%%%%%

%def\d{\mathord{\rm d}}                      % d as in dx
%def\e{{\rm e}}                              % e as in e^x

%def\Ell{\hbox{\it\char`\$}}

\def\sfmath#1{{\mathchoice%
{{\sf #1}}{{\sf #1}}{{\S\sf #1}}{{\SS\sf #1}}}}
\def\Stalkset#1{\sfmath{I\kern-.12em#1}}
\def\Bset{\Stalkset B}
\def\Nset{\Stalkset N}
\def\Rset{\Stalkset R}
\def\Hset{\Stalkset H}
\def\Fset{\Stalkset F}
\def\kset{\Stalkset k}
\def\In@set{\raise.14ex\hbox{\i}\kern-.237em\raise.43ex\hbox{\i}}
\def\Roundset#1{\sfmath{\kern.14em\In@set\kern-.4em#1}}
\def\Qset{\Roundset Q}
\def\Cset{\Roundset C}
\def\Oset{\Roundset O}
\def\Zset{\sfmath{Z\kern-.44emZ}}

% \frac overwrites LaTeX's one (use TeX \over instead)
%def\fraq#1#2{{}^{#1}\!/\!{}_{\,#2}}
\def\frac#1#2{\mathord{\mathchoice%
{\T{#1\over#2}}
{\T{#1\over#2}}
{\S{#1\over#2}}
{\SS{#1\over#2}}}}
%def\half{\frac12}

\mathcode`\<="4268         % < now is \langle, \lt is <
\mathcode`\>="5269         % > now is \rangle, \gt is >

%def\biggg#1{{\hbox{$\left#1\vbox %to20.5\p@{}\right.\n@space$}}}
%def\Biggg#1{{\hbox{$\left#1\vbox %to23.5\p@{}\right.\n@space$}}}

\let\epsi=\varepsilon
\def\omikron{o}

\def\Alpha{{\rm A}}
\def\Beta{{\rm B}}
\def\Epsilon{{\rm E}}
\def\Zeta{{\rm Z}}
\def\Eta{{\rm H}}
\def\Iota{{\rm I}}
\def\Kappa{{\rm K}}
\def\Mu{{\rm M}}
\def\Nu{{\rm N}}
\def\Omikron{{\rm O}}
\def\Rho{{\rm P}}
\def\Tau{{\rm T}}
\def\Ypsilon{{\rm Y}} % differs from \Upsilon
\def\Chi{{\rm X}}

%def\dg{^{\circ}}                   % degrees

%def\1{^{-1}}                       % inverse

\def\*#1{{\bf #1}}                  % boldface e.g. vector
%def\vi{\mathord{\hbox{\bf\i}}}     % boldface vector \i
%def\vj{\mathord{\,\hbox{\bf\j}}}   % boldface vector \j

%def\union{\mathbin\cup}
%def\isect{\mathbin\cap}

\let\so\Longrightarrow
\let\oso\Longleftrightarrow
\let\os\Longleftarrow

% := and :<=>
%def\isdef{\mathrel{\smash{\stackrel{\SS\rm def}{=}}}}
%def\iffdef{\mathrel{\smash{stackrel{\SS\rm def}{\oso}}}}

\def\isdef{\mathrel{\mathop{=}\limits^{\smash{\hbox{\tiny def}}}}}
%def\iffdef{\mathrel{\mathop{\oso}\limits^{\smash{\hbox{\tiny %def}}}}}

%def\tr{\mathop{\rm tr}}            % tr[ace]
%def\ter#1{\mathop{^#1\rm ter}}     % k-ter[minant]

%let\.=\cdot
%let\x=\times                % ><Ã (direct product)

%def\qed{ ${\S\circ}\!{}^\circ\!{\S\circ}$}
%def\qed{\vrule height 6pt width 6pt depth 0pt}

%def\edots{\mathinner{\mkern1mu
%   \raise7pt\vbox{\kern7pt\hbox{.}}\mkern1mu   %  .shorter
%   \raise4pt\hbox{.}\mkern1mu                  %     .
%   \raise1pt\hbox{.}\mkern1mu}}                %        .
%def\fdots{\mathinner{\mkern1mu
%   \raise7pt\vbox{\kern7pt\hbox{.}}            %   . ~45ÃÃÃÃÃÃÃÃÃð
%   \raise4pt\hbox{.}                           %     .
%   \raise1pt\hbox{.}\mkern1mu}}                %       .

\def\mod#1{\allowbreak \mkern 10mu({\rm mod}\,\,#1)}
% redefines TeX's one using less space

%def\int{\intop\displaylimits}
%def\oint{\ointop\displaylimits}

%def\intoi{\int_0^1}
%def\intall{\int_{-\infty}^\infty}

%def\su#1{\mathop{\sum\raise0.7pt\hbox{$\S\!\!\!\!\!#1\,$}}}

%let\frakR\Re
%let\frakI\Im
%def\Re{\mathop{\rm Re}\nolimits}
%def\Im{\mathop{\rm Im}\nolimits}
%def\conj#1{\overline{#1\vphantom1}}
%def\cj#1{\overline{#1\vphantom+}}

%def\forAll{\mathop\forall\limits}
%def\Exists{\mathop\exists\limits}

%%%% PICTURES %%%%%%%%%%%%%%%%%%%%%%%%%%%%%%%%%%%%%%%%%%%

\def\cent{\makebox(0,0)}

\def\node{\circle*4}
\def\nOde{\circle4}

%%%% REFERENCES %%%%%%%%%%%%%%%%%%%%%%%%%%%%%%%%%%%%%%%%%

%def\opcit{[{\it op.\,cit.}]}
\def\bitem#1{\bibitem[#1]{#1}}
\def\name#1{{\sc #1}}
\def\book#1{{\sl #1\/}}
\def\paper#1{``#1''}
\def\mag#1{{\it #1\/}}
\def\vol#1{{\bf #1}}
\def\isbn#1{{\small\tt ISBN\,\,#1}}
\def\seq#1{{\small\tt #1}}
%def\url<{\verb>}
%def\@cite#1#2{[{#1\if@tempswa\ #2\fi}]}

%%%% VERBATIM CODE %%%%%%%%%%%%%%%%%%%%%%%%%%%%%%%%%%%%%%

%def\"{\verb"}

%%%% AD HOC %%%%%%%%%%%%%%%%%%%%%%%%%%%%%%%%%%%%%%%%%%%%%

%%%% WORDS %%%%%%%%%%%%%%%%%%%%%%%%%%%%%%%%%%%%%%%%%%%%%%

% \hyphenation{pre-sent pre-sents pre-sent-ed pre-sent-ing
% re-pre-sent re-pre-sents re-pre-sent-ed re-pre-sent-ing
% re-fer-ence re-fer-ences re-fer-enced re-fer-encing
% ge-o-met-ry re-la-ti-vi-ty Gauss-ian Gauss-ians
% Des-ar-gues-ian}

%def\oord/{o{\trema o}rdin\-ate}
% usage: C\oord/s, c\oord/.
% output: co\"ord... except when linebreak at co-ord...

%%%%%%%%%%%%%%%%%%%%%%%%%%%%%%%%%%%%%%%%%%%%%%%%%%%%%%%%%

                                          %
                                          %          ^ ^
\catcode`\@=12                            %          ` '
                                          %         ->*<-
                                          %           ~
\begin{document}
\PMlinkescapeword{columns}
\PMlinkescapeword{components}
\PMlinkescapeword{entire}
\PMlinkescapeword{equivalent}
\PMlinkescapeword{fixed}
\PMlinkescapeword{index}
\PMlinkescapeword{matching}
\PMlinkescapeword{order}
\PMlinkescapeword{place}
\PMlinkescapeword{planar}
\PMlinkescapeword{potential}
\PMlinkescapeword{relation}
\PMlinkescapeword{represent}
\PMlinkescapeword{ring}
\PMlinkescapeword{rows}
\PMlinkescapeword{satisfies}
\PMlinkescapeword{satisfy}
\PMlinkescapeword{simple}
\PMlinkescapeword{states}
\PMlinkescapeword{structure}
\PMlinkescapeword{structures}
\PMlinkescapeword{terms}
\PMlinkescapeword{time}
\PMlinkescapeword{line}
\PMlinkescapeword{point}

\subsection*{Finite projective planes from axioms}

Let us start from the beginning and assume we are given a set of $n$ things called {\sc point}s and a set of $\nu$ things called {\sc line}s, with an
incidence relation between them (a {\sc point} is incident with a {\sc line} iff the {\sc line} is incident with the {\sc point}) that satisfies only two {\bf axioms}, the two unique incidence properties:
%
\begin{itemize}

\item For any two distinct {\sc point}s,
      there is exactly one {\sc line} through both of them.

\item For any two distinct {\sc line}s,
      there is exactly one {\sc point} on both of them.

\end{itemize}
%
Given these \PMlinkid{projective plane}{6940} axioms we can apply the De Bruijn--Erd\H{o}s theorem twice (once with the r\^ole of {\sc point}s and {\sc line}s swapped) which tells us we must have $n=\nu$, and one of two possible configurations:
%
\begin{itemize}

\item[$\circ$] One ``degenerate'' solution has one special line
               $\ell_0$ and $n-1$ ordinary lines $\ell_i$, and a special point
               P$_0$ and $n-1$ ordinary points P$_i$. Each P$_i$ ($i\ne0$) is
               incident with its own $\ell_i$ and with $\ell_0$; each $\ell_i$
               ($i\ne0$) with its own P$_i$ and with P$_0$. Now $\ell_0$ is
               incident with every point except $\ell_0$ while P$_0$ is
               incident with every line except $\ell_0$.

\item[$\circ$] The other solution is a projective plane:

               \begin{itemize}

               \item[$\bullet$] the number of {\sc point}s and the number of
                    {\sc line}s are of the form $q^2+q+1$ for some integer $q$;

               \item[$\bullet$] each {\sc point} is incident with $q+1$
                    {\sc line}s and each {\sc line} with $q+1$ {\sc point}s.

               \end{itemize}

\end{itemize}
%
One usually avoids the degenerate case by demanding there is a {\bf quadrangle}
in the plane: four {\sc point}s no three of which lie on the same {\sc line}. This, or a
similar non-triviality clause, can then be adopted as a third axiom.
Unfortunately it also has the effect of excluding projective planes of orders $q=0$ and $1$. The
\PMlinkname{De Bruijn--Erd\"os}{DeBruijnErdHosTheorem}
theorem actually makes an additional assumption: that
no two lines are incident with the same set of points (and when applied dually
that no two points are incident with the same set of lines). That is not a
problem for a {\sc line} incident with more than one {\sc point} or vice versa, because
then the axioms forbid it. We just get some more degenerate cases consisting
of multiple lines intersecting one point or none, or vice versa; the insistence
on a quadrangle catches these too.

I will not prove the whole theorem here, but under the assumption
that a quadrangle exists it is not hard show the interesting part,
that those numbers $q+1$ and $q^2+q+1$ drop out by necessity.
The same construction can also be used to give coordinates to the
{\sc point}s and {\sc line}s of the plane (see below).

\clearpage
\subsection*{Projective planes as bipartite graphs}

Projective planes are, apart from anything else, also {\bf bipartite graphs}: let the
$q^2+q+1$ {\sc point}s be represented by ``black'' nodes (vertices) of the graph,
and let the $q^2+q+1$ {\sc line}s {\em also\/} be nodes, ``white'' ones say.
Edges will represent incidence between a ``black'' and a ``white'' node,
so there are $(q^2+q+1)(q+1)=q^3+2q^2+2q+1$ edges in all.
%
\begin{itemize}

\item In a bipartite graph, nodes of the same color have even distance.
      This is 0 for any node and itself and {\bf must be 2} for any two
      distinct nodes of the same color: for any two {\sc point}s there is a path
      via the {\sc line} they share, and for any two {\sc line}s a path via the {\sc point}
      they share. So distance 4 does not occur.

      For nodes of opposite color we can have 1 (if the {\sc point} lies on the
      {\sc line}) or 3 (otherwise); we can't have distance 5 which would imply
      4 for some nodes. So the graph has {\bf diameter} (maximum distance) 3.

\item In a bipartite graph, all cycles (if any) have even length. We
      don't have 2-cycles (this is not a multigraph) but no 4-cycles
      either, by the following argument: such a cycle P$\ell$Q$m$
      would mean two {\sc line}s intersecting in two different {\sc point}s, and
      vice versa. So the {\bf girth} (minimum cycle length) is 6
      (showing there indeed are 6-cycles for $q\ge2$ is left as
      exercise for the reader).

\end{itemize}
%
These two conditions, diameter~3 and girth~6, are not only necessary for
the graph to represent a projective plane, they are also sufficient, and therefore form
an alternative formulation of what it means to be a projective plane.

These graphs could be called the bipartite analogue of Moore graphs of
diameter 2 and girth 5; many of the same arguments can be used~\cite{vG03}.

\clearpage
\subsection*{Co\uml{o}rdinates for finite projective planes}

The details on how to impose coordinates on a projective plane can be found in \cite{Hal59,dRe96}, or in \PMlinkname{this entry}{IntroducingCoordinatesInAProjectivePlane}.  We summarize it for a finite projective plane below:

Setup: Use a set $Q$ of $q$ different symbols for coordinate values, two of which are 0 and 1.  Start with four {\sc point}s O, I, X, Y forming a quadrangle.  XY is the ``{\sc line} at infinity'' $\ell$.

Construction: O and I are given coordinates $(0,0)$ and $(1,1)$, and the other {\sc point}s on OI (except its intersection with $\ell$) arbitrarily $(c,c)$ with unique $c\in Q$.  For any {\sc point} P not on $\ell$, let YP
intersect OI at $(a,a)$, and let XP intersect OI at $(b,b)$.  Then P will be given $(a,b)$.  For the coordinates of {\sc point}s on $\ell$, intersect the {\sc line}s joining $(0,0)$ with $(1,m)$, for all $m \in Q$, with $\ell$, calling the intersection $(m)$.  Finally, Y gets coordinate $(\,)$ (instead of $(\infty)$ in the literature).

{\sc Lines} not through Y are given coordinates $[\,m,k\,]$ where $(m)$ and $(0,k)$ are their intersections with $\ell$ and with OY respectively; {\sc line}s through Y (other than $\ell$) get coordinates $[\,k\,]$ going by their
intersection $(k,0)$ with OX.  The {\sc line} $\ell$ gets $[\;]$ (instead of $[\,\infty\,]$ in the literature).

\clearpage
\subsection*{Coordinates from bipartite graphs}

The whole business with coordinates becomes a bit more transparent when we try
to actually construct the thing as a bipartite graph, starting from one edge,
Y$\ell$ say.

There are $q$ more {\sc point}s (other than Y) incident to $\ell$, let's call
them Y$_m$ where $m$ ranges over the $q$ different values of the set $Q$.
Again, each Y$_m$ is incident to $q$ more {\sc line}s (apart from $\ell$), let's
call those $\ell_{mb}$ where $b$ is another subscript from $Q$.
Likewise there are $q$ more {\sc line}s (other than $\ell$) incident to Y, let's
call them $\ell_x$ and to each of them are incident $q$ more {\sc point}s (apart
from Y) which we can label Y$_{xy}$.

This immediately produces the incidence list of the preceding section.
Pictorially, for $q=2$ (with the ``to be assessed'' portion filled in
as thinner lines):

\begin{center}
\begin{picture}(100,140)(-50,-70)

\put(-40,-30){\circle{4}}
\put(  0,-60){\circle*{4}}
\put(+20,-60){\circle*{4}}
\put(+20,-30){\circle{4}}
\put(+30,-30){\circle{4}}
\put(+40,-30){\circle{4}}
\put(+50,-30){\circle{4}}
\put(+20,+30){\circle*{4}}
\put(+30,+30){\circle*{4}}
\put(+40,+30){\circle*{4}}
\put(+50,+30){\circle*{4}}
\put(+20,+60){\circle{4}}
\put(  0,+60){\circle{4}}
\put(-40,+30){\circle*{4}}

\put(+55,+30){\cent[l]{Y$_{xy}$}}
\put( 10,+70){\cent[b]{$\ell_x$}}
\put(-50,+30){\cent[r]{Y}}
\put(-50,-30){\cent[r]{$\ell$}}
\put( 10,-70){\cent[t]{Y$_m$}}
\put(+55,-30){\cent[l]{$\ell_{mb}$}}

\thicklines
\put(-40,-30){\line( 0,+1){60}}
\put(-40,+30){\line(+4,+3){40}}
\put(-40,+30){\line(+2,+1){60}}
\put(  0,+60){\line(+2,-3){20}}
\put(  0,+60){\line(+4,-3){40}}
\put(+20,+60){\line(+1,-3){10}}
\put(+20,+60){\line(+1,-1){30}}
\put(  0,-60){\line(+2,+3){20}}
\put(  0,-60){\line(+4,+3){40}}
\put(+20,-60){\line(+1,+3){10}}
\put(+20,-60){\line(+1,+1){30}}
\put(-40,-30){\line(+2,-1){60}}
\put(-40,-30){\line(+4,-3){40}}

\thinlines
\put(+20,-30){\line( 0,+1){60}}
\put(+30,-30){\line( 0,+1){60}}
\put(+40,-30){\line( 0,+1){60}}
\put(+50,-30){\line( 0,+1){60}}
\put(+20,-30){\line(+1,+6){10}}
\put(+30,-30){\line(+1,+6){10}}
\put(+40,-30){\line(+1,+6){10}}
\put(+50,-30){\line(-1,+2){30}}

\end{picture}
\end{center}

We already have $\ell_0$ containing all points Y$_{0y}$, and in addition if we
wanted we could easily arrange for $\ell_{00}$ to be incident with all the
Y$_{x0}$ (just renumber the labels $y$ in the sets Y$_{x0}$ for every $x$),
and for $\ell_{10}$ to be incident with all the Y$_{cc}$ (just look at which
Y$_{xy}$ are incident with $\ell_{10}$, then exercise the freedom we still have
to relabel all the $x$, matching each time the corresponding $y$ there). This
reproduces the entire coordinatisation.

\subsection*{Projective planes as permutation systems}

Five sides of the hexagon are self-evident. On the left we have the
single edge Y$\ell$, along the top we fan out in two stages to $q$ and $q^2$
items, along the bottom we also fan out in two stages to $q$ and $q^2$ items,
and the hard part is the right. We have by now accounted for $2q^2+2q+1$ edges
so we know there must be exactly $q^3$ more edges, between {\em some\/} of
those $q^2$ {\sc point}s and {\em some\/} of those $q^2$ {\sc line}s. But where?

For $q=2$ it is simple enough to avoid 4-cycle clashes (and the diagram shows
what is, up to isomorphism, the only solution). In general, it is an open
problem to find all possible solutions.

From the projective plane properties it is easy to see that any point Y$_{xy}$ must, for
every value of $m$, be incident to no more than one of the $\ell_{mb}$ (because
if it was incident to two they would intersect both in our Y$_{xy}$ and in
their shared Y$_m$ at the bottom). On the other hand, Y$_{xy}$ needs $q$ more
mates so it needs to choose exactly one $\ell_{mb}$ for every value of $m$.

By exactly the same reasoning, any $\ell_{mb}$ needs to be incident with, for
every value of $x$, at most one and then exactly one Y$_{xy}$. In other words,
the projective plane structure takes the form of, for each pair of values $x$ and $m$,
a {\bf bijection} $\psi_{xm}$ between the $y$ indices and the $b$ indices.

This is not yet the only constraint. A closer analysis~\cite{vG03} shows
that, in order to prevent any two {\sc line}s meeting in more than one {\sc point} or any
two {\sc point}s in more than two {\sc line}s, all compound bijections of the form
$$
  \psi_{xm}^{\vphantom1} \psi_{x'm}^{-1}
  \psi_{x'm'}^{\vphantom1} \psi_{xm'}^{-1}
$$
(for $m \ne m'$ and $x \ne x'$) must be {\bf derangements}, i.e.\
permutations that do not leave any index in place. It is not hard to see
this is just a device to prevent 4-cycles. And careful scrutiny of the graph
shows that, after arranging 5 of 6 edges along trees fanning out as above, and
after recognising there are those bijections along the sixth edge, these
derangement shenanigans stop the only remaining gap for 4-cycles to lurk.

This could be called the permutation group interpretation of projective plane structure.
The algebraic version appears in the next section.

\clearpage
\subsection*{Planar ternary rings}

We've been giving our {\sc point}s and {\sc line}s cooordinates from a set $Q$ of $q$ different labels.   We can give $Q$ the structure of a {\bf ternary ring}, that is, a set with a ternary operation $p\bullet q\circ r$, by defining
%%
\begin{equation}\label{x*mob}
  y = x\bullet m\circ b \;\oso\; (x,y) \;\hbox{is on}\; [\,m,b\,]
\end{equation}
%%
Thus, geometric properties of a projective plane may not be analyzed algebraically.  See \PMlinkname{this entry}{TernaryRingOfAProjectivePlane} for more detail.  Conversely, every ternary ring determines a projective plane with coordinates as above, so that algebraic properties of the ring can be interpreted geometrically.  For more detail on this, see \PMlinkname{this entry}{ProjectivePlaneOfATernaryRing}.

Fields and division rings are examples of ternary rings.  But there are other classes of algebraic structures intermediate between fields and ternary rings.  See \cite{dRe96} for a survey. There are four different projective planes known of order 9, many more of order 16, and so on.

Finding all PTRs in general remains of course just as hard as finding projective planes!

\clearpage
\subsection*{Finite projective planes as incidence structures}

Finite projective planes can be defined as
({\bf simple square}) 2-$(q^2+q+1\comma q+1\comma 1)$ {\bf designs},
or what's the same thing, {\bf Steiner systems} $S(2\comma q+1\comma q^2+q+1)$.
This way, we already {\em assume\/} in the definition that each ``block''
({\sc line}) contains exactly $q+1$ {\sc point}s, from which it follows by calculation that
the number of {\sc line}s that contain a given {\sc point} is also $q+1$, and that the
total number of {\sc line}s equals the total number of {\sc point}s.

Also, the unique incidence property that every two {\sc point}s lie together on
exactly one {\sc line} is given, and the dual one that every two {\sc line}s meet in
exactly one {\sc point} then follows from the numbers and the properties that define
a design or Steiner system.

We took a different route to projective planes above: we saw we only need to start with
these two unique incidence properties, and can {\em prove\/} from them that
the number of {\sc point}s on a {\sc line} and the number of {\sc line}s through a {\sc point} are both
fixed (i.e.\ that the projective plane and its dual are both designs).

\clearpage
\subsection*{Finite projective planes as incidence matrices}

Consider an arbitrary projective plane of order $q$.
We know there are $n=q^2+q+1$ points and the same number of lines. Let $A$
be an $n\times n$ matrix of which the rows and columns represent lines and
points of the plane, and give the value 1 to entries at intersections where
the point (represented by the column) lies on the line (represented by the
row), and leave the other entries zeros. Such an $A$ is known as the
{\bf incidence matrix} of the plane.

$AA^\top$ has rows and columns both representing lines, and the entries
here are the dot products of any two rows of $A$, i.e.\ the number of points
any two lines $l$ and $m$ have in common, which is $q+1$ when $l=m$ and 1
otherwise, by the projective plane properties.\vspace{-10pt}
$$
  AA^\top \;=\;
  \left\lgroup
  \begin{array}{cccc}
          q+1& 1 & 1 &\cdots \\
          1 &q+1& 1 &\cdots \\
          1 & 1 &q+1&\cdots \\
         \vdots&\vdots&\vdots&\ddots 
  \end{array}
  \right\rgroup
  \;=\; J + qI
$$
where $I$ is the identity matrix (ones along the diagonal, zeros everywhere
else) and $J$ is the matrix filled with ones in every entry. Note $AA^\top$
has this same value for {\em every\/} projective plane with $n$ points and $n$ lines,
while $A$ on its own of course varies if there is more than one plane of the
same order, as the plane is fully determined by its incidence matrix. Also,
$AA^\top$ having this value is the {\em only\/} thing needed to make the plane
specified by $A$ satisfy the projective plane properties; it is a fully equivalent
formulation of those properties.

The {\bf determinant} of $A$ is best evaluated via that of $AA^\top$,
because $(\det A)^2 = (\det A)(\det A^\top) = \det AA^\top$, and we know
every entry of $AA^\top$. The {\bf eigenvalues} of any $n\times n$ matrix of
the form $J + qI$ are easily seen to be
%
\begin{itemize}

\item $q+n$ (once) with eigenvector ${\bf x}$ for which $x_0=x_1=x_2\dots$

\item $q$ (multiplicity $n-1$) for the complementary eigenspace
      $x_0+x_1+x_2\cdots=0$

\end{itemize}
%
by letting the matrix act on these eigenvectors. This gives
$$
  \det AA^\top \;=\; q^{n-1}\,(q+n)
$$
which in our case ($n=q^2+q+1$) amounts to
$$
  \det AA^\top \;=\; q^{q^2+q}\,(q^2+2q+1)
$$
As $q^2+q$ is even and the other factor a square, we get an integer
$$
  \det A \;=\; \pm\;q^{\T{q^2+q\over2}}\,(q+1)
         \;=\; \pm\;q^{\T{q+1\choose2}}\,(q+1)
$$
where the sign is moot, as $A$ has no correspondence between its columns and
rows.

\clearpage
\subsection*{Existence}

The {\bf Bruck--Ryser theorem}~\cite{BR49} states that

{\em If $q$ is the order of a projective plane, and $q\equiv 1$ or $2\mod4$, then $q$ is
the sum of two integer squares (one of which can be $0^2$).}

It can be proven~\cite{BR49,Cam94} using the incidence matrix $A$, by a
clever rearrangement of ${\bf x}AA^\top{\bf x}$ (plus one extra term) for an
arbitrary vector ${\bf x}$, on which gradually constraints are imposed four
components at a time using the fact that each integer is a sum of four
squares. The argument only works when $q^2+q+1 = n \equiv -1 \mod4$, hence the
constraint on $q$ (mod~4).

Interestingly, there is a partial converse~\cite{HR54}: whenever $q$ is
allowed by the theorem (either congruent 0 or 3 (mod~4), or congruent 1
or 2 (mod~4) and a sum of two squares), there always is a {\bf rational}
$n\times n$ matrix $A$ (where $n=q^2+q+1$) such that $AA^\top = M$. Of course,
the matrix is only a proper incidence matrix if that rational matrix only has
entries 0 and 1.

The theorem doesn't rule out any potential projective plane orders $q\equiv 0$ or $3\mod4$,
but does rule out a large number of $q\equiv 1$ or $2\mod4$, starting with
$n=6, 14, 21, 22\dots$. Note in passing that it is easy to prove no prime
power falls foul of Bruck--Ryser, which is just as well, as (classical)
planes exist for all those orders.

The only other order ruled out to date is 10, via an epic computer search
by Lam, Swiercz and Thiel (read
{\tt\PMlinkexternal{http://www.cecm.sfu.ca/organics/papers/lam/index.html}{http://www.cecm.sfu.ca/organics/papers/lam/index.html}}
for Lam's account).

All projective planes actually found to date, even the non-classical ones,
have an order $q$ that is the power of a prime. The {\bf prime power conjecture}
says this is the case for all projective planes. It is still open.

\clearpage
\raggedright
\begin{thebibliography}{MST73}

\bitem{BR49}  \name{R.\,H.\,Bruck} \& \name{H.\,J.\,Ryser},
              \paper{The non-existence of certain finite projective planes}
              \mag{Can.\,J.\,Math.}
              \vol{12} (1949) pp~189--203

\bitem{HR54}  \name{M.\,Hall, Jr.} \& \name{H.\,J.\,Ryser},
              \paper{Normal completions of incidence matrices}
              \mag{Amer.\,J.\,Math.} \vol{76} (1954) pp~581--9

\bitem{Hal59} \name{Marshall Hall, Jr.}, \book{The Theory of Groups},
              Macmillan~1959

\bitem{Cam94} \name{Peter J.\,Cameron},
              \book{Combinatorics: topics, techniques, algorithms},\\
              Camb.~Univ.~Pr.~1994, \isbn{0\,521\,45761\,0}\\
{\tt\PMlinkexternal{http://www.maths.qmul.ac.uk/~pjc/comb/}{http://www.maths.qmul.ac.uk/~pjc/comb/}}
              (solutions, errata \&c.)

\bitem{CD96}  \name{Charles J.\,Colbourn} and \name{Jeffrey H.\,Dinitz}, eds.\\
              \book{The CRC Handbook of Combinatorial Designs},\\
              CRC~Press~1996, \isbn{0\,8493\,8948\,8}\\
{\tt\PMlinkexternal{http://www.emba.uvm.edu/~dinitz/hcd.html}{http://www.emba.uvm.edu/~dinitz/hcd.html}}
(errata, new results)\\
              {\em the reference work on designs incl.\ Steiner systems,
               proj.~planes}

\bitem{dRe96} \name{Marialuisa J.\,de~Resmini},
              \paper{Projective Planes, Nondesarguesian},
              pp~708--718 in the previous reference

\bitem{vG03} \name{Marijke van Gans}, M.Phil.(Qual.) thesis,
             Birmingham~2003

\end{thebibliography}
%%%%%
%%%%%
\end{document}
