\documentclass[12pt]{article}
\usepackage{pmmeta}
\pmcanonicalname{AlternateProofOfParallelogramLaw}
\pmcreated{2013-03-22 12:43:52}
\pmmodified{2013-03-22 12:43:52}
\pmowner{drini}{3}
\pmmodifier{drini}{3}
\pmtitle{alternate proof of parallelogram law}
\pmrecord{5}{33031}
\pmprivacy{1}
\pmauthor{drini}{3}
\pmtype{Proof}
\pmcomment{trigger rebuild}
\pmclassification{msc}{51-00}
\pmrelated{ProofOfParallelogramLaw2}

\endmetadata

% this is the default PlanetMath preamble.  as your knowledge
% of TeX increases, you will probably want to edit this, but
% it should be fine as is for beginners.

% almost certainly you want these
\usepackage{amssymb}
\usepackage{amsmath}
\usepackage{amsfonts}

% used for TeXing text within eps files
%\usepackage{psfrag}
% need this for including graphics (\includegraphics)
%\usepackage{graphicx}
% for neatly defining theorems and propositions
%\usepackage{amsthm}
% making logically defined graphics
%%%\usepackage{xypic} 

% there are many more packages, add them here as you need them

% define commands here
\begin{document}
Proof of this is simple, given the cosine law:
\[
c^2 = a^2 + b^2 - 2ab\cos\phi
\]
where $a$, $b$, and $c$ are the lengths of the sides of the triangle, and angle $\phi$ is the corner angle opposite the side of length $c$.


Let us define the largest interior angles as angle $\theta$.
Applying this to the parallelogram, we find that
\begin{eqnarray*}
d_1^2 &=& u^2 + v^2 - 2uv\cos\theta\\
d_2^2 &=& u^2+v^2 - 2uv\cos\left(\pi-\theta\right)
\end{eqnarray*}
 


Knowing that
\[
\cos\left(\pi-\theta\right) = - \cos\theta
\]
we can add the two expressions together, and find ourselves with
\begin{eqnarray*}
d_1^2+d_2^2 &=& 2u^2 + 2v^2 - 2uv\cos\theta + 2uv\cos\theta\\
d_1^2+d_2^2 &=& 2u^2 + 2v^2
\end{eqnarray*}
which is the theorem we set out to prove.
%%%%%
%%%%%
\end{document}
