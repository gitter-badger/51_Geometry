\documentclass[12pt]{article}
\usepackage{pmmeta}
\pmcanonicalname{OrthicTriangle}
\pmcreated{2013-03-22 12:11:00}
\pmmodified{2013-03-22 12:11:00}
\pmowner{drini}{3}
\pmmodifier{drini}{3}
\pmtitle{orthic triangle}
\pmrecord{8}{31447}
\pmprivacy{1}
\pmauthor{drini}{3}
\pmtype{Definition}
\pmcomment{trigger rebuild}
\pmclassification{msc}{51-00}
\pmrelated{Triangle}
\pmrelated{Orthocenter}
\pmrelated{EulerLine}
\pmrelated{CevasTheorem}
\pmrelated{CyclicQuadrilateral}
\pmrelated{TrigonometricVersionOfCevasTheorem}
\pmrelated{BaseAndHeightOfTriangle}

\usepackage{graphicx}
%%%%\usepackage{xypic} 
\usepackage{bbm}
\newcommand{\Z}{\mathbbmss{Z}}
\newcommand{\C}{\mathbbmss{C}}
\newcommand{\R}{\mathbbmss{R}}
\newcommand{\Q}{\mathbbmss{Q}}
\newcommand{\mathbb}[1]{\mathbbmss{#1}}
\begin{document}
If $ABC$ is a triangle and $AD, DE, CF$ are its three \PMlinkname{heights}{BaseAndHeightOfTriangle}, then the triangle $DEF$ is called the \emph{orthic triangle} of $ABC$.

A remarkable property of orthic triangles says that the orthocenter of $ABC$ is also the incenter of the orthic triangle $DEF$. That is, the heights of $ABC$ are the angle bisectors of $DEF$.

\begin{center}
\includegraphics{ortho}
\end{center}
%%%%%
%%%%%
%%%%%
\end{document}
