\documentclass[12pt]{article}
\usepackage{pmmeta}
\pmcanonicalname{Angle}
\pmcreated{2013-03-22 15:32:36}
\pmmodified{2013-03-22 15:32:36}
\pmowner{CWoo}{3771}
\pmmodifier{CWoo}{3771}
\pmtitle{angle}
\pmrecord{22}{37440}
\pmprivacy{1}
\pmauthor{CWoo}{3771}
\pmtype{Definition}
\pmcomment{trigger rebuild}
\pmclassification{msc}{51F20}
\pmclassification{msc}{51G05}
\pmsynonym{supplement}{Angle}
\pmrelated{PaschsTheorem}
\pmrelated{BetweennessInRays}
\pmrelated{SupplementaryAngles}
\pmdefines{supplementary}
\pmdefines{right angle}
\pmdefines{between rays}
\pmdefines{crossbar theorem}
\pmdefines{free angle}
\pmdefines{acute angle}
\pmdefines{obtuse angle}
\pmdefines{angle measure}
\pmdefines{side}
\pmdefines{vertex}

\endmetadata

\usepackage{amssymb,amscd}
\usepackage{amsmath}
\usepackage{amsfonts}

% used for TeXing text within eps files
%\usepackage{psfrag}
% need this for including graphics (\includegraphics)
%\usepackage{graphicx}
% for neatly defining theorems and propositions
\usepackage{amsthm}
% making logically defined graphics
%%%\usepackage{xypic}

% define commands here

\renewcommand{\line}[1]{\overleftrightarrow{#1}}
\newcommand{\ray}[1]{\overrightarrow{#1}}
\begin{document}
\section{Definition}
In an ordered geometry $S$, given a point $p$ let $\Pi(p)$ be the family of
all rays emanating from it.  Let $\alpha,\beta\in\Pi(p)$ such that
$\alpha\neq\beta$ and $\alpha\neq-\beta$.  The \emph{angle} \PMlinkname{between rays}{BetweennessInRays} 
$\alpha$ and $\beta$ at $p$ is
$$\lbrace \rho\in\Pi(p)\mid
\rho\mbox{ is between }\alpha\mbox{ and }\beta\rbrace.$$  This angle
is denoted by $\angle \alpha p\beta$.  The two rays $\alpha$ and
$\beta$ are the \emph{sides} of the angle, and $p$ the \emph{vertex} of the 
angle.  Since any point (other
than the source $p$) on a ray uniquely determines the ray, we may
also write the angle by $\angle apb$, whenever we have points
$a\in\alpha$ and $b\in\beta$.  

The notational device given for the angle suggests 
the possibility of defining an angle between two line segments satisfying certain conditions:
let $\overline{pq}$ and $\overline{qr}$ be two open line segments with a common endpoint 
$q$.  The angle between the two open line segments is the angle between the 
rays $\ray{qp}$ and $\ray{qr}$.  In this case, we may denote the angle by $\angle pqr$.

Suppose $\ell$ is a line and $p$ a point lying on $\ell$.  We have
two opposite rays emanating from $p$ that lie on $\ell$.  Call them
$\sigma$ and $-\sigma$.  Any ray $\rho$ emanating from a point $p$
that does not lie on $\ell$ produces two angles at $p$, one between
$\rho$ and $\sigma$ and the other between $\rho$ and $-\sigma$.
These two angles are said to be \emph{supplement} of one another, or
that $\angle \sigma p \rho$ is \emph{supplementary} of $\angle
(-\sigma) p \rho$.  Every angle has exactly two supplements.

\section{Ordering of Angles}
Let $S$ be an ordered geometry and $\rho$ a ray in $S$ with source
point $p$.  Consider the set $E$ of all angles whose one side is
$\rho$.  Define an ordering on $E$ by the following rule: for
$\angle \alpha p\rho,\angle \beta p\rho\in E$,
\begin{enumerate}
\item $\angle\alpha p\rho=\angle\beta p\rho$ if $\alpha=\beta$,
\item $\angle\alpha p\rho<\angle\beta p\rho$ if $\alpha\in\angle\beta p\rho$, and
\item $\angle\alpha p\rho>\angle\beta p\rho$ if $\beta\in\angle\alpha p\rho$.
\end{enumerate}
The ordering relation above is well-defined.  However, it is quite
limited, because there is no way to compare angles if the pair (of
angles) do not share a common side.  This can be remedied with an
additional set of axioms on the geometry: the axioms of congruence.
\\\\
In an ordered geometry satisfying the congruence axioms, we have the
concept of angle congruence. This binary relation turns out to be an
equivalence relation, so we can form the set of equivalence classes
on angles. Each equivalence class of angles is called a \emph{free
angle}.  Any member of a free angle $\mathfrak{a}$ is called a
representative of $\mathfrak{a}$, which is simply an angle of form
$\angle abc$, where $b$ is the source of two rays $\ray{ba}$ and
$\ray{bc}$.  We write $\mathfrak{a}=[\angle abc]$.  It is easy to
see that given a point $p$ and a ray $\rho$ emanating from $p$, we
can find, in each free angle, a representative whose one side is
$\rho$. In other words, for any free angle $\mathfrak{a}$, it is
possible to write $\mathfrak{a}=[\angle \alpha p\rho]$ for some ray
$\alpha$.
\\\\
Now we are ready to define orderings on angles in general.  In fact,
this this done via free angles.  Let $\mathfrak{A}$ be the set of
all free angles in an ordered geometry satisfying the congruence
axioms, and $\mathfrak{a},\mathfrak{b}\in\mathfrak{A}$.  Write
$\mathfrak{a}=[\angle \alpha p\rho]$ and $\mathfrak{b}=[\angle \beta
p\rho]$. We say that $\mathfrak{a}<\mathfrak{b}$ if ray $\alpha$ is
between $\beta$ and $\rho$.  The other inequality is dually defined.
This is a well-defined binary relation.  Given the ordering on free
angles, we define $\angle \alpha p\beta<\angle \gamma q\delta$ if
$[\angle \alpha p\beta]<[\angle \gamma q\delta]$.
\\\\
Let $\ell$ be a line and $p$ a point lying on $\ell$.  The point $p$
determines two opposite rays $\rho$ and $-\rho$ lying on $\ell$. Any
ray $\sigma$ emanating from $p$ that is distinct from either $\rho$
and $-\rho$ determines exactly two angles: $\angle \rho p\sigma$ and
$\angle (-\rho)p\sigma$.  These two angles are said to be
supplements of one another, or that one is supplementary of the
other.
\\\\
In an ordered geometry satisfying the congruence axioms,
supplementary free angles are defined if each contains a
representative that is supplementary to one another.  Given two
supplementary free angles $\mathfrak{a},\mathfrak{b}$, we may make
comparisons of the two:
\begin{itemize}
\item if $\mathfrak{a}=\mathfrak{b}$, then we say that $\mathfrak{a}$ is a
\emph{right free angle}, or simple a \emph{right angle}.  Clearly
$\mathfrak{b}$ is a right angle if $\mathfrak{a}$ is;
\item if $\mathfrak{a}>\mathfrak{b}$, then $\mathfrak{a}$ is called
an \emph{obtuse free angle}, or an \emph{obtuse angle}.  The
supplement of an obtuse angle is called an \emph{acute free angle},
or an \emph{acute angle}.  Thus, $\mathfrak{b}$ is acute if
$\mathfrak{a}$ is obtuse.
\end{itemize}
Given any two free angles, we can always compare them.  In other
words, the law of trichotomy is satisfied by the ordering of free
angles: for any $\mathfrak{a}$ and $\mathfrak{b}$, exactly one of
$$\mathfrak{a}>\mathfrak{b}\qquad\qquad\mathfrak{a}=\mathfrak{b}
\qquad\qquad\mathfrak{a}<\mathfrak{b}$$ is true.

\section{Operations on Angles}
Let $S$ be an ordered geometry satisfying the congruence axioms and
$\mathfrak{a}$ and $\mathfrak{b}$ are two free angles.  Write
$\mathfrak{a}=[\angle\alpha p\beta]$ and $\mathfrak{b}=[\angle\beta
p\gamma]$.  If $\beta$ is between $\alpha$ and $\gamma$, we define
an ``addition'' of $\mathfrak{a}$ and $\mathfrak{b}$, written
$\mathfrak{a}+\mathfrak{b}$ as the free angle $\mathfrak{c}$ with
representative $\angle\alpha p\gamma$.  In symbol, this says that if
$\beta$ is between $\alpha$ and $\gamma$, then $$[\angle \alpha
p\beta]+[\angle \beta p\gamma]=[\angle \alpha p\gamma].$$ This is a
well-defined binary operation, provided that \emph{one free angle is
between the other two}.  Therefore, the sum of a pair of
supplementary angles is not defined!  In addition, if $\mathfrak{a}$
and $\mathfrak{c}$ are two free angles, such that there exists a
free angle $\mathfrak{b}$ with
$\mathfrak{a}+\mathfrak{b}=\mathfrak{c}$, then $\mathfrak{b}$ is
unique and we denote it by $\mathfrak{c}-\mathfrak{a}$.  It is also
possible to define the multiplication of a free angle by a positive
integer, provided that the resulting angle is a well-defined free
angle. Finally, division of a free angle by positive integral powers
of 2 can also be defined.

\section{Angle Measurement}
An angle measure $\mathcal{A}$ is a function defined on free angles
of an ordered geometry $S$ with the congruence axioms, such that
\begin{enumerate}
\item $\mathcal{A}$ is real-valued and positive,
\item $\mathcal{A}$ is additive; in other words,
$\mathcal{A}(\mathfrak{a}+\mathfrak{b})=
\mathcal{A}(\mathfrak{a})+\mathcal{A}(\mathfrak{b})$, if
$\mathfrak{a}+\mathfrak{b}$ is defined;
\end{enumerate}
Here are some properties:
\begin{itemize}
\item if $\mathcal{A}(\mathfrak{a})=\mathcal{A}(\mathfrak{b})$, then
$\mathfrak{a}=\mathfrak{b}$.
\item $\mathfrak{a}>\mathfrak{b}$ iff $\mathcal{A}(\mathfrak{a})>
\mathcal{A}(\mathfrak{b})$.
\item for any free angle $\mathfrak{a}$, denote its supplement by
$\mathfrak{a}^s$.  Then $\mathcal{A}(\mathfrak{a})+
\mathcal{A}(\mathfrak{a}^s)$ is a positive constant
$r_{\mathcal{A}}$ that does not depend on $\mathfrak{a}$.
\item $\mathcal{A}$ is bounded above by $r_{\mathcal{A}}$.
\item if $\mathcal{A}$ and $\mathcal{B}$ are angle measures, then
$\mathcal{A+B}$ defined by $(\mathcal{A+B})(\mathfrak{a})=
\mathcal{A}(\mathfrak{a})+\mathcal{B}(\mathfrak{a})$ is an angle
measure too.
\item if $\mathcal{A}$ is an angle measure, then for any positive
real number $r$, $r\mathcal{A}$ defined by
$(r\mathcal{A})(\mathfrak{a})=r(\mathcal{A}(\mathfrak{a}))$ is also
an angle measure.  In the event that $r$ is an integer such that
$r\mathfrak{a}$ makes sense, we also have
$r(\mathcal{A}(\mathfrak{a}))=\mathcal{A}(r\mathfrak{a})$.
\end{itemize}
If $S$ is a neutral geometry, then we impose a third requirement for
a function to be an angle measure:
\begin{enumerate}
\item[3.] for any real number $r$ with $0<r<r_{\mathcal{A}}$, there is a
free angle $\mathfrak{a}$ such that
$\mathcal{A}(\mathfrak{a})=r$.
\end{enumerate}
Once the measure of a free angle is defined, one can next define the
measure of an angle: let $\mathcal{A}$ be a measure of the free
angles, define $\mathcal{A}^{\prime}$ on angles by
$\mathcal{A}^{\prime}(\angle \alpha p\beta)=\mathcal{A}([\angle
\alpha p\beta])$.  This is a well-defined function.  It is easy to
see that $\mathcal{A}^{\prime}(\angle \alpha p\beta)=
\mathcal{A}^{\prime}(\angle \gamma q\delta)$ iff $\angle \alpha
p\beta\cong\angle \gamma q\delta$, and $\mathcal{A}^{\prime}(\angle
\alpha p\beta)>\mathcal{A}^{\prime}(\angle \gamma q\delta)$ iff
$\angle \alpha p\beta>\angle \gamma q\delta$.
\\\\
Two popular angle measures are the degree measure and the radian
measure.  In the degree measure, $r_\mathcal{A}=180^{\circ}$.  In
the radian measure, $r_\mathcal{A}=\pi$.

\begin{thebibliography}{6}
\bibitem{dh} D. Hilbert, {\it Foundations of Geometry}, Open Court Publishing Co. (1971)
\bibitem{bs} K. Borsuk and W. Szmielew, {\it Foundations of Geometry}, North-Holland Publishing Co. Amsterdam (1960)
\bibitem{mg} M. J. Greenberg, {\it Euclidean and Non-Euclidean Geometries, Development and History}, W. H. Freeman and Company, San Francisco (1974)
\end{thebibliography}
%%%%%
%%%%%
\end{document}
