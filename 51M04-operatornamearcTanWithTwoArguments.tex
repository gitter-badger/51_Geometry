\documentclass[12pt]{article}
\usepackage{pmmeta}
\pmcanonicalname{operatornamearcTanWithTwoArguments}
\pmcreated{2013-03-22 15:18:19}
\pmmodified{2013-03-22 15:18:19}
\pmowner{matte}{1858}
\pmmodifier{matte}{1858}
\pmtitle{$\operatorname{arc tan}$ with two arguments}
\pmrecord{11}{37105}
\pmprivacy{1}
\pmauthor{matte}{1858}
\pmtype{Definition}
\pmcomment{trigger rebuild}
\pmclassification{msc}{51M04}
\pmclassification{msc}{51-01}
\pmsynonym{angle function}{operatornamearcTanWithTwoArguments}
%\pmkeywords{arc tan}
%\pmkeywords{arctan}
%\pmkeywords{$d \theta$}

\endmetadata

% this is the default PlanetMath preamble.  as your knowledge
% of TeX increases, you will probably want to edit this, but
% it should be fine as is for beginners.

% almost certainly you want these
\usepackage{amssymb}
\usepackage{amsmath}
\usepackage{amsfonts}
\usepackage{amsthm}

\usepackage{mathrsfs}

% used for TeXing text within eps files
%\usepackage{psfrag}
% need this for including graphics (\includegraphics)
%\usepackage{graphicx}
% for neatly defining theorems and propositions
%
% making logically defined graphics
%%%\usepackage{xypic}

% there are many more packages, add them here as you need them

% define commands here

\newcommand{\sR}[0]{\mathbb{R}}
\newcommand{\sC}[0]{\mathbb{C}}
\newcommand{\sN}[0]{\mathbb{N}}
\newcommand{\sZ}[0]{\mathbb{Z}}

 \usepackage{bbm}
 \newcommand{\Z}{\mathbbmss{Z}}
 \newcommand{\C}{\mathbbmss{C}}
 \newcommand{\F}{\mathbbmss{F}}
 \newcommand{\R}{\mathbbmss{R}}
 \newcommand{\Q}{\mathbbmss{Q}}



\newcommand*{\norm}[1]{\lVert #1 \rVert}
\newcommand*{\abs}[1]{| #1 |}



\newtheorem{thm}{Theorem}
\newtheorem{defn}{Definition}
\newtheorem{prop}{Proposition}
\newtheorem{lemma}{Lemma}
\newtheorem{cor}{Corollary}
\begin{document}
When inverting the polar coordinates, one needs the 
\PMlinkname{arc tan function}{CyclometricFunctions} $\arctan$
with two arguments.
If $(x,y)\in \R^2\setminus\{0\}$, then 
$
  \arctan(x,y)
$
is defined as the angle $(x,y)$ makes with 
the positive $x$-axis. 

One usually sees expressions like $\arctan(y/x)$,
which is equal to $\arctan(x,y)$ when $(x,y)$ is in the first quadrant.
However, $\arctan(y/x)$ does not give the correct angle when $(x,y)$
is in the third quadrant (since $y/x=(-y)/(-x)$).
Also, the quotient $y/x$ involves a division by zero when $x=0$,
which is damaging both numerically and mathematically.

In most mathematical software and programming languages the two-argument $\arctan$
is directly implemented.

In Python language the functions \texttt{atan(x)} and \texttt{atan2(x,y)} are the respective one and two argument versions of $\arctan$. The point of having the two argument version is to determine the correct quadrant of the point. For instance, $1/1 = 1 = -1/-1$, so \texttt{atan(x)} cannot distinguish between $(1,1)$ and $(-1,-1)$, but \texttt{atan2(x,y)} can, as the following Python code illustrates:
\begin{verbatim}
\PMlinkescapetext{
>>> from math import *
>>> print atan(1)
0.785398163397
>>> print atan2(1,1)
0.785398163397
>>> print atan2(-1,-1)
-2.3619449019
}
\end{verbatim}
because $(1,1)$ has argument $\pi/4=0.7853\ldots$ but $(-1,-1)$ has argument $-3\pi/4=-2.3619\ldots$.

\section*{Analytic properties}
In mathematical works, $\arctan(x,y)$
is simply denoted by $\theta(x, y)$.  The symbol $\theta$ obviously refers to the angle, but it is really
the function $h_2$, where
\[
g(r, \theta) = (r \cos \theta, r \sin \theta)\,, \quad h(x,y) = g^{-1}(x, y) = (r, \theta)\,.
\]
The function $g \colon \R^2 \to \R^2$ is the polar-to-Cartesian coordinate transformation.
By the inverse function theorem, the function $h$ (the Cartesian-to-polar coordinate transformation) exists and is smooth wherever it is defined. 
Note that $h$ cannot be defined continuously everywhere, because of the multi-valued nature of $\theta$ --- $(r, \theta)$ and $(r, \theta + 2\pi n)$ always map to the same point under $g$.
(Similarly, $\theta$ cannot defined when $r = \sqrt{x^2 + y^2} = 0$.)
This means, if one chases a loop (say a circle) around the origin, $\theta$ would move
from $0$ to $2\pi$, even though the image point $g(r, \theta)$ winds back to the starting point.

Technically, a ``largest'' possible domain of $h$ (and $\theta$) can only be taken to be some simply connected open subset of $\R^2 \setminus \{0\}$. (Note: $\R^2 \setminus \{0\}$ itself is not  simply connected.)
For example, such a domain might be $\R^2 \setminus \{ (x, y) : x \leq 0 \}$, i.e. delete the negative real axis from $\R^2$.

The exterior derivative of $\theta$
is
\[
d\theta = \frac{-y}{x^2 + y^2} \, dx + \frac{x}{x^2 + y^2} \, dy \,,
\]
(found by implicit differentiation),
and hence
\[
\frac{\partial \theta}{\partial x} = \frac{-y}{x^2 + y^2}\,, \quad
\frac{\partial \theta}{\partial y} = \frac{x}{x^2 + y^2}
\]
(which can also be found by differentiating $\arctan(y/x)$ directly
and piecing the results for each quadrant).

Of course, the formulas above are only valid wherever $\theta$ is defined,
but the analytical expressions do not change no matter which domain of definition is taken for $\theta$.
This allows for the following neat formula
to find the total variation of angle of a smooth curve $\gamma \colon [a, b] \to \R^2 \setminus \{0 \}$:
\[
\int_{\gamma} d\theta = \int_a^b \gamma^* d\theta
= \int_a^b \left( \frac{-y \dot{x} }{x^2 + y^2} + \frac{x \dot{y}}{x^2 + y^2} \right) \, dt\,.
\]
(This is related to the formula for the winding number
and the argument principle in complex analysis.)

For example, if $\gamma(t) = (r \cos t, r \sin t)$, for $t \in [0, 2\pi n]$,
is the circle that winds around the origin $n$ times, then $\int_{\gamma} d\theta = 2\pi n$.
%%%%%
%%%%%
\end{document}
