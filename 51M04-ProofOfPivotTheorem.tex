\documentclass[12pt]{article}
\usepackage{pmmeta}
\pmcanonicalname{ProofOfPivotTheorem}
\pmcreated{2013-03-22 13:12:37}
\pmmodified{2013-03-22 13:12:37}
\pmowner{pbruin}{1001}
\pmmodifier{pbruin}{1001}
\pmtitle{proof of pivot theorem}
\pmrecord{4}{33674}
\pmprivacy{1}
\pmauthor{pbruin}{1001}
\pmtype{Proof}
\pmcomment{trigger rebuild}
\pmclassification{msc}{51M04}

\endmetadata

% this is the default PlanetMath preamble.  as your knowledge
% of TeX increases, you will probably want to edit this, but
% it should be fine as is for beginners.

% almost certainly you want these
\usepackage{amssymb}
\usepackage{amsmath}
\usepackage{amsfonts}

% used for TeXing text within eps files
%\usepackage{psfrag}
% need this for including graphics (\includegraphics)
\usepackage{graphicx}
% for neatly defining theorems and propositions
%\usepackage{amsthm}
% making logically defined graphics
%%%\usepackage{xypic}

% there are many more packages, add them here as you need them

% define commands here
\begin{document}
Let $\triangle ABC$ be a triangle, and let $D$, $E$, and $F$ be points
on $BC$, $CA$, and $AB$, respectively.  The circumcircles of
$\triangle AEF$ and $\triangle BFD$ intersect in $F$ and in another
point, which we call $P$.  Then $AEPF$ and $BFPD$ are cyclic
quadrilaterals, so
$$
\angle A+\angle EPF=\pi
$$
and
$$
\angle B+\angle FPD=\pi
$$
Combining this with $\angle A+\angle B+\angle C=\pi$ and $\angle
EPF+\angle FPD+\angle DPE=2\pi$, we get
$$
\angle C+\angle DPE=\pi.
$$
This implies that $CDPE$ is a cyclic quadrilateral as well, so that
$P$ lies on the circumcircle of $\triangle CDE$.  Therefore, the
circumcircles of the triangles $AEF$, $BFD$, and $CDE$ have a common
point, $P$.
\begin{center}
\includegraphics{pivot.eps}
\end{center}
%%%%%
%%%%%
\end{document}
