\documentclass[12pt]{article}
\usepackage{pmmeta}
\pmcanonicalname{Diagonal}
\pmcreated{2013-03-22 17:34:41}
\pmmodified{2013-03-22 17:34:41}
\pmowner{CWoo}{3771}
\pmmodifier{CWoo}{3771}
\pmtitle{diagonal}
\pmrecord{7}{39990}
\pmprivacy{1}
\pmauthor{CWoo}{3771}
\pmtype{Definition}
\pmcomment{trigger rebuild}
\pmclassification{msc}{51N05}
\pmrelated{BasicPolygon}
\pmrelated{Polyhedron}
\pmdefines{adjacent vertices}

\endmetadata

\usepackage{amssymb,amscd}
\usepackage{amsmath}
\usepackage{amsfonts}
\usepackage{mathrsfs}

% used for TeXing text within eps files
%\usepackage{psfrag}
% need this for including graphics (\includegraphics)
%\usepackage{graphicx}
% for neatly defining theorems and propositions
\usepackage{amsthm}
% making logically defined graphics
%%\usepackage{xypic}
\usepackage{pst-plot}
\usepackage{psfrag}

% define commands here
\newtheorem{prop}{Proposition}
\newtheorem{thm}{Theorem}
\newtheorem{ex}{Example}
\newcommand{\real}{\mathbb{R}}
\newcommand{\pdiff}[2]{\frac{\partial #1}{\partial #2}}
\newcommand{\mpdiff}[3]{\frac{\partial^#1 #2}{\partial #3^#1}}
\begin{document}
\PMlinkescapeword{adjacent}

Let $P$ be a polygon or a polyhedron.  Two vertices on $P$ are \emph{adjacent} if the line segment joining them is an edge of $P$.  A \emph{diagonal} of $P$ is a line segment joining two non-adjacent vertices.

Below is a figure showing a hexagon and all its diagonals (in red) with $X$ as one of its endpoints.

\begin{center}
\begin{pspicture}(-8,0)(0,3)
\pspolygon(-5,0)(-3,0)(-2,1.4)(-3,3)(-5,3)(-6,1.5)
\psline[linecolor=red](-6,1.5)(-3,0)
\psline[linecolor=red](-3,0)(-3,3)
\psline[linecolor=red](-3,0)(-5,3)
\rput[b](-2.7,-0.3){$X$}
\rput[l](-6,1.5){.}
\rput[a](-3,3){.}
\rput[r](-2,1.4){.}
\end{pspicture}
\end{center}
\textbf{Remarks}.  
\begin{itemize}
\item
If $P$ is convex, then the relative interior of a diagonal lies in the relative interior of $P$.  Below is a figure showing that a diagonal may partially lie outside of $P$.
\begin{center}
\begin{pspicture}(-8,0)(0,2)
\pspolygon(-5,0)(-4,0.5)(-2,0)(-2,2)(-3,1)(-4,1.3)(-5,1.3)(-6,2)(-6,0.7)
\psline[linecolor=red](-6,0.7)(-2,2)
\end{pspicture}
\end{center}
\item
If a polygon $P$ has $n$ (distinct) vertices, then it has $\displaystyle{\frac{n(n-3)}{2}}$ diagonals.
\end{itemize}
%%%%%
%%%%%
\end{document}
