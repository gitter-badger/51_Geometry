\documentclass[12pt]{article}
\usepackage{pmmeta}
\pmcanonicalname{DeterminingSignsOfTrigonometricFunctions}
\pmcreated{2013-03-22 16:06:08}
\pmmodified{2013-03-22 16:06:08}
\pmowner{Wkbj79}{1863}
\pmmodifier{Wkbj79}{1863}
\pmtitle{determining signs of trigonometric functions}
\pmrecord{11}{38166}
\pmprivacy{1}
\pmauthor{Wkbj79}{1863}
\pmtype{Topic}
\pmcomment{trigger rebuild}
\pmclassification{msc}{51-01}
\pmclassification{msc}{97D40}
\pmsynonym{all snow tastes cold}{DeterminingSignsOfTrigonometricFunctions}
\pmsynonym{all students take calculus}{DeterminingSignsOfTrigonometricFunctions}
\pmrelated{Trigonometry}
\pmrelated{CalculatorTrigonometricFunctions}

\usepackage{amssymb}
\usepackage{amsmath}
\usepackage{amsfonts}
\usepackage{pstricks}
\begin{document}
There are at least two mnemonic devices for determining the sign \PMlinkescapetext{of} a trigonometric function at a given angle.  They are ``all snow tastes cold'' and (the more mathematical version) ``all students take calculus''.

The first \PMlinkescapetext{word} in both of these, ``all'', indicates that, if an angle lies in the first quadrant, then, when any trigonometric function is applied to it, the result is positive.

The second \PMlinkescapetext{word} in both of these starts with the letter ``s'', which indicates that, if the terminal ray of an angle lies in the second quadrant, then the only trigonometric functions that can be applied to it that yield a positive result are $\sin$ and its reciprocal $\csc$.

The third \PMlinkescapetext{word} in both of these starts with the letter ``t'', which indicates that, if the terminal ray of an angle lies in the third quadrant, then the only trigonometric functions that can be applied to it that yield a positive result are $\tan$ and its reciprocal $\cot$.

The fourth \PMlinkescapetext{word} in both of these starts with the letter ``c'', which indicates that, if the terminal ray of an angle lies in the fourth quadrant, then the only trigonometric functions that can be applied to it that yield a positive result are $\cos$ and its reciprocal $\sec$.

Because of how these mnemonic devices work, it is clear that they are in \PMlinkescapetext{terms} of the calculator trigonometric functions.

Below is a picture that illustrates how the mnemonic device ``all students take calculus'' works:

\begin{center}
\begin{pspicture}(-3,-3)(3,3)
\psline{<->}(-3,0)(3,0)
\psline{<->}(0,-3)(0,3)
\rput[l](0.5,2.5){all are}
\rput[l](0.5,2){positive}
\rput[l](0.5,1.5){here}
\rput[l](0.3,0.3){I ``all''}
\rput[r](-0.5,2.5){$\sin$ and $\csc$}
\rput[r](-0.5,2){are positive}
\rput[r](-0.5,1.5){here}
\rput[r](-0.3,0.3){``students'' II}
\rput[r](-0.3,-0.3){``take'' III}
\rput[r](-0.5,-1.5){$\tan$ and $\cot$}
\rput[r](-0.5,-2){are positive}
\rput[r](-0.5,-2.5){here}
\rput[l](0.3,-0.3){IV ``calculus''}
\rput[l](0.5,-1.5){$\cos$ and $\sec$}
\rput[l](0.5,-2){are positive}
\rput[l](0.5,-2.5){here}
\rput[a](3,-0.2){$x$}
\rput[r](-0.2,3){$y$}
\rput[l](-3,0){.}
\rput[b](0,-3){.}
\end{pspicture}
\end{center}

For angles whose terminal ray lies on the boundary of two quadrants, the matter of determining sign is not as \PMlinkescapetext{simple}, but it is still possible to do so through use of the mnemonic device.  If, for both of the boundary quadrants, the sign \PMlinkescapetext{of} the trigonometric function is positive, then the value of the trigonometric function applied to the angle is $1$.  If, for both of the boundary quadrants, the sign \PMlinkescapetext{of} the trigonometric function is negative, then the value of the trigonometric function applied to the angle is $-1$.  If the sign \PMlinkescapetext{of} the trigonometric function is different in the two boundary quadrants, then the value of the trigonometric function applied to the angle is either $0$ or undefined.

{\sl Example:\/}  Since the terminal ray of $\displaystyle \frac{2\pi}{3}$ lies in the second quadrant, we have that $\displaystyle \sin \left( \frac{2\pi}{3} \right)>0$, and $\displaystyle \cos \left( \frac{2\pi}{3} \right)<0$.

$\bigg($ In fact, $\displaystyle \sin \left( \frac{2\pi}{3} \right)=\frac{\sqrt{3}}{2}$ and $\displaystyle \cos \left( \frac{2\pi}{3} \right)=\frac{-1}{2}$. $\bigg)$

{\sl Example:\/} Since the terminal ray of $\displaystyle \frac{7\pi}{2}$ lies on the boundary of the third and fourth quadrants and, when $\csc$ is applied to any angle whose terminal ray lies in either the third or fourth quadrant, the result is negative, we have that $\displaystyle \csc \left( \frac{7\pi}{2} \right)=-1$.
%%%%%
%%%%%
\end{document}
