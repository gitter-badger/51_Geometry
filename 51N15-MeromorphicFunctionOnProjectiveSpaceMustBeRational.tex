\documentclass[12pt]{article}
\usepackage{pmmeta}
\pmcanonicalname{MeromorphicFunctionOnProjectiveSpaceMustBeRational}
\pmcreated{2013-03-22 17:52:13}
\pmmodified{2013-03-22 17:52:13}
\pmowner{jirka}{4157}
\pmmodifier{jirka}{4157}
\pmtitle{meromorphic function on projective space must be rational}
\pmrecord{5}{40349}
\pmprivacy{1}
\pmauthor{jirka}{4157}
\pmtype{Theorem}
\pmcomment{trigger rebuild}
\pmclassification{msc}{51N15}
\pmclassification{msc}{32A20}
\pmrelated{ChowsTheorem}

\endmetadata

% this is the default PlanetMath preamble.  as your knowledge
% of TeX increases, you will probably want to edit this, but
% it should be fine as is for beginners.

% almost certainly you want these
\usepackage{amssymb}
\usepackage{amsmath}
\usepackage{amsfonts}

% used for TeXing text within eps files
%\usepackage{psfrag}
% need this for including graphics (\includegraphics)
%\usepackage{graphicx}
% for neatly defining theorems and propositions
\usepackage{amsthm}
% making logically defined graphics
%%%\usepackage{xypic}

% there are many more packages, add them here as you need them

% define commands here
\theoremstyle{theorem}
\newtheorem*{thm}{Theorem}
\newtheorem*{lemma}{Lemma}
\newtheorem*{conj}{Conjecture}
\newtheorem*{cor}{Corollary}
\newtheorem*{example}{Example}
\newtheorem*{prop}{Proposition}
\theoremstyle{definition}
\newtheorem*{defn}{Definition}
\theoremstyle{remark}
\newtheorem*{rmk}{Remark}

\begin{document}
To define a rational function on complex projective space ${\mathbb{P}}^n$, we just take
two homogeneous polynomials of the same degree $p$ and $q$ on ${\mathbb{C}}^{n+1},$ and we 
note that $p/q$ induces a meromorphic function on ${\mathbb{P}}^n.$  In fact, every
meromorphic function on ${\mathbb{P}}^n$ is rational.

\begin{thm}
Let $f$ be a meromorphic function on ${\mathbb{P}}^n$.
Then $f$ is rational.
\end{thm}

\begin{proof}
Note that the zero set of $f$ and the pole set are analytic subvarieties of ${\mathbb{P}}^n$
and hence algebraic by Chow's theorem.  $f$ induces a meromorphic function $\tilde{f}$ on
${\mathbb{C}}^{n+1} \setminus \{ 0 \}$.  Let $p$ and $q$ be two homogeneous polynomials
such that $q = 0$ are the poles and $p=0$ are the zeros of $\tilde{f}.$  We can assume we can take $p$ and $q$ such that if we multiply $\tilde{f}$
by $q/p$ we have a holomorphic function outside the origin.  Hence $(q/p)\tilde{f}$ extends through the origin
by Hartogs' theorem.   Further since $\tilde{f}$ was 
constant on complex lines through the origin, it is not hard to see that $(q/p)\tilde{f}$ is homogeneous
and hence a homogeneous polynomial, by the same argument as in the proof of Chow's theorem.
Since it is not zero outside the origin, it can't be zero at the origin, and hence $(q/p)\tilde{f}$
must be a constant, and the proof is finished.
\end{proof}

%FIXME: decide on the best easily accessible reference that includes this theorem
%%%%%
%%%%%
\end{document}
