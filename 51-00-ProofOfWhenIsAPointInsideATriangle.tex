\documentclass[12pt]{article}
\usepackage{pmmeta}
\pmcanonicalname{ProofOfWhenIsAPointInsideATriangle}
\pmcreated{2013-03-22 17:57:08}
\pmmodified{2013-03-22 17:57:08}
\pmowner{joen235}{18354}
\pmmodifier{joen235}{18354}
\pmtitle{proof of when is a point inside a triangle}
\pmrecord{4}{40451}
\pmprivacy{1}
\pmauthor{joen235}{18354}
\pmtype{Proof}
\pmcomment{trigger rebuild}
\pmclassification{msc}{51-00}

% this is the default PlanetMath preamble.  as your knowledge
% of TeX increases, you will probably want to edit this, but
% it should be fine as is for beginners.

% almost certainly you want these
\usepackage{amssymb}
\usepackage{amsmath}
\usepackage{amsfonts}

% used for TeXing text within eps files
%\usepackage{psfrag}
% need this for including graphics (\includegraphics)
%\usepackage{graphicx}
% for neatly defining theorems and propositions
%\usepackage{amsthm}
% making logically defined graphics
%%%\usepackage{xypic}

% there are many more packages, add them here as you need them

% define commands here

\begin{document}
Let $\mathbf{u}\in\mathbb{R}^{2}$, $\mathbf{v}\in\mathbb{R}^{2}$
and $\mathbf{0}\in\mathbb{R}^{2}$. Let's consider the convex hull
of the set $T=\left\{ \mathbf{u},\mathbf{v},\mathbf{0}\right\} $.
By definition, the convex hull of $T$, noted $coT$, is the smallest
convex set that contains $T$. Now, the triangle $\Delta_{T}$ spanned
by $T$ is convex and contains $T$. Then $coT\subseteq\Delta_{T}$.
Now, every convex $C$ set containing $T$ must satisfy that $t\mathbf{u}+\left(1-t\right)\mathbf{v}\in C$,
$t\mathbf{u}\in C$ and $t\mathbf{v}\in C$ for $0\leq t\leq1$ (at
least the convex combination of the points of $T$ are contained in
$C$). This means that the boundary of $\Delta_{T}$ is contained
in $C$. But then every convex combination of points of $\partial\Delta_{T}$
must also be contained in $C$, meaning that $\Delta_{T}\subseteq C$
for every convex set containing $T$. In particular, $\Delta_{T}\subseteq coT$.

Since the convex hull is exactly the set containing all convex combinations
of points of $T$,\[
\Delta_{T}=coT=\left\{ \mathbf{x}\in\mathbb{R}^{2}:\mathbf{x}=\lambda\mathbf{u}+\mu\mathbf{v}+\left(1-\lambda-\mu\right)\mathbf{0},0\leq\lambda,\mu,\leq1,0\leq1-\lambda-\mu\leq1\right\} \]
we conclude that $\mathbf{x}\in\mathbb{R}^{2}$ is in the triangle
spanned by $T$ if and only if $\mathbf{x}=\lambda\mathbf{u}+\mu\mathbf{v}$
with $0\leq\lambda,\mu,\leq1$ and $0\leq\lambda+\mu\leq1$.
%%%%%
%%%%%
\end{document}
