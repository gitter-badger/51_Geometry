\documentclass[12pt]{article}
\usepackage{pmmeta}
\pmcanonicalname{Collinear}
\pmcreated{2013-03-22 12:24:40}
\pmmodified{2013-03-22 12:24:40}
\pmowner{Wkbj79}{1863}
\pmmodifier{Wkbj79}{1863}
\pmtitle{collinear}
\pmrecord{17}{32281}
\pmprivacy{1}
\pmauthor{Wkbj79}{1863}
\pmtype{Definition}
\pmcomment{trigger rebuild}
\pmclassification{msc}{51-00}
\pmsynonym{collinearity}{Collinear}
\pmrelated{PappussTheorem}
\pmrelated{EulerLine}
\pmrelated{MenelausTheorem}

\usepackage{pstricks} 
\usepackage{bbm}
\newcommand{\Z}{\mathbbmss{Z}}
\newcommand{\C}{\mathbbmss{C}}
\newcommand{\R}{\mathbbmss{R}}
\newcommand{\Q}{\mathbbmss{Q}}
\newcommand{\mathbb}[1]{\mathbbmss{#1}}
\begin{document}
In a \PMlinkescapetext{geometry}, points are said to be \emph{collinear} if they all lie on a straight line.

In the following picture, $A$, $P$, and $B$ are collinear.
\begin{center}
\begin{pspicture}(-2,-0.3)(3,0.1)
\psline(-2,0)(3,0)
\psdots(-2,0)(0,0)(3,0)
\rput[a](-2,-0.3){$A$}
\rput[a](0,-0.3){$P$}
\rput[a](3,-0.3){$B$}
\rput[a](0,0.03){.}
\end{pspicture}
\end{center}
%%%%%
%%%%%
\end{document}
