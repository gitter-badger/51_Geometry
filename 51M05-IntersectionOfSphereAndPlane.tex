\documentclass[12pt]{article}
\usepackage{pmmeta}
\pmcanonicalname{IntersectionOfSphereAndPlane}
\pmcreated{2013-03-22 18:18:39}
\pmmodified{2013-03-22 18:18:39}
\pmowner{pahio}{2872}
\pmmodifier{pahio}{2872}
\pmtitle{intersection of sphere and plane}
\pmrecord{22}{40934}
\pmprivacy{1}
\pmauthor{pahio}{2872}
\pmtype{Theorem}
\pmcomment{trigger rebuild}
\pmclassification{msc}{51M05}
\pmrelated{ConformalityOfStereographicProjection}
\pmdefines{zero circle}

% this is the default PlanetMath preamble.  as your knowledge
% of TeX increases, you will probably want to edit this, but
% it should be fine as is for beginners.

% almost certainly you want these
\usepackage{amssymb}
\usepackage{amsmath}
\usepackage{amsfonts}
\usepackage{amsthm}

\usepackage{mathrsfs}
\usepackage{pstricks}
\usepackage{pst-plot}

% used for TeXing text within eps files
%\usepackage{psfrag}
% need this for including graphics (\includegraphics)
%\usepackage{graphicx}
% for neatly defining theorems and propositions
%
% making logically defined graphics
%%%\usepackage{xypic}

% there are many more packages, add them here as you need them

% define commands here

\newcommand{\sR}[0]{\mathbb{R}}
\newcommand{\sC}[0]{\mathbb{C}}
\newcommand{\sN}[0]{\mathbb{N}}
\newcommand{\sZ}[0]{\mathbb{Z}}

 \usepackage{bbm}
 \newcommand{\Z}{\mathbbmss{Z}}
 \newcommand{\C}{\mathbbmss{C}}
 \newcommand{\F}{\mathbbmss{F}}
 \newcommand{\R}{\mathbbmss{R}}
 \newcommand{\Q}{\mathbbmss{Q}}



\newcommand*{\norm}[1]{\lVert #1 \rVert}
\newcommand*{\abs}[1]{| #1 |}



\newtheorem{thm}{Theorem}
\newtheorem{defn}{Definition}
\newtheorem{prop}{Proposition}
\newtheorem{lemma}{Lemma}
\newtheorem{cor}{Corollary}
\begin{document}
\textbf{Theorem.}\, The intersection curve of a sphere and a plane is a circle.

{\em Proof.}\, We prove the theorem without the equation of the sphere.\, Let $c$ be the intersection curve, $r$ the radius of the sphere and $OQ$ be the distance of the centre $O$ of the sphere and the plane.\, If $P$ is an arbitrary point of $c$, then $OPQ$ is a right triangle.\, By the Pythagorean theorem,
$$PQ = \varrho = \sqrt{r^2\!-\!OQ^2} = \mbox{\;constant}.$$

\begin{center}
\begin{pspicture}(-3.5,-3.5)(3.5,3.5)
\psdots(0,0)(0,1.23)
\psdot[linecolor=blue](-2.41,0.95)

\rput(0.3,0){$O$}
\rput(2,0.68){$c$}
\rput(-2.45,0.68){$P$}
\rput(0.3,1.23){$Q$}
\rput(-1.1,0.22){$r$}
\rput(-1.1,1.28){$\varrho$}
\psline[linestyle=dashed](0,0)(0,0.6)
\psline[linestyle=dotted](0,0.75)(0,1.23)
\psline[linestyle=dashed](0,1.23)(-2.41,0.95)
\psline[linestyle=dashed](0,0)(-2.41,0.95)
\psline(0,1.09)(-0.12,1.07)
\psline(-0.12,1.07)(-0.12,1.22)
\psellipse[linecolor=blue](0,1.23)(2.7,0.6)
\pscircle[linecolor=blue](0,0){3}
\rput(-3.5,-3.5){.}
\rput(3.5,3.5){.}
\end{pspicture}
\end{center}

Thus any point of the curve $c$ is in the plane at a \PMlinkescapetext{constant} distance $\varrho$ from the point $Q$, whence $c$ is a circle.\\

\textbf{Remark.}\, There are two special cases of the intersection of a sphere and a plane:\, the empty set of points ($OQ > r$) and a single point ($OQ = r$); these of course are not curves.\, In the former case one usually says that the sphere does not intersect the plane, in the latter one sometimes calls the common point a {\em zero circle} (it can be thought a circle with radius 0).
%%%%%
%%%%%
\end{document}
