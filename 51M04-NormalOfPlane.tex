\documentclass[12pt]{article}
\usepackage{pmmeta}
\pmcanonicalname{NormalOfPlane}
\pmcreated{2013-04-19 15:03:42}
\pmmodified{2013-04-19 15:03:42}
\pmowner{pahio}{2872}
\pmmodifier{pahio}{2872}
\pmtitle{normal of plane}
\pmrecord{17}{41595}
\pmprivacy{1}
\pmauthor{pahio}{2872}
\pmtype{Theorem}
\pmcomment{trigger rebuild}
\pmclassification{msc}{51M04}
\pmsynonym{plane normal}{NormalOfPlane}
\pmrelated{AngleBetweenLineAndPlane}
\pmrelated{NormalLine}
\pmrelated{NormalVector}
\pmrelated{Congruence}
\pmrelated{ParallelAndPerpendicularPlanes}
\pmrelated{ParallelismOfTwoPlanes}
\pmdefines{normal plane}
\pmdefines{center normal plane}

% this is the default PlanetMath preamble.  as your knowledge
% of TeX increases, you will probably want to edit this, but
% it should be fine as is for beginners.

% almost certainly you want these
\usepackage{amssymb}
\usepackage{amsmath}
\usepackage{amsfonts}

% used for TeXing text within eps files
%\usepackage{psfrag}
% need this for including graphics (\includegraphics)
%\usepackage{graphicx}
% for neatly defining theorems and propositions
 \usepackage{amsthm}
% making logically defined graphics
%%%\usepackage{xypic}
\usepackage{pstricks}
\usepackage{pst-plot}

% there are many more packages, add them here as you need them

% define commands here

\theoremstyle{definition}
\newtheorem*{thmplain}{Theorem}

\begin{document}
\PMlinkescapeword{normal}

The perpendicular or normal line of a plane is a special case of the surface normal, but may be defined separately as follows:

A line $l$ is a {\em normal of a plane} $\pi$, if it intersects the plane and is perpendicular to all lines passing through the intersection point in the plane.\, Then the plane $\pi$ is a {\em normal plane} of the line $l$.\, The normal plane passing through the \PMlinkname{midpoint}{Midpoint3} of a line segment is the {\em center normal plane} of the segment.\\

There is the

\textbf{Theorem.}\, If a line ($l$) \PMlinkescapetext{cuts} a plane ($\pi$) and is perpendicular to two distinct lines ($m$ and $n$) passing through the cutting point ($L$) in the plane, then the line is a normal of the plane.
\begin{center}
\begin{pspicture}(-3,-3.5)(3,4.1)
\rput(-3,-3.5){.}
\rput(3,4){.}
\psline(-1.5,1)(-2.5,-1)
\psline(-1.5,1)(1.43,1)
\psline(1.55,1)(3,1)
\psline[linewidth=0.06](-2.5,-1)(2,-1)
\psline[linewidth=0.05](2,-1)(3,1)
\psline(1.5,3.8)(1.5,0.3)
\psdot(1.5,0.3)
\psline(1.5,-1)(1.5,-3)
\psline(-1.5,1)(2,-1)
\psline(1.5,0.3)(-1.8,0.4)
\psline(1.5,0.3)(-2.5,-1)
\psline(1.5,0.3)(0.5,-1)
\rput(1.35,3.8){$l$}
\rput(2.7,0.8){$\pi$}
\rput(1.75,0.37){$L$}
\rput(-0.43,0.15){$M$}
\rput(0.30,-0.35){$N$}
\rput(1.15,-0.35){$A$}
\psdot(-0.37,0.35)
\psdot(0.36,-0.08)
\psdot(0.95,-0.4)
\psdots(1.5,3.3)(1.5,-2.7)
\psline(1.5,3.3)(-0.37,0.35)
\psline(1.5,3.3)(0.36,-0.08)
\psline(1.5,3.3)(0.95,-0.40)
\rput(1.75,3.3){$P$}
\rput(1.76,-2.7){$Q$}
\rput(-1.45,0.21){$m$}
\rput(-1.7,-0.60){$n$}
\rput(0.45,-0.77){$a$}
\end{pspicture}
\end{center}
{\em Proof.}\, Let $a$ be an arbitrary line passing through the point $L$ in the plane $\pi$.\, We need to show that\, $a \perp l$.\; Set another line of the plane cutting the lines $m$, $n$ and $a$ at the points $M$, $N$ and $A$, respectively.\, Separate from $l$ the equally \PMlinkescapetext{long line} segments $LP$ and $LQ$.\, Then
$$PM \;=\; QM \quad \mbox{and} \quad PN \;=\; QN,$$
since any point of the center normal of a line segment ($PQ$) is equidistant from the end points of the segment.\, Consequently, 
$$\Delta MNP \cong \Delta MNQ \quad(\mbox{SSS}).$$
Thus the segments $PA$ and $QA$, being corresponding parts of two congruent triangles, are equally long.\, I.e., the point $A$ is equidistant from the end points of the segment $PQ$, and it must be on the \PMlinkname{perpendicular bisector}{CenterNormal} of $PQ$.\, Therefore\, $AL \bot PQ$, i.e.\, $a \bot l$.\\


\textbf{Proposition 1.}\, All \PMlinkescapetext{normals} of a plane are parallel.\, If a line is parallel to a normal of a plane, then it is a normal of the plane, too.


\textbf{Proposition 2.}\, All normal planes of a line are parallel.\, If a plane is parallel to a normal plane of a line, then also it is a normal plane of the line.

%%%%%
%%%%%
\end{document}
