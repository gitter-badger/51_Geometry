\documentclass[12pt]{article}
\usepackage{pmmeta}
\pmcanonicalname{TheoremsOfEuclid}
\pmcreated{2013-03-22 17:13:23}
\pmmodified{2013-03-22 17:13:23}
\pmowner{rspuzio}{6075}
\pmmodifier{rspuzio}{6075}
\pmtitle{theorems of Euclid}
\pmrecord{9}{39549}
\pmprivacy{1}
\pmauthor{rspuzio}{6075}
\pmtype{Topic}
\pmcomment{trigger rebuild}
\pmclassification{msc}{51M04}

% this is the default PlanetMath preamble.  as your knowledge
% of TeX increases, you will probably want to edit this, but
% it should be fine as is for beginners.

% almost certainly you want these
\usepackage{amssymb}
\usepackage{amsmath}
\usepackage{amsfonts}

% used for TeXing text within eps files
%\usepackage{psfrag}
% need this for including graphics (\includegraphics)
%\usepackage{graphicx}
% for neatly defining theorems and propositions
%\usepackage{amsthm}
% making logically defined graphics
%%%\usepackage{xypic}

% there are many more packages, add them here as you need them

% define commands here

\begin{document}
This is a list of  theorems contained in Euclid's {\em The Elements}.  
At present, the list is incomplete, but as time goes on, more theorems will
systematically be added until it is complete.  Note that both
theorems and constructions are combined together in the elements; for a listing
of the constructions, please see \PMlinkid{their entry}{9531}.

\begin{itemize}
\item I 4 SAS theorem
\item I 5 (the pons asinorum)  The base angles of an
isosceles triangle are equal to each other.  Moreover, the angles under 
the bases are also equal.
\item I 6 If two angles of a triangle are equal, then that triangle is
isosceles.  (converse of the preceding proposition)
\item I 7   Given a line segment, it is not possible to construct
upon it two distinct triangles which lie upon the same side of the segment
whose remaining sides are equal to each other pairwise.
\item I 8 SSS theorem
\end{itemize}
%%%%%
%%%%%
\end{document}
