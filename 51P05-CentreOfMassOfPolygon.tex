\documentclass[12pt]{article}
\usepackage{pmmeta}
\pmcanonicalname{CentreOfMassOfPolygon}
\pmcreated{2013-03-22 17:33:13}
\pmmodified{2013-03-22 17:33:13}
\pmowner{pahio}{2872}
\pmmodifier{pahio}{2872}
\pmtitle{centre of mass of polygon}
\pmrecord{11}{39959}
\pmprivacy{1}
\pmauthor{pahio}{2872}
\pmtype{Result}
\pmcomment{trigger rebuild}
\pmclassification{msc}{51P05}
\pmclassification{msc}{51M04}
\pmclassification{msc}{26B15}
\pmclassification{msc}{15A72}
\pmsynonym{centroid of polygon}{CentreOfMassOfPolygon}
\pmrelated{ArithmeticMean}
\pmrelated{AreaOfPolygon}
\pmrelated{CentreOfMassOfHalfDisc}
\pmrelated{BarycentricSubdivision}
\pmrelated{CoordinatesOfMidpoint}

\endmetadata

% this is the default PlanetMath preamble.  as your knowledge
% of TeX increases, you will probably want to edit this, but
% it should be fine as is for beginners.

% almost certainly you want these
\usepackage{amssymb}
\usepackage{amsmath}
\usepackage{amsfonts}

% used for TeXing text within eps files
%\usepackage{psfrag}
% need this for including graphics (\includegraphics)
%\usepackage{graphicx}
% for neatly defining theorems and propositions
 \usepackage{amsthm}
% making logically defined graphics
%%%\usepackage{xypic}

% there are many more packages, add them here as you need them

% define commands here

\theoremstyle{definition}
\newtheorem*{thmplain}{Theorem}

\begin{document}
Let $A_1A_2{\ldots}A_n$ be an \PMlinkname{$n$-gon}{Polygon} which is supposed to have a \PMlinkescapetext{constant} surface-density in all of its points, $M$ the centre of mass of the polygon and $O$ the origin.  Then the position vector of $M$ with respect to $O$ is
\begin{align}
\overrightarrow{OM} = \frac{1}{n}\sum_{i=1}^n\overrightarrow{OA_i}.
\end{align}
We can of course take especially\, $O = A_1$,\, and thus
$$\overrightarrow{A_1M} = \frac{1}{n}\sum_{i=1}^n\overrightarrow{A_1A_i} =  
\frac{1}{n}\sum_{i=2}^n\overrightarrow{A_1A_i}.$$

In the special case of the triangle $ABC$ we have
\begin{align}
\overrightarrow{AM} = \frac{1}{3}(\overrightarrow{AB}+\overrightarrow{AC}).
\end{align}
The centre of mass of a triangle is the common point of its medians.\\

\textbf{Remark.}  An analogical result with (2) concerns also the \PMlinkescapetext{homogeneous} tetrahedron $ABCD$,
$$\overrightarrow{AM} = \frac{1}{4}(\overrightarrow{AB}+\overrightarrow{AC}+\overrightarrow{AD}),$$
and any $n$-dimensional simplex (cf. the \PMlinkname{midpoint}{Midpoint} of line segment:\, $\overrightarrow{AM} = \frac{1}{2}\overrightarrow{AB}$).

%%%%%
%%%%%
\end{document}
