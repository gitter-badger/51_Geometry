\documentclass[12pt]{article}
\usepackage{pmmeta}
\pmcanonicalname{PonsAsinorum}
\pmcreated{2013-03-22 17:17:31}
\pmmodified{2013-03-22 17:17:31}
\pmowner{Wkbj79}{1863}
\pmmodifier{Wkbj79}{1863}
\pmtitle{pons asinorum}
\pmrecord{6}{39635}
\pmprivacy{1}
\pmauthor{Wkbj79}{1863}
\pmtype{Topic}
\pmcomment{trigger rebuild}
\pmclassification{msc}{51M04}
\pmclassification{msc}{51-03}
\pmclassification{msc}{51-00}
\pmclassification{msc}{01A20}
\pmclassification{msc}{01A35}
\pmrelated{AnglesOfAnIsoscelesTriangle}

\endmetadata

\usepackage{amssymb}
\usepackage{amsmath}
\usepackage{amsfonts}
\usepackage{pstricks}
\usepackage{psfrag}
\usepackage{graphicx}
\usepackage{amsthm}
%%\usepackage{xypic}
\begin{document}
\PMlinkescapeword{base}
\PMlinkescapeword{bridge}
\PMlinkescapeword{translation}

\emph{Pons asinorum} is Latin for ``bridge of asses''.  During medieval times, this name was given to the fifth proposition in the first book of Euclid's \emph{The Elements}.  In the original Greek, this proposition reads:

\begin{quote}
$T\tilde{\omega}\nu$ $\iota\sigma o\sigma\kappa\varepsilon\lambda\tilde{\omega}\nu$ $\tau\rho\iota\gamma\acute{\omega}\nu\omega\nu$ $\alpha\iota$ $\pi\rho\grave{o}\varsigma$ $\tau\tilde{\eta}$ $\beta\acute{\alpha}\sigma\varepsilon\iota$ $\gamma\omega\nu\acute{\iota}\alpha\iota$ $\iota\sigma\alpha\iota$ $\alpha\lambda\lambda\acute{\eta}\lambda\alpha\iota\varsigma$ $\varepsilon\iota\sigma\acute{\iota}\nu,$ $\kappa\alpha\grave{\iota}$ $\pi\rho o\sigma\varepsilon\kappa\beta\lambda\eta\theta\varepsilon\iota\sigma\tilde{\omega}\nu$ $\tau\tilde{\omega}\nu$ $\iota\sigma\omega\nu$ $\varepsilon\upsilon\theta\varepsilon\iota\tilde{\omega}\nu$ $\alpha\iota$ $\upsilon\pi\grave{o}$ $\tau\grave{\eta}\nu$ $\beta\acute{\alpha}\sigma\iota\nu$ $\gamma\omega\nu\acute{\iota}\alpha\iota$ $\iota\sigma\alpha\iota$ $\alpha\lambda\lambda\acute{\eta}\lambda\alpha\iota\varsigma$ $\varepsilon\sigma o\nu\tau\alpha\iota.$
% Τῶν ἰσοσκελῶν Ï„Ï?ιγώνων αἱ Ï€Ï?ὸς τῇ βάσει γωνίαι ἴσαι ἀλλήλαις εἰσίν, καὶ Ï€Ï?οσεκβληθεισῶν τῶν ἴσων εá½?θειῶν αἱ ὑπὸ τὴν βάσιν γωνίαι ἴσαι ἀλλήλαις ἔσονται.
\end{quote}

A translation of this proposition is:

\begin{quote}
In isosceles triangles, the angles at the base equal one another, and, if the equal straight lines are produced further, then the angles under the base equal one another.
\end{quote}

There are a couple of reasons why this proposition was named pons asinorum:

\begin{itemize}
\item Euclid's diagram for this proposition looks like a bridge.
\item This is the first nontrivial proposition in \emph{The Elements} and thus tests a student's ability to understand more advanced concepts in Euclidean geometry.  Therefore, this proposition serves as a bridge from from the trivial portion of Euclidean geometry to the nontrivial portion, and the people who cannot cross this bridge are considered to be unintelligent.
\end{itemize}

For more details, please see \PMlinkexternal{a post written by rspuzio}{http://planetmath.org/?op=getmsg&id=15847} and \PMlinkexternal{a post written by Wkbj79}{http://planetmath.org/?op=getmsg&id=15849}.

\begin{thebibliography}{9}
\bibitem{mourmouras} Mourmouras, Dimitrios.  \emph{The Elements: The original Greek text}.  URL: \PMlinkexternal{http://www.physics.ntua.gr/Faculty/mourmouras/euclid}{http://www.physics.ntua.gr/Faculty/mourmouras/euclid}
\bibitem{wikipedia} Wikipedia.  \emph{Pons asinorum}.  URL: \PMlinkexternal{http://en.wikipedia.org/wiki/Pons_Asinorum}{http://en.wikipedia.org/wiki/Pons_Asinorum}
\end{thebibliography}
%%%%%
%%%%%
\end{document}
