\documentclass[12pt]{article}
\usepackage{pmmeta}
\pmcanonicalname{ProofOfBisectorsTheorem}
\pmcreated{2013-03-22 14:49:25}
\pmmodified{2013-03-22 14:49:25}
\pmowner{drini}{3}
\pmmodifier{drini}{3}
\pmtitle{proof of bisectors theorem}
\pmrecord{6}{36487}
\pmprivacy{1}
\pmauthor{drini}{3}
\pmtype{Proof}
\pmcomment{trigger rebuild}
\pmclassification{msc}{51A05}

\usepackage{graphicx}
%%%\usepackage{xypic} 
\usepackage{bbm}
\newcommand{\Z}{\mathbbmss{Z}}
\newcommand{\C}{\mathbbmss{C}}
\newcommand{\R}{\mathbbmss{R}}
\newcommand{\Q}{\mathbbmss{Q}}
\newcommand{\mathbb}[1]{\mathbbmss{#1}}
\newcommand{\figura}[1]{\begin{center}\includegraphics{#1}\end{center}}
\newcommand{\figuraex}[2]{\begin{center}\includegraphics[#2]{#1}\end{center}}
\newtheorem{dfn}{Definition}
\begin{document}
Notice that the triangles $\triangle BAP$ and $\triangle CAP$ have the same common height $h$, and if $(BAP)$  and $(CAP)$ denote their respective areas, we have
\[
\frac{(BAP)}{(CAP)} = \frac{BP\cdot h/2}{PC\cdot h/2} = \frac{BP}{PC}.
\]
\figuraex{genbisector}{scale=0.75}
On the other hand 
\[(BAP) = \frac{BA\cdot AP \sin BAP}{2},\qquad (CAP)=\frac{CA\cdot AP\sin CAP}{2}\]
and so
\[
\frac{(BAP)}{(CAP)} = \frac{BA\cdot AP \sin BAP/2}{CA\cdot AP\sin CAP/2} =\frac{ BA\sin BAP}{CA\sin CAP}.
\]

We have obtained
\[ \frac{BP}{PC} =\frac{ BA\sin BAP}{CA\sin CAP},\]
which is the generalization to the theorem. In the particular case when $AP$ is the bisector, $\angle BAP = \angle CAP$, and thus $\sin BAP=\sin CAP$. Cancelling out the sines proves the bisector theorem.
%%%%%
%%%%%
\end{document}
