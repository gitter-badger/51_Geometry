\documentclass[12pt]{article}
\usepackage{pmmeta}
\pmcanonicalname{Radius}
\pmcreated{2013-03-22 12:21:02}
\pmmodified{2013-03-22 12:21:02}
\pmowner{akrowne}{2}
\pmmodifier{akrowne}{2}
\pmtitle{radius}
\pmrecord{5}{32006}
\pmprivacy{1}
\pmauthor{akrowne}{2}
\pmtype{Definition}
\pmcomment{trigger rebuild}
\pmclassification{msc}{51-00}
%\pmkeywords{circle}
%\pmkeywords{sphere}
%\pmkeywords{radius}
\pmrelated{Diameter2}

\usepackage{amssymb}
\usepackage{amsmath}
\usepackage{amsfonts}

%\usepackage{psfrag}
%\usepackage{graphicx}
%%%\usepackage{xypic}
\begin{document}
The \emph{radius} of a circle or sphere is the distance from the center of the figure to the outer edge (or surface.)  

This definition actually holds in $n$ dimensions; so 4th and 5th and $k$-dimensional ``spheres'' have radii.  Since a circle is really a 2-dimensional sphere, its ``radius'' is merely an instance of the general definition.
%%%%%
%%%%%
\end{document}
