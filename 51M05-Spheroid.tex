\documentclass[12pt]{article}
\usepackage{pmmeta}
\pmcanonicalname{Spheroid}
\pmcreated{2013-03-22 14:56:48}
\pmmodified{2013-03-22 14:56:48}
\pmowner{rspuzio}{6075}
\pmmodifier{rspuzio}{6075}
\pmtitle{spheroid}
\pmrecord{5}{36638}
\pmprivacy{1}
\pmauthor{rspuzio}{6075}
\pmtype{Definition}
\pmcomment{trigger rebuild}
\pmclassification{msc}{51M05}
\pmsynonym{ellipsoid of revolution}{Spheroid}
\pmdefines{triaxial}
\pmdefines{oblate}
\pmdefines{prolate}

% this is the default PlanetMath preamble.  as your knowledge
% of TeX increases, you will probably want to edit this, but
% it should be fine as is for beginners.

% almost certainly you want these
\usepackage{amssymb}
\usepackage{amsmath}
\usepackage{amsfonts}

% used for TeXing text within eps files
%\usepackage{psfrag}
% need this for including graphics (\includegraphics)
%\usepackage{graphicx}
% for neatly defining theorems and propositions
%\usepackage{amsthm}
% making logically defined graphics
%%%\usepackage{xypic}

% there are many more packages, add them here as you need them

% define commands here
\begin{document}
A \emph{spheroid} or \emph{ellipsoid of revolution} is an ellipsoid such that two of the axes are equal in length.  The name ellipsoid of revolution refers to the fact that such a shape is rotationally symmetric about the third axis.  By way of contrast, the term \emph{triaxial ellipsoid} is used to specify that the three axes are of different lengths.  Also, the term \emph{oblate} is used to describe spheriods in which the two axes of equal length are longer than the third (think of a doorknob) whilst the term \emph{prolate} is used to describe spheriods for which the two equally long axes are shorter than the third axis (think of a cigar).
%%%%%
%%%%%
\end{document}
