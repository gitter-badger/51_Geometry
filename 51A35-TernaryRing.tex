\documentclass[12pt]{article}
\usepackage{pmmeta}
\pmcanonicalname{TernaryRing}
\pmcreated{2013-03-22 18:30:52}
\pmmodified{2013-03-22 18:30:52}
\pmowner{CWoo}{3771}
\pmmodifier{CWoo}{3771}
\pmtitle{ternary ring}
\pmrecord{6}{41200}
\pmprivacy{1}
\pmauthor{CWoo}{3771}
\pmtype{Definition}
\pmcomment{trigger rebuild}
\pmclassification{msc}{51A35}
\pmclassification{msc}{51E15}
\pmclassification{msc}{51A25}
\pmsynonym{planar ternary ring}{TernaryRing}
\pmsynonym{Hall ternary ring}{TernaryRing}
\pmdefines{linear ternary ring}
\pmdefines{left distributive}
\pmdefines{right distributive}
\pmdefines{distributive}

\endmetadata

\usepackage{amssymb,amscd}
\usepackage{amsmath}
\usepackage{amsfonts}
\usepackage{mathrsfs}

% used for TeXing text within eps files
%\usepackage{psfrag}
% need this for including graphics (\includegraphics)
%\usepackage{graphicx}
% for neatly defining theorems and propositions
\usepackage{amsthm}
% making logically defined graphics
%%\usepackage{xypic}
\usepackage{pst-plot}

% define commands here
\newcommand*{\abs}[1]{\left\lvert #1\right\rvert}
\newtheorem{prop}{Proposition}
\newtheorem{thm}{Theorem}
\newtheorem{ex}{Example}
\newcommand{\real}{\mathbb{R}}
\newcommand{\pdiff}[2]{\frac{\partial #1}{\partial #2}}
\newcommand{\mpdiff}[3]{\frac{\partial^#1 #2}{\partial #3^#1}}
\begin{document}
Let $R$ be a set containing at least two distinct elements $0$ and $1$, and $*$ a ternary operation on $R$.  Write $a*b*c$ the image of $(a,b,c)$ under $*$.  We call $(R,*,0,1)$, or simply just $R$, a \emph{ternary ring} if
\begin{enumerate}
\item $a*0*b=0*a*b=b$ for any $a,b\in R$, 
\item $1*a*0=a*1*0=a$ for any $a\in R$,
\item given $a,b,c,d\in R$ with $a\ne c$, the equation $x*a*b=x*c*d$ has a unique solution for $x$,
\item given $a,b,c\in R$, the equation $a*b*x=c$ has a unique solution for $x$,
\item given $a,b,c,d\in R$, with $a\ne c$, the system of equations
\begin{displaymath}
\left\{
\begin{array}{ll}
a * x * y = b \\
c * x * y = d
\end{array}
\right.
\end{displaymath}
has a unique solution for $(x,y)$.
\end{enumerate}

Given a ternary ring $R$, we may form two binary operations on $R$, one called the \emph{addition} $+$, and the other the \emph{multiplication} $\cdot$ on $R$:
\begin{eqnarray*}
a+b &:=& a*1*b, \\
a\cdot b &:=& a*b*0.
\end{eqnarray*}

\begin{prop} $(R,+,0)$ and $(R-\lbrace 0\rbrace,\cdot,1)$ are loops, and $0$ is the zero element in $R$ under the multiplication $\cdot$.\end{prop}
\begin{proof}
We first show that $(R,+,0)$ is a loop.  Given $a,b\in R$, there is a unique $c\in R$ such that $a*1*c = b$, but this is exactly $a+c=b$.  In addition, there is a unique $d\in R$ such that $d*1*a = b = d*0*b$.  But $d*1*a=b$ means $d+a=b$.  This shows that $(R,+)$ is a quasigroup.  Now, $a+0=a*1*0=a$ and $0+a=0*1*a=a$, so $0$ is the identity with respect to $+$.  Therefore, $(R,+,0)$ is a loop.

Next we show that $(S,\cdot,1)$ is a loop, where $S=R-\lbrace 0\rbrace$.  Given $a,b\in S$, there is a unique $c\in R$ such that $c*a*0=b=c*0*b=b$, since $a\ne 0$.  From $c*a*0=b$, we get $c\cdot a=b$.  Furthermore, $c\ne 0$, for otherwise $b=c*a*0=0*a*0=0$, contradicting the fact that $b\in S$.  In addition, there is a unique $d\in R$ such that we have a system of equations $a*d*0=b$ and $0*d*0=0$.  From $a*d*0=b$ we get $a\cdot d=b$.  Furthermore $d\ne 0$, for otherwise $b=a*d*0=a*0*0=0$, contradicting the fact that $b\in S$.  Thus, $(S,\cdot,1)$ is a quasigroup.  Now, $a\cdot 1 = a*1*0=a$ and $1\cdot a=1*a*0=a$, showing that $(S,\cdot,1)$ is a loop.

Finally, for any $a\in R$, $a\cdot 0 = a*0*0=0$ and $0\cdot a= 0*a*0=0$.
\end{proof}

Another property of a ternary ring is that, if the ternary ring is finite, then conditions 4 and 5 are equivalent in the presence of the first three.

Let $R$ be a ternary ring, and $a,b,c$ are arbitrary elements of $R$.  $R$ is said to be \emph{linear} if $a*b*c = a\cdot b+c$ for all $a,b,c\in R$, \emph{left distributive} if $a\cdot (b+c)=a\cdot b+a\cdot c$, \emph{right distributive} if $(a+b)\cdot c = a\cdot c +b\cdot c$, and \emph{distributive} if it is both left and right distributive.

For example, any division ring $D$, associative or not, is a linear ternary ring if we define the ternary operation $*$ on $D$ by $a*b*c:=a\cdot b+c$.  Any associative division ring $D$ is a distributive ternary ring.  This easy verification is left to the reader.  Semifields, near fields are also examples of ternary rings.

\textbf{Remark}.  Ternary rings were invented by Marshall Hall in his studies of axiomatic projective and affine planes.  Therefore, a ternary ring is also called a \emph{Hall ternary ring}, or a \emph{planar ternary ring}.  It can be shown that in every affine plane, one can set up a coordinate system, and from this coordinate system, one can construct a ternary ring.  Conversely, given any ternary ring, one can define an affine plane so that its coordinate system corresponds to this ternary ring.

\begin{thebibliography}{7}
\bibitem{RA} R. Artzy, {\it Linear Geometry}, Addison-Wesley (1965)
\end{thebibliography}
%%%%%
%%%%%
\end{document}
