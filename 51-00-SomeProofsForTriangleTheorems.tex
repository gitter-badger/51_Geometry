\documentclass[12pt]{article}
\usepackage{pmmeta}
\pmcanonicalname{SomeProofsForTriangleTheorems}
\pmcreated{2013-03-22 14:03:55}
\pmmodified{2013-03-22 14:03:55}
\pmowner{Wkbj79}{1863}
\pmmodifier{Wkbj79}{1863}
\pmtitle{some proofs for triangle theorems}
\pmrecord{13}{35422}
\pmprivacy{1}
\pmauthor{Wkbj79}{1863}
\pmtype{Proof}
\pmcomment{trigger rebuild}
\pmclassification{msc}{51-00}

% this is the default PlanetMath preamble.  as your knowledge
% of TeX increases, you will probably want to edit this, but
% it should be fine as is for beginners.

% almost certainly you want these
\usepackage{amssymb}
\usepackage{amsmath}
\usepackage{amsfonts}

% used for TeXing text within eps files
\usepackage{psfrag}
% need this for including graphics (\includegraphics)
\usepackage{graphicx}
% for neatly defining theorems and propositions
%\usepackage{amsthm}
% making logically defined graphics
%%%\usepackage{xypic}

% there are many more packages, add them here as you need them

% define commands here
\begin{document}
In the following, only Euclidean geometry is considered.

The sum of three angles $A$, $B$, and $C$ of a triangle is $A+B+C=180^\circ$.

The following triangle shows how the angles can be found to make a half revolution, which equals $180^\circ$.

\begin{center}
\includegraphics{triangleangles.eps}
\end{center}

$\Box$

The area \PMlinkescapetext{formula} $A=rs$ where $s$ is the semiperimeter $\displaystyle s=\frac{a+b+c}{2}$ and $r$ is the radius of the inscribed circle can be proven by creating the triangles $\triangle BAO$, $\triangle BCO$, and $\triangle ACO$ from the original triangle $\triangle ABC$, where $O$ is the center of the inscribed circle.

\begin{center}
\includegraphics{trianglearea1.eps}
\end{center}

\vspace{5mm}

\begin{center}
$\begin{array}{rl}
A_{\triangle ABC} & =A_{\triangle ABO}+A_{\triangle BCO}+A_{\triangle ACO} \\
& \\
& \displaystyle =\frac{rc}{2}+\frac{ra}{2}+\frac{rb}{2} \\
& \\
& \displaystyle =\frac{r(a+b+c)}{2} \\
& \\
& =rs \end{array}$
\end{center}

$\Box$
%%%%%
%%%%%
\end{document}
