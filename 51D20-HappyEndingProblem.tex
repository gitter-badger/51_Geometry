\documentclass[12pt]{article}
\usepackage{pmmeta}
\pmcanonicalname{HappyEndingProblem}
\pmcreated{2013-03-22 16:16:21}
\pmmodified{2013-03-22 16:16:21}
\pmowner{PrimeFan}{13766}
\pmmodifier{PrimeFan}{13766}
\pmtitle{happy ending problem}
\pmrecord{13}{38383}
\pmprivacy{1}
\pmauthor{PrimeFan}{13766}
\pmtype{Definition}
\pmcomment{trigger rebuild}
\pmclassification{msc}{51D20}
\pmsynonym{happy end problem}{HappyEndingProblem}

% this is the default PlanetMath preamble.  as your knowledge
% of TeX increases, you will probably want to edit this, but
% it should be fine as is for beginners.

% almost certainly you want these
\usepackage{amssymb}
\usepackage{amsmath}
\usepackage{amsfonts}

% used for TeXing text within eps files
%\usepackage{psfrag}
% need this for including graphics (\includegraphics)
\usepackage{graphicx}
% for neatly defining theorems and propositions
%\usepackage{amsthm}
% making logically defined graphics
%%%\usepackage{xypic}

% there are many more packages, add them here as you need them

% define commands here

\begin{document}
The {\em happy ending problem} asks, for a given integer $n > 2$, to find the smallest number $g(n)$ of points laid on a plane in general position such that a subset of $n$ points can be made the vertices of a convex $n$-agon.

\begin{figure}
\includegraphics[width=3in]{happy_end_4}
\caption{Examples showing that $g(4)\le 5$.}
\end{figure}

It is obvious that for $n = 3$, just three points in general position
are sufficient to create a triangle. For $n = 4$, Paul Erd\H{o}s and
Esther Klein (later Szekeres) determined that at least five points are
necessary, and Kalbfleisch later determined $g(5) = 9$.  Much later,
George Szekeres (posthumously) and Lindsay Peters published a computer
proof~\cite{cite:PS2006} that $g(6) = 17$.

For higher $n$, Erd\H{o}s and George Szekeres in 1935 gave the upper bound $$g(n) \le \binom{2n - 4}{n - 2} + 1.$$ Later, in 1961 they gave the lower bound $g(n) \geq 1 + 2^{n - 2}$. 

New ideas for the upper bound were in the air in the late 1990s, with Chung and Graham showing that if $n \ge 4$, then $$g(n) \le \binom{2n - 4}{n - 2},$$ while Kleitman and Pachter showed that then $$g(n) \le \binom{2n - 4}{n - 2} + 7 - 2n.$$ And G\'eza T\'oth and Pavel Valtr in 1998 gave the upper bound $$g(n) \leq {2n - 5 \choose n - 3} + 2,$$ which in 2005 they refined to $$g(n) \le \binom{2n-5}{n-3} + 1.$$

\begin{thebibliography}{9}
\bibitem{cite:CG1998}
F.~R.~K.~Chung and R.~L.~Graham, Forced convex $n$-gons in the plane, {\it Discrete Comput. Geom.} {\bf 19} (1996), 229--233.

\bibitem{cite:ES1935}
P.~Erd\H{o}s and G.~Szekeres, A combinatorial problem in geometry, {\it Compositio Math.} {\bf 2} (1935), 463--470.

\bibitem{cite:ES1961}
P.~Erd\H{o}s and G.~Szekeres, On some extremum problems in elementary geometry, {\it Ann. Univ. Sci. Budapest E\"otv\"os Sect. Math.} {\bf 3--4} (1961), 53--62.

\bibitem{cite:KP1998}
D.~Kleitman and L.~Pachter, Finding convex sets among points in the plane, {\it Discrete Comput. Geom.} {\bf 19} (1998), 405--410.

\bibitem{cite:MS2000}
W.~Morris and V.~Soltan, The Erd\H{o}s-Szekeres problem on points in convex position -- a survey, {\it Bull. Amer. Math. Soc.} {\bf 37} (2000), 437--458.

\bibitem{cite:PS2006}
L.~Peters and G.~Szekeres, Computer solution to the 17-point Erd\H{o}s-Szekeres problem, {\it ANZIAM J.} {\bf 48} (2006), 151--164.

\bibitem{cite:TV1998}
G.~T\'{o}th and P.~Valtr, Note on the Erd\H{o}s-Szekeres theorem, {\it Discrete Comput. Geom.} {\bf 19} (1998), 457--459.

\bibitem{cite:TV2005}
G.~T\'{o}th and P.~Valtr, The Erd\H{o}s-Szekeres theorem: upper bounds and related results. Appearing in J.~E.~Goodman, J.~Pach, and E.~Welzl, eds., {\it Combinatorial and computational geometry}, Mathematical Sciences Research Institute Publications {\bf 52} (2005) 557--568.
\end{thebibliography}
%%%%%
%%%%%
\end{document}
