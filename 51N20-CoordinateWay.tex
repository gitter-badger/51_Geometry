\documentclass[12pt]{article}
\usepackage{pmmeta}
\pmcanonicalname{CoordinateWay}
\pmcreated{2013-03-22 17:28:45}
\pmmodified{2013-03-22 17:28:45}
\pmowner{pahio}{2872}
\pmmodifier{pahio}{2872}
\pmtitle{coordinate way}
\pmrecord{11}{39865}
\pmprivacy{1}
\pmauthor{pahio}{2872}
\pmtype{Definition}
\pmcomment{trigger rebuild}
\pmclassification{msc}{51N20}
\pmclassification{msc}{51G05}
\pmclassification{msc}{51-00}
\pmsynonym{coordinate broken line}{CoordinateWay}

\endmetadata

% this is the default PlanetMath preamble.  as your knowledge
% of TeX increases, you will probably want to edit this, but
% it should be fine as is for beginners.

% almost certainly you want these
\usepackage{amssymb}
\usepackage{amsmath}
\usepackage{amsfonts}

% used for TeXing text within eps files
%\usepackage{psfrag}
% need this for including graphics (\includegraphics)
%\usepackage{graphicx}
% for neatly defining theorems and propositions
 \usepackage{amsthm}
% making logically defined graphics
%%%\usepackage{xypic}
\usepackage{pstricks}
\usepackage{pst-plot}

% there are many more packages, add them here as you need them

% define commands here

\theoremstyle{definition}
\newtheorem*{thmplain}{Theorem}

\begin{document}
Let\, $P = (x,\,y,\,z)$\, be a point in $\mathbb{R}^3$.  If $N$ is the \PMlinkname{projection}{ProjectionOfPoint} of $P$ on the $xy$-plane, $A$ the projection of $N$ on the $x$-axis and $O$ the origin, then the broken line \,$PNAO$\, is the {\em coordinate way} of the point $P$.  The lengths of the line segments forming the coordinate way are
        $$AO = |x|,\quad NA = |y|,\quad PN = |z|.$$
\begin{center}
\begin{pspicture}(-5.5,-4.5)(5.5,4)
\rput(-0.3,3.5){$z$}
\rput(4.55,-0.3){$y$}
\rput(-2.9,-2.3){$x$}
\psline{<-}(-3,-2)(-1.5,-1)
\psline{-}(0,0)(1.43,0)
\psline{->}(1.56,0)(4.5,0)
\psline{->}(0,0)(0,3.5)
\psline[linecolor=red](-1.5,-1)(0,0)
\psline[linecolor=red](-1.5,-1)(1.5,-1)
\psline[linecolor=red](1.5,-1)(1.5,2.5)
\psdot[linecolor=red](1.5,2.5)
\psdot[linecolor=red](1.5,-1)
\psdot[linecolor=red](-1.5,-1)
\psdot[linecolor=red](0,0)
\rput(1.8,2.6){$P$}
\rput(1.8,-1){$N$}
\rput(-1.74,-0.9){$A$}
\rput(-0.3,0.1){$O$}
\rput(0.5,-4){\mbox{The coordinate way of}\;\;$P$\;\;\mbox{(red)}}
\end{pspicture}
\end{center}

%%%%%
%%%%%
\end{document}
