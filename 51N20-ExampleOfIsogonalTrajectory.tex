\documentclass[12pt]{article}
\usepackage{pmmeta}
\pmcanonicalname{ExampleOfIsogonalTrajectory}
\pmcreated{2013-03-22 18:59:23}
\pmmodified{2013-03-22 18:59:23}
\pmowner{pahio}{2872}
\pmmodifier{pahio}{2872}
\pmtitle{example of isogonal trajectory}
\pmrecord{10}{41856}
\pmprivacy{1}
\pmauthor{pahio}{2872}
\pmtype{Example}
\pmcomment{trigger rebuild}
\pmclassification{msc}{51N20}
\pmclassification{msc}{34A26}
\pmclassification{msc}{34A09}
\pmsynonym{isogonal trajectories of concentric circles}{ExampleOfIsogonalTrajectory}
%\pmkeywords{logarithmic spiral}
\pmrelated{IsogonalTrajectory}

\endmetadata

% this is the default PlanetMath preamble.  as your knowledge
% of TeX increases, you will probably want to edit this, but
% it should be fine as is for beginners.

% almost certainly you want these
\usepackage{amssymb}
\usepackage{amsmath}
\usepackage{amsfonts}

% used for TeXing text within eps files
%\usepackage{psfrag}
% need this for including graphics (\includegraphics)
%\usepackage{graphicx}
% for neatly defining theorems and propositions
 \usepackage{amsthm}
% making logically defined graphics
%%%\usepackage{xypic}

% there are many more packages, add them here as you need them

% define commands here

\theoremstyle{definition}
\newtheorem*{thmplain}{Theorem}

\begin{document}
Determine the curves which intersect the origin-centered circles at an angle of $45^\circ$.\\


The differential equation of the circles \,$x^2\!+\!y^2 = R^2$\, is\, $2x\,dx+2y\,dy = 0$,\, i.e.
$$\frac{x}{y}+\frac{dy}{dx} \;=\; 0.$$
Thus, by the model (2) of the \PMlinkname{parent entry}{IsogonalTrajectory}, the differential equation of the isogonal trajectory reads
\begin{align}
\frac{x}{y}+\frac{y'-\tan\frac{\pi}{4}}{1+y'\tan\frac{\pi}{4}} \;=\; 0,
\end{align}
which can be rewritten as
$$y' \;=\; \frac{y\!-\!x}{y\!+\!x} \;=\, \frac{\frac{y}{x}\!-\!1}{\frac{y}{x}\!+\!1}.$$
Here, one may take\, $\frac{y}{x} := t$\, as a new variable (see ODE types reductible to the variables separable case), when
$$y \;=\; xt, \quad y' \;=\; \frac{dy}{dx} \;=\; t+x\frac{dt}{dx},$$
and in the resulting equation
$$t+x\frac{dt}{dx} \;=\; \frac{t\!-\!1}{t\!+\!1}$$
one can \PMlinkname{separate the variables}{SeparationOfVariables}:
$$\frac{1\!+\!t}{1\!+\!t^2}\,dt \;=\; -\frac{dx}{x}$$
Multiplying here by 2 and integrating then give
$$2\arctan{t}+\ln(1\!+\!t^2) \;=\; -2\ln{x}+\ln{C^2} \;\equiv\; -\ln\frac{x^2}{C^2},$$
or equivalently
$$\ln\frac{x^2\!+\!x^2t^2\!}{C^2} \;=\; -2\arctan{t}.$$
This is
$$\ln\frac{\sqrt{x^2\!+\!y^2}}{C} \;=\; -\arctan\frac{y}{x},$$
i.e.
$$\sqrt{x^2\!+\!y^2} \;=\; Ce^{-\arctan\frac{y}{x}}.$$
Expressing this in the polar coordinates $r,\,\varphi$ gives the family of the integral curves of the equation (1) in the form
$$r \;=\; Ce^{-\varphi}.$$
Consequently, the family of the isogonal trajectories consists of logarithmic spirals.



%%%%%
%%%%%
\end{document}
