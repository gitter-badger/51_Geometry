\documentclass[12pt]{article}
\usepackage{pmmeta}
\pmcanonicalname{ProofOfBisectorsTheorem1}
\pmcreated{2013-03-22 14:49:28}
\pmmodified{2013-03-22 14:49:28}
\pmowner{drini}{3}
\pmmodifier{drini}{3}
\pmtitle{proof of bisectors theorem}
\pmrecord{5}{36488}
\pmprivacy{1}
\pmauthor{drini}{3}
\pmtype{Proof}
\pmcomment{trigger rebuild}
\pmclassification{msc}{51A05}

\endmetadata

\usepackage{graphicx}
%%%\usepackage{xypic} 
\usepackage{bbm}
\newcommand{\Z}{\mathbbmss{Z}}
\newcommand{\C}{\mathbbmss{C}}
\newcommand{\R}{\mathbbmss{R}}
\newcommand{\Q}{\mathbbmss{Q}}
\newcommand{\mathbb}[1]{\mathbbmss{#1}}
\newcommand{\figura}[1]{\begin{center}\includegraphics{#1}\end{center}}
\newcommand{\figuraex}[2]{\begin{center}\includegraphics[#2]{#1}\end{center}}
\newtheorem{dfn}{Definition}
\begin{document}
Consider sines law in triangles $\triangle APB$ and $\triangle APC$. 
\figuraex{genbisector}{scale=0.75}

On $\triangle APB$ we have
\[
\frac{BP}{\sin BAP } = \frac{AB}{\sin APB}
\]
and on $\triangle APC$ we have
\[
\frac{PC}{\sin PAC} = \frac{AC}{\sin CPA}.
\]

Combining the two relation gives
\[
\frac{BP}{PC} = \frac{ AB\sin BAP / \sin APB}{AC\sin PAC / \sin CPA}.
\]
However, $\angle APB + \angle CPA = 180^\circ$, and so $\sin APB = \sin CPA$. Cancelling gives
\[\frac{BP}{PC} = \frac{ AB\sin BAP}{AC\sin PAC },\]
which is the generalization of the theorem. When $AP$ is a bisector, $\angle BAP = \angle PAC$ and we can cancel further to obtain the bisector theorem.
%%%%%
%%%%%
\end{document}
