\documentclass[12pt]{article}
\usepackage{pmmeta}
\pmcanonicalname{ExactTrigonometryTables}
\pmcreated{2013-03-22 15:31:26}
\pmmodified{2013-03-22 15:31:26}
\pmowner{stevecheng}{10074}
\pmmodifier{stevecheng}{10074}
\pmtitle{exact trigonometry tables}
\pmrecord{11}{37395}
\pmprivacy{1}
\pmauthor{stevecheng}{10074}
\pmtype{Example}
\pmcomment{trigger rebuild}
\pmclassification{msc}{51M04}
\pmclassification{msc}{33B10}
\pmrelated{ConstructibleAnglesWithIntegerValuesInDegrees}
\pmrelated{Cotangent2}
\pmrelated{Trigonometry}
\pmrelated{TheoremOnConstructibleAngles}
\pmrelated{GraphOfEquationXyConstant}

% this is the default PlanetMath preamble.  as your knowledge
% of TeX increases, you will probably want to edit this, but
% it should be fine as is for beginners.

% almost certainly you want these
\usepackage{amssymb}
\usepackage{amsmath}
\usepackage{amsfonts}

% used for TeXing text within eps files
%\usepackage{psfrag}
% need this for including graphics (\includegraphics)
%\usepackage{graphicx}
% for neatly defining theorems and propositions
%\usepackage{amsthm}
% making logically defined graphics
%%%\usepackage{xypic}

% there are many more packages, add them here as you need them

% define commands here
\begin{document}
\subsection{Basic angles}

Since the trigonometric ratios for most angles cannot be
calculated exactly in closed algebraic form,
a few well-known angles 
that can be calculated
often comprise the bulk of textbook
exercises involving trigonometry.  

The basic angles are given in Table~\ref{table: basic angles}.

\begin{table}
\caption{Basic angles encountered in trigonometry}
\label{table: basic angles}
\begin{tabular}{| c | c | c | c |}
\hline
$\theta$ &$\sin \theta $ &$\cos \theta$ &$\tan \theta $\\
\hline
$0^\circ$ &0 &1 &0 \\
$30^\circ$ &1/2 &$\sqrt{3}/2$ &$1/\sqrt{3}$ \\
$45^\circ$ &$\sqrt{2}/2$ &$\sqrt{2}/2$ &1 \\
$60^\circ$ &$\sqrt{3}/2$ &1/2 &$\sqrt{3}$ \\
$90^\circ$ &1 &0 & $\infty$ \\
\hline
\end{tabular}
\end{table}

\subsection{Other angles by addition and halving}

These basic angles can be easily extended to obtain more angles of interest.  Adding multiples of $90^\circ$ merely rotates these angles into other quadrants; the appropriate values of $\sin$ and $\cos$ can be obtained through symmetry.

The values for $15^\circ$ can be obtained by using the \PMlinkname{formula for the difference of angles}{AngleSumIdentity}:
\begin{align*}
\sin 15^\circ &=\sin(45^\circ-30^\circ)\\
  &=\sin 45^\circ \cos 30^\circ -\cos 45^\circ \sin 30^\circ  \\
  &=\frac{\sqrt{2}}{2} \cdot \frac{\sqrt{3}}{2} - \frac{\sqrt{2}}{2} \cdot \frac{1}{2} \\
  &=\frac{\sqrt{6}-\sqrt{2}}{4}\,.
\end{align*}
Likewise, we can find that 
\begin{align*}
\cos(15^\circ) &=\frac{\sqrt{6}+\sqrt{2}}{4} \\
\sin(75^\circ) &= \sin(45^\circ+30^\circ)=\frac{\sqrt{6}+\sqrt{2}}{4} \\
\cos(75^\circ) &= \frac{\sqrt{6}-\sqrt{2}}{4}\,.
\end{align*}

More exact angles can be obtained by solving the double angle identity:
\[
\sin \frac{\theta}{2} =\pm\sqrt{\frac{1-\cos \theta}{2}}\,, \quad 
\cos \frac{\theta}{2} =\pm\sqrt{\frac{1+\cos \theta}{2}}\,.
\]
So for example, $\sin{7.5^\circ}=\sqrt{(4-\sqrt{6}-\sqrt{2})/8}$.  These angles can be further added and subdivided to obtain a dense subset of exactly known angles.  However, such effort is not generally useful.  Computers and calculators use a combination of lookup-tables and numeric iteration to obtain their values.

\subsection{The angles $18^\circ$, $36^\circ$, $54^\circ$, $72^\circ$}

The $18^\circ$-$36^\circ$-$54^\circ$-$72^\circ$ series of angles cannot be obtained
by halving, doubling, adding or subtracting the previous angles.  Nevertheless, they are constructible, and their exact values can be derived by the following elementary procedure:

Consider an isosceles triangle with the angles $72^\circ$, $54^\circ$ and $54^\circ$.  
From the triangle we derive the relation:
\[
\sin \frac{72^\circ}{2} = \cos 54^\circ
\]
Notice that $72=4 \times 18$ and $54 = 3\times 18$, so if $x = 18^\circ$,
then
\begin{align*}
\sin 2x &= \cos 3x \\
2 \sin x \cos x &= 4 \cos^3 x - 3 \cos x \\
2 \sin x &= 4 \cos^2 x - 3 \\
2 \sin x &= 4 (1-\sin^2 x) -3 
\end{align*}
The last equation is a quadratic equation that can be solved for $\sin 18^\circ$.  Carrying out the calculations, we obtain the values in Table~\ref{table: eighteen degrees}.

\begin{table}
\caption{Other constructible angles in trigonometry}
\label{table: eighteen degrees}
\begin{tabular}{| c | c | c |}
\hline
$\theta$ &$\sin \theta $ &$\cos \theta $ \\
\hline
$18^\circ$ & $\dfrac{\sqrt{5}-1}{4}$ & $\dfrac{\sqrt{5 + \sqrt{5}}}{2\sqrt{2}}$  \\
$36^\circ$ & $\dfrac{\sqrt{5-\sqrt{5}}}{2\sqrt{2}}$ & $\dfrac{\sqrt{5}+1}{4}$ \\
$54^\circ$ & $\dfrac{\sqrt{5}+1}{4}$ & $\dfrac{\sqrt{5-\sqrt{5}}}{2\sqrt{2}}$  \\
$72^\circ$ & $\dfrac{\sqrt{5+\sqrt{5}}}{2\sqrt{2}}$ & $\dfrac{\sqrt{5}-1}{4}$ \\
\hline
\end{tabular}
\end{table}


%%%%%
%%%%%
\end{document}
