\documentclass[12pt]{article}
\usepackage{pmmeta}
\pmcanonicalname{ClassificationOfPlatonicSolids}
\pmcreated{2013-03-22 15:53:07}
\pmmodified{2013-03-22 15:53:07}
\pmowner{mps}{409}
\pmmodifier{mps}{409}
\pmtitle{classification of Platonic solids}
\pmrecord{5}{37885}
\pmprivacy{1}
\pmauthor{mps}{409}
\pmtype{Result}
\pmcomment{trigger rebuild}
\pmclassification{msc}{51-00}

\endmetadata

% this is the default PlanetMath preamble.  as your knowledge
% of TeX increases, you will probably want to edit this, but
% it should be fine as is for beginners.

% almost certainly you want these
\usepackage{amssymb}
\usepackage{amsmath}
\usepackage{amsfonts}

% used for TeXing text within eps files
%\usepackage{psfrag}
% need this for including graphics (\includegraphics)
%\usepackage{graphicx}
% for neatly defining theorems and propositions
\usepackage{amsthm}
% making logically defined graphics
%%%\usepackage{xypic}

% there are many more packages, add them here as you need them

% define commands here
\newtheorem*{proposition*}{Proposition.}
\begin{document}
\PMlinkescapeword{incident}
\PMlinkescapeword{flat}
\PMlinkescapeword{bounds}
\PMlinkescapeword{meet}
\PMlinkescapeword{measure}
\begin{proposition*}
The regular tetrahedron, regular octahedron, regular icosahedron, cube, and regular dodecahedron
are the only Platonic solids.
\end{proposition*}

\begin{proof}
Each vertex of a Platonic solid is incident with at least three 
faces.  The interior angles incident with that vertex must sum to
less than $2\pi$, for otherwise the solid would be flat at that 
vertex.  Since all faces of the solid have the same number of sides,
this implies bounds on the number of faces which could meet at a 
vertex.

The interior angle of an equilateral triangle has measure $\frac{\pi}{3}$,
so a Platonic solid could only have three, four, or five triangles
meeting at each vertex.  By similar reasoning, a Platonic solid could
only have three squares or three pentagons meeting at each vertex.  But 
the interior angle of a regular hexagon has measure $\frac{2\pi}{3}$.  
To avoid flatness a solid with hexagons as faces would thus have
to have only two faces meeting at each vertex, which is impossible.  For
polygons with more sides it only gets worse.  

Since a Platonic solid is uniquely determined by the number and kind of 
faces meeting at each vertex, there are at most five Platonic solids, 
with the numbers and kinds of faces listed above.  But
these correspond to the five known Platonic solids.  Hence there are
exactly five Platonic solids.
\end{proof}
%%%%%
%%%%%
\end{document}
