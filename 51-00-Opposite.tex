\documentclass[12pt]{article}
\usepackage{pmmeta}
\pmcanonicalname{Opposite}
\pmcreated{2013-03-22 15:59:28}
\pmmodified{2013-03-22 15:59:28}
\pmowner{Wkbj79}{1863}
\pmmodifier{Wkbj79}{1863}
\pmtitle{opposite}
\pmrecord{13}{38012}
\pmprivacy{1}
\pmauthor{Wkbj79}{1863}
\pmtype{Definition}
\pmcomment{trigger rebuild}
\pmclassification{msc}{51-00}
\pmclassification{msc}{51-01}
\pmsynonym{opposite side}{Opposite}
\pmrelated{Sohcahtoa}

\usepackage{amssymb}
\usepackage{amsmath}
\usepackage{amsfonts}

\usepackage{amsthm}
\usepackage{pstricks}

\begin{document}
Given a right triangle with an acute angle $\theta$, the side of the triangle that is {\sl opposite} of $\theta$ is the side of the triangle that is not a \PMlinkname{side}{Angle} of $\theta$.

\begin{center}
\begin{pspicture}(0,-2)(8,4)
\pspolygon(0,0)(4,4)(4,0)
\rput[l](4.1,2.1){opposite}
\psline(3.8,0)(3.8,0.2)
\psline(3.8,0.2)(4,0.2)
\psarc(0,0){0.3}{0}{45}
\rput[b](0.5,0.15){$\theta$}
\rput[l](0,0){.}
\rput[a](4,0){.}
\rput[b](4,4){.}
\end{pspicture}
\end{center}

This definition can be generalized to any triangle and any angle $\theta$:  The side of the triangle that is \emph{opposite} of $\theta$ is the side of the triangle that is not a side of $\theta$.  This more general definition is not used as commonly as the definition for right triangles.
%%%%%
%%%%%
\end{document}
