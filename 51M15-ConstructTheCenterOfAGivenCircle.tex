\documentclass[12pt]{article}
\usepackage{pmmeta}
\pmcanonicalname{ConstructTheCenterOfAGivenCircle}
\pmcreated{2013-03-22 17:13:41}
\pmmodified{2013-03-22 17:13:41}
\pmowner{rm50}{10146}
\pmmodifier{rm50}{10146}
\pmtitle{construct the center of a given circle}
\pmrecord{9}{39555}
\pmprivacy{1}
\pmauthor{rm50}{10146}
\pmtype{Derivation}
\pmcomment{trigger rebuild}
\pmclassification{msc}{51M15}
\pmclassification{msc}{51-00}
\pmrelated{CompassAndStraightedgeConstructionOfCenterOfGivenCircle}

% this is the default PlanetMath preamble.  as your knowledge
% of TeX increases, you will probably want to edit this, but
% it should be fine as is for beginners.

% almost certainly you want these
\usepackage{amssymb}
\usepackage{amsmath}
\usepackage{amsfonts}

% used for TeXing text within eps files
%\usepackage{psfrag}
% need this for including graphics (\includegraphics)
%\usepackage{graphicx}
% for neatly defining theorems and propositions
%\usepackage{amsthm}
% making logically defined graphics
%%%\usepackage{xypic}
\usepackage{pstricks}

% there are many more packages, add them here as you need them

% define commands here
\newcommand{\ol}{\overline}
\begin{document}
\PMlinkescapeword{center}
\PMlinkescapeword{addition}
\emph{[Euclid, Book III, Prop. 1]} Find the \PMlinkname{center}{Center8} of a given circle.

Since, in Euclidean geometry, a circle has one center only, it suffices to construct a point that is a center of the given circle.

Draw any chord $\ol{AB}$ in the circle, and construct the perpendicular bisector of $\ol{AB}$, intersecting $\ol{AB}$ in $C$, and the circle in $D,E$. 

Let $O$ be the center of the circle; we will show that $O$ is the midpoint of $\ol{DE}$. Note that in the diagram below, $O$ is purposely drawn not to lie on $\ol{DE}$; the proof shows that this position is impossible and that in fact $O$ lies on $\ol{DE}$. It then follows easily that in fact $O$ is the midpoint of $\ol{DE}$. 

\begin{center}
% Generated by eukleides 1.0.3
%--eukleides
%frame(-2.8,-2.8,2.6,2.6)
%c=circle(point(0,0),2); draw(c)
%A=point(c,230:); draw(A)
%B=point(c,330:); draw(B)
%l=segment(A,B); draw(l)
%b=bisector(l); draw(b,dotted)
%D E intersection(b,c); draw(D); draw(E); draw(segment(D,E))
%C=intersection(b,line(A,B)); draw(C)
%mark(A,C,E,right,0.75)
%mark(segment(A,C),double)
%mark(segment(C,B),double)
%draw("$C$",C,135:); draw("$A$",A,230:); draw("$B$",B,330:); draw("$D$",D,315:); draw("$E$",E,45:)
%O=point(0.6,0.5); draw(O); draw("$O$",O,45:)
%draw(segment(O,A)); mark(segment(O,A),simple)
%draw(segment(O,B)); mark(segment(O,B),simple)
%draw(segment(O,C))
%--end
\begin{pspicture*}(-2.8000,-2.8000)(2.6000,2.6000)
\rput(-2.81,-2.81){.}
\rput(2.61,2.61){.}
\pscircle(0.0000,0.0000){2.0000}
\psdots[dotstyle=*, dotscale=1.0000](-1.2856,-1.5321)
\psdots[dotstyle=*, dotscale=1.0000](1.7321,-1.0000)
\psline(-1.2856,-1.5321)(1.7321,-1.0000)
\psline[linestyle=dotted](0.4585,-2.6000)(-0.4585,2.6000)
\psdots[dotstyle=*, dotscale=1.0000](0.3473,-1.9696)
\psdots[dotstyle=*, dotscale=1.0000](-0.3473,1.9696)
\psline(0.3473,-1.9696)(-0.3473,1.9696)
\psdots[dotstyle=*, dotscale=1.0000](0.2232,-1.2660)
\psline(0.0017,-1.3051)(-0.0374,-1.0835)(0.1842,-1.0445)
\psline(-0.5868,-1.2566)(-0.5347,-1.5520)
\psline(-0.5277,-1.2461)(-0.4756,-1.5416)
\psline(0.9221,-0.9905)(0.9741,-1.2860)
\psline(0.9811,-0.9801)(1.0332,-1.2755)
\uput{0.3000}[135.0000](0.2232,-1.2660){$C$}
\uput{0.3000}[230.0000](-1.2856,-1.5321){$A$}
\uput{0.3000}[330.0000](1.7321,-1.0000){$B$}
\uput{0.3000}[315.0000](0.3473,-1.9696){$D$}
\uput{0.3000}[45.0000](-0.3473,1.9696){$E$}
\psdots[dotstyle=*, dotscale=1.0000](0.6000,0.5000)
\uput{0.3000}[45.0000](0.6000,0.5000){$O$}
\psline(0.6000,0.5000)(-1.2856,-1.5321)
\psline(-0.2328,-0.6181)(-0.4527,-0.4140)
\psline(0.6000,0.5000)(1.7321,-1.0000)
\psline(1.2858,-0.1596)(1.0463,-0.3404)
\psline(0.6000,0.5000)(0.2232,-1.2660)
\end{pspicture*}
% End of figure
\end{center}
Since $O$ is the center of the circle, it follows that $OA=OB$. Since $\ol{DE}$ bisects $\ol{AB}$, we see in addition that $AC=BC$. $\triangle ACO$ and $\triangle BCO$ share their third side, $\ol{OC}$. So by SSS, $\triangle ACO \cong \triangle BCO$, and thus, using CPCTC, $\angle ACO\cong\angle BCO$. But $\angle ACO+\angle BCO=180^{\circ}$, so $\angle ACO$ and $\angle BCO$ are each right angles. Thus $O$ in fact lies on $\ol{DE}$.

However, since $O$ is the center of the circle, it must be equidistant from $D$ and $E$, and thus $O$ is the midpoint of $\ol{DE}$.

%%%%%
%%%%%
\end{document}
