\documentclass[12pt]{article}
\usepackage{pmmeta}
\pmcanonicalname{ProofOfVanAubelTheorem}
\pmcreated{2013-03-22 14:03:28}
\pmmodified{2013-03-22 14:03:28}
\pmowner{drini}{3}
\pmmodifier{drini}{3}
\pmtitle{proof of Van Aubel theorem}
\pmrecord{6}{35413}
\pmprivacy{1}
\pmauthor{drini}{3}
\pmtype{Proof}
\pmcomment{trigger rebuild}
\pmclassification{msc}{51N20}
\pmsynonym{Van Aubel's theorem}{ProofOfVanAubelTheorem}
\pmrelated{ProofOfVanAubelsTheorem}
\pmrelated{CevasTheorem}
\pmrelated{VanAubelTheorem}
\pmrelated{TrigonometricVersionOfCevasTheorem}

\endmetadata

\usepackage{graphicx}
%%%\usepackage{xypic} 
\usepackage{bbm}
\newcommand{\Z}{\mathbbmss{Z}}
\newcommand{\C}{\mathbbmss{C}}
\newcommand{\R}{\mathbbmss{R}}
\newcommand{\Q}{\mathbbmss{Q}}
\newcommand{\mathbb}[1]{\mathbbmss{#1}}
\newcommand{\figura}[1]{\begin{center}\includegraphics{#1}\end{center}}
\newcommand{\figuraex}[2]{\begin{center}\includegraphics[#2]{#1}\end{center}}
\usepackage{amsmath}
\begin{document}
We want to prove
\[\frac{CP}{PF} =\frac{CD}{DB} +\frac{CE}{EA}\]
\figura{aubel}

On the picture, let us call $\phi$ to the angle $\angle ABE$ and $\psi$ to the angle $\angle EBC$.

A generalization of bisector's theorem states
\[\frac{CE}{EA} = \frac{CB \sin \psi}{AB\sin\phi} \quad\mbox{on }\triangle ABC\]
and
\[\frac{CP}{PF} = \frac{CB \sin \psi}{FB\sin\phi} \quad\mbox{on }\triangle FBC.
\]
From the two equalities we can get
\[\frac{CE\cdot AB}{EA}=\frac{CP\cdot FB}{PF}\]
and thus
\[\frac{CP}{PF}=\frac{CE\cdot AB}{EA\cdot FB}.\]
Since $AB=AF+FB$, substituting leads to
\begin{align*}
\frac{CE\cdot AB}{EA\cdot FB}&=\frac{CE(AF+FB)}{EA\cdot FB}\\
&=\frac{CE\cdot AF}{EA\cdot FB}+\frac{CE\cdot FB}{EA\cdot FB}\\
&=\frac{CE\cdot AF}{EA\cdot FB}+\frac{CE}{EA}
\end{align*}

But Ceva's theorem states 
\[\frac{CE}{EA}\cdot\frac{AF}{FB}\cdot\frac{BD}{DC}=1\]
and so 
\[\frac{CE\cdot AF}{EA\cdot FB}=\frac{CD}{DB}\]
Subsituting the last equality gives the desired result.
%%%%%
%%%%%
\end{document}
