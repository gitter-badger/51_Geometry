\documentclass[12pt]{article}
\usepackage{pmmeta}
\pmcanonicalname{Legs}
\pmcreated{2013-03-22 12:06:02}
\pmmodified{2013-03-22 12:06:02}
\pmowner{akrowne}{2}
\pmmodifier{akrowne}{2}
\pmtitle{legs}
\pmrecord{8}{31205}
\pmprivacy{1}
\pmauthor{akrowne}{2}
\pmtype{Definition}
\pmcomment{trigger rebuild}
\pmclassification{msc}{51-00}

\endmetadata

\usepackage{amssymb}
\usepackage{amsmath}
\usepackage{amsfonts}
\usepackage{graphicx}
\begin{document}
The {\it legs} of a right triangle are the two sides which are not the hypotenuse.

\begin{center}

\includegraphics[scale=.72]{legs.eps}

{\tiny Above: Various right triangles, with legs in red. }

\end{center}

Note that there are no legs for scalene or non-right isosceles triangles, just as there is no notion of hypotenuse for these special triangles.  An exception to this is that the two equal sides of an isosceles triangle may occasionally be referred to as ``legs'', but this is uncommon.
%%%%%
%%%%%
%%%%%
\end{document}
