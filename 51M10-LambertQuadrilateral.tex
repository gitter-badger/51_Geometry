\documentclass[12pt]{article}
\usepackage{pmmeta}
\pmcanonicalname{LambertQuadrilateral}
\pmcreated{2013-03-22 17:08:04}
\pmmodified{2013-03-22 17:08:04}
\pmowner{Wkbj79}{1863}
\pmmodifier{Wkbj79}{1863}
\pmtitle{Lambert quadrilateral}
\pmrecord{24}{39440}
\pmprivacy{1}
\pmauthor{Wkbj79}{1863}
\pmtype{Definition}
\pmcomment{trigger rebuild}
\pmclassification{msc}{51M10}
\pmclassification{msc}{51-00}
\pmsynonym{Lambert's quadrilateral}{LambertQuadrilateral}
\pmrelated{RightTrapezoid}

\usepackage{amssymb}
\usepackage{amsmath}
\usepackage{amsfonts}
\usepackage{pstricks}
\usepackage{psfrag}
\usepackage{graphicx}
\usepackage{amsthm}
%%\usepackage{xypic}

\begin{document}
In hyperbolic geometry, a \emph{Lambert quadrilateral} is a quadrilateral with exactly three right angles.  Since the angle sum of a triangle in hyperbolic geometry is strictly less than $\pi$ radians, the angle sum of a quadrilateral in hyperbolic geometry is strictly less than $2\pi$ radians.  Thus, in any Lambert quadrilateral, the angle that is not a right angle must be acute.

The discovery of Lambert quadrilaterals is attributed to Johann Lambert.

Both pairs of opposite sides of a Lambert quadrilateral are disjointly parallel since, in both cases, they have a common perpendicular.  Therefore, Lambert quadrilaterals are parallelograms.  Note also that Lambert quadrilaterals are right trapezoids.

Below are some examples of Lambert quadrilaterals in various models.  In each example, the Lambert quadrilateral is labelled as $ABCD$.

\begin{itemize}
\item The Beltrami-Klein model:

In each of these examples, blue lines indicate verification of right angles by using the poles, and green lines indicate verification of acute angles by using the poles.  (Recall that most other models of hyperbolic geometry are angle preserving.  Thus, verification of angle measures is unnecessary in those models.)

\begin{center}
\begin{pspicture}(-3,-3)(4,4)
\pscircle[linestyle=dashed](0,0){2}
\psline[linewidth=1.5pt]{o-o}(-2,0)(2,0)
\psline[linewidth=1.5pt]{o-o}(0,-2)(0,2)
\psline{o-o}(-1.6,1.2)(1.6,1.2)
\psline{<->}(-2.8,-0.4)(0.5,4)
\psline{<->}(2.8,-0.4)(-0.5,4)
\psline{o-o}(1.2,-1.6)(1.2,1.6)
\psline{<->}(-0.3,-2.725)(4,0.5)
\psline{<->}(-0.3,2.725)(4,-0.5)
\psline[linewidth=0.1pt, linecolor=blue]{<->}(0,-3)(0,4)
\psline[linewidth=0.1pt, linecolor=blue]{<->}(-3,0)(4,0)
\psline[linecolor=green]{<->}(0.5,1.59375)(4,-0.375)
\rput[a](-0.2,-0.2){$A$}
\rput[b](-0.2,0.8){$B$}
\rput[b](1,0.8){$C$}
\rput[a](1,-0.2){$D$}
\rput[l](-3,0){.}
\rput[r](4,0){.}
\rput[a](0,4){.}
\rput[b](0,-3){.}
\psdots(0,3.3333)(3.3333,0)(0,0)(0,1.2)(1.2,1.2)(1.2,0)
\end{pspicture}
\end{center}

\begin{center}
\begin{pspicture}(-4,-6)(4,6)
\pscircle[linestyle=dashed](0,0){2}
\psline{o-o}(-1.2,-1.6)(-1.2,1.6)
\psline{<->}(-4,-0.5)(2,4)
\psline{<->}(-4,0.5)(2,-4)
\psline[linewidth=1.5pt]{o-o}(-1.972414138,0.3310345)(1.6,1.2)
\psline{<->}(-2.7,-4)(-1.021,6)
\psline{<->}(-2,6)(4,-2)
\psline[linewidth=1.5pt]{o-o}(-1.972414138,-0.3310345)(1.6,-1.2)
\psline{<->}(-2.7,4)(-1.021,-6)
\psline{<->}(-2,-6)(4,2)
\psline[linewidth=1.5pt]{o-o}(-0.9557,1.7574)(-0.6757,-1.8826)
\psline[linewidth=0.1pt, linecolor=blue]{<->}(-3.95,-0.15)(3.75,1.723)
\psline[linewidth=0.1pt, linecolor=blue]{<->}(-3.95,0.15)(3.75,-1.723)
\psline[linewidth=0.1pt, linecolor=blue]{<->}(-1.282,6)(-0.358977,-6)
\psline[linecolor=green]{<->}(-1.3058,-6)(-0.7108577,0)
\rput[a](-1.4,-0.7){$A$}
\rput[b](-1.4,0.6){$B$}
\rput[b](-0.7,0.8){$C$}
\rput[a](-0.6,-0.4){$D$}
\rput[l](-4,0.5){.}
\rput[r](4,2){.}
\rput[a](-2,6){.}
\rput[b](-2,-6){.}
\psdots(-3.333333,0)(-1.2,4.9333)(-1.2,-4.9333)(-1.2,0.51892)(-1.2,-0.51892)(-0.86667,0.6)(-0.7726,-0.6229)
\end{pspicture}
\end{center}

\item The Poincar\'e disc model:

\begin{center}
\begin{pspicture}(-2,-2)(2,2)
\pscircle[linestyle=dashed](0,0){2}
\psline{o-o}(-2,0)(2,0)
\psline{o-o}(0,-2)(0,2)
\psarc{o-o}(0,3.3333){2.6667}{233.13}{306.87}
\psarc{o-o}(3.3333,0){2.6667}{143.13}{216.87}
\rput[a](-0.2,-0.2){$A$}
\rput[b](-0.2,0.8){$B$}
\rput[b](0.5,0.8){$C$}
\rput[a](0.9,-0.2){$D$}
\rput[r](-2,0){.}
\rput[l](2,0){.}
\rput[b](0,2){.}
\rput[a](0,-2){.}
\psdots(0,0)(0,0.6667)(0.78475,0.78475)(0.6667,0)
\end{pspicture}
\end{center}

\item The upper half plane model:

\begin{center}
\begin{pspicture}(-5,-0.1)(5,4)
\psline[linestyle=dashed]{<->}(-5,0)(5,0)
\psline{o->}(-1,0)(-1,4)
\psarc{o-o}(-1,0){2}{0}{180}
\psarc{o-o}(-1,0){3}{0}{180}
\psarc{o-o}(2,0){2}{0}{180}
\rput[l](-0.9,1.7){$A$}
\rput[l](-0.9,2.7){$B$}
\rput[l](1.4,2.2){$C$}
\rput[b](0.5,1.5){$D$}
\rput[l](-5,0){.}
\rput[r](5,0){.}
\rput[a](-1,4){.}
\psdots(-1,2)(-1,3)(0.5,1.323)(1.3333,1.8856)
\end{pspicture}
\end{center}
\end{itemize}
%%%%%
%%%%%
\end{document}
