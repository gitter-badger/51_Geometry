\documentclass[12pt]{article}
\usepackage{pmmeta}
\pmcanonicalname{Parallelogram}
\pmcreated{2013-03-22 12:02:21}
\pmmodified{2013-03-22 12:02:21}
\pmowner{drini}{3}
\pmmodifier{drini}{3}
\pmtitle{parallelogram}
\pmrecord{11}{31081}
\pmprivacy{1}
\pmauthor{drini}{3}
\pmtype{Definition}
\pmcomment{trigger rebuild}
\pmclassification{msc}{51-00}
\pmrelated{ParallelogramTheorems}
\pmrelated{Quadrilateral}
\pmrelated{Rectangle}
\pmrelated{Rhombus}
\pmrelated{Square}
\pmrelated{ParallelogramLaw}
\pmrelated{Kite}
\pmrelated{Rhomboid}
\pmrelated{ProofOfParallelogramTheorems}
\pmrelated{Parallelotope}

\usepackage{amssymb}
\usepackage{amsmath}
\usepackage{amsfonts}
\usepackage{graphicx}
%%%\usepackage{xypic}

\begin{document}
A {\em parallelogram} is a quadrilateral whose opposite sides are parallel.

Some special parallelograms have their own names: squares, rectangles, rhombuses.
A rectangle is a parallelogram whose all angles are equal (i.e. $90^\circ$), a rhombus is a parallelogram whose all sides are equal, and a square is a parallelogram that is a rectangle and a rhombus at the same time.

\includegraphics{parallelograms}

Every parallelogram have their opposite sides and opposite angles equal. Also, adjacent angles of a parallelogram always add up to $180^\circ$, and the diagonals bisect each other.

A \emph{base} of a parallelogram is one of its sides.  Any side of a
parallelogram can be selected as its base.  Once a base has been
selected, the \emph{height} of the parallelogram is defined to be the
length of any line segment perpendicular to the base which extends
from the base to the opposite side parallel to the base.  The area of
a parallelogram with base length $b$ and height $h$ is given by the
formula  \,$A = bh$.

There is also a neat relation between the length of the sides and the lengths of the diagonals called the parallelogram law.

\PMlinkescapeword{base}
\PMlinkescapeword{height}
%%%%%
%%%%%
%%%%%
\end{document}
