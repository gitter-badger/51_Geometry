\documentclass[12pt]{article}
\usepackage{pmmeta}
\pmcanonicalname{VolumeAsIntegral}
\pmcreated{2013-03-22 17:20:44}
\pmmodified{2013-03-22 17:20:44}
\pmowner{pahio}{2872}
\pmmodifier{pahio}{2872}
\pmtitle{volume as integral}
\pmrecord{9}{39702}
\pmprivacy{1}
\pmauthor{pahio}{2872}
\pmtype{Topic}
\pmcomment{trigger rebuild}
\pmclassification{msc}{51M25}
\pmclassification{msc}{51-00}
\pmrelated{Volume}
\pmrelated{VolumeOfSolidOfRevolution}
\pmrelated{RiemannMultipleIntegral}
\pmrelated{ExampleOfRiemannTripleIntegral}

% this is the default PlanetMath preamble.  as your knowledge
% of TeX increases, you will probably want to edit this, but
% it should be fine as is for beginners.

% almost certainly you want these
\usepackage{amssymb}
\usepackage{amsmath}
\usepackage{amsfonts}

% used for TeXing text within eps files
%\usepackage{psfrag}
% need this for including graphics (\includegraphics)
%\usepackage{graphicx}
% for neatly defining theorems and propositions
 \usepackage{amsthm}
% making logically defined graphics
%%%\usepackage{xypic}

% there are many more packages, add them here as you need them

% define commands here

\theoremstyle{definition}
\newtheorem*{thmplain}{Theorem}

\begin{document}
\PMlinkescapeword{formula} \PMlinkescapeword{cut}
The \PMlinkname{volume of a solid of revolution}{VolumeOfSolidOfRevolution} can be obtained from
                $$V \;=\; \int_a^b\pi[f(x)]^2\,dx,$$
where the integrand is the area of the intersection disc of the solid of revolution and a plane perpendicular to the axis of revolution at a certain value of $x$.\, This volume formula may be generalized to an analogous formula containing instead of the area $\pi[f(x)]^2$ a more general intersection area $A(t)$ obtained from a given solid by cutting it with a set of parallel planes determined by the parameter $t$ on a certain axis.\, One must assume that the function \,$t \mapsto A(t)$\, is continuous on an interval \,$[a,\,b]$\, where $a$ and $b$ correspond to the ``ends'' of the solid.\, If the $t$-axis \PMlinkname{forms an angle}{AngleBetweenTwoLines} $\omega$ with the normal line of those planes, then we have the volume formula of the form
      $$V \;=\; \int_a^b\!A(t)\,dt\,\cos\omega.$$


%%%%%
%%%%%
\end{document}
