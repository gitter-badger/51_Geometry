\documentclass[12pt]{article}
\usepackage{pmmeta}
\pmcanonicalname{AngleBetweenLineAndPlane}
\pmcreated{2013-03-22 17:30:14}
\pmmodified{2013-03-22 17:30:14}
\pmowner{pahio}{2872}
\pmmodifier{pahio}{2872}
\pmtitle{angle between line and plane}
\pmrecord{13}{39893}
\pmprivacy{1}
\pmauthor{pahio}{2872}
\pmtype{Definition}
\pmcomment{trigger rebuild}
\pmclassification{msc}{51N20}
\pmsynonym{slant}{AngleBetweenLineAndPlane}
\pmsynonym{inclination}{AngleBetweenLineAndPlane}
\pmrelated{AngleBetweenTwoLines}
\pmrelated{DotProduct}
\pmrelated{EquationOfPlane}
\pmrelated{AngleBetweenTwoPlanes}
\pmrelated{NormalOfPlane}
\pmrelated{ProjectionOfRightAngle}
\pmdefines{angle between plane and line}

% this is the default PlanetMath preamble.  as your knowledge
% of TeX increases, you will probably want to edit this, but
% it should be fine as is for beginners.

% almost certainly you want these
\usepackage{amssymb}
\usepackage{amsmath}
\usepackage{amsfonts}

% used for TeXing text within eps files
%\usepackage{psfrag}
% need this for including graphics (\includegraphics)
%\usepackage{graphicx}
% for neatly defining theorems and propositions
 \usepackage{amsthm}
% making logically defined graphics
%%%\usepackage{xypic}

% there are many more packages, add them here as you need them

\usepackage{pstricks}

% define commands here

\theoremstyle{definition}
\newtheorem*{thmplain}{Theorem}

\begin{document}
The {\em angle between a line $l$ and a plane $\tau$} is defined as the \textbf{least possible} angle $\omega$ between $l$ and a line contained by $\tau$.

It is apparent that $\omega$ satisfies always\, $0 \leqq \omega \leqq 90^\circ$.

Let the plane $\tau$ be given by the \PMlinkname{equation}{EquationOfPlane} \,$Ax\!+\!By\!+\!Cz\!+\!D = 0$,\, i.e. its normal vector has the components $A,\,B,\,C$.  Let a direction vector of the line $l$ have the components $a,\,b,\,c$.  Then the angle $\omega$ between $l$ and $\tau$ is obtained from the equation
    $$\sin\omega = \frac{|Aa\!+\!Bb\!+\!Cc|}{\sqrt{A^2\!+\!B^2\!+\!C^2}\sqrt{a^2\!+\!b^2\!+\!c^2}}.$$
In fact, the \PMlinkname{right hand side}{Equation} is the cosine of the angle $\alpha$ between $l$ and the surface normal of $\tau$ (see angle between two lines), and $\omega$ is the complementary angle of $\alpha$.\\
\begin{center}
\begin{pspicture}(-1,-0.5)(8,5)
\psline(2,3)(0,0)(5,0)(7,3)
\psline(2,3)(3.44,3)
\psline(3.73,3)(7,3)
\psline[linecolor=blue](1,1.5)(4.7,1.5)
\psline(4.7,1.5)(6,1.5)
\psline[linecolor=blue](4.7,1.5)(2.5,4.5)
\rput[a](4.2,1.7){$\omega$}
\rput[a](1.7,0.7){$\tau$}
\rput[a](3,4.2){$l$}
\psline(0,-0.032)(5.023,-0.032)(7.023,3)
\psarc(4.7,1.5){0.25}{130}{180}
\rput(-1,-0.5){.}
\rput(8,5){.}
\end{pspicture}
\end{center}

\textbf{Example.}\, Consider the $xy$-plane and the line $l$ through the origin and the point \,$(1,\,1,\,1)$.  We can use the components $1,\,1,\,1$ for the direction vector of $l$ and the components $0,\,0,\,1$ for the normal vector of the plane.  We have
$$\omega \;=\; \arcsin\frac{1\!\cdot\!0\!+\!1\!\cdot\!0\!+\!1\!\cdot\!1}{\sqrt{1^2\!+\!1^2\!+\!1^2}\sqrt{0^2\!+\!0^2\!+\!1^2}} 
\;=\; \arcsin\frac{1}{\sqrt{3}} \approx 35.26^\circ.$$


%%%%%
%%%%%
\end{document}
