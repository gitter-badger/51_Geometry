\documentclass[12pt]{article}
\usepackage{pmmeta}
\pmcanonicalname{GeneralizedPythagoreanTheorem}
\pmcreated{2013-03-22 17:13:58}
\pmmodified{2013-03-22 17:13:58}
\pmowner{yogis}{15158}
\pmmodifier{yogis}{15158}
\pmtitle{generalized Pythagorean theorem}
\pmrecord{12}{39561}
\pmprivacy{1}
\pmauthor{yogis}{15158}
\pmtype{Theorem}
\pmcomment{trigger rebuild}
\pmclassification{msc}{51-00}
\pmsynonym{Pythagorean theorem}{GeneralizedPythagoreanTheorem}
%\pmkeywords{right triangle}
%\pmkeywords{polygons}
%\pmkeywords{hypothenuse}
%\pmkeywords{generalized theorem}
%\pmkeywords{Pythagora}
\pmrelated{RightTriangle}
\pmrelated{Polygon}
\pmrelated{PythagorasTheorem}
\pmrelated{ProofOfPythagoreanTheorem2}

\endmetadata

% this is the default PlanetMath preamble.  as your knowledge
% of TeX increases, you will probably want to edit this, but
% it should be fine as is for beginners.

% almost certainly you want these
\usepackage{amssymb}
\usepackage{amsmath}
\usepackage{amsfonts}

% used for TeXing text within eps files
%\usepackage{psfrag}
% need this for including graphics (\includegraphics)
%\usepackage{graphicx}
% for neatly defining theorems and propositions
%\usepackage{amsthm}
% making logically defined graphics
%%%\usepackage{xypic}
\usepackage{pstricks}
% there are many more packages, add them here as you need them

% define commands here
\newtheorem{thm}{Theorem} 
\begin{document}
% Beginning graphics and declaring dimensions of drawing window
\begin{thm} If three similar polygons are constructed on the sides
of a right triangle, then the area of the polygon constructed on the hypotenuse is 
equal to the sum of the areas of the polygons constructed on the legs.
\end{thm}


\begin{center}
\begin{pspicture}(-1,-2)(6,5)

% Drawing similar triangles in colors
\pspolygon[linecolor=blue](0,0)(0,4)(3,4)
\pspolygon[linecolor=red](3,4)(4.92,2.56)(3,0)
\pspolygon[linecolor=green](0,0)(3,0)(1.92,-1.44)

% Drawing height of original triangle in cyan
\psline[linecolor=cyan](1.08,1.44)(3,0)

% Drawing original triangle in black
\pspolygon(0,0)(3,0)(3,4)

% Marking critical points
\psdots(0,0)(0,4)(1.08,1.44)(1.92,-1.44)(3,0)(3,4)(4.92,2.56)

% Ending the picture
\end{pspicture}
\end{center}
We when say that the polygon is constructed on a side of the right triangle, we mean
that the polygon shares an entire side with the polygon.

Beginning of proof.
First, it suffices to prove the theorem for polygons of only one shape.
Suppose that the areas of two polygons $P$ and $P'$ of different shapes
constructed on some side of the triangle have a ratio $k$. Then the areas
of polygons similar to them (say $R$ and $R'$) and constructed on another side
which is $m$ times longer, will be $m^2$ times larger for both shapes. Therefore,
they will have the same ratio, $k$. Hence if the areas of $P', Q', R'$ satisfy
the property that the first two add up to the third one, then the same
will hold true for the areas of $P,Q$ and $R$ where are $k$ times greater.

So instead of constructing a square on each side, as Euclid did, we use a right
triangle that is similar to the original right triangle. And instead of constructing
the triangle on the outside, we use the inside of the triangle.

Drop an altitude of the right triangle to its hypotenuse. This altitude divides the 
triangle into two triangles and each is similar to the original triangle. 
We now have three similar right triangles constructed on the sides of the original 
right triangle, and two of them add up to the third one.

End of proof. 
%%%%%
%%%%%
\end{document}
