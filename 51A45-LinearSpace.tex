\documentclass[12pt]{article}
\usepackage{pmmeta}
\pmcanonicalname{LinearSpace}
\pmcreated{2013-03-22 19:14:57}
\pmmodified{2013-03-22 19:14:57}
\pmowner{CWoo}{3771}
\pmmodifier{CWoo}{3771}
\pmtitle{linear space}
\pmrecord{8}{42174}
\pmprivacy{1}
\pmauthor{CWoo}{3771}
\pmtype{Definition}
\pmcomment{trigger rebuild}
\pmclassification{msc}{51A45}
\pmclassification{msc}{51A05}
\pmclassification{msc}{05C65}
\pmrelated{FinitePlane}
\pmrelated{ProjectivePlane2}
\pmrelated{DeBruijnErdHosTheorem}
\pmdefines{near-pencil}
\pmdefines{star}
\pmdefines{cross}

\endmetadata

\usepackage{amssymb,amscd}
\usepackage{amsmath}
\usepackage{amsfonts}
\usepackage{mathrsfs}

% used for TeXing text within eps files
%\usepackage{psfrag}
% need this for including graphics (\includegraphics)
%\usepackage{graphicx}
% for neatly defining theorems and propositions
\usepackage{amsthm}
% making logically defined graphics
%%\usepackage{xypic}
\usepackage{pst-plot}

% define commands here
\newcommand*{\abs}[1]{\left\lvert #1\right\rvert}
\newtheorem{prop}{Proposition}
\newtheorem{thm}{Theorem}
\newtheorem{ex}{Example}
\newcommand{\real}{\mathbb{R}}
\newcommand{\pdiff}[2]{\frac{\partial #1}{\partial #2}}
\newcommand{\mpdiff}[3]{\frac{\partial^#1 #2}{\partial #3^#1}}
\begin{document}
A {\em \PMlinkescapetext{linear space}} is a near-linear space $\mathscr{S}=(\mathcal{P},\mathcal{L})$ in which every pair of distinct points are on exactly one line. 

Note: The usage of the term has little relation to its occasional appearance in linear algebra as a synonym for a vector space.

Note: The convention for drawing finite linear spaces is to ignore two-point lines.  Thus, a linear space with three points and three lines is drawn
\begin{center}
\begin{pspicture}(-1,0)(1,1)
\psset{unit=20pt}
\psdots[linecolor=blue,dotsize=5pt](0,1)
\psdots[linecolor=blue,dotsize=5pt](-1,0)
\psdots[linecolor=blue,dotsize=5pt](1,0)
\end{pspicture}
\hspace{1cm}
\begin{pspicture}(-1,0)(1,1)
\rput(0,0.5){rather than}
\end{pspicture}
\hspace{1cm}
\begin{pspicture}(-1,0)(1,1)
\psset{unit=20pt}
\psdots[linecolor=blue,dotsize=5pt](0,1)
\psdots[linecolor=blue,dotsize=5pt](-1,0)
\psdots[linecolor=blue,dotsize=5pt](1,0)
\psline(-1,0)(1,0)
\psline(0,1)(1,0)
\psline(0,1)(-1,0)
\end{pspicture}
\end{center}

\paragraph{Examples:}
\begin{enumerate}
\item We have seen in the parent entry that a graph can be thought of as a near-linear space in which every line contains two points.  A complete graph is then a linear space.

\item Let $\mathbb{F}$ be a finite field. Let $\mathcal{P}$ be the elements in the Cartesian product $\mathbb{F}\times \mathbb{F}$.  The solutions to a linear equation
\[
  \{ (x,y)\in\mathcal{P} \mid ax+by =c\}
\]
for some $a, b, c\in \mathbb{F}$, where $a$ and $b$ are not both zero, form a line in $\mathcal{L}$. Since any two points determine a unique line,
$\mathscr{A} = (\mathcal{P},\mathcal{L})$ is a linear space,
called the affine plane over $\mathbb{F}$.

\item Any projective plane is a linear space.  Conversely, any linear space in which every two lines meet, and there exists a quadrangle is a projective plane.

\item A \emph{near-pencil} is a finite linear space with the following diagram:
\begin{center}
\begin{pspicture}(-3,-1)(3,1.5)
\psset{unit=20pt}
\psdots[linecolor=blue,dotsize=5pt](0,1.5)
\psdots[linecolor=blue,dotsize=5pt](-3,0)
\psdots[linecolor=blue,dotsize=5pt](-1.5,0)
\psdots[linecolor=blue,dotsize=5pt](1.5,0)
\psdots[linecolor=blue,dotsize=5pt](3,0)
\psline(-3,0)(3,0)
\uput[d](-3,0){$P_1$}
\uput[d](-1.5,0){$P_2$}
\uput[d](0,0){$\cdots$}
\uput[d](1.5,0){$P_{n-1}$}
\uput[d](3,0){$P_n$}
\rput[l](0.25,1.5){$P$}
\rput[l](3.5,0){$\ell$}
\end{pspicture}
\end{center}
In other words, a near-pencil consists of $n+1$ points and $n+1$ lines, where $n\ge 2$, such that $n$ points lie on one line $\ell$, and the remaining point $P$ lies on $n$ 2-point lines.  Of course, on any 2-point line, the point other than $P$ must be on $\ell$.

A fundamental fact concerning finite linear space is the De Bruijn Erd\H{o}s theorem, which states that given a finite linear space $\mathscr{S}=(\mathcal{P},\mathcal{L})$ with at least two lines, then $|\mathcal{P}|\le |\mathcal{L}|$, and if the equality occurs, then $\mathscr{S}$ is either a projective plane or a near-pencil.

\item
Let $k_1,\ldots,k_n$ be positive integers greater than 2.  A \emph{$(k_1,\ldots,k_n)$-star} is a linear space such that there are $n$ lines of sizes $k_1,\ldots, k_n$ respectively, which are concurrent, and all other lines are 2-point lines.  By convention, we order the integers so that $k_1\le \cdots \le k_n$.  When $n=2$, we call the space a $(k_1,k_2)$-cross.  Below are diagrams of a $(3,3,3,3)$-star and a $(4,5)$-cross:
\begin{center}
\begin{pspicture}(-1,-1)(1,1)
\psset{unit=20pt}
\psdots[linecolor=blue,dotsize=5pt](-1,0)
\psdots[linecolor=blue,dotsize=5pt](0,0)
\psdots[linecolor=blue,dotsize=5pt](1,0)
\psdots[linecolor=blue,dotsize=5pt](0,-1)
\psdots[linecolor=blue,dotsize=5pt](0,1)
\psdots[linecolor=blue,dotsize=5pt](-0.7,-0.7)
\psdots[linecolor=blue,dotsize=5pt](0.7,0.7)
\psdots[linecolor=blue,dotsize=5pt](-0.7,0.7)
\psdots[linecolor=blue,dotsize=5pt](0.7,-0.7)
\psline(-1,0)(1,0)
\psline(0,-1)(0,1)
\psline(-0.7,-0.7)(0.7,0.7)
\psline(-0.7,0.7)(0.7,-0.7)
\end{pspicture}
\hspace{2cm}
\begin{pspicture}(-2,-1)(2,1)
\psset{unit=20pt}
\psdots[linecolor=blue,dotsize=5pt](-2,0)
\psdots[linecolor=blue,dotsize=5pt](-1,0)
\psdots[linecolor=blue,dotsize=5pt](0,0)
\psdots[linecolor=blue,dotsize=5pt](1,0)
\psdots[linecolor=blue,dotsize=5pt](2,0)
\psdots[linecolor=blue,dotsize=5pt](-1,-0.5)
\psdots[linecolor=blue,dotsize=5pt](1,0.5)
\psdots[linecolor=blue,dotsize=5pt](2,1)
\psline(-2,0)(2,0)
\psline(-1,-0.5)(2,1)
\end{pspicture}
\end{center}

\item There are five non-isomorphic five-point linear spaces:
\begin{center}
\begin{pspicture}(-2,-1)(2,1.5)
\psset{unit=20pt}
\psdots[linecolor=blue,dotsize=5pt](-2,0)
\psdots[linecolor=blue,dotsize=5pt](-1,0)
\psdots[linecolor=blue,dotsize=5pt](0,0)
\psdots[linecolor=blue,dotsize=5pt](1,0)
\psdots[linecolor=blue,dotsize=5pt](2,0)
\psline(-2,0)(2,0)
\end{pspicture}
\begin{pspicture}(-1.25,-1)(1.25,1.5)
\psset{unit=20pt}
\psdots[linecolor=blue,dotsize=5pt](-1.5,0)
\psdots[linecolor=blue,dotsize=5pt](-0.5,0)
\psdots[linecolor=blue,dotsize=5pt](0,1)
\psdots[linecolor=blue,dotsize=5pt](0.5,0)
\psdots[linecolor=blue,dotsize=5pt](1.5,0)
\psline(-1.5,0)(1.5,0)
\end{pspicture}
\begin{pspicture}(-1.5,-1)(1.5,1.5)
\psset{unit=20pt}
\psdots[linecolor=blue,dotsize=5pt](-1,0)
\psdots[linecolor=blue,dotsize=5pt](0,0)
\psdots[linecolor=blue,dotsize=5pt](1,0)
\psdots[linecolor=blue,dotsize=5pt](0,0.5)
\psdots[linecolor=blue,dotsize=5pt](1,1)
\psline(-1,0)(1,0)
\psline(-1,0)(1,1)
\end{pspicture}
\begin{pspicture}(-1,-1)(1,1.5)
\psset{unit=20pt}
\psdots[linecolor=blue,dotsize=5pt](-1,0)
\psdots[linecolor=blue,dotsize=5pt](0,0)
\psdots[linecolor=blue,dotsize=5pt](1,0)
\psdots[linecolor=blue,dotsize=5pt](-0.5,1)
\psdots[linecolor=blue,dotsize=5pt](0.5,1)
\psline(-1,0)(1,0)
\end{pspicture}
\begin{pspicture}(-1.25,-1)(1.25,1.5)
\psset{unit=20pt}
\psdots[linecolor=blue,dotsize=5pt](-0.6,0)
\psdots[linecolor=blue,dotsize=5pt](0.6,0)
\psdots[linecolor=blue,dotsize=5pt](-1.15,0.8)
\psdots[linecolor=blue,dotsize=5pt](1.15,0.8)
\psdots[linecolor=blue,dotsize=5pt](0,1.5)
\end{pspicture}
\end{center}
Corresponding to the diagrams above, the spaces have respectively 1, 5, 6, 8, and 10 lines.  Of these, the second is a near-pencil, the third is a $(3,3)$-cross, and the last is a complete graph.
\end{enumerate}

\paragraph{Some properties:}
\begin{enumerate}
\item In a linear space $\mathscr{S}=(\mathcal{P},\mathcal{L})$, 
$$\sum_{\ell \in \mathcal{L}} \binom{|\ell|}{2} = \binom{|\mathcal{P}|}{2}.$$

\item Let $p$ be an arbitrary point in a linear space, 
\[\sum_{\ell \ni p} (|\ell| - 1) = |\mathcal{P}| - 1\]
where the sum is taken over all lines containing $p$. This holds because given any point, this point forms exactly one line with every other point, so $|\ell| - 1$ counts the number of points $p$ shares in line $\ell$. Summing over all lines gives all the points except $p$.
\end{enumerate}

\begin{thebibliography}{7}
\bibitem{BB} L. M. Batten, A. Beutelspacher {\it The Theory of Finite Linear Spaces}, Cambridge University Press (2009)
\end{thebibliography}
%%%%%
%%%%%
\end{document}
