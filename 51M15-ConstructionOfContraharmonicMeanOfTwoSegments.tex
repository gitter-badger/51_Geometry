\documentclass[12pt]{article}
\usepackage{pmmeta}
\pmcanonicalname{ConstructionOfContraharmonicMeanOfTwoSegments}
\pmcreated{2013-03-22 19:12:34}
\pmmodified{2013-03-22 19:12:34}
\pmowner{pahio}{2872}
\pmmodifier{pahio}{2872}
\pmtitle{construction of contraharmonic mean of two segments}
\pmrecord{8}{42125}
\pmprivacy{1}
\pmauthor{pahio}{2872}
\pmtype{Example}
\pmcomment{trigger rebuild}
\pmclassification{msc}{51M15}
\pmclassification{msc}{51-00}
\pmrelated{ContraharmonicProportion}
\pmrelated{HarmonicMeanInTrapezoid}

\endmetadata

% this is the default PlanetMath preamble.  as your knowledge
% of TeX increases, you will probably want to edit this, but
% it should be fine as is for beginners.

% almost certainly you want these
\usepackage{amssymb}
\usepackage{amsmath}
\usepackage{amsfonts}
\usepackage{pstricks}

% used for TeXing text within eps files
%\usepackage{psfrag}
% need this for including graphics (\includegraphics)
%\usepackage{graphicx}
% for neatly defining theorems and propositions
 \usepackage{amsthm}
% making logically defined graphics
%%\usepackage{xypic}

% there are many more packages, add them here as you need them

% define commands here

\theoremstyle{definition}
\newtheorem*{thmplain}{Theorem}

\begin{document}
Let $a$ and $b$ \PMlinkescapetext{mean} two line segments (and their \PMlinkescapetext{lengths}).\, The contraharmonic mean
$$x \;:=\; \frac{a^2\!+\!b^2}{a\!+\!b} \;=\; \frac{\left(\sqrt{a^2\!+\!b^2}\,\right)^2}{a\!+\!b},$$
satisfying the proportion equation
$$\frac{a\!+\!b}{\sqrt{a^2\!+\!b^2}} \;=\; \frac{\sqrt{a^2\!+\!b^2}}{x},$$
can be \PMlinkname{constructed geometrically}{GeometricConstruction} as the third proportional of the segments
$a\!+\!b$ and $\sqrt{a^2\!+\!b^2}$, the latter of which is gotten as the hypotenuse of the right triangle with catheti 
$a$ and $b$.\, See the construction of fourth proportional.\\

\begin{center}
\begin{pspicture}(-0.5,-0.5)(9.5,4)
\psline(0,0)(9,0)
\psline(0,0)(6,4)
\psline(3,0)(3,2)
\psline[linestyle=dotted](5,0)(3,2)
\psline[linestyle=dotted](8.61,0)(5.17,3.44)
\rput(1.6,-0.25){$b$}
\rput(3.17,0.9){$a$}
\rput(4,-0.2){$a$}
\rput(1,1.2){$\sqrt{a^2\!+\!b^2}$}
\rput(6.6,-0.3){$\sqrt{a^2\!+\!b^2}$}
\rput(4.1,3){$x$}
\psline(2.8,0)(2.8,0.2)(3,0.2)
\psdots(5,0)(8.61,0)(3,2)(5.17,3.44)
\rput(-0.5,-0.5){.}
\rput(9.5,4){.}
\end{pspicture}
\end{center}
(The dotted lines are parallel, with declivity $45^\circ$.)
%%%%%
%%%%%
\end{document}
