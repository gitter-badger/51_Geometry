\documentclass[12pt]{article}
\usepackage{pmmeta}
\pmcanonicalname{CompassAndStraightedgeConstructionOfSquare}
\pmcreated{2013-03-22 17:19:13}
\pmmodified{2013-03-22 17:19:13}
\pmowner{Wkbj79}{1863}
\pmmodifier{Wkbj79}{1863}
\pmtitle{compass and straightedge construction of square}
\pmrecord{5}{39671}
\pmprivacy{1}
\pmauthor{Wkbj79}{1863}
\pmtype{Algorithm}
\pmcomment{trigger rebuild}
\pmclassification{msc}{51M15}
\pmclassification{msc}{51-00}

\usepackage{amssymb}
\usepackage{amsmath}
\usepackage{amsfonts}
\usepackage{pstricks}
\usepackage{psfrag}
\usepackage{graphicx}
\usepackage{amsthm}
%%\usepackage{xypic}

\begin{document}
\PMlinkescapeword{label}

One can construct a square with sides of a given length $s$ using compass and straightedge as follows:

\begin{enumerate}

\item Draw a line segment of length s. Label its endpoints $P$ and $Q$.

\begin{center}
\begin{pspicture}(-3,-1)(4,1)
\rput[a](-2,0.04){.}
\psline[linecolor=blue](-2,0)(2,0)
\psdots(-2,0)(2,0)
\rput[a](-2.2,-0.2){$P$}
\rput[a](2.2,-0.2){$Q$}
\end{pspicture}
\end{center}

\item Extend the line segment past $Q$.

\begin{center}
\begin{pspicture}(-3,-1)(4,1)
\rput[a](-2,0.04){.}
\rput[r](3.5,0){.}
\psline(-2,0)(2,0)
\psline[linecolor=blue]{->}(2,0)(3.5,0)
\psdots(-2,0)(2,0)
\rput[a](-2.2,-0.2){$P$}
\rput[a](2.2,-0.2){$Q$}
\end{pspicture}
\end{center}

\item Erect the perpendicular to $\overrightarrow{PQ}$ at $Q$.

\begin{center}
\begin{pspicture}(-3,-2)(4,5)
\rput[b](2,-2){.}
\rput[a](2,4.9){.}
\rput[r](3.5,0){.}
\psline{->}(-2,0)(3.5,0)
\psarc[linecolor=blue](2,0){1.3}{-10}{190}
\psarc[linecolor=blue](0.7,0){1.5}{-60}{60}
\psarc[linecolor=blue](3.3,0){1.5}{120}{240}
\psline[linecolor=blue]{<->}(2,-2)(2,5)
\psdots(-2,0)(2,0)
\rput[a](-2.2,-0.2){$P$}
\rput[a](2.2,-0.2){$Q$}
\end{pspicture}
\end{center}

\item Using the line drawn in the previous step, mark off a line segment of length $s$ such that one of its endpoints is $Q$.  Label the other endpoint as $R$.

\begin{center}
\begin{pspicture}(-3,-2)(4,5)
\rput[b](2,-2){.}
\rput[a](2,4.9){.}
\rput[r](3.5,0){.}
\psline{->}(-2,0)(3.5,0)
\psarc(2,0){1.3}{-10}{190}
\psarc(0.7,0){1.5}{-60}{60}
\psarc(3.3,0){1.5}{120}{240}
\psline{<->}(2,-2)(2,5)
\psline[linecolor=blue](2,0)(2,4)
\psdots(-2,0)(2,0)(2,4)
\rput[a](-2.2,-0.2){$P$}
\rput[a](2.2,-0.2){$Q$}
\rput[b](2.2,4.2){$R$}
\end{pspicture}
\end{center}

\item Draw an arc of the circle with center $P$ and radius $\overline{PQ}$.

\begin{center}
\begin{pspicture}(-3,-2)(4,5)
\rput[l](-2.6946,3.94){.}
\rput[b](2,-2){.}
\rput[a](2,4.9){.}
\rput[r](3.5,0){.}
\psline{->}(-2,0)(3.5,0)
\psarc(2,0){1.3}{-10}{190}
\psarc(0.7,0){1.5}{-60}{60}
\psarc(3.3,0){1.5}{120}{240}
\psline{<->}(2,-2)(2,5)
\psline(2,0)(2,4)
\psarc[linecolor=blue](-2,0){4}{80}{100}
\psdots(-2,0)(2,0)(2,4)
\rput[a](-2.2,-0.2){$P$}
\rput[a](2.2,-0.2){$Q$}
\rput[b](2.2,4.2){$R$}
\end{pspicture}
\end{center}

\item Draw an arc of the circle with center $R$ and radius $\overline{QR}$ to find the point $S$ where it intersects the arc from the previous step such that $S \neq Q$.

\begin{center}
\begin{pspicture}(-3,-2)(4,5)
\rput[l](-2.6946,3.94){.}
\rput[b](2,-2){.}
\rput[a](2,4.9){.}
\rput[r](3.5,0){.}
\psline{->}(-2,0)(3.5,0)
\psarc(2,0){1.3}{-10}{190}
\psarc(0.7,0){1.5}{-60}{60}
\psarc(3.3,0){1.5}{120}{240}
\psline{<->}(2,-2)(2,5)
\psline(2,0)(2,4)
\psarc(-2,0){4}{80}{100}
\psarc[linecolor=blue](2,4){4}{170}{190}
\psdots(-2,0)(2,0)(2,4)(-2,4)
\rput[a](-2.2,-0.2){$P$}
\rput[a](2.2,-0.2){$Q$}
\rput[b](2.2,4.2){$R$}
\rput[b](-2.2,4.2){$S$}
\end{pspicture}
\end{center}

\item Draw the square $PQRS$.

\begin{center}
\begin{pspicture}(-3,-2)(4,5)
\rput[l](-2.6946,3.94){.}
\rput[b](2,-2){.}
\rput[a](2,4.9){.}
\rput[r](3.5,0){.}
\psline{->}(-2,0)(3.5,0)
\psarc(2,0){1.3}{-10}{190}
\psarc(0.7,0){1.5}{-60}{60}
\psarc(3.3,0){1.5}{120}{240}
\psline{<->}(2,-2)(2,5)
\psline(2,0)(2,4)
\psarc(-2,0){4}{80}{100}
\psarc(2,4){4}{170}{190}
\pspolygon[linecolor=blue](-2,0)(2,0)(2,4)(-2,4)
\psdots(-2,0)(2,0)(2,4)(-2,4)
\rput[a](-2.2,-0.2){$P$}
\rput[a](2.2,-0.2){$Q$}
\rput[b](2.2,4.2){$R$}
\rput[b](-2.2,4.2){$S$}
\end{pspicture}
\end{center}

\end{enumerate}

This construction is justified because $PS=PQ=QR=QS$, yielding that $PQRS$ is a rhombus.  Since $\angle PQR$ is a right angle, it follows that $PQRS$ is a square.

If you are interested in seeing the rules for compass and straightedge constructions, click on the \PMlinkescapetext{link} provided.
%%%%%
%%%%%
\end{document}
