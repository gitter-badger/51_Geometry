\documentclass[12pt]{article}
\usepackage{pmmeta}
\pmcanonicalname{CommonPointOfTriangleMedians}
\pmcreated{2013-03-22 17:46:54}
\pmmodified{2013-03-22 17:46:54}
\pmowner{pahio}{2872}
\pmmodifier{pahio}{2872}
\pmtitle{common point of triangle medians}
\pmrecord{8}{40241}
\pmprivacy{1}
\pmauthor{pahio}{2872}
\pmtype{Theorem}
\pmcomment{trigger rebuild}
\pmclassification{msc}{51M04}
\pmrelated{MutualPositionsOfVectors}
\pmrelated{ParallelogramPrinciple}
\pmrelated{DifferenceOfVectors}
\pmrelated{TriangleMidSegmentTheorem}
\pmrelated{LengthsOfTriangleMedians}

\endmetadata

% this is the default PlanetMath preamble.  as your knowledge
% of TeX increases, you will probably want to edit this, but
% it should be fine as is for beginners.

% almost certainly you want these
\usepackage{amssymb}
\usepackage{amsmath}
\usepackage{amsfonts}

% used for TeXing text within eps files
%\usepackage{psfrag}
% need this for including graphics (\includegraphics)
%\usepackage{graphicx}
% for neatly defining theorems and propositions
 \usepackage{amsthm}
% making logically defined graphics
%%%\usepackage{xypic}

% there are many more packages, add them here as you need them

\usepackage{pstricks}

% define commands here

\theoremstyle{definition}
\newtheorem*{thmplain}{Theorem}

\begin{document}
\PMlinkescapeword{length} \PMlinkescapeword{median} \PMlinkescapeword{medians}

\textbf{Theorem.}\, The three \PMlinkname{medians}{Median} of a triangle intersect one another in one point, which divides each median in the ratio $2\!:\!1$.

\begin{center}
\begin{pspicture}(-1,-0.5)(5,5.5)
\psline(4,0)(3,5)
\psdot[linecolor=blue](0,0)
\rput[a](-0.3,-0.1){$A$}
\rput[a](4.2,-0.1){$B$}
\rput[a](2.9,5.24){$C$}
\rput[a](3.7,2.6){$D$}
\rput[a](1.34,2.6){$E$}
\rput[a](1.9,-0.25){$F$}
\psline[arrows=->,arrowsize=5pt,linecolor=blue](0,0)(2.33,1.67)
\psline[linestyle=dotted](2.33,1.67)(3.5,2.5)
\psline[arrows=->,arrowsize=5pt,linecolor=green](0,0)(4,0)
\psline[arrows=->,arrowsize=5pt,linecolor=green](4,0)(2.33,1.67)
\psline[linestyle=dotted](2.33,1.67)(1.5,2.5)
\psline[arrows=->,arrowsize=5pt,linecolor=red](0,0)(3,5)
\psline[arrows=->,arrowsize=5pt,linecolor=red](3,5)(2.33,1.67)
\psline[linestyle=dotted](2.33,1.67)(2,0)
\end{pspicture}
\end{center}
{\em Proof.}\, Let the medians of a triangle\, $ABC$\, be $AD$, $BE$ and $CF$.\, Any median vector is the arithmetic mean of the side vectors emanating from the same vertex.\, Using vectors, let us form three ways all beginning from the vertex $A$, the first going simply $2/3$ of the median vector $\overrightarrow{AD}$ (\PMlinkescapetext{blue} in the picture):
\begin{align}
\frac{2}{3}\overrightarrow{AD} = \frac{2}{3}\cdot\frac{1}{2}(\overrightarrow{AB}+\overrightarrow{AC}) = \frac{1}{3}(\overrightarrow{AB}+\overrightarrow{AC})
\end{align}
The second way goes first the side vector $\overrightarrow{AB}$ and then $2/3$ of the median vector $\overrightarrow{BE}$ (green in the picture):
\begin{align}
\overrightarrow{AB}+\frac{2}{3}\overrightarrow{BE} = 
\overrightarrow{AB}+\frac{2}{3}\!\cdot\!\frac{1}{2}\!\left[-\overrightarrow{AB}+(\overrightarrow{AC}-\overrightarrow{AB})\right]
= \frac{1}{3}(\overrightarrow{AB}+\overrightarrow{AC})
\end{align}
Similarly, the third way goes first the side vector $\overrightarrow{AC}$ and then $2/3$ of the median vector $\overrightarrow{CF}$ (red in the picture):
\begin{align}
\overrightarrow{AC}+\frac{2}{3}\overrightarrow{CF} = 
\overrightarrow{AC}+\frac{2}{3}\!\cdot\!\frac{1}{2}\!\left[-\overrightarrow{AC}+(\overrightarrow{AB}-\overrightarrow{AC})\right]
= \frac{1}{3}(\overrightarrow{AB}+\overrightarrow{AC})
\end{align}
Thus the ways (2) and (3), where one goes from $A$ to another vertex and continues along the corresponding median $2/3$ of its length, lead to the point $M$ which is attained directly along $AD$.\, This means that all medians intersect in $M$.\, The distance of $M$ from any vertex is $2/3$ of the corresponding median, and so the rest of the median is $1/3$ of its length, i.e. the ratio of the parts of any median is $2\!:\!1$.


%%%%%
%%%%%
\end{document}
