\documentclass[12pt]{article}
\usepackage{pmmeta}
\pmcanonicalname{UnitDisc}
\pmcreated{2013-03-22 15:47:01}
\pmmodified{2013-03-22 15:47:01}
\pmowner{PrimeFan}{13766}
\pmmodifier{PrimeFan}{13766}
\pmtitle{unit disc}
\pmrecord{7}{37741}
\pmprivacy{1}
\pmauthor{PrimeFan}{13766}
\pmtype{Definition}
\pmcomment{trigger rebuild}
\pmclassification{msc}{51-00}
\pmclassification{msc}{54-00}
\pmsynonym{unit ball}{UnitDisc}
\pmrelated{UnitDisk}
\pmrelated{OpenBall}
\pmrelated{Ball}
\pmdefines{closed unit disk}
\pmdefines{open unit disk}

\endmetadata

\usepackage{graphicx}
%%%\usepackage{xypic} 
\usepackage{bbm}
\newcommand{\Z}{\mathbbmss{Z}}
\newcommand{\C}{\mathbbmss{C}}
\newcommand{\R}{\mathbbmss{R}}
\newcommand{\Q}{\mathbbmss{Q}}
\newcommand{\mathbb}[1]{\mathbbmss{#1}}
\newcommand{\figura}[1]{\begin{center}\includegraphics{#1}\end{center}}
\newcommand{\figuraex}[2]{\begin{center}\includegraphics[#2]{#1}\end{center}}
\newtheorem{dfn}{Definition}
\begin{document}
The \emph{open  unit disc} around $P$ (where $P$ is a given point on the plane), is the set of points whose distance from $P$ is less than one:
\[D_1(P) = \{ Q : \vert P-Q\vert<1\}.\]

A \emph{closed unit  disc} is the set of points whose distance from $P$ is less than or equal to one:
\[\overline D_1(P)=\{Q:|P-Q| \leq 1\}\]

Unit discs are a special case of unit ball. Without further specifications, the term ''unit disc'' is used for the open unit disc about the origin, $D_1(0)$.

The specific set of points in the unit disc, and hence, its visual appearance, depends on the metric being used.
\figuraex{unitdisc}{scale=0.75}

With the standard metric, unit discs look like circles of radius one, but changing the metric changes also the corresponding set of points and therefore the shape of the discs.  	 
 		 
For instance, with the taxicab metric discs look like squares, while on the Chebyshev metric a disc is shaped like a rhombus (even though the underlying topologies are the same as the Euclidean one).
%%%%%
%%%%%
\end{document}
