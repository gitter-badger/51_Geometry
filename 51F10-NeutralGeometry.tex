\documentclass[12pt]{article}
\usepackage{pmmeta}
\pmcanonicalname{NeutralGeometry}
\pmcreated{2013-03-22 15:33:49}
\pmmodified{2013-03-22 15:33:49}
\pmowner{CWoo}{3771}
\pmmodifier{CWoo}{3771}
\pmtitle{neutral geometry}
\pmrecord{9}{37467}
\pmprivacy{1}
\pmauthor{CWoo}{3771}
\pmtype{Definition}
\pmcomment{trigger rebuild}
\pmclassification{msc}{51F10}
\pmclassification{msc}{51F05}
\pmsynonym{absolute geometry}{NeutralGeometry}
\pmsynonym{Dedekind axiom}{NeutralGeometry}
\pmdefines{hyperbolic axiom}
\pmdefines{Bolyai-Lobachevsky geometry}
\pmdefines{continuity axiom}
\pmdefines{categorical}

\usepackage{amssymb,amscd}
\usepackage{amsmath}
\usepackage{amsfonts}

% used for TeXing text within eps files
%\usepackage{psfrag}
% need this for including graphics (\includegraphics)
%\usepackage{graphicx}
% for neatly defining theorems and propositions
%\usepackage{amsthm}
% making logically defined graphics
%%%\usepackage{xypic}

% define commands here

\renewcommand{\line}[1]{\overleftrightarrow{#1}}
\newcommand{\ray}[1]{\overrightarrow{#1}}
\begin{document}
\textbf{Dedekind Cuts.}  Let $\ell$ be a line in a linear ordered
geometry $S$ and let $A,B$ be two subsets on $\ell$.  A point $p$ is
said to be between $A$ and $B$ if for any pair of points $q\in A$
and $r\in B$, $p$ is between $q$ and $r$.  Note that $p$ necessarily
lies on $\ell$.
\\\\
For example, given a ray $\rho$ on a line $\ell$.  If $p$ is the
source of $\rho$, then $p$ is a point between $\rho$ and its
opposite ray $-\rho$, regardless whether the ray is defined to be
open or closed.  It is easy to see that $p$ is the unique point
between $\rho$ and $-\rho$.
\\\\
Given a line $\ell$, a Dedekind cut on $\ell$ is a pair of subsets
$A,B\subseteq\ell$ such that $A\cup B=\ell$ and there is a unique
point $p$ between $A$ and $B$.  A ray $\rho$ on a line $\ell$ and
its compliment $\overline{\rho}$ constitute a Dedekind cut on
$\ell$.
\\\\
If $A,B$ form a Dedekind cut on $\ell$, then $A$ and $B$ have two additional properties:
\begin{enumerate}
\item no point on $A$ is strictly between two points on $B$, and
\item no point on $B$ is strictly between two points on $A$.
\end{enumerate}

Conversely, if $A,B$ satisfy the above two conditions, can we say that $A$ and $B$ constitute a Dedekind cut?  In a neutral geometry, the answer is yes.

\textbf{Neutral Geometry.} A neutral geometry is a linear ordered
geometry satisfying
\begin{enumerate}
\item the congruence axioms, and
\item the continuity axiom:
given any line $\ell$ with $\ell=A\cup B$ such that
\begin{enumerate}
\item no point on $A$ is (strictly) between two points on $B$, and
\item no point on $B$ is (strictly) between two points on $A$.
\end{enumerate}
then $A$ and $B$ constitute a Dedekind cut on $\ell$.  In other
words, there is a unique point $o$ between $A$ and $B$.
\end{enumerate}
Clearly,
$A\cap B$ contains at most one point.  The continuity axiom is also known as Dedekind's Axiom.

\textbf{Properties.}
\begin{enumerate}
\item Let $\ell=A\cup B$ be a line, satisfying (a) and (b)
above and let $p\in A$.  Suppose $\rho$ lying on $\ell$ is a ray
emanating from $p$.  Then either $\rho\subseteq A$ or $B\subseteq
\rho$.
\item Let $\ell=A\cup B$ be a line, satisfying (a) and (b)
above and let $o$ be the unique point as mentioned above.  Then a
closed ray emanating from $o$ is either $A$ or $B$.
This implies that every Dedekind cut on a line
$\ell$ consists of at least one ray.
\item We can similarly propose a continuity axiom on a ray as follows: given any ray $\rho$ with $\rho=A\cup B$ such that
\begin{itemize}
\item no point on $A$ is strictly between two points on $B$, and
\item no point on $B$ is strictly between two points on $A$.
\end{itemize}
then there is a unique point $o$ on $\rho$ between $A$ and $B$.
It turns out that the two continuity axioms are equivalent.
\item \textbf{Archimedean Property}  Given two line segments $\overline{pq}$ and $\overline{rs}$, then there is a unique natural number $n$ and a unique point $t$, such that
\begin{enumerate}
\item $t$ lies on the line segment $n\cdot\overline{rs}\subseteq\ray{rs}$,
\item $t$ does not lie on the line segment
$(n-1)\cdot\overline{rs}$, and
\item $\overline{pq}\cong\overline{rt}$.
\end{enumerate}
This property usually appears in the study of ordered fields.
\item For any given line $\ell$ and any point $p$, there exists a line $m$ passing through $p$ that is perpendicular to $\ell$.
\item Consequently, for any given line $\ell$ and any point $p$ not lying on $\ell$, there exists at leaast one line passing through $p$ that is parallel to $\ell$.  If there is more than one line passing through $p$ parallel to $\ell$, then there are infinitely many of these lines.
\end{enumerate}


\textbf{Examples}.
\begin{itemize}
\item A Euclidean geometry is a neutral geometry satisfying the Euclid's
parallel axiom: for any given line and any given point not lying on
the line, there is a unique line passing through the point and
parallel to the given line.
\item A hyperbolic geometry (or Bolyai-Lobachevsky geometry) is a neutral
geometry satisfying the hyperbolic axiom: for any given line and any
given point not lying on the line, there are at least two distinct (hence infinitely many) 
lines passing through the point and parallel to the given line.
\item In fact, one can replace the \emph{indefinite article} ``a'' in the
first letter of each of the above examples by the \emph{definite
article} ``the''.  It can be shown that any two Euclidean geometries
are geometrically isomorphic (preserving incidence, order,
congruence, and continuity).  Similarly, any two hyperbolic
geometries are isomorphic.  Such geometries are said to be
\emph{categorical}.
\item An elliptic geometry is not a neutral geometry, because pairwise distinct parallel lines do not exist.
\end{itemize}
%%%%%
%%%%%
\end{document}
