\documentclass[12pt]{article}
\usepackage{pmmeta}
\pmcanonicalname{EuclideanGeometryOfPlane}
\pmcreated{2013-03-22 15:12:13}
\pmmodified{2013-03-22 15:12:13}
\pmowner{matte}{1858}
\pmmodifier{matte}{1858}
\pmtitle{Euclidean geometry of plane}
\pmrecord{30}{36962}
\pmprivacy{1}
\pmauthor{matte}{1858}
\pmtype{Topic}
\pmcomment{trigger rebuild}
\pmclassification{msc}{51-01}
\pmsynonym{plane geometry}{EuclideanGeometryOfPlane}
\pmrelated{EuclideanGeometryOfTheSpace}
\pmrelated{AnalyticGeometry}

\endmetadata

% this is the default PlanetMath preamble.  as your knowledge
% of TeX increases, you will probably want to edit this, but
% it should be fine as is for beginners.

% almost certainly you want these
\usepackage{amssymb}
\usepackage{amsmath}
\usepackage{amsfonts}
\usepackage{amsthm}

\usepackage{mathrsfs}

% used for TeXing text within eps files
%\usepackage{psfrag}
% need this for including graphics (\includegraphics)
%\usepackage{graphicx}
% for neatly defining theorems and propositions
%
% making logically defined graphics
%%%\usepackage{xypic}

% there are many more packages, add them here as you need them

% define commands here

\newcommand{\sR}[0]{\mathbb{R}}
\newcommand{\sC}[0]{\mathbb{C}}
\newcommand{\sN}[0]{\mathbb{N}}
\newcommand{\sZ}[0]{\mathbb{Z}}

 \usepackage{bbm}
 \newcommand{\Z}{\mathbbmss{Z}}
 \newcommand{\C}{\mathbbmss{C}}
 \newcommand{\R}{\mathbbmss{R}}
 \newcommand{\Q}{\mathbbmss{Q}}



\newcommand*{\norm}[1]{\lVert #1 \rVert}
\newcommand*{\abs}[1]{| #1 |}



\newtheorem{thm}{Theorem}
\newtheorem{defn}{Definition}
\newtheorem{prop}{Proposition}
\newtheorem{lemma}{Lemma}
\newtheorem{cor}{Corollary}
\begin{document}
\subsubsection*{Basic objects}
\begin{enumerate}
\item point, 
\item line, ray, line segment, 
\item plane,
\item angle,
\item convex angle,
\item circle,
\item tangent line,
\item secant line,
\item normal line
\item center normal
\item angle bisector
\item triangle,
\item \PMlinkname{congruent triangles}{Congruence},
\item triangle solving,
\item polygon,
\item regular polygon 
\end{enumerate}

\subsubsection*{Basic concepts}
\begin{enumerate}
\item \PMlinkname{incidence}{IncidenceGeometry}, 
\item betweenness,
\item \PMlinkname{congruence}{CongruenceAxioms},
\item similarity,
\item continuity,
\item parallelism,
\item perpendicularity,
\item corresponding angles in transversal cutting
\item intercept theorem
\item locus
\item dilation and translation,
\item rotation and reflection,
\item inversion of plane,
\item harmonic division,
\item angle of view,
\item compass and straightedge constructions
\end{enumerate}

\subsubsection*{Analytic geometry}
\begin{enumerate}
\item Cartesian coordinates,
\item polar coordinates, 
\item line in plane
\item slope, slope angle, condition of orthogonality,
\item parabola,
\item \PMlinkname{circle}{Circle},
\item ellipse,
\item hyperbola,
\item conjugate hyperbola,
\item unit hyperbola,
\item tangent of conic section,
\item hyperbolas orthogonal to ellipses
\item power of point,
\item chordal,
\item confocal,
\item transition to skew-angled coordinates,
\item conjugate diameters of ellipse,
\item perimeter of ellipse
\end{enumerate}
%%%%%
%%%%%
\end{document}
