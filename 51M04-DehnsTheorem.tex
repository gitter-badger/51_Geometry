\documentclass[12pt]{article}
\usepackage{pmmeta}
\pmcanonicalname{DehnsTheorem}
\pmcreated{2013-03-22 16:18:04}
\pmmodified{2013-03-22 16:18:04}
\pmowner{paolini}{1187}
\pmmodifier{paolini}{1187}
\pmtitle{Dehn's theorem}
\pmrecord{11}{38422}
\pmprivacy{1}
\pmauthor{paolini}{1187}
\pmtype{Theorem}
\pmcomment{trigger rebuild}
\pmclassification{msc}{51M04}
\pmclassification{msc}{52B45}
\pmrelated{BanachTarskiParadox}
\pmrelated{HilbertsProblems}
\pmrelated{RegularTetrahedron3}
\pmdefines{scissor-equivalent}

\endmetadata

% this is the default PlanetMath preamble.  as your knowledge
% of TeX increases, you will probably want to edit this, but
% it should be fine as is for beginners.

% almost certainly you want these
\usepackage{amssymb}
\usepackage{amsmath}
\usepackage{amsfonts}

% used for TeXing text within eps files
%\usepackage{psfrag}
% need this for including graphics (\includegraphics)
%\usepackage{graphicx}
% for neatly defining theorems and propositions
\usepackage{amsthm}
% making logically defined graphics
%%%\usepackage{xypic}

% there are many more packages, add them here as you need them

% define commands here
\newcommand{\R}{\mathbb R}
\newtheorem{theorem}{Theorem}
\newtheorem{definition}{Definition}
\theoremstyle{remark}
\newtheorem{example}{Example}
\begin{document}
We all know the elementary formula to compute the area of a triangle: basis times height divided by two. This formula can be justified with a scissor type argument: 
one divides the triangle into smaller polygons and rearranges these polygons to obtain a rectangle which should have the same area.

Can we use the same argument to compute the volume of a pyramid? This is 
the third Hilbert's problem. Quite surprisingly the answer is negative, as states the theorem below. This means that the formulae to compute the volume of polyhedra cannot be proved without a limiting process (for example using integrals).

\begin{definition}
We say that two polyhedra $P$ and $Q$ are \emph{scissor-equivalent} if there exists a finite number $P_1,\ldots, P_N$ of polyhedra and $\theta_1,\ldots,\theta_N$ isometries such that 
\begin{enumerate}
\item $P=\bigcup_{k=1}^N P_k$ and $Q=\bigcup_{k=1}^N \theta_k(P_k)$;
\item $P_j\cap P_k$ and $\theta_j(P_j) \cap \theta_k(P_k)$ have empty interior
for every $k\neq j$
\end{enumerate}
\end{definition}

The properties given above assure that two scissor-equivalent polyhedra 
must have the same volume. It is also simple to prove that the scissor-equivalence is indeed an equivalence relation.

\begin{theorem}
The regular tetrahedron is not \emph{scissor-equivalent} to any parallelepiped.
\end{theorem}
%%%%%
%%%%%
\end{document}
