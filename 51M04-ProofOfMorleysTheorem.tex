\documentclass[12pt]{article}
\usepackage{pmmeta}
\pmcanonicalname{ProofOfMorleysTheorem}
\pmcreated{2013-03-22 13:45:44}
\pmmodified{2013-03-22 13:45:44}
\pmowner{mathcam}{2727}
\pmmodifier{mathcam}{2727}
\pmtitle{proof of Morley's theorem}
\pmrecord{6}{34465}
\pmprivacy{1}
\pmauthor{mathcam}{2727}
\pmtype{Proof}
\pmcomment{trigger rebuild}
\pmclassification{msc}{51M04}

\endmetadata

\usepackage{amssymb}
\usepackage{amsmath}
\usepackage{amsfonts}
\usepackage{graphicx}
\begin{document}
\PMlinkescapeword{vertices} \PMlinkescapeword{scheme}
\PMlinkescapeword{symmetric} \PMlinkescapeword{symmetry}
\PMlinkescapeword{segments} \PMlinkescapeword{identity}
\begin{center}
\includegraphics{morley}
\end{center}
The scheme of this proof, due to A. Letac,
is to use the sines law to get formulas for the segments
$AR$, $AQ$, $BP$, $BR$, $CQ$, and $CP$,
and then to apply the cosines law to the triangles $ARQ$, $BPR$, and $CQP$,
getting $RQ$, $PR$, and $QP$.

To simplify some formulas, let us denote the angle $\pi/3$, or 60 degrees,
by $\sigma$.
Denote the angles at $A$, $B$, and $C$ by $3a$, $3b$, and $3c$ respectively,
and let $R$ be the circumradius of $ABC$. We have
$BC = 2R \sin(3a)$. Applying the sines law to the triangle $BPC$,
\begin{eqnarray}
BP/\sin(c) &=& BC/\sin(\pi - b - c) = 2R \sin(3a)/\sin(b + c) \\
           &=& 2R\sin(3a)/\sin(\sigma - a)
\end{eqnarray}
so
$$BP = 2R\sin(3a)\sin(c)/\sin(\sigma - a)\;.$$
Combining that with the identity
$$\sin(3a) = 4\sin(a)\sin(\sigma + a)\sin(\sigma -a)$$
we get
$$BP = 8R\sin(a)\sin(c)\sin(\sigma + a)\;.$$
Similarly,
$$BR = 8R\sin(c)\sin(a)\sin(\sigma + c)\;.$$

Using the cosines law now,
$$PR^2 = BP^2 + BR^2 - 2 BP\cdot BR\cos(b)$$
$$= 64 \sin^2(a)\sin^2(c)[
\sin^2(\sigma + a) + \sin^2(\sigma+ c) -
 2\sin(\sigma + a)\sin(\sigma + c)\cos(b)]\;.$$
But we have
$$(\sigma + a) + (\sigma + c) + b = \pi\;.$$
whence the cosines law can be applied to those three angles, getting
$$\sin^2(b) = \sin^2(\sigma + a) + \sin^2(\sigma + c) -
2\sin(\sigma + a)\sin(\sigma + c)\cos(b)$$
whence
$$PR = 8R\sin(a)\sin(b)\sin(c)\;.$$
Since this expression is symmetric in $a$, $b$, and $c$, we deduce
$$PR = RQ = QP$$
as claimed.

\noindent
\textbf{Remarks: }It is not hard to show that the triangles $RYP$, $PZQ$,
and $QXR$ are isoscoles.

By the sines law we have
$$\frac{AR}{\sin b}=\frac{BR}{\sin a}\qquad
\frac{BP}{\sin c}=\frac{CP}{\sin b}\qquad
\frac{CQ}{\sin a}=\frac{AQ}{\sin c}$$
whence
$$AR\cdot BP\cdot CQ = AQ\cdot BR\cdot CP\;.$$
This implies that if we identify the various vertices
with complex numbers, then
$$(P-C)(Q-A)(R-B)=\frac{-1+i\sqrt{3}}{2}(P-B)(Q-C)(R-A)$$
provided that the triangle $ABC$ has positive orientation, i.e.
$$\Re\left(\frac{C-A}{B-A}\right)>0\;.$$

I found Letac's proof at
\PMlinkexternal{cut-the-knot.org}{http://www.cut-the-knot.org/triangle/Morley/index.shtml},
with the reference \emph{Sphinx}, \textbf{9} (1939) 46.
Several shorter and prettier proofs of Morley's theorem can also be
seen at cut-the-knot.
%%%%%
%%%%%
\end{document}
