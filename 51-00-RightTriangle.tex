\documentclass[12pt]{article}
\usepackage{pmmeta}
\pmcanonicalname{RightTriangle}
\pmcreated{2013-10-30 8:57:24}
\pmmodified{2013-10-30 8:57:24}
\pmowner{drini}{3}
\pmmodifier{pahio}{2872}
\pmtitle{right triangle}
\pmrecord{7}{30650}
\pmprivacy{1}
\pmauthor{drini}{2872}
\pmtype{Definition}
\pmcomment{trigger rebuild}
\pmclassification{msc}{51-00}
\pmrelated{EquilateralTriangle}
\pmrelated{Triangle}
\pmrelated{Hypotenuse}
\pmrelated{PythagorasTheorem}
\pmrelated{IsoscelesTriangle}
\pmrelated{GeneralizedPythagoreanTheorem}

\endmetadata

\usepackage{amssymb}
\usepackage{amsmath}
\usepackage{amsfonts}
\usepackage{graphicx}
%%%%\usepackage{xypic}

\begin{document}
A triangle $ABC$ is right when one of its angles is equal to $90^\circ$ (and therefore has two perpendicular sides).

\begin{center}
\includegraphics{righttriangle}
\end{center}

The sides of a right triangle satisfy the Pythagorean theorem.

%%%%%
%%%%%
%%%%%
%%%%%
\end{document}
