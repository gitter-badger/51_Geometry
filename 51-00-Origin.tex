\documentclass[12pt]{article}
\usepackage{pmmeta}
\pmcanonicalname{Origin}
\pmcreated{2013-03-22 15:04:31}
\pmmodified{2013-03-22 15:04:31}
\pmowner{jirka}{4157}
\pmmodifier{jirka}{4157}
\pmtitle{origin}
\pmrecord{7}{36797}
\pmprivacy{1}
\pmauthor{jirka}{4157}
\pmtype{Definition}
\pmcomment{trigger rebuild}
\pmclassification{msc}{51-00}

\endmetadata

% this is the default PlanetMath preamble.  as your knowledge
% of TeX increases, you will probably want to edit this, but
% it should be fine as is for beginners.

% almost certainly you want these
\usepackage{amssymb}
\usepackage{amsmath}
\usepackage{amsfonts}

% used for TeXing text within eps files
%\usepackage{psfrag}
% need this for including graphics (\includegraphics)
%\usepackage{graphicx}
% for neatly defining theorems and propositions
\usepackage{amsthm}
% making logically defined graphics
%%%\usepackage{xypic}

% there are many more packages, add them here as you need them

% define commands here
\begin{document}
In the vector space ${\mathbb{R}}^n$, the word {\em origin} refers to the
zero point, that is the point $(0,\ldots,0)$.  Similarly for ${\mathbb{C}}^n$.  \PMlinkescapetext{Similar} definitions can be made for any vector space.  Often the notation $0$ or $O$ is used for the origin.

In some contexts the choice of ``origin'' can be arbitrary and thus not natural.
For example, if we think of Euclidean space as an affine space or as a Riemannian manifold, it has no natural origin.
Many theorems about local properties of manifolds are stated for values near the origin in some vector space.  This is because any point on the manifold can be the origin in some set of local coordinates.
%%%%%
%%%%%
\end{document}
