\documentclass[12pt]{article}
\usepackage{pmmeta}
\pmcanonicalname{LimitingTriangle}
\pmcreated{2013-03-22 17:06:35}
\pmmodified{2013-03-22 17:06:35}
\pmowner{Wkbj79}{1863}
\pmmodifier{Wkbj79}{1863}
\pmtitle{limiting triangle}
\pmrecord{15}{39408}
\pmprivacy{1}
\pmauthor{Wkbj79}{1863}
\pmtype{Definition}
\pmcomment{trigger rebuild}
\pmclassification{msc}{51M10}
\pmrelated{AreaOfASphericalTriangle}
\pmrelated{IdealTriangle}

\endmetadata

\usepackage{amssymb}
\usepackage{amsmath}
\usepackage{amsfonts}
\usepackage{pstricks}

\usepackage{psfrag}
\usepackage{graphicx}
\usepackage{amsthm}
%%\usepackage{xypic}
\begin{document}
\PMlinkescapeword{connected}
\PMlinkescapeword{entire}
\PMlinkescapeword{equivalent}

In spherical geometry, a \emph{limiting triangle} is a great circle of the sphere that is serving as the model for the geometry.

The motivation for this definition is as follows:  In Euclidean geometry and hyperbolic geometry, if three collinear points are connected, the result is \emph{always} a line segment, which does not contain any area.  In spherical geometry, if the three points are close to each other, this procedure will produce a great arc (the equivalent to a line segment in this geometry).  On the other hand, if the three points are sufficiently spaced from each other, this procedure will yield an entire great circle (the equivalent to a line in this geometry).  For example, imagine that the circle shown below is a great circle of a sphere.  Then connecting the three plotted points yields the entire great circle.

\begin{center}
\begin{pspicture}(-2,-2)(2,2)
\pscircle(0,0){2}
\psdots(0,2)(-1.732,-1)(1.732,-1)
\end{pspicture}
\end{center}

Thus, limiting triangles are geodesic triangles determined by three collinear points that are sufficiently spaced from each other.

\PMlinkescapetext{Strictly} speaking, the resulting figure is not a triangle in spherical geometry; however, it is useful for demonstrating the following facts in spherical geometry:

\begin{itemize}
\item $540^{\circ}$ is the least upper bound of the angle sum of a triangle;
\item $2\pi$ is the least upper bound of the area of a \PMlinkescapetext{triangle}.
\end{itemize}
%%%%%
%%%%%
\end{document}
