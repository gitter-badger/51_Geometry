\documentclass[12pt]{article}
\usepackage{pmmeta}
\pmcanonicalname{ConstructingNearlinearSpacesFromExistingOnes}
\pmcreated{2013-03-22 19:14:53}
\pmmodified{2013-03-22 19:14:53}
\pmowner{CWoo}{3771}
\pmmodifier{CWoo}{3771}
\pmtitle{constructing near-linear spaces from existing ones}
\pmrecord{10}{42173}
\pmprivacy{1}
\pmauthor{CWoo}{3771}
\pmtype{Definition}
\pmcomment{trigger rebuild}
\pmclassification{msc}{51A45}
\pmclassification{msc}{51A05}
\pmclassification{msc}{05C65}
\pmdefines{restriction}
\pmdefines{subspace}
\pmdefines{subplane}
\pmdefines{subline}
\pmdefines{hyperplane}

\usepackage{amssymb,amscd}
\usepackage{amsmath}
\usepackage{amsfonts}
\usepackage{mathrsfs}

% used for TeXing text within eps files
%\usepackage{psfrag}
% need this for including graphics (\includegraphics)
%\usepackage{graphicx}
% for neatly defining theorems and propositions
\usepackage{amsthm}
% making logically defined graphics
%%\usepackage{xypic}
\usepackage{pst-plot}

% define commands here
\newcommand*{\abs}[1]{\left\lvert #1\right\rvert}
\newtheorem{prop}{Proposition}
\newtheorem{thm}{Theorem}
\newtheorem{ex}{Example}
\newcommand{\real}{\mathbb{R}}
\newcommand{\pdiff}[2]{\frac{\partial #1}{\partial #2}}
\newcommand{\mpdiff}[3]{\frac{\partial^#1 #2}{\partial #3^#1}}
\begin{document}
One can construct new near-linear spaces from old.  We name a few of the constructions in this entry.

Let $\mathscr{S}=(\mathcal{P},\mathcal{L})$ be a near-linear space.

\textbf{Restrictions}.  A pair $\mathscr{S}':=(\mathcal{P}',\mathcal{L}')$, where $\mathcal{P}'\subseteq \mathcal{P}$, and $\mathcal{L}':= \lbrace \mathcal{P}'\cap \ell \mid \ell \in \mathcal{L}\rbrace$, is called a \emph{restriction} of $\mathscr{S}$.

It is easy to verify that a restriction of a near-linear space is near-linear, and a restriction of a linear space is linear.

For example, let $\mathscr{S}$ be a near-linear space given by the diagram below:
\begin{center}
\begin{pspicture}(-1,0)(1,1)
\psset{unit=20pt}
\psdots[linecolor=blue,dotsize=5pt](0,1)
\psdots[linecolor=blue,dotsize=5pt](-1,0)
\psdots[linecolor=blue,dotsize=5pt](0,0)
\psdots[linecolor=blue,dotsize=5pt](1,0)
\psline(-1,0)(1,0)
\psline(0,1)(1,0)
\psline(0,1)(-1,0)
\end{pspicture}
\end{center}
Then $\mathscr{S}$ has 7 restrictions, no two of which are isomorphic.  Besides the empty space $\varnothing$ and $\mathscr{S}$ itself, the following diagram shows the other five, with increasing number of lines:
\begin{center}
\begin{pspicture}(-1,0)(1,1)
\psset{unit=20pt}
\psdots[linecolor=blue,dotsize=5pt](0,0)
\end{pspicture}
\begin{pspicture}(-1,0)(1,1)
\psset{unit=20pt}
\psdots[linecolor=blue,dotsize=5pt](0,1)
\psdots[linecolor=blue,dotsize=5pt](0,0)
\end{pspicture}
\begin{pspicture}(-1,0)(1,1)
\psset{unit=20pt}
\psdots[linecolor=blue,dotsize=5pt](0,1)
\psdots[linecolor=blue,dotsize=5pt](-1,0)
\psline(0,1)(-1,0)
\end{pspicture}
\begin{pspicture}(-1,0)(1,1)
\psset{unit=20pt}
\psdots[linecolor=blue,dotsize=5pt](-1,0)
\psdots[linecolor=blue,dotsize=5pt](0,0)
\psdots[linecolor=blue,dotsize=5pt](1,0)
\psline(-1,0)(1,0)
\end{pspicture}
\begin{pspicture}(-1,0)(1,1)
\psset{unit=20pt}
\psdots[linecolor=blue,dotsize=5pt](0,1)
\psdots[linecolor=blue,dotsize=5pt](-1,0)
\psdots[linecolor=blue,dotsize=5pt](0,0)
\psline(-1,0)(0,0)
\psline(0,1)(-1,0)
\end{pspicture}
\end{center}

Here is another example.  The (affine) Euclidean plane $\mathbb{E}^2$ is a linear space.  The rational plane, formed by all points with rational coordinates, and lines with rational coefficients, is a restriction of $\mathbb{E}^2$.  More generally, let $A(V)$ be an affine space over some vector space over some field $K$.  If $k$ is a subfield of $K$, then $A(V')$ is a restriction of $A(V)$, where $V'$ is the restriction of $V$ to $k$, consisting of vectors in $V$ whose components are elements of $k$.

In the study of projective planes, restrictions of finite projective planes play an important role.  A restriction of a projective plane is called a \emph{subplane} if the restriction itself is a projective plane.  \emph{Sublines} are defined similarly.  Bruck showed that if a projective plane has order $n$ and a subplane has order $m$, then $n=m^2$ or $n\ge m^2+m$.

\textbf{Subspaces}.  Call a subset $\mathcal{P}'$ of $\mathcal{P}$ \emph{linear} if for any distinct $P, Q\in \mathcal{P}'$, $PQ\in \mathcal{L}$ implies $PQ \subseteq \mathcal{P}'$.  In other words, if two points on a line are in $\mathcal{P}'$, all of the points on the line are in $\mathcal{P}'$.

A \emph{subspace} of $\mathscr{S}$ is a pair $\mathscr{S}':=(\mathcal{P}',\mathcal{L}')$, where $\mathcal{P}'\subseteq \mathcal{P}$ is linear, and $\mathcal{L}':= \lbrace \ell \in \mathcal{L} \mid \ell\subseteq \mathcal{P}'\rbrace$.

Subspace are restrictions.  Thus, subspaces of near-linear and linear spaces are near-linear and linear, respectively.

In the first example above, all of the restrictions of $\mathscr{S}$, except the last one with two lines, are subspaces of $\mathscr{S}$.  The last restriction is not a subspace because one of its lines is not a line of $\mathscr{S}$.

In the second example above, the affine space $A(V')$, while a restriction of $A(V)$, is not a subspace of $A(V)$.

The subspaces of affine (projective) spaces coincide with the usual definitions of subspaces of affine (projective) spaces.  If a restriction $A'$ of an affine space $A$ is a subspace of $A$ with the same dimension, then $A'=A$.

In the set $\mathbf{S}$ of all subspaces of $\mathscr{S}$, we write $\mathscr{U}\le \mathscr{V}$ if $\mathscr{U}$ is a subspace of $\mathscr{V}$.  Then $\le$ is a partial order on $\mathbf{S}$.  A maximal element in $\mathbf{S}$ is called a \emph{hyperplane}.

\textbf{Dual Spaces}.  Another way of constructing a new space out of the existing one is that of a dual space.  Let $\mathscr{S}=(\mathcal{P},\mathcal{L})$ be a near-linear space.  The dual space $\mathscr{S}^*$ of $\mathscr{S}$ is a pair $(\mathcal{P}^*,\mathcal{L}^*)$ such that
\begin{itemize}
\item $\mathcal{P}^*=\mathcal{L}$, and
\item $\mathcal{L}^*=\lbrace P^* \mid P\in \mathcal{P}\mbox{ and }|P^*|\ge 2\rbrace$, where $P^*$ is the set of all lines passing through $P$.
\end{itemize}
The incidence relation on $\mathscr{S}^*$ is the usual set membership $\in$.  One easily checks that $\mathscr{S}^*$ is a near-linear space.

Below are diagrams of the example $\mathscr{S}$ above, now with labels, and its dual space $\mathscr{S}^*$ (on the right):
\begin{center}
\begin{pspicture}(-1,0)(1,2)
\psset{unit=2cm}
\psdots[linecolor=blue,dotsize=5pt](0,1)
\psdots[linecolor=blue,dotsize=5pt](-1,0)
\psdots[linecolor=blue,dotsize=5pt](0,0)
\psdots[linecolor=blue,dotsize=5pt](1,0)
\psline(-1,0)(1,0)
\psline(0,1)(1,0)
\psline(0,1)(-1,0)
\rput[l](0.1,1){$P$}
\rput[r](-1.1,0){$Q$}
\rput[u](0,0.15){$R$}
\rput[l](1.1,0){$S$}
\rput[r](-0.65,0.5){$\ell_1$}
\rput[l](0.65,0.5){$\ell_2$}
\rput[d](-0.5,-0.15){$\ell_3$}
\end{pspicture}
\hspace{5cm}
\begin{pspicture}(-1,0)(1,2)
\psset{unit=2cm}
\psdots[linecolor=blue,dotsize=5pt](0,1)
\psdots[linecolor=blue,dotsize=5pt](-1,0)
%\psdots[linecolor=blue,dotsize=5pt](0,0)
\psdots[linecolor=blue,dotsize=5pt](1,0)
\psline(-1,0)(1,0)
\psline(0,1)(1,0)
\psline(0,1)(-1,0)
\rput[l](0.1,1){$\ell_1$}
\rput[r](-1.1,0){$\ell_2$}
%\rput[u](0,0.15){$$}
\rput[l](1.1,0){$\ell_3$}
\rput[r](-0.65,0.5){$P^*$}
\rput[l](0.65,0.5){$Q^*$}
\rput[d](0,-0.15){$S^*$}
\end{pspicture}
\end{center}
Note that the dual space $\mathscr{S}^{**}$ of $\mathscr{S}^*$ is isomorphic to $\mathscr{S}^*$.  This example shows that, in general, the double dual of a near-linear space is not isomorphic to itself.

\begin{thebibliography}{7}
\bibitem{LB} L. M. Batten, {\it Combinatorics of Finite Geometries}, 2nd edition, Cambridge University Press (1997)
\end{thebibliography}
%%%%%
%%%%%
\end{document}
