\documentclass[12pt]{article}
\usepackage{pmmeta}
\pmcanonicalname{AlternativeProofOfNecessityDirectionOfEquivalentConditionsForTriangleshyperbolicAndSpherical}
\pmcreated{2013-03-22 17:12:55}
\pmmodified{2013-03-22 17:12:55}
\pmowner{Wkbj79}{1863}
\pmmodifier{Wkbj79}{1863}
\pmtitle{alternative proof of necessity direction of equivalent conditions for triangles (hyperbolic and spherical)}
\pmrecord{5}{39539}
\pmprivacy{1}
\pmauthor{Wkbj79}{1863}
\pmtype{Proof}
\pmcomment{trigger rebuild}
\pmclassification{msc}{51-00}

\endmetadata

\usepackage{amssymb}
\usepackage{amsmath}
\usepackage{amsfonts}
\usepackage{pstricks}
\usepackage{psfrag}
\usepackage{graphicx}
\usepackage{amsthm}
%%\usepackage{xypic}

\begin{document}
\PMlinkescapeword{equiangular}
\PMlinkescapeword{equilateral}

The following is a proof that, in hyperbolic geometry and spherical geometry, an equiangular triangle $\triangle ABC$ is automatically \PMlinkname{equilateral}{EquilateralTriangle} (and therefore \PMlinkname{regular}{RegularTriangle}).  It better \PMlinkescapetext{parallels} the proof of sufficiency supplied in the entry equivalent conditions for triangles and is slightly shorter than the proof of necessity supplied in the same entry.

\begin{proof}

Assume that $\triangle ABC$ is equiangular.

\begin{center}
\begin{pspicture}(-0.2,-0.2)(5.2,5.2)
\pspolygon(0,0)(5,0)(2.5,4.33)
\rput[b](2.5,4.5){$A$}
\rput[a](0,-0.2){$B$}
\rput[a](5,-0.2){$C$}
\psarc(0,0){0.5}{0}{60}
\psarc(5,0){0.5}{120}{180}
\psarc(2.5,4.33){0.5}{240}{300}
\end{pspicture}
\end{center}

Since $\angle A \cong \angle B \cong \angle C$, AAA yields that $\triangle ABC \cong \triangle BCA$.  By CPCTC, $\overline{AB} \cong \overline{AC} \cong \overline{BC}$.  Hence, $\triangle ABC$ is equilateral.

\end{proof}
%%%%%
%%%%%
\end{document}
