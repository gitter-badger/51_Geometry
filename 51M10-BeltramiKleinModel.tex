\documentclass[12pt]{article}
\usepackage{pmmeta}
\pmcanonicalname{BeltramiKleinModel}
\pmcreated{2013-03-22 17:06:37}
\pmmodified{2013-03-22 17:06:37}
\pmowner{Wkbj79}{1863}
\pmmodifier{Wkbj79}{1863}
\pmtitle{Beltrami-Klein model}
\pmrecord{18}{39409}
\pmprivacy{1}
\pmauthor{Wkbj79}{1863}
\pmtype{Definition}
\pmcomment{trigger rebuild}
\pmclassification{msc}{51M10}
\pmclassification{msc}{51-00}
\pmsynonym{Klein-Beltrami model}{BeltramiKleinModel}
\pmsynonym{Klein model}{BeltramiKleinModel}
\pmrelated{ConvertingBetweenTheBeltramiKleinModelAndThePoincareDiscModel}
\pmdefines{pole}

\usepackage{amssymb}
\usepackage{amsmath}
\usepackage{amsfonts}
\usepackage{pstricks}
\usepackage{amsthm}
\begin{document}
The \emph{Beltrami-Klein model} for $\mathbb{H}^2$ is the disc $\{(x,y) \in \mathbb{R}^2 : x^2+y^2<1 \}$ in which a point is similar to the Euclidean point and a line is defined to be a chord (excluding its endpoints) of the (circular) boundary.

\begin{center}
\begin{pspicture}(-2,-2)(2,2)
\pscircle[linestyle=dashed](0,0){2}
\psline{o-o}(-2,0)(1.414,1.414)
\end{pspicture}
\end{center}

The Beltrami-Klein model has the advantage that lines in the model resemble Euclidean lines; however, it has the drawback that it is not angle preserving.  That is, the Euclidean \PMlinkescapetext{measure} of an angle within the model is not necessarily the angle measure in hyperbolic geometry.

Some points outside of the Beltrami-Klein model are important for constructions within the model.  The following is an example of such:

Let $\ell$ be a line in the Beltrami-Klein model that is not a diameter of the circle.  The \emph{pole} of $\ell$ is the intersection of the Euclidean lines that are \PMlinkname{tangent}{TangentLine} to the circle at the endpoints of $\ell$.

\begin{center}
\begin{pspicture}(-3,-2)(3,5)
\pscircle[linestyle=dashed](0,0){2}
\psline{<->}(-2,-2)(-2,5)
\psline{<->}(-2.7172,5)(2.828,0.18)
\psline{o-o}(-2,0)(1.414,1.414)
\rput[a](0,0.7){$\ell$}
\psdots(-2,4.4)
\rput[l](-1.9,4.4){$P(\ell)$}
\rput[b](-2,-2){.}
\rput[b](-2.7172,5){.}
\rput[b](2.828,0.18){.}
\end{pspicture}
\end{center}

Poles are important for the following reason:  Given a line $\ell$ that is not a diameter of the Beltrami-Klein model, one constructs a line perpendicular to $\ell$ by considering Euclidean lines passing through $P(\ell)$.  Thus, given two disjointly parallel lines $\ell$ and $m$ that are not diameters of the Beltrami-Klein model, one constructs their common perpendicular by connecting their poles.

\begin{center}
\begin{pspicture}(-3,-3)(3,5.1)
\pscircle[linestyle=dashed](0,0){2}
\psline{<->}(-2,-3)(-2,5)
\psline{<->}(-2.7172,5)(2.828,0.18)
\psline{<->}(-2.4,-0.7)(0.4,-2.8)
\psline{<->}(-0.4,-2.8)(2.4,-0.7)
\psline{o-o}(-2,0)(1.414,1.414)
\rput[a](0,0.7){$\ell$}
\psline{o-o}(-1.2,-1.6)(1.2,-1.6)
\rput[b](0.8,-1.6){$m$}
\psdots(-2,4.4)(0,-2.5)
\rput[l](-1.9,4.4){$P(\ell)$}
\rput[l](0.2,-2.5){$P(m)$}
\rput[b](-2,-3){.}
\rput[b](-2.7172,5){.}
\rput[b](2.828,0.18){.}
\psline{<->}(-2.2,5.09)(0.1,-2.845)
\rput[b](-1,1.4){$n$}
\end{pspicture}
\end{center}

In the above picture, $n$ is the common perpendicular of $\ell$ and $m$.
%%%%%
%%%%%
\end{document}
