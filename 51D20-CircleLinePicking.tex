\documentclass[12pt]{article}
\usepackage{pmmeta}
\pmcanonicalname{CircleLinePicking}
\pmcreated{2013-03-22 16:26:05}
\pmmodified{2013-03-22 16:26:05}
\pmowner{PrimeFan}{13766}
\pmmodifier{PrimeFan}{13766}
\pmtitle{circle line picking}
\pmrecord{5}{38587}
\pmprivacy{1}
\pmauthor{PrimeFan}{13766}
\pmtype{Definition}
\pmcomment{trigger rebuild}
\pmclassification{msc}{51D20}
\pmrelated{BertrandsProblem}

\endmetadata

% this is the default PlanetMath preamble.  as your knowledge
% of TeX increases, you will probably want to edit this, but
% it should be fine as is for beginners.

% almost certainly you want these
\usepackage{amssymb}
\usepackage{amsmath}
\usepackage{amsfonts}

% used for TeXing text within eps files
%\usepackage{psfrag}
% need this for including graphics (\includegraphics)
%\usepackage{graphicx}
% for neatly defining theorems and propositions
%\usepackage{amsthm}
% making logically defined graphics
%%%\usepackage{xypic}

% there are many more packages, add them here as you need them

% define commands here

\begin{document}
{\em Circle line picking} is the activity of choosing two points on the circumference a unit circle and calculating probabilities and averages about the chord formed by connecting those two points.

The length of the line segment between the two points is $\sqrt{2 - 2\cos\theta}$, where $\theta$ is the angle of the central vertex of a triangle formed by connecting the two points to the center. On average, this works out to $\frac{4}{\pi}$.

%%%%%
%%%%%
\end{document}
