\documentclass[12pt]{article}
\usepackage{pmmeta}
\pmcanonicalname{ComplementaryAngles}
\pmcreated{2013-03-22 17:18:07}
\pmmodified{2013-03-22 17:18:07}
\pmowner{pahio}{2872}
\pmmodifier{pahio}{2872}
\pmtitle{complementary angles}
\pmrecord{7}{39649}
\pmprivacy{1}
\pmauthor{pahio}{2872}
\pmtype{Definition}
\pmcomment{trigger rebuild}
\pmclassification{msc}{51G05}
\pmclassification{msc}{51F20}
\pmsynonym{complement angles}{ComplementaryAngles}
\pmsynonym{complementary}{ComplementaryAngles}
\pmrelated{ExplementaryAngle}
\pmrelated{SupplementaryAngles}
\pmrelated{GoniometricFormulae}
\pmrelated{ConvexAngle}
\pmrelated{Explementary}

\endmetadata

% this is the default PlanetMath preamble.  as your knowledge
% of TeX increases, you will probably want to edit this, but
% it should be fine as is for beginners.

% almost certainly you want these
\usepackage{amssymb}
\usepackage{amsmath}
\usepackage{amsfonts}

% used for TeXing text within eps files
%\usepackage{psfrag}
% need this for including graphics (\includegraphics)
%\usepackage{graphicx}
% for neatly defining theorems and propositions
 \usepackage{amsthm}
% making logically defined graphics
%%%\usepackage{xypic}

% there are many more packages, add them here as you need them

% define commands here

\theoremstyle{definition}
\newtheorem*{thmplain}{Theorem}

\begin{document}
Two angles are called {\em complementary angles} of each other, if their sum is the right angle $\displaystyle\frac{\pi}{2}$, i.e. $90^\circ$.  

For example, the acute angles of a right triangle are complement angles of each other, since the angle sum in any triangle is $180^\circ$.

The sine of an angle is equal to the cosine of the complement angle, and vice versa. 

The tangent of an angle equals to the cotangent of the complement angle, and vice versa (provided that no one of the angles is a multiple of the straight angle).
%%%%%
%%%%%
\end{document}
