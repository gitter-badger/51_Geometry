\documentclass[12pt]{article}
\usepackage{pmmeta}
\pmcanonicalname{Polyhedron}
\pmcreated{2013-03-22 12:14:43}
\pmmodified{2013-03-22 12:14:43}
\pmowner{Mathprof}{13753}
\pmmodifier{Mathprof}{13753}
\pmtitle{polyhedron}
\pmrecord{24}{31627}
\pmprivacy{1}
\pmauthor{Mathprof}{13753}
\pmtype{Definition}
\pmcomment{trigger rebuild}
\pmclassification{msc}{51M20}
\pmclassification{msc}{57Q05}
\pmrelated{RegularPolygon}
\pmrelated{Polytope}
\pmrelated{Diagonal}
\pmrelated{CubicallyThinHomotopy}
\pmdefines{vertex}
\pmdefines{corner point}
\pmdefines{finite polyhedron}
\pmdefines{locally finite}
\pmdefines{polyhedra}
\pmdefines{bounded polyhedron}
\pmdefines{normal polyhedron}
\pmdefines{regular polyhedron}
\pmdefines{Euler polyhedron}
\pmdefines{convex polyhedron}
\pmdefines{simple polyhedron}

%\usepackage{graphicx}
%%%%\usepackage{xypic} 
\usepackage{bbm}
\newcommand{\Z}{\mathbbmss{Z}}
\newcommand{\C}{\mathbbmss{C}}
\newcommand{\R}{\mathbbmss{R}}
\newcommand{\Q}{\mathbbmss{Q}}
\newcommand{\mathbb}[1]{\mathbbmss{#1}}
\begin{document}
At least four  definitions of a polyhedron are used. 
\subsection*{Combinatorics}
In combinatorics a \emph{polyhedron} is the solution set of a finite system
of linear inequalities. The solution set is in ${\mathbb{R}}^n$ for integer
$n$. Hence, it is a convex set. Each extreme point of such a polyhedron is also called a \emph{vertex} (or \emph{corner point}) of the polyhedron. A solution 
set could be empty. If the solution set is bounded (that is, is contained in
some sphere) the polyhedron is said to be \emph{bounded}.


\subsection*{Elementary Geometry}

In elementary geometry a polyhedron is a  solid bounded by a finite number of plane faces, 
each of which is a polygon.  This of course is not a precise definition as it
relies on the undefined term ``solid''. Also, this definition allows a polyhedron
to be non-convex.

\subsection*{Careful Treatments of Geometry}
In treatments of geometry that are carefully done a definition due to Lennes is 
sometimes used \cite{LE}. The intent is to rule out certain objects that one does not want
to consider and to simplify the theory of dissection.
A polyhedron is a set of points consisting of a finite set of
triangles $T$, not all coplanar,   and their interiors such that
\begin{itemize}
\item[(i)] every side of a triangle is common to an even number of triangles of the
set, and
\item[(ii)] there is no subset $T'$ of $T$ such that (i) is true of a proper subset
of $T'$. 
\end{itemize}
Notice that condition (ii) excludes, for example, two cubes that are disjoint. But two
tetrahedra having a common edge are allowed. The faces of the polyhedron are the insides
of the triangles. Note that the condition that the faces be triangles 
is not important, since a polygon an be dissected into triangles. 
Also note since a triangle meets an \emph{even} number of other triangles,
it is possible to meet 4,6 or any other even number of triangles. So for example,
a configuration of 6 tetrahedra all sharing a common edge is allowed. 

By dissections of the triangles one can create a set of triangles in which
no face intersects another face, edge or vertex. If this done the
polyhedron is said to be \emph{\PMlinkescapetext{normal}}.
 
A \emph{convex polyhedron} is one such that all its inside points lie on one side of
each of the planes of its faces. 

An \emph{Euler polyhedron}  $P$ is a set of points consisting of a finite set
of \emph{polygons}, not all coplanar, and their insides such that
\begin{itemize}
\item[(i)] each edge is common to just \emph{two} polygons, 
\item[(ii)] there is a way using edges of $P$ from a given vertex to each vertex, and
\item[(iii)] any simple polygon $p$ made up of edges of $P$, divides the polygons
of $P$ into two sets  $A$ and $B$ such that any way, whose points are on $P$
from any point inside a polygon of $A$ to a point inside a polygon of $B$,
meets $p$. 
\end{itemize}

A \emph{regular polyhedron} is a convex Euler polyhedron whose faces are congruent 
regular polygons and whose dihedral angles are congruent.

It is a theorem, proved \PMlinkname{here}{ClassificationOfPlatonicSolids}, that for a regular polyhedron, the number of polygons with the same
vertex is the same for each vertex and that there are 5 types of regular polyhedra.


Notice that a cone, and a cylinder are not polyhedra since they have ``faces'' that are not polygons.

A \emph{simple polyhedron} is one that is homeomorphic to a sphere. For such a polyhedron
one has $V-E+F = 2$, where $V$ is the number of vertices, $E$ is the number of edges
and $F$ is the number of faces. This is called Euler's formula. 

\subsection*{Algebraic Topology}
In algebraic topology another definition is used:

If $K$ is a simplicial complex in ${\mathbb{R}}^n$, then $|K|$ denotes the union of the elements of 
$K$, with the subspace topology induced by the topology of ${\mathbb{R}}^n$.
$|K|$ is called a \emph{polyhedron}. If $K$ is a finite complex, then
$|K|$ is called a \emph{finite polyhedron}.

It should be noted that we allow the complex to have an infinite number of
simplexes. As a result, spaces such as $\mathbb{R}$ and
${\mathbb{R}}^n$ are polyhedra.

Some authors require the simplicial complex to be \emph{locally finite}.
That is, given $x \in \sigma \in K$ there is a neighborhood of $x$ that meets only finitely many $\tau \in K$.

\begin{thebibliography}{99}
\bibitem{FO}
Henry George Forder, \emph{The Foundations of Euclidean Geometry}, Dover Publications, New York , 1958.
\bibitem{LE} N.J. Lennes, \emph{On the simple finite polygon and polyhedron}, Amer. J. Math. \textbf{33}, (1911), p. 37
\end{thebibliography}

%%%%%
%%%%%
%%%%%
\end{document}
