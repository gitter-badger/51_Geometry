\documentclass[12pt]{article}
\usepackage{pmmeta}
\pmcanonicalname{Heptahedron}
\pmcreated{2013-03-22 14:26:24}
\pmmodified{2013-03-22 14:26:24}
\pmowner{CWoo}{3771}
\pmmodifier{CWoo}{3771}
\pmtitle{heptahedron}
\pmrecord{9}{35953}
\pmprivacy{1}
\pmauthor{CWoo}{3771}
\pmtype{Definition}
\pmcomment{trigger rebuild}
\pmclassification{msc}{51E99}
\pmsynonym{septahedron}{Heptahedron}
%\pmkeywords{polyhedron}
\pmrelated{Grafix}

\endmetadata

% this is the default PlanetMath preamble.  as your knowledge
% of TeX increases, you will probably want to edit this, but
% it should be fine as is for beginners.

% almost certainly you want these
\usepackage{amssymb}
\usepackage{amsmath}
\usepackage{amsfonts}

% used for TeXing text within eps files
%\usepackage{psfrag}
\usepackage{pstricks}
% need this for including graphics (\includegraphics)
%\usepackage{graphicx}
% for neatly defining theorems and propositions
%\usepackage{amsthm}
% making logically defined graphics
%%%\usepackage{xypic}

% there are many more packages, add them here as you need them

% define commands here
\begin{document}
A \emph{heptahedron} is a polyhedron with seven faces.

According to the Steinitz classification of f-vectors of 3-polytopes and the Euler polyhedron formula together imply that a convex heptahedron will have one of the followng f-vectors:

\begin{itemize}
\item $(6,11,7)$
\begin{center}
\begin{pspicture}(-7,-1)(7,1.3)
\psset{unit=0.8cm}
\pspolygon[linestyle=dashed, dash=2pt 2pt](-5,-1)(-3,-1)(-2,0)(-4,0)
\pspolygon(-4.3,1)(-3.1,1.2)(-3,-1)
\pspolygon(-5,-1)(-3,-1)(-4.3,1)
\pspolygon(-3,-1)(-2,0)(-3.1,1.2)
\pspolygon[linestyle=dashed, dash=2pt 2pt](-4.3,1)(-3.1,1.2)(-4,0)
\end{pspicture}
\end{center}
\item $(7,12,7)$
\begin{center}
\begin{pspicture}(-8,-1)(7,1.4)
\psset{unit=0.8cm}
\pspolygon[linestyle=dashed, dash=2pt 2pt](-5,-1)(-3.6,-1)(-2.6,-0.5)(-3.6,0)(-5,0)(-6,-0.5)
\pspolygon(-4.3,1.4)(-5,-1)(-6,-0.5)
\pspolygon(-4.3,1.4)(-2.6,-0.5)(-3.6,-1)
\psline(-5,-1)(-3.6,-1)
\pspolygon[linestyle=dashed, dash=2pt 2pt](-4.3,1.4)(-5,0)(-3.6,0)
\end{pspicture}
\end{center}
\item $(8,13,7)$
\begin{center}
\begin{pspicture}(-8,-1)(7,1.4)
\psset{unit=0.8cm}
\pspolygon[linestyle=dashed, dash=2pt 2pt](-5,-1)(-4,-1)(-2.5,-0.5)(-4,0)(-5,0)(-6.5,-0.5)
\pspolygon(-6,1.2)(-3,1.2)(-4,-1)(-5,-1)
\pspolygon(-6,1.2)(-6.5,-0.5)(-5,-1)
\pspolygon(-3,1.2)(-4,-1)(-2.5,-0.5)
\pspolygon[linestyle=dashed, dash=2pt 2pt](-6,1.2)(-3,1.2)(-4,0)(-5,0)
\end{pspicture}
\end{center}
\item $(9,14,7)$
\begin{center}
\begin{pspicture}(-8,-1)(7,1.4)
\psset{unit=0.8cm}
\pspolygon[linestyle=dashed, dash=2pt 2pt](-6,-0.5)(-4,-0.5)(-4,1.5)(-6,1.5)
\pspolygon[linestyle=dashed, dash=2pt 2pt](-4,-0.5)(-4,1.5)(-3,1)(-3,-1)
\pspolygon(-5,-1)(-3,-1)(-3,1)(-4.5,1)(-5,0.5)
\pspolygon(-4.5,1)(-5,0.5)(-6,1.5)
\pspolygon(-4.5,1)(-6,1.5)(-4,1.5)(-3,1)
\pspolygon(-5,0.5)(-6,1.5)(-6,-0.5)(-5,-1)
\end{pspicture}
\end{center}
\item $(10,15,7)$
\begin{center}
\begin{pspicture}(-8,-1)(7,1.4)
\psset{unit=0.8cm}
\pspolygon[linestyle=dashed, dash=2pt 2pt](-5,-1)(-3,-1)(-2,-0.5)(-4,0)(-6,-0.5)
\pspolygon(-5,0.5)(-3,0.5)(-2,1)(-4,1.5)(-6,1)
\psline(-5,-1)(-5,0.5)
\psline(-3,-1)(-3,0.5)
\psline(-2,-0.5)(-2,1)
\psline(-6,-0.5)(-5,-1)(-3,-1)(-2,-0.5)
\psline[linestyle=dashed, dash=2pt 2pt](-4,0)(-4,1.5)
\psline(-6,-0.5)(-6,1)
\end{pspicture}
\end{center}
\end{itemize}
%%%%%
%%%%%
\end{document}
