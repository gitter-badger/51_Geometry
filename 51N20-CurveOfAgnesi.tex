\documentclass[12pt]{article}
\usepackage{pmmeta}
\pmcanonicalname{CurveOfAgnesi}
\pmcreated{2013-03-22 16:40:56}
\pmmodified{2013-03-22 16:40:56}
\pmowner{CompositeFan}{12809}
\pmmodifier{CompositeFan}{12809}
\pmtitle{curve of Agnesi}
\pmrecord{5}{38892}
\pmprivacy{1}
\pmauthor{CompositeFan}{12809}
\pmtype{Definition}
\pmcomment{trigger rebuild}
\pmclassification{msc}{51N20}
\pmsynonym{Agnesi's curve}{CurveOfAgnesi}
\pmsynonym{witch of Agnesi}{CurveOfAgnesi}
\pmsynonym{Agnesi's witch}{CurveOfAgnesi}
\pmsynonym{averisera}{CurveOfAgnesi}
\pmsynonym{avversiera}{CurveOfAgnesi}
\pmsynonym{cubique d'Agnesi}{CurveOfAgnesi}
\pmsynonym{agn\'esienne}{CurveOfAgnesi}
\pmsynonym{agnesienne}{CurveOfAgnesi}
\pmrelated{AsymptoteOfLamesCubic}

\endmetadata

% this is the default PlanetMath preamble.  as your knowledge
% of TeX increases, you will probably want to edit this, but
% it should be fine as is for beginners.

% almost certainly you want these
\usepackage{amssymb}
\usepackage{amsmath}
\usepackage{amsfonts}

% used for TeXing text within eps files
%\usepackage{psfrag}

% need this for including graphics (\includegraphics)
\usepackage{graphicx}

% for neatly defining theorems and propositions
%\usepackage{amsthm}
% making logically defined graphics
%%%\usepackage{xypic}

% there are many more packages, add them here as you need them

% define commands here

\begin{document}
Given a real constant $c$, the {\em curve of Agnesi} (often called {\em witch of Agnesi} in English) is the result of plotting the equation $$y = \frac{8c}{4c^2 + x^2}$$ in the Cartesian plane. If we set $c = \frac{1}{2}$, the equation simplifies to $y = \frac{1}{1 + x^2}$. Another way of drawing the curve employs a circle of radius $c$.

In the following diagram, the associated circle is shown in light gray.

\begin{center}
\includegraphics{AgnesiCurve}
\end{center}

(This diagram was made with Grapher 1.1 for Mac OS X).

This curve was first studied by Pierre de Fermat, but Maria Gaetana Agnesi later studied it in greater detail and mentioned it in her book {\it Instituzioni Analitiche}.
%%%%%
%%%%%
\end{document}
