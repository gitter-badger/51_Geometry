\documentclass[12pt]{article}
\usepackage{pmmeta}
\pmcanonicalname{NinepointCircle}
\pmcreated{2013-03-22 13:11:20}
\pmmodified{2013-03-22 13:11:20}
\pmowner{mathwizard}{128}
\pmmodifier{mathwizard}{128}
\pmtitle{nine-point circle}
\pmrecord{6}{33641}
\pmprivacy{1}
\pmauthor{mathwizard}{128}
\pmtype{Definition}
\pmcomment{trigger rebuild}
\pmclassification{msc}{51-00}
\pmsynonym{Euler circle}{NinepointCircle}
\pmsynonym{Feuerbach circle}{NinepointCircle}
\pmsynonym{nine point circle}{NinepointCircle}

\endmetadata

% this is the default PlanetMath preamble.  as your knowledge
% of TeX increases, you will probably want to edit this, but
% it should be fine as is for beginners.

% almost certainly you want these
\usepackage{amssymb}
\usepackage{amsmath}
\usepackage{amsfonts}

% used for TeXing text within eps files
%\usepackage{psfrag}
% need this for including graphics (\includegraphics)
\usepackage{graphicx}
% for neatly defining theorems and propositions
%\usepackage{amsthm}
% making logically defined graphics
%%%\usepackage{xypic}

% there are many more packages, add them here as you need them

% define commands here
\begin{document}
\PMlinkescapeword{name}
The \textbf{nine point circle} also known as the \textbf{Euler's circle} or the \textbf{Feuerbach circle} is the circle that passes through the feet of perpendiculars from the vertices $A, B$ and $C$ of a triangle $\triangle ABC.$
\begin{center}
\includegraphics{feuerbachkreis.eps}
\end{center}

Some of the properties of this circle are:

\textbf{Property 1 : }
This circle also passes through the midpoints of the sides $AB, BC$ and $CA$ of $\triangle ABC.$ This was shown by Euler.

\textbf{Property 2 : }
Feuerbach showed that this circle also passes through the midpoints of the line segments $AH, BH$ and $CH$ which are drawn from the vertices of $\triangle ABC$ to its orthocenter $H.$

These three triples of points make nine in all, giving the circle its name.

\textbf{Property 3 : }
The radius of the nine-point cirlce is $R/2,$ where $R$ is the circumradius (radius of the circumcircle).

\textbf{Property 4 : }
The center of the nine-point circle is the midpoint of the line segment joining the orthocenter and the circumcenter, and hence lies on the Euler line.

\textbf{Property 5 : }
All triangles inscribed in a given circle and having the same orthocenter, have the same nine-point circle.
%%%%%
%%%%%
\end{document}
