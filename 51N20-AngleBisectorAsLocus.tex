\documentclass[12pt]{article}
\usepackage{pmmeta}
\pmcanonicalname{AngleBisectorAsLocus}
\pmcreated{2013-03-22 17:10:41}
\pmmodified{2013-03-22 17:10:41}
\pmowner{pahio}{2872}
\pmmodifier{pahio}{2872}
\pmtitle{angle bisector as locus}
\pmrecord{11}{39492}
\pmprivacy{1}
\pmauthor{pahio}{2872}
\pmtype{Definition}
\pmcomment{trigger rebuild}
\pmclassification{msc}{51N20}
\pmrelated{DistanceFromPointToALine}
\pmrelated{ConverseOfIsoscelesTriangleTheorem}
\pmrelated{ConstructionOfTangent}
\pmrelated{LengthsOfAngleBisectors}
\pmrelated{Incenter}
\pmrelated{CenterNormalAndCenterNormalPlaneAsLoci}

\endmetadata

% this is the default PlanetMath preamble.  as your knowledge
% of TeX increases, you will probably want to edit this, but
% it should be fine as is for beginners.

% almost certainly you want these
\usepackage{amssymb}
\usepackage{amsmath}
\usepackage{amsfonts}

% used for TeXing text within eps files
%\usepackage{psfrag}
% need this for including graphics (\includegraphics)
%\usepackage{graphicx}
% for neatly defining theorems and propositions
 \usepackage{amsthm}
% making logically defined graphics
%%%\usepackage{xypic}

% there are many more packages, add them here as you need them

% define commands here

\theoremstyle{definition}
\newtheorem*{thmplain}{Theorem}

\begin{document}
If\, $0 < \alpha < 180^{\mathrm{o}}$,\, then the angle bisector of $\alpha$ is the locus of all such points which are equidistant from both sides of the angle (it is proved by using the AAS and SSA theorems).

The equation of the angle bisectors of all four angles formed by two intersecting lines
\begin{align}
a_1x\!+\!b_1y\!+\!c_1 \;=\; 0,\qquad a_2x\!+\!b_2y\!+\!c_2 \;=\; 0
\end{align}
is
\begin{align}
\frac{a_1x\!+\!b_1y\!+\!c_1}{\sqrt{a_1^2\!+\!b_1^2}} \;=\; \pm\frac{a_2x\!+\!b_2y\!+\!c_2}{\sqrt{a_2^2\!+\!b_2^2}},
\end{align}
which may be written in the form
\begin{align}
x\sin\alpha_1-y\cos\alpha_1+h_1 \;=\; \pm(x\sin\alpha_2-y\cos\alpha_2+h_2)
\end{align}
after performing the divisions in (2) termwise; the angles $\alpha_1$ and $\alpha_2$ \PMlinkescapetext{mean} then the slope angles of the lines.

\textbf{Note.}\, The two lines in (2) are perpendicular, since their slopes
$\displaystyle\frac{\sin\alpha_1\pm\sin\alpha_2}{\cos\alpha_1\pm\cos\alpha_2}$ are opposite inverses of each other.

%%%%%
%%%%%
\end{document}
