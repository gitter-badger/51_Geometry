\documentclass[12pt]{article}
\usepackage{pmmeta}
\pmcanonicalname{LinearOrderedGeometry}
\pmcreated{2013-03-22 17:19:35}
\pmmodified{2013-03-22 17:19:35}
\pmowner{Mathprof}{13753}
\pmmodifier{Mathprof}{13753}
\pmtitle{linear ordered geometry}
\pmrecord{4}{39678}
\pmprivacy{1}
\pmauthor{Mathprof}{13753}
\pmtype{Definition}
\pmcomment{trigger rebuild}
\pmclassification{msc}{51G05}

% this is the default PlanetMath preamble.  as your knowledge
% of TeX increases, you will probably want to edit this, but
% it should be fine as is for beginners.

% almost certainly you want these
\usepackage{amssymb}
\usepackage{amsmath}
\usepackage{amsfonts}

% used for TeXing text within eps files
%\usepackage{psfrag}
% need this for including graphics (\includegraphics)
%\usepackage{graphicx}
% for neatly defining theorems and propositions
%\usepackage{amsthm}
% making logically defined graphics
%%%\usepackage{xypic}

% there are many more packages, add them here as you need them

% define commands here

\begin{document}
An incidence geometry $A=(P,n,I)$ is a
\emph{linear ordered geometry} if there is a strict betweenness
relation $B$ defined on the points $P_0$ of $A$, such that
\begin{itemize}
\item[Col1] $(p,q,r)\in B$ only if $p,q$, and $r$ are collinear (all incident with a common
line $\ell\in P_1$);
\item[Col2] for any pairwise distinct collinear points $p,q,r$, at least one of
$(p,q,r)$, $(q,r,p)$, or $(r,p,q)\in B$,
\end{itemize}
We denote the linear ordered geometry by $(A,B)$.\\
%%%%%
%%%%%
\end{document}
