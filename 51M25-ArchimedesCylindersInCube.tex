\documentclass[12pt]{article}
\usepackage{pmmeta}
\pmcanonicalname{ArchimedesCylindersInCube}
\pmcreated{2013-03-22 17:20:51}
\pmmodified{2013-03-22 17:20:51}
\pmowner{pahio}{2872}
\pmmodifier{pahio}{2872}
\pmtitle{Archimedes' cylinders in cube}
\pmrecord{7}{39704}
\pmprivacy{1}
\pmauthor{pahio}{2872}
\pmtype{Example}
\pmcomment{trigger rebuild}
\pmclassification{msc}{51M25}
\pmclassification{msc}{51-00}
\pmsynonym{perpendicular cylinders}{ArchimedesCylindersInCube}
\pmsynonym{cylinders inscribed in cube}{ArchimedesCylindersInCube}
%\pmkeywords{Archimedes' cylinder}
\pmrelated{SubstitutionNotation}

\endmetadata

% this is the default PlanetMath preamble.  as your knowledge
% of TeX increases, you will probably want to edit this, but
% it should be fine as is for beginners.

% almost certainly you want these
\usepackage{amssymb}
\usepackage{amsmath}
\usepackage{amsfonts}

% used for TeXing text within eps files
%\usepackage{psfrag}
% need this for including graphics (\includegraphics)
%\usepackage{graphicx}
% for neatly defining theorems and propositions
%\usepackage{amsthm}
% making logically defined graphics
%%%\usepackage{xypic}

% there are many more packages, add them here as you need them

% define commands here
\newcommand{\sijoitus}[2]%
{\operatornamewithlimits{\Big/}_{\!\!\!#1}^{\,#2}}
\begin{document}
\PMlinkescapeword{cut}
The following problem has been solved by \textbf{Archimedes}:

Two distinct circular cylinders are inscribed in a cube; the axes thus intersect each other perpendicularly.\, Determine the volume common to both cylinders, when the radius of the base of the cylinders is $r$.

If the solid common to both cylinders is cut with a plane parallel to the axes of both cylinders, the figure of intersection is a square.\, Denote the distance of the plane from the center of the cube be $x$.\, By the Pythagorean theorem, half of the side of the square is $\sqrt{r^2\!-\!x^2}$ and the area of the square is 
$4(\sqrt{r^2\!-\!x^2})^2$.  Accordingly, we have the function 
$$A(x) \;:=\; 4(r^2\!-\!x^2)$$
for the area of the intersection square.\, If we let $x$ here to grow from $0$ to $r$, then half of the given solid is got.\, By the volume \PMlinkescapetext{formula} of the \PMlinkname{parent entry}{VolumeAsIntegral}, the half volume of the solid is
$$\frac{1}{2}V \;=\; \int_0^r\!4(r^2\!-\!x^2)\,dx \;=\; 4\!\sijoitus{x=0}{\quad r}\!\left(r^2x-\frac{x^3}{3}\right) \;=\; \frac{8}{3}r^3.$$
So the volume in the question is $\frac{16}{3}r^3$.\, It is $\displaystyle\frac{2}{3}$ of the volume of the cube.

%%%%%
%%%%%
\end{document}
