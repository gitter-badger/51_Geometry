\documentclass[12pt]{article}
\usepackage{pmmeta}
\pmcanonicalname{AsymptoteOfAnArctangent}
\pmcreated{2013-03-22 18:51:01}
\pmmodified{2013-03-22 18:51:01}
\pmowner{PrimeFan}{13766}
\pmmodifier{PrimeFan}{13766}
\pmtitle{asymptote of an arctangent}
\pmrecord{4}{41659}
\pmprivacy{1}
\pmauthor{PrimeFan}{13766}
\pmtype{Example}
\pmcomment{trigger rebuild}
\pmclassification{msc}{51N99}
\pmsynonym{asymptote of an inverse tangent}{AsymptoteOfAnArctangent}

\endmetadata

% this is the default PlanetMath preamble.  as your knowledge
% of TeX increases, you will probably want to edit this, but
% it should be fine as is for beginners.

% almost certainly you want these
\usepackage{amssymb}
\usepackage{amsmath}
\usepackage{amsfonts}

% used for TeXing text within eps files
%\usepackage{psfrag}

% need this for including graphics (\includegraphics)
\usepackage{graphicx}

% for neatly defining theorems and propositions
%\usepackage{amsthm}
% making logically defined graphics
%%%\usepackage{xypic}

% there are many more packages, add them here as you need them

% define commands here

\begin{document}
Given the function $y = \tan^{-1}(x)$,

\begin{center}
\includegraphics{AsympArcTan}
\end{center}

we see that it has two asymptotes, namely $\frac{\pi}{2}$ and $-(\frac{\pi}{2})$. This specific example of an asymptote was mentioned in an episode of the comedy TV show {\it The Big Bang Theory}, in which the mathematician Sheldon Eppes tries to explain his inability to reach the top of a rock wall, getting stuck at the same height each time. According to the Mathematica manual, the results of \verb=ArcTan= for real $z$ "are always in the range $- \pi / 2$ to $\pi / 2$. 

\begin{thebibliography}{1}
\bibitem{cl} Chuck Lorre, \PMlinkexternal{Chuck Lorre Productions Vanity Card \#237}{http://chucklorre.com/index.php?p=237}
\end{thebibliography}
%%%%%
%%%%%
\end{document}
