\documentclass[12pt]{article}
\usepackage{pmmeta}
\pmcanonicalname{ALineSegmentHasAtMostOneMidpoint}
\pmcreated{2013-03-22 17:17:37}
\pmmodified{2013-03-22 17:17:37}
\pmowner{Mathprof}{13753}
\pmmodifier{Mathprof}{13753}
\pmtitle{a line segment has at most one midpoint}
\pmrecord{9}{39637}
\pmprivacy{1}
\pmauthor{Mathprof}{13753}
\pmtype{Theorem}
\pmcomment{trigger rebuild}
\pmclassification{msc}{51G05}

\endmetadata

% this is the default PlanetMath preamble.  as your knowledge
% of TeX increases, you will probably want to edit this, but
% it should be fine as is for beginners.

% almost certainly you want these
\usepackage{amssymb}
\usepackage{amsmath}
\usepackage{amsfonts}

% used for TeXing text within eps files
%\usepackage{psfrag}
% need this for including graphics (\includegraphics)
%\usepackage{graphicx}
% for neatly defining theorems and propositions
\usepackage{amsthm}
% making logically defined graphics
%%%\usepackage{xypic}

% there are many more packages, add them here as you need them

% define commands here
\newtheorem{thm}{Theorem}
\begin{document}
(this proof is not correct yet)
\begin{thm} In an ordered geometry a line segment has at most one midpoint.
\end{thm}
\begin{proof}
Let $[p,q]$ be a closed line segment and suppose $m$ and $m'$ are midpoints.
If $m:p:q$ then $[m,p] < [m,q]$ so $m$ is not a midpoint. Similarly we cannot have
$p:q:m$, so we have $p:m:q$. And also, $p:m':q$. Suppose $m \not = m'$. Without loss of
generality we can assume $p:m:m'$ and $m:m':q$. But then $[p,m'] > [p,m] \cong [m,q] > [m',q]$ so that
$[p,m'] \not \cong [m',q]$, a contradiction. Hence $m = m'$.
\end{proof}

%%%%%
%%%%%
\end{document}
