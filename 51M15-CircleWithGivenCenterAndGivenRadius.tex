\documentclass[12pt]{article}
\usepackage{pmmeta}
\pmcanonicalname{CircleWithGivenCenterAndGivenRadius}
\pmcreated{2013-03-22 17:14:03}
\pmmodified{2013-03-22 17:14:03}
\pmowner{Wkbj79}{1863}
\pmmodifier{Wkbj79}{1863}
\pmtitle{circle with given center and given radius}
\pmrecord{30}{39563}
\pmprivacy{1}
\pmauthor{Wkbj79}{1863}
\pmtype{Algorithm}
\pmcomment{trigger rebuild}
\pmclassification{msc}{51M15}
\pmclassification{msc}{51-00}

\endmetadata

\usepackage{amssymb}
\usepackage{amsmath}
\usepackage{amsfonts}
\usepackage{pstricks}
\usepackage{psfrag}
\usepackage{graphicx}
\usepackage{amsthm}
%%\usepackage{xypic}

% there are many more packages, add them here as you need them

% define commands here

\theoremstyle{definition}
\newtheorem*{thmplain}{Theorem}

\begin{document}
\PMlinkescapeword{solution}

\textbf{Task.}\, Draw the circle having a given point $O$ as its center and a given line segment of length $AB$ as its radius.  This construction must be performed with constraints in the spirit of Euclid: One must not take the length of $\overline{AB}$ between the tips of the compass (\PMlinkname{i.e.}{Ie}, one must pretend that the compass is \PMlinkname{collapsible}{CollapsibleCompass}).  This means than one may only draw arcs that are of circles with the center and one point of the circumference known.

\begin{center}
\begin{pspicture}(-2,-3)(4,1)
\psline(-2,-2)(3,-2)
\psdots(-2,-2)(3,-2)(0,1)
\rput[a](-2,-2.4){$A$}
\rput[a](3.3,-2.3){$B$}
\rput[a](0,0.7){$O$}
\rput[b](0,1){.}
\end{pspicture}
\end{center}

\textbf{Solution.}

\begin{enumerate}

\item Draw an arc of the circle $a$ through $A$ with center $O$ and an arc of the circle $o$ through $O$ with center $A$.  These arcs must intersect each other. Let one of the intersection points be $C$.

\begin{center}
\begin{pspicture}(-4,-4)(4,2)
\psline(-2,-2)(3,-2)
\psarc[linecolor=blue](0,1){3.60555}{150}{330}
\rput[r](0,-2.8){$a$}
\psarc[linecolor=blue](-2,-2){3.60555}{-30}{150}
\rput[l](-2,1.8){$o$}
\psdots(-2,-2)(3,-2)(0,1)(-3.598,1.23205)
\rput[a](-2,-2.4){$A$}
\rput[a](3.3,-2.3){$B$}
\rput[a](0,0.7){$O$}
\rput[a](-4,1.5){$C$}
\end{pspicture}
\end{center}

\item Draw the lines $\overleftrightarrow{CA}$ and $\overleftrightarrow{CO}$.

\begin{center}
\begin{pspicture}(-6,-6)(6,6)
\psline(-2,-2)(3,-2)
\psarc(0,1){3.60555}{150}{330}
\rput[r](0,-2.8){$a$}
\psarc(-2,-2){3.60555}{-30}{150}
\rput[l](-2,1.8){$o$}
\psline[linecolor=blue](-5,1.32247)(5,0.67753)
\psline[linecolor=blue](-5,4.06768)(-0.0223,-6)
\psdots(-2,-2)(3,-2)(0,1)(-3.598,1.23205)
\rput[a](-2,-2.4){$A$}
\rput[a](3.3,-2.3){$B$}
\rput[a](0,0.7){$O$}
\rput[a](-4,1.5){$C$}
\end{pspicture}
\end{center}

\item Draw an arc of the circle $b$ through $B$ and with center $A$. Let $D$ be the intersection point of $b$ and the line $\overleftrightarrow{CA}$.

\begin{center}
\begin{pspicture}(-6,-7)(6,6)
\psline(-2,-2)(3,-2)
\psarc(0,1){3.60555}{150}{330}
\rput[r](0,-2.8){$a$}
\psarc(-2,-2){3.60555}{-30}{150}
\rput[l](-2,1.8){$o$}
\psline(-5,1.32247)(5,0.67753)
\psline(-5,4.06768)(0.472115,-7)
\psarc[linecolor=blue](-2,-2){5}{-75}{45}
\rput[l](2.1,0.3){$b$}
\psdots(-2,-2)(3,-2)(0,1)(-3.598,1.23205)(0.216,-6.482)
\rput[a](-2,-2.4){$A$}
\rput[a](3.3,-2.3){$B$}
\rput[a](0,0.7){$O$}
\rput[a](-4,1.5){$C$}
\rput[l](0.4,-6.6){$D$}
\end{pspicture}
\end{center}

\item Draw an arc of the circle $c$ through $C$ and with center $D$. Let $E$ be the intersection point of $d$ and the line $\overleftrightarrow{CO}$ with $E \neq C$.

\begin{center}
\begin{pspicture}(-6,-7)(7,6)
\rput[l](-5.869,-0.397){.}
\rput[r](6.3,-0.397){.}
\psline(-2,-2)(3,-2)
\psarc(0,1){3.60555}{150}{330}
\rput[r](0,-2.8){$a$}
\psarc(-2,-2){3.60555}{-30}{150}
\rput[l](-2,1.4){$o$}
\psline(-5,1.32247)(5,0.67753)
\psline(-5,4.06768)(0.472115,-7)
\psarc(-2,-2){5}{-75}{45}
\rput[l](2.1,0.3){$b$}
\psarc[linecolor=blue](0.216,-6.482){8.6054}{45}{135}
\rput[l](0.216,2.3){$c$}
\psdots(-2,-2)(3,-2)(0,1)(-3.598,1.23205)(0.216,-6.482)(4.989,0.67822)
\rput[a](-2,-2.4){$A$}
\rput[a](3.3,-2.3){$B$}
\rput[a](0,0.7){$O$}
\rput[a](-4,1.5){$C$}
\rput[l](0.4,-6.6){$D$}
\rput[a](5.05,0.3){$E$}
\end{pspicture}
\end{center}

\item Draw the circle $e$ through $E$ and with center $O$.  This is the required circle.

\begin{center}
\begin{pspicture}(-6,-7)(7,6)
\rput[l](-5.869,-0.397){.}
\rput[r](6.3,-0.397){.}
\psline(-2,-2)(3,-2)
\psarc(0,1){3.60555}{150}{330}
\rput[r](0,-2.8){$a$}
\psarc(-2,-2){3.60555}{-30}{150}
\rput[l](-2,1.4){$o$}
\psline(-5,1.32247)(5,0.67753)
\psline(-5,4.06768)(0.472115,-7)
\psarc(-2,-2){5}{-75}{45}
\rput[l](2.1,0.3){$b$}
\psarc(0.216,-6.482){8.6054}{45}{135}
\rput[l](0.216,2.3){$c$}
\pscircle[linecolor=blue](0,1){5}
\rput[b](0,6.2){$e$}
\psdots(-2,-2)(3,-2)(0,1)(-3.598,1.23205)(0.216,-6.482)(4.989,0.67822)
\rput[a](-2,-2.4){$A$}
\rput[a](3.3,-2.3){$B$}
\rput[a](0,0.7){$O$}
\rput[a](-4,1.5){$C$}
\rput[l](0.4,-6.6){$D$}
\rput[a](5.05,0.3){$E$}
\end{pspicture}
\end{center}

\end{enumerate}

A justification for this construction is that $OE=CE-CO=CD-CA=AD=AB$.

If you are interested in seeing the rules for compass and straightedge constructions, click on the \PMlinkescapetext{link} provided.

\begin{thebibliography}{9}
\bibitem{NP}{\sc E. J. Nystr\"om}: {\em Korkeamman geometrian alkeet sovellutuksineen}.\, Kustannusosakeyhti\"o Otava, Helsinki (1948).
\end{thebibliography}
%%%%%
%%%%%
\end{document}
