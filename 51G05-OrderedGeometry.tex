\documentclass[12pt]{article}
\usepackage{pmmeta}
\pmcanonicalname{OrderedGeometry}
\pmcreated{2013-03-22 15:28:21}
\pmmodified{2013-03-22 15:28:21}
\pmowner{CWoo}{3771}
\pmmodifier{CWoo}{3771}
\pmtitle{ordered geometry}
\pmrecord{43}{37326}
\pmprivacy{1}
\pmauthor{CWoo}{3771}
\pmtype{Definition}
\pmcomment{trigger rebuild}
\pmclassification{msc}{51G05}
\pmsynonym{open interval}{OrderedGeometry}
\pmsynonym{closed interval}{OrderedGeometry}
\pmsynonym{interval}{OrderedGeometry}
\pmrelated{PaschsTheorem}
\pmdefines{half plane}
\pmdefines{side of line}
\pmdefines{open line segment}
\pmdefines{closed line segment}
\pmdefines{opposite sides}
\pmdefines{open half plane}
\pmdefines{closed half plane}
\pmdefines{end points}
\pmdefines{open line segment}
\pmdefines{closed line segment}

\endmetadata

\usepackage{amssymb,amscd}
\usepackage{amsmath}
\usepackage{amsfonts}

% used for TeXing text within eps files
%\usepackage{psfrag}
% need this for including graphics (\includegraphics)
\usepackage{graphicx}
% for neatly defining theorems and propositions
\usepackage{amsthm}
% making logically defined graphics
%%%\usepackage{xypic}

% define commands here
\renewcommand{\line}[1]{\overleftrightarrow{#1}}
\begin{document}
\PMlinkescapeword{simple}
\PMlinkescapeword{ray}
\PMlinkescapephrase{line segment}
\PMlinkescapeword{words}
\PMlinkescapeword{order}
\PMlinkescapeword{state}
\PMlinkescapeword{even}
\PMlinkescapeword{reduced}
\PMlinkescapephrase{generated by}
\PMlinkescapephrase{opposite sides}
\PMlinkescapeword{open}
\PMlinkescapeword{closed}
\PMlinkescapephrase{end points}
\section{Definition}
Let $(A,B)$ be a linear ordered geometry, where $A= (P,n,I)$ is an incidence geometry,
and $B$ is a strict betweenness relation. 
Recall that $P$ is partitioned into disjoint sets $P_0, \ldots, P_n$, where $n$ is a positive integer.

For $0\le i<n$, let 
\begin{eqnarray*}
B_i &=& \{(p,a,q)\in P_0 \times P_i \times P_0 \mid   
 p,q \textrm{ do not lie on } a, \\ && \textrm{and there
exists a point } r \textrm{ lying on  } a \textrm{ such that }(p,r,q)\in B\},
\end{eqnarray*}
and
$$
B_i(a)=\lbrace (p,q)\mid (p,a,q)\in
B_i\rbrace .
$$

For any $a\in P_i$, the set  is symmetric and anti-reflexive.  \\ 

We say that the \emph{hyperplane $a \in P_i$ is
between $p$ and $q$} if $(p,q)\in B_i(a)$. \\
We see that $B_0=B$. \\
Let's look at the case when $i=1$. If $(p,\ell,q)\in B_1$ where
$\ell$ is a line, then $p,q$ and $\ell$ necessarily lie on a common
plane $\pi$.
\begin{center}
\includegraphics{betweenness}
\end{center}
The above diagram seems to suggest that $\ell$ ``separates $\pi$ into two regions''.  However, this is not true in general without the next axiom.  \\
An \emph{ordered geometry} $(A,B)$ is a linear ordered geometry such that
\begin{itemize}
\item[S1] for any three non-collinear points $p,q,r$, and any line $\ell$
lying on the same plane $\pi$ generated by $p,q,r$, if $(p,q)\in
B_1(\ell)$ and if $r$ does not lie on $\ell$, then at least one of
$(q,r),(r,p)\in B_1(\ell)$.
\end{itemize}
In fact, in axiom S1, it can be shown that exactly one of $(q,r)$ and $(r,p)$ is in $B_1(\ell)$.  This axiom says that ``a line lying on a plane separates the plane into two mutually exclusive subsets''. \\
 Each subset is called an (open) \emph{half plane} of the line.  \\
A closed half plane is just the union of one of its open half planes and the line itself. \\
 Suppose points $p,q$ and line $\ell$ lie on plane $\pi$ and that $\ell$ is between $p$ and $q$.  Then we say that $p$ and $q$ are on the \emph{opposite sides of line} $\ell$.  Two points are on the \emph{same} side of line $\ell$ if they are not on the opposite sides of $\ell$.  If $r$ is a third point (distinct from $p,q$) that lies on $\pi$ and not on $\ell$, then according to axiom S1 above, $r$ must be on the same side of either $p$ or $q$ (but not both!).  Same sidedness is an equivalence relation on points of $A$.
\begin{center}
\includegraphics{betweenness1}
\end{center}3
An equivalent characterization of axiom S1 is in the form of Pasch's theorem.
\\\\
The ten conditions or axioms (seven betweenness, two collinearity,
and one ``separation'' axioms) are sometimes called the ``order axioms''
of $A$.
\\\\
It is customary, in an ordered geometry, to identify each element of
$P$ by its \PMlinkname{shadow}{IncidenceGeometry} (a subset of $P_0$), and we shall do so in this discussion. A line, for example, will then consist of points that are incident with it, as opposed to an abstract element of $P_1$.
Hence, we shall also confuse the notation $\line{pq}$ with $I_0(\line{pq})$.
\\\\
\section{Remarks}
\begin{itemize}
\item \textbf{Law of Trichotomy} on a strict betweenness relation:  Let $B$
be a strict betweenness relation. If $p,q,r$ are collinear, then
exactly one of $(p,q,r)$, $(q,r,p)$, or $(r,p,q)\in B$.
\item In an ordered geometry, one can define familiar concepts, such as a
line segment, a ray, even an angle, using the order axioms above.  For example, $B_{p*q}$ is called the \emph{open line segment} between $p$ and $q$, and is more commonly denoted by $\overline{pq}$, or $(p,q)$. A \emph{closed line segment} between $p$ and $q$ is just $\lbrace p\rbrace \cup \overline{pq}\cup \lbrace q\rbrace$, denoted by $[p,q]$.  From the third remark under betweenness relation, $\overline{pq}=B_{p*q}=B_{q*p}=\overline{qp}$. The points $p$ and $q$ are called the \emph{end points} of $\overline{pq}$.  
\item A \emph{ray} is defined to be $B_{pq}$.  For a more detailed discussion, see the \PMlinkname{entry on ray}{Ray}.
\item $\line{pq}=B(p,q)$.
\item $\overline{pq}\subset\line{pq}$.  The inclusion is strict,
since there exists a point $r$ such that $(p,q,r)\in B$ by order
axiom S1.  $r$ lies on the $\line{pq}$ and is clearly distinct from
both $p$ and $q$.
\item Any line segment $\overline{pq}$ in an ordered geometry, in \PMlinkescapetext{addition} to being orderable, is linearly orderable, thanks to the Law of Trichotomy.
\item It fact, $\le$, defined on a line segment, can be extended to a linear order defined on the line that includes the segment (see the last remark above on betweenness relation).  This shows that every line in an ordered geometry can be linearly ordered.
\end{itemize}

\begin{thebibliography}{6}
\bibitem{dh} D. Hilbert, {\it Foundations of Geometry}, Open Court Publishing Co. (1971)
\bibitem{bs} K. Borsuk and W. Szmielew, {\it Foundations of Geometry}, North-Holland Publishing Co. Amsterdam (1960)
\bibitem{rh} R. Hartshorne, {\it Geometry: Euclid and Beyond}, Springer (2000)
\end{thebibliography}
%%%%%
%%%%%
\end{document}
