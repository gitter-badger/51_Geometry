\documentclass[12pt]{article}
\usepackage{pmmeta}
\pmcanonicalname{OpposingAnglesInACyclicQuadrilateralAreSupplementary}
\pmcreated{2013-03-22 17:13:31}
\pmmodified{2013-03-22 17:13:31}
\pmowner{rm50}{10146}
\pmmodifier{rm50}{10146}
\pmtitle{opposing angles in a cyclic quadrilateral are supplementary}
\pmrecord{8}{39552}
\pmprivacy{1}
\pmauthor{rm50}{10146}
\pmtype{Theorem}
\pmcomment{trigger rebuild}
\pmclassification{msc}{51M04}

% this is the default PlanetMath preamble.  as your knowledge
% of TeX increases, you will probably want to edit this, but
% it should be fine as is for beginners.

% almost certainly you want these
\usepackage{amssymb}
\usepackage{amsmath}
\usepackage{amsfonts}

% used for TeXing text within eps files
%\usepackage{psfrag}
% need this for including graphics (\includegraphics)
%\usepackage{graphicx}
% for neatly defining theorems and propositions
\usepackage{amsthm}
% making logically defined graphics
%%%\usepackage{xypic}
\usepackage{pstricks}

% there are many more packages, add them here as you need them

% define commands here
\newtheorem{thm}{Theorem}

\begin{document}
\begin{thm} \emph{[Euclid, Book III, Prop. 22]} If a quadrilateral is inscribed in a circle, then opposite angles of the quadrilateral sum to $180^{\circ}$.
\end{thm}
\begin{proof}
Let $ABCD$ be a quadrilateral inscribed in a circle
\begin{center}
%--eukleides
%frame(-3,-3,3,3)
%O=point(0,0); draw(O); draw("$O$",O,45:)
%c=circle(O,2); draw(c)
%A=point(c,0:); draw(A); draw("$A$",A,0:)
%B=point(c,75:); draw(B); draw("$B$",B,75:)
%C=point(c,190:); draw(C); draw("$C$",C,190:)
%D=point(c,320:); draw(D); draw("$D$",D,320:)
%draw(A,B,C,D)
%draw(segment(O,B))
%draw(segment(O,D))
%--end
% Generated by eukleides 1.0.3
\begin{pspicture*}(-3.0000,-3.0000)(3.0000,3.0000)
\rput(-3.1,-3.1){.}
\rput(3.1,3.1){.}
\psdots[dotstyle=*, dotscale=1.0000](0.0000,0.0000)
\uput{0.3000}[45.0000](0.0000,0.0000){$O$}
\pscircle(0.0000,0.0000){2.0000}
\psdots[dotstyle=*, dotscale=1.0000](2.0000,0.0000)
\uput{0.3000}[0.0000](2.0000,0.0000){$A$}
\psdots[dotstyle=*, dotscale=1.0000](0.5176,1.9319)
\uput{0.3000}[75.0000](0.5176,1.9319){$B$}
\psdots[dotstyle=*, dotscale=1.0000](-1.9696,-0.3473)
\uput{0.3000}[190.0000](-1.9696,-0.3473){$C$}
\psdots[dotstyle=*, dotscale=1.0000](1.5321,-1.2856)
\uput{0.3000}[320.0000](1.5321,-1.2856){$D$}
\pspolygon(2.0000,0.0000)(0.5176,1.9319)(-1.9696,-0.3473)(1.5321,-1.2856)
\psline(0.0000,0.0000)(0.5176,1.9319)
\psline(0.0000,0.0000)(1.5321,-1.2856)
\end{pspicture*}
% End of figure
\end{center}
Note that $\angle BAD$ subtends arc $BCD$ and $\angle BCD$ subtends arc $BAD$. Now, since a circumferential angle is half the corresponding central angle, we see that $\angle BAD + \angle BCD$ is one half of the sum of the two angles $BOD$ at $O$. But the sum of these two angles is $360^{\circ}$, so that
\[\angle BAD + \angle BCD = 180^{\circ}\]
Similarly, the sum of the other two opposing angles is also $180^{\circ}$.
\end{proof}
%%%%%
%%%%%
\end{document}
