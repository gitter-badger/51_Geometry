\documentclass[12pt]{article}
\usepackage{pmmeta}
\pmcanonicalname{Endpoint}
\pmcreated{2013-03-22 16:06:05}
\pmmodified{2013-03-22 16:06:05}
\pmowner{Wkbj79}{1863}
\pmmodifier{Wkbj79}{1863}
\pmtitle{endpoint}
\pmrecord{12}{38165}
\pmprivacy{1}
\pmauthor{Wkbj79}{1863}
\pmtype{Definition}
\pmcomment{trigger rebuild}
\pmclassification{msc}{51-00}
\pmclassification{msc}{03-00}
\pmsynonym{end point}{Endpoint}

\endmetadata

\usepackage{amssymb}
\usepackage{amsmath}
\usepackage{amsfonts}
\usepackage{pstricks}
\usepackage{psfrag}
\usepackage{graphicx}
\usepackage{amsthm}
%%\usepackage{xypic}

\begin{document}
An \emph{endpoint} of a line segment $L$ is a point that belongs to the boundary of $L$.  Note that every line segment has two distinct endpoints.  For example, if $V$ is a vector space and $a,b \in V$ with $b \neq 0$, then the endpoints of the line segment $\displaystyle L = \{ a+tb : t\in[0,1]\}$ are $a$ and $a+b$.

\begin{center}
\begin{pspicture}(0,-0.5)(4,0.1)
\rput[a](0,0.06){.}
\psline(0,0)(4,0)
\psdots(0,0)(4,0)
\rput[a](0,-0.3){$a$}
\rput[a](4,-0.3){$a+b$}
\end{pspicture}
\end{center}

Note that the endpoints of the open line segment $\displaystyle L=\{a+tb:t\in(0,1)\}$ are also $a$ and $a+b$.

\begin{center}
\begin{pspicture}(0,-0.5)(4,0.1)
\rput[a](0,0.06){.}
\psline{o-o}(0,0)(4,0)
\rput[a](0,-0.3){$a$}
\rput[a](4,-0.3){$a+b$}
\end{pspicture}
\end{center}

Endpoints can be defined in a \PMlinkescapetext{similar} manner for other geometric \PMlinkescapetext{objects}.  These include rays, \PMlinkname{arcs}{PathConnected}, and \PMlinkname{intervals}{Interval}.

\begin{itemize}
\item Rays have one endpoint.

\begin{center}
\begin{pspicture}(0,-0.1)(4,0.1)
\psline{->}(0,0)(4,0)
\psdot(0,0)
\end{pspicture}
\end{center}

\item \PMlinkescapetext{Arcs} have two endpoints.

\begin{center}
\begin{pspicture}(0,0)(4,2)
\psarc(2,0){2}{0}{180}
\psdots(0,0)(4,0)
\end{pspicture}
\end{center}

\item \PMlinkescapetext{Intervals} can have zero, one, or two endpoints, depending on whether they are bounded above and/or below.  See the entry on \PMlinkname{intervals}{Interval} for more details.
\end{itemize}
%%%%%
%%%%%
\end{document}
