\documentclass[12pt]{article}
\usepackage{pmmeta}
\pmcanonicalname{EquilateralTriangle}
\pmcreated{2013-03-22 11:44:04}
\pmmodified{2013-03-22 11:44:04}
\pmowner{Mathprof}{13753}
\pmmodifier{Mathprof}{13753}
\pmtitle{equilateral triangle}
\pmrecord{22}{30143}
\pmprivacy{1}
\pmauthor{Mathprof}{13753}
\pmtype{Definition}
\pmcomment{trigger rebuild}
\pmclassification{msc}{51-00}
\pmclassification{msc}{81-00}
\pmclassification{msc}{08-01}
%\pmkeywords{Triangle}
\pmrelated{Triangle}
\pmrelated{IsoscelesTriangle}
\pmrelated{EquivalentConditionsForTriangles}
\pmrelated{EquiangularTriangle}
\pmrelated{RegularTriangle}

\endmetadata

\usepackage{amssymb}
\usepackage{amsmath}
\usepackage{amsfonts}
\usepackage{pstricks}
\usepackage{graphicx}
%%%%%%%\usepackage{xypic}

\begin{document}
An \emph{equilateral triangle} is one for which all 3  sides are congruent.


\begin{center}
\begin{pspicture}(-0.2,-0.2)(5.2,5.2)
\pspolygon(0,0)(5,0)(2.5,4.33)
\rput[b](2.5,4.5){A}
\rput[a](0,-0.2){B}
\rput[a](5,-0.2){C}
\psline(2.5,-0.2)(2.5,0.2)
\psline(1.15,2.2)(1.35,2.1)
\psline(3.65,2.1)(3.85,2.2)
\end{pspicture}
\end{center}



The following statements hold in Euclidean geometry for an equilateral triangle.

\begin{itemize}
\item
It is a regular polygon.
\item
The  bisector of any angle coincides with the height, the median and the perpendicular bisector of the \PMlinkescapetext{opposite side}.
\item 
If $r$ is the length of the side, then the height is equal to $\displaystyle \frac{r\sqrt{3}}{2}$.
\item
If $r$ is the length of the side, then the area is equal to  $\displaystyle \frac{r^2\sqrt{3}}{4}$.
\end{itemize}
%%%%%
%%%%%
%%%%%
%%%%%
%%%%%
%%%%%
%%%%%
\end{document}
