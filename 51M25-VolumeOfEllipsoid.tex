\documentclass[12pt]{article}
\usepackage{pmmeta}
\pmcanonicalname{VolumeOfEllipsoid}
\pmcreated{2013-03-22 17:20:41}
\pmmodified{2013-03-22 17:20:41}
\pmowner{pahio}{2872}
\pmmodifier{pahio}{2872}
\pmtitle{volume of ellipsoid}
\pmrecord{11}{39700}
\pmprivacy{1}
\pmauthor{pahio}{2872}
\pmtype{Result}
\pmcomment{trigger rebuild}
\pmclassification{msc}{51M25}
\pmsynonym{ellipsoid volume}{VolumeOfEllipsoid}
\pmrelated{Ellipsoid}
\pmrelated{SubstitutionNotation}
\pmrelated{SqueezingMathbbRn}

% this is the default PlanetMath preamble.  as your knowledge
% of TeX increases, you will probably want to edit this, but
% it should be fine as is for beginners.

% almost certainly you want these
\usepackage{amssymb}
\usepackage{amsmath}
\usepackage{amsfonts}

% used for TeXing text within eps files
%\usepackage{psfrag}
% need this for including graphics (\includegraphics)
%\usepackage{graphicx}
% for neatly defining theorems and propositions
%\usepackage{amsthm}
% making logically defined graphics
%%%\usepackage{xypic}

% there are many more packages, add them here as you need them

% define commands here
\newcommand{\sijoitus}[2]%
{\operatornamewithlimits{\Big/}_{\!\!\!#1}^{\,#2}}
\begin{document}
\PMlinkescapeword{formula} \PMlinkescapeword{cut}
Let us determine the volume of the ellipsoid
$$\frac{x^2}{a^2}+\frac{y^2}{b^2}+\frac{z^2}{c^2} \;=\; 1.$$

Suppose\, $-a \leqq x \leqq a$.  When we cut the ellipsoid with a plane parallel to the $yz$-plane, that is, let $x$ be \PMlinkescapetext{constant}, we get the ellipse
$$\frac{y^2}{b^2}+\frac{z^2}{c^2} \;=\; 1\!-\!\frac{x^2}{a^2},$$
i.e. 
$$\frac{y^2}{b^2\left(1\!-\!\frac{x^2}{a^2}\right)}+\frac{z^2}{c^2\left(1\!-\!\frac{x^2}{a^2}\right)} \;=\; 1,$$
with the semiaxes
$$b_1 := b\sqrt{1\!-\!\frac{x^2}{a^2}},\quad c_1 \;:=\; c\sqrt{1\!-\!\frac{x^2}{a^2}}.$$
The area of this ellipse is $\pi b_1 c_1$ (see area of plane region), and thus we have the function
$$A(x) \;:=\; \pi b c \left(1-\frac{x^2}{a^2}\right)$$
expressing the area cut of the ellipsoid by parallel planes.  By the volume formula of the \PMlinkname{parent entry}{VolumeAsIntegral} we can calculate the volume of the ellipsoid as
$$V \;=\; \int_{-a}^a\!A(x)\,dx = \pi b c \int_{-a}^a\!\left(1\!-\!\frac{x^2}{a^2}\right)\,dx 
 \;=\; \pi b c\! \sijoitus{x\,=-a}{\quad a}\left(x-\frac{x^3}{3a^2}\right) \;=\; \frac{4}{3}\pi a b c.$$
The special case\, $a = b = c = r$\, of a sphere is the well-known expression $\frac{4}{3}\pi r^3.$
%%%%%
%%%%%
\end{document}
