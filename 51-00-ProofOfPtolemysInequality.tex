\documentclass[12pt]{article}
\usepackage{pmmeta}
\pmcanonicalname{ProofOfPtolemysInequality}
\pmcreated{2013-03-22 12:46:09}
\pmmodified{2013-03-22 12:46:09}
\pmowner{mathwizard}{128}
\pmmodifier{mathwizard}{128}
\pmtitle{proof of Ptolemy's inequality}
\pmrecord{5}{33079}
\pmprivacy{1}
\pmauthor{mathwizard}{128}
\pmtype{Proof}
\pmcomment{trigger rebuild}
\pmclassification{msc}{51-00}

\endmetadata

% this is the default PlanetMath preamble.  as your knowledge
% of TeX increases, you will probably want to edit this, but
% it should be fine as is for beginners.

% almost certainly you want these
\usepackage{amssymb}
\usepackage{amsmath}
\usepackage{amsfonts}

% used for TeXing text within eps files
%\usepackage{psfrag}
% need this for including graphics (\includegraphics)
\usepackage{graphicx}
% for neatly defining theorems and propositions
%\usepackage{amsthm}
% making logically defined graphics
%%%\usepackage{xypic}

% there are many more packages, add them here as you need them

% define commands here
\begin{document}
Looking at the quadrilateral $ABCD$ we construct a point $E$, such that the triangles $ACD$ and $AEB$ are similar ($\angle ABE=\angle CDA$ and $\angle BAE=\angle CAD$).
\begin{center}
\includegraphics{ptolomaeus.eps}
\end{center}
This means that:
$$\frac{AE}{AC}=\frac{AB}{AD}=\frac{BE}{DC},$$
from which follows that
$$BE=\frac{AB\cdot DC}{AD}.$$
Also because $\angle EAC=\angle BAD$ and 
$$\frac{AD}{AC}=\frac{AB}{AE}$$
the triangles $EAC$ and $BAD$ are similar. So we get:
$$EC=\frac{AC\cdot DB}{AD}.$$
Now if $ABCD$ is cyclic we get
$$\angle ABE+\angle CBA=\angle ADC+\angle CBA=180^\circ.$$
This means that the points $C$, $B$ and $E$ are on one line and thus:
$$EC=EB+BC$$
Now we can use the formulas we already found to get:
$$\frac{AC\cdot DB}{AD}=\frac{AB\cdot DC}{AD}+BC.$$
Multiplication with $AD$ gives:
$$AC\cdot DB=AB\cdot DC+BC\cdot AD.$$

Now we look at the case that $ABCD$ is not cyclic. Then
$$\angle ABE+\angle CBA=\angle ADC+\angle CBA\neq 180^\circ,$$
so the points $E$, $B$ and $C$ form a triangle and from the triangle inequality we know:
$$EC<EB+BC.$$
Again we use our formulas to get:
$$\frac{AC\cdot DB}{AD}<\frac{AB\cdot DC}{AD}+BC.$$
From this we get:
$$AC\cdot DB<AB\cdot DC+BC\cdot AD.$$
Putting this together we get Ptolomy's inequality:
$$AC\cdot DB\leq AB\cdot DC+BC\cdot AD,$$
with equality iff $ABCD$ is cyclic.
%%%%%
%%%%%
\end{document}
