\documentclass[12pt]{article}
\usepackage{pmmeta}
\pmcanonicalname{VeblenWedderburnSystem}
\pmcreated{2013-03-22 19:15:06}
\pmmodified{2013-03-22 19:15:06}
\pmowner{CWoo}{3771}
\pmmodifier{CWoo}{3771}
\pmtitle{Veblen-Wedderburn system}
\pmrecord{9}{42177}
\pmprivacy{1}
\pmauthor{CWoo}{3771}
\pmtype{Definition}
\pmcomment{trigger rebuild}
\pmclassification{msc}{51A35}
\pmclassification{msc}{51A40}
\pmclassification{msc}{51E15}
\pmclassification{msc}{51A25}
\pmsynonym{Veblen-Wedderburn ring}{VeblenWedderburnSystem}
\pmsynonym{quasifield}{VeblenWedderburnSystem}
\pmsynonym{VW system}{VeblenWedderburnSystem}
\pmsynonym{V-W system}{VeblenWedderburnSystem}

\usepackage{amssymb,amscd}
\usepackage{amsmath}
\usepackage{amsfonts}
\usepackage{mathrsfs}

% used for TeXing text within eps files
%\usepackage{psfrag}
% need this for including graphics (\includegraphics)
%\usepackage{graphicx}
% for neatly defining theorems and propositions
\usepackage{amsthm}
% making logically defined graphics
%%\usepackage{xypic}
\usepackage{pst-plot}

% define commands here
\newcommand*{\abs}[1]{\left\lvert #1\right\rvert}
\newtheorem{prop}{Proposition}
\newtheorem{thm}{Theorem}
\newtheorem{ex}{Example}
\newcommand{\real}{\mathbb{R}}
\newcommand{\pdiff}[2]{\frac{\partial #1}{\partial #2}}
\newcommand{\mpdiff}[3]{\frac{\partial^#1 #2}{\partial #3^#1}}
\begin{document}
A \emph{Veblen-Wedderburn system} is an algebraic system over a set $R$ with two binary operations $+$ (called addition) and $\cdot$ (called multiplication) on $R$ such that
\begin{enumerate}
\item there is a $0\in R$, and that $R$ is an abelian group under $+$, with $0$ the additive identity
\item $R-\lbrace 0\rbrace$, together with $\cdot$, is a loop (we denote $1$ as its identity element)
\item $\cdot$ is right distributive over $+$; that is, $(a+b)\cdot c=a\cdot c+ b\cdot c$
\item if $a\ne b$, then the equation $x\cdot a=x\cdot b+c$ has a unique solution in $x$
\end{enumerate}

A Veblen-Wedderburn system is also called a \emph{quasifield}.

Usually, we write $ab$ instead of $a\cdot b$.

For any $a,b,c\in R$, by defining a ternary operation $*$ on $R$, given by $$a*b*c:=a b+c,$$ it is not hard to see that $(R,*,0,1)$ is a ternary ring.  In fact, it is a linear ternary ring because $ab=a*b*0$ and $a+c=a*1*c$.

For example, any field, or more generally, any division ring, associative or not, is Veblen-Wedderburn.  An example of a Veblen-Wedderburn system that is not a division ring is the Hall quasifield.

A well-known fact about Veblen-Wedderburn systems is that, the projective plane of a Veblen-Wedderburn system is a translation plane, and, conversely, every translation plane can be coordinatized by a Veblen-Wedderburn system.  This is the reason why a translation plane is also called a Veblen-Wedderburn plane.

\textbf{Remark}.  Let $R$ be a Veblen-Wedderburn system.  If the multiplication $\cdot$, in addition to be right distributive over $+$, is also left distributive over $+$, then $R$ is a semifield.  If $\cdot$, on the other hand, is associative, then $R$ is an abelian nearfield (a nearfield such that $+$ is commutative).

\begin{thebibliography}{7}
\bibitem{RC} R. Casse, {\it Projective Geometry, An Introduction}, Oxford University Press (2006)
\end{thebibliography}
%%%%%
%%%%%
\end{document}
