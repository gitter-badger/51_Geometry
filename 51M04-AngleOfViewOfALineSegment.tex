\documentclass[12pt]{article}
\usepackage{pmmeta}
\pmcanonicalname{AngleOfViewOfALineSegment}
\pmcreated{2013-03-22 17:34:11}
\pmmodified{2013-03-22 17:34:11}
\pmowner{pahio}{2872}
\pmmodifier{pahio}{2872}
\pmtitle{angle of view of a line segment}
\pmrecord{14}{39980}
\pmprivacy{1}
\pmauthor{pahio}{2872}
\pmtype{Topic}
\pmcomment{trigger rebuild}
\pmclassification{msc}{51M04}
\pmclassification{msc}{51F20}
\pmrelated{CircumferentialAngleIsHalfCorrespondingCentralAngle}
\pmrelated{ThalesTheorem}
\pmrelated{CalculatingTheSolidAngleOfDisc}
\pmrelated{ExampleOfCalculusOfVariations}
\pmrelated{ProjectionOfRightAngle}
\pmdefines{angle of view}

% this is the default PlanetMath preamble.  as your knowledge
% of TeX increases, you will probably want to edit this, but
% it should be fine as is for beginners.

% almost certainly you want these
\usepackage{amssymb}
\usepackage{amsmath}
\usepackage{amsfonts}

% used for TeXing text within eps files
%\usepackage{psfrag}
% need this for including graphics (\includegraphics)
%\usepackage{graphicx}
% for neatly defining theorems and propositions
 \usepackage{amsthm}
% making logically defined graphics
%%%\usepackage{xypic}

% there are many more packages, add them here as you need them

\usepackage{pstricks}

% define commands here

\theoremstyle{definition}
\newtheorem*{thmplain}{Theorem}

\begin{document}
Let $PQ$ be a line segment and $A$ a point not belonging to $PQ$.\, Let the magnitude of the angle $PAQ$ be $\alpha$.\,  One says that the line segment $PQ$ {\em is seen from the point $A$ in an angle of $\alpha$}; one may also speak of the {\em angle of view} of $PQ$.

The locus of the points from which a given line segment $PQ$ is seen in an angle of $\alpha$ (with\, $0 < \alpha < 180^\circ$) consists of two congruent circular arcs having the line segment as the common chord and containing the circumferential angles equal to $\alpha$.  

Especially, the locus of the points from which the line segment is seen in an angle of $90^\circ$ is the circle having the line segment as its diameter.


\begin{center}
\begin{pspicture}(-3,-3)(3,3)
\psline[linecolor=blue](-1.73,0)(1.73,0)
\rput[a](-2.1,-0.1){$P$}
\rput[a](2.1,-0.1){$Q$}
\psarc[linecolor=red](0,1){2}{-30}{210}
\psarc[linecolor=red](0,-1){2}{-210}{30}
\psline(-1.73,0)(-1.2,2.6)
\psline(1.73,0)(-1.2,2.6)
\psline(-1.73,0)(2,1)
\psline(1.73,0)(2,1)
\rput[a](-1.08,2.25){$\alpha$}
\rput[a](1.72,0.75){$\alpha$}
\psdots[linecolor=blue linewidth=0.5](-1.73,0)(1.73,0)
\psdots[linecolor=red](-1.2,2.6)(2,1)
\rput(-3,-3){.}
\rput(3,3){.}
\end{pspicture}
\end{center}

\textbf{Note.}  The explementary arcs of the above mentioned two arcs form the locus of the points from which the segment $PQ$ is seen in the angle $180^\circ\!-\!\alpha$.

%%%%%
%%%%%
\end{document}
