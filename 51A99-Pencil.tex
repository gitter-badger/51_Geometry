\documentclass[12pt]{article}
\usepackage{pmmeta}
\pmcanonicalname{Pencil}
\pmcreated{2013-03-22 15:21:31}
\pmmodified{2013-03-22 15:21:31}
\pmowner{CWoo}{3771}
\pmmodifier{CWoo}{3771}
\pmtitle{pencil}
\pmrecord{5}{37182}
\pmprivacy{1}
\pmauthor{CWoo}{3771}
\pmtype{Definition}
\pmcomment{trigger rebuild}
\pmclassification{msc}{51A99}
\pmrelated{FanOfLines}
\pmrelated{HyperbolasOrthogonalToEllipses}
\pmdefines{flat pencil}

\usepackage{amssymb,amscd}
\usepackage{amsmath}
\usepackage{amsfonts}

% used for TeXing text within eps files
%\usepackage{psfrag}
% need this for including graphics (\includegraphics)
%\usepackage{graphicx}
% for neatly defining theorems and propositions
%\usepackage{amsthm}
% making logically defined graphics
%%%\usepackage{xypic}

% define commands here
\begin{document}
A \emph{pencil} is a set of geometric objects, usually either congruent or
similar to each other, that share a common incidence property. Below are some of the most commonly encountered pencils:
\begin{enumerate}
\item A pencil of lines usually means a set of straight lines that are incident with one point.  If the lines all lie in the same plane, the pencil is sometimes called a \emph{flat pencil}.
\item In some cases, a pencil of lines denotes a set of parallel lines in a plane.  If a point of infinity is added to the plane, then we are back to the previous example.
\item A pencil of circles can mean that either these circles all intersect at exactly one point (or share the same tangent line)
\item A pencil of circles can also mean that the circles have two common points of intersection.
\end{enumerate}

\begin{thebibliography}{8}
\bibitem{art} E. Artin, {\em Geometric Algebra}, Wiley-Interscience, Reprint (1988).
\bibitem{cox} H. S. M. Coxeter, {\em Projective Geometry}, Springer-Verlag, 2nd Edition (2003).
\end{thebibliography}
%%%%%
%%%%%
\end{document}
