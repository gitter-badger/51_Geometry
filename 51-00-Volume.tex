\documentclass[12pt]{article}
\usepackage{pmmeta}
\pmcanonicalname{Volume}
\pmcreated{2013-03-22 17:12:58}
\pmmodified{2013-03-22 17:12:58}
\pmowner{Wkbj79}{1863}
\pmmodifier{Wkbj79}{1863}
\pmtitle{volume}
\pmrecord{11}{39540}
\pmprivacy{1}
\pmauthor{Wkbj79}{1863}
\pmtype{Definition}
\pmcomment{trigger rebuild}
\pmclassification{msc}{51-00}
\pmclassification{msc}{51M25}
\pmrelated{Area2}
\pmrelated{Prismatoid}
\pmrelated{BasicLength}
\pmrelated{ExampleOfRiemannTripleIntegral}
\pmdefines{solid}

\usepackage{amssymb}
\usepackage{amsmath}
\usepackage{amsfonts}
\usepackage{pstricks}
\usepackage{psfrag}
\usepackage{graphicx}
\usepackage{amsthm}
%%\usepackage{xypic}

\begin{document}
\PMlinkescapeword{formula}
\PMlinkescapeword{formulas}
\PMlinkescapeword{base}
\PMlinkescapeword{height}
\PMlinkescapeword{cut}
\PMlinkescapeword{cube}
\PMlinkescapeword{cubes}

Note that a \emph{solid} is a three-dimensional figure.

The \emph{volume} of a solid is the amount of space contained within the solid.  Volume is typically measured in \PMlinkescapetext{cubic units} (also called \PMlinkescapetext{cube units}); \PMlinkname{i.e.}{Ie}, if the volume of a solid is 5 $\text{in}^3$, this means that, if five 1 inch by 1 inch by 1 inch cubes are cut appropriately, they can be arranged so that they exactly fill the solid without any overlapping. In formulas, volume is almost always denoted using the letter $V$.

All examples provided within this entry are in Euclidean geometry.

For certain solids, volume is quite commonly found by multiplying the lengths of three (not necessarily distinct) line segments which are related to the solid.  Some examples are:

\begin{itemize}
\item cylinders
\item prisms
\item cones
\item pyramids
\item spheres
\end{itemize}

With the exception of spheres, each of these solids has at least one face that serves as its base.  Cylinders and prisms have two bases; for these solids, the height is defined to be the line segment that is perpendicular to both bases and whose endpoints lie on the bases.  If a cylinder or prism has a base of area $B$ and a height of length $h$, then its volume is:

$$V=Bh$$

Cones and pyramids have one base; for these solids, the height is defined to be the line segment that is perpendicular to the base, has one endpoint on the base, and has the apex of the figure as its other endpoint.  If a cone or pyramid has a base of area $B$ and a height of length $h$, then its volume is:

$$V=\frac{1}{3}Bh$$

The volume of a sphere is given by $\displaystyle V=\frac{4}{3} \pi r^3$.

\textbf{Remarks}
\begin{itemize}
\item Finding the volume of more complicated geometric figures requires knowledge of calculus.  In fact, the formulas given above can all be derived using calculus.
\item The concept of volume is a special case of a more general concept called measure, or more appropriately, product measure.
\end{itemize}
%%%%%
%%%%%
\end{document}
