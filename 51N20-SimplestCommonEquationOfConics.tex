\documentclass[12pt]{article}
\usepackage{pmmeta}
\pmcanonicalname{SimplestCommonEquationOfConics}
\pmcreated{2015-03-12 8:24:02}
\pmmodified{2015-03-12 8:24:02}
\pmowner{pahio}{2872}
\pmmodifier{pahio}{2872}
\pmtitle{simplest common equation of conics}
\pmrecord{11}{41724}
\pmprivacy{1}
\pmauthor{pahio}{2872}
\pmtype{Derivation}
\pmcomment{trigger rebuild}
\pmclassification{msc}{51N20}
\pmsynonym{common equation of conics}{SimplestCommonEquationOfConics}
%\pmkeywords{eccentricity}
\pmrelated{ConicSection}
\pmrelated{QuadraticCurves}
\pmrelated{BodyInCentralForceField}

\endmetadata

% this is the default PlanetMath preamble.  as your knowledge
% of TeX increases, you will probably want to edit this, but
% it should be fine as is for beginners.

% almost certainly you want these
\usepackage{amssymb}
\usepackage{amsmath}
\usepackage{amsfonts}

% used for TeXing text within eps files
%\usepackage{psfrag}
% need this for including graphics (\includegraphics)
%\usepackage{graphicx}
% for neatly defining theorems and propositions
 \usepackage{amsthm}
% making logically defined graphics
%%%\usepackage{xypic}

% there are many more packages, add them here as you need them

% define commands here

\theoremstyle{definition}
\newtheorem*{thmplain}{Theorem}

\begin{document}
In the plane, the locus of the points having the ratio of their distances from a certain point (the focus) and from a certain line (the directrix) equal to a given constant $\varepsilon$, is a conic section, which is an ellipse, a \PMlinkname{parabola}{ConicSection} or a hyperbola depending on whether $\varepsilon$ is less than, equal to or greater than 1.\\

For showing this, we choose the $y$-axis as the directrix and the point \,$(q,\,0)$\, as the focus.\, The locus condition reads then
$$\sqrt{(x-q)^2+y^2} \;=\; \varepsilon x.$$
This is simplified to
\begin{align}
(1-\varepsilon^2)x^2-2qx+y^2+q^2 \;=\; 0.
\end{align}
If\, $\varepsilon = 1$,\, we obtain the parabola
$$y^2 \;=\; 2qx-q^2.$$
In the following, we thus assume that\, $\varepsilon \neq 1$.

Setting\, $y := 0$\, in (1) we see that the $x$-axis cuts the locus in two points with the midpoint of the segment connecting them having the abscissa
$$x_0 \;=\; \frac{q}{1-\varepsilon^2}.$$
We take this point as the new origin (replacing $x$ by $x+x_0$); then the equation (1) changes to
\begin{align}
(1-\varepsilon^2)x^2+y^2 \;=\; \frac{\varepsilon^2q^2}{1-\varepsilon^2}.
\end{align}
From this we infer that the locus is
\begin{enumerate}
\item in the case\, $\varepsilon < 1$\, an \PMlinkname{ellipse}{Ellipse2} with the semiaxes
$$a \;=\; \frac{\varepsilon q}{1-\varepsilon^2}, \qquad b \;=\; \frac{\varepsilon q}{\sqrt{1-\varepsilon^2}}$$
and with eccentricity $\varepsilon$;
\item in the case\, $\varepsilon > 1$\, a \PMlinkname{hyperbola}{Hyperbola2} with semiaxes
$$a \;=\; \frac{\varepsilon q}{\varepsilon^2-1}, \qquad b \;=\; \frac{\varepsilon q}{\sqrt{\varepsilon^2-1}}$$
and also now with the eccentricity $\varepsilon$.\\
\end{enumerate}

\textbf{\PMlinkescapetext{Polar} equation}

\PMlinkescapetext{Place} the origin into a focus of a conic section (and in the cases of ellipse and hyperbola, the abscissa axis through the other focus).\, As before, let $q$ be the distance of the focus from the corresponding directrix.\, Let $r$ and $\varphi$ be the polar coordinates of an arbitrary point of the conic.\, Then the locus condition may be expressed as
$$\frac{r}{q\pm r\cos{\varphi}} \;=\; \varepsilon.$$
Solving this equation for the \PMlinkid{polar radius}{6968} $r$ yields the form
\begin{align}
r \;=\; \frac{\varepsilon q}{1\mp \varepsilon\cos{\varphi}}
\end{align}
for the common polar equation of the conic.\, The sign alternative ($\mp$) depends on whether the polar axis ($\varphi = 0$) intersects the directrix or not.


%%%%%
%%%%%
\end{document}
