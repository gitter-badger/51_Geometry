\documentclass[12pt]{article}
\usepackage{pmmeta}
\pmcanonicalname{Midpoint1}
\pmcreated{2015-04-25 17:39:00}
\pmmodified{2015-04-25 17:39:00}
\pmowner{pahio}{2872}
\pmmodifier{pahio}{2872}
\pmtitle{midpoint}
\pmrecord{9}{41235}
\pmprivacy{1}
\pmauthor{pahio}{2872}
\pmtype{Definition}
\pmcomment{trigger rebuild}
\pmclassification{msc}{51M15}
\pmclassification{msc}{51-00}
\pmsynonym{centre}{Midpoint1}
\pmsynonym{center}{Midpoint1}

% this is the default PlanetMath preamble.  as your knowledge
% of TeX increases, you will probably want to edit this, but
% it should be fine as is for beginners.

% almost certainly you want these
\usepackage{amssymb}
\usepackage{amsmath}
\usepackage{amsfonts}

% used for TeXing text within eps files
%\usepackage{psfrag}
% need this for including graphics (\includegraphics)
%\usepackage{graphicx}
% for neatly defining theorems and propositions
 \usepackage{amsthm}
% making logically defined graphics
%%%\usepackage{xypic}

% there are many more packages, add them here as you need them

% define commands here

\theoremstyle{definition}
\newtheorem*{thmplain}{Theorem}

\begin{document}
\PMlinkescapeword{degree}
The concept of \PMlinkname{midpoint of line segment}{Midpoint} is a special case of the midpoint of a curve or arbitrary figure in $\mathbb{R}^2$ or $\mathbb{R}^3$.

A point $T$ is a {\em midpoint} of the figure $f$, if for each point $A$ of $f$ there is a point $B$ of $f$ such that $T$ is the midpoint of the line segment $AB$.\, One says also that $f$ is symmetric about the point $T$.\\


Given the equation of a curve in $\mathbb{R}^2$ or of a surface $f$ in $\mathbb{R}^3$, one can, if \PMlinkescapetext{necessary}, take a new point $T$ for the origin by using the linear substitutions of the form
$$x \;:=\; x'\!+\!a, \qquad y \;:=\; y'\!+\!b \quad \mbox{etc.}$$
Thus one may test whether the origin is the midpoint of $f$ by checking whether $f$ always contains along with any point\, $(x,\,y,\,z)$\, also the point\, $(-x,\,-y,\,-z)$.\\

It is easily verified the

\textbf{Theorem.}\, If the origin is the midpoint of a quadratic curve or a quadratic surface, then its equation has no \PMlinkname{terms of degree}{BasicPolynomial} 1.\\

Similarly one can verify the generalisation, that if the origin is the midpoint of an algebraic curve or surface of degree $n$, the equation has no terms of degree $n\!-\!1$,\, $n\!-\!3$\, and so on.\\

\textbf{Note.}\, Some curves and surfaces have infinitely many midpoints (see \PMlinkname{quadratic surfaces}{QuadraticSurfaces}).

\begin{thebibliography}{8}
\bibitem{IF}{\sc Felix Iversen}: {\em Analyyttisen geometrian oppikirja}. Tiedekirjasto Nr. 19.\, Second edition.\, Kustannusosakeyhti\"o Otava, Helsinki (1963).
\end{thebibliography}





%%%%%
%%%%%
\end{document}
