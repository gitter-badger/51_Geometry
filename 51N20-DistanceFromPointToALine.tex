\documentclass[12pt]{article}
\usepackage{pmmeta}
\pmcanonicalname{DistanceFromPointToALine}
\pmcreated{2013-03-22 15:24:30}
\pmmodified{2013-03-22 15:24:30}
\pmowner{acastaldo}{8031}
\pmmodifier{acastaldo}{8031}
\pmtitle{distance from point to a line}
\pmrecord{7}{37248}
\pmprivacy{1}
\pmauthor{acastaldo}{8031}
\pmtype{Result}
\pmcomment{trigger rebuild}
\pmclassification{msc}{51N20}
\pmrelated{DistanceOfNonParallelLines}
\pmrelated{DistanceBetweenTwoLinesInR3}
\pmrelated{Envelope}
\pmrelated{AngleBisectorAsLocus}

\endmetadata

% this is the default PlanetMath preamble.  as your knowledge
% of TeX increases, you will probably want to edit this, but
% it should be fine as is for beginners.

% almost certainly you want these
\usepackage{amssymb}
\usepackage{amsmath}
\usepackage{amsfonts}

% used for TeXing text within eps files
%\usepackage{psfrag}
% need this for including graphics (\includegraphics)
%\usepackage{graphicx}
% for neatly defining theorems and propositions
%\usepackage{amsthm}
% making logically defined graphics
%%%\usepackage{xypic}

% there are many more packages, add them here as you need them

% define commands here
\begin{document}
The distance from a point P with coordinates $(x_p, y_p) \in \mathbb{R}^2$ to the line with equation $ax + by + c = 0$ is given by $|ax_p + by_p+c|/\sqrt{a^2+b^2}$.

\textbf{Proof}  Every point $x,y$ on the line is at some distance $\sqrt{(x-x_p)^2+(y-y_p)^2}$ from P.  What we need to find is the minimum such distance.  Our problem is
$$
\min (x-x_p)^2+(y-y_p)^2
$$
subject to
$$
ax + by +c = 0
$$
This problem is solvable using the Lagrange multiplier method.  We minimize
$$
(x-x_p)^2+(y-y_p)^2 + \lambda(ax + by +c)
$$
Calculating the derivatives with respect to $x,y$ and $\lambda$ and setting them to zero we get three equations:
\begin{eqnarray}
2x - 2x_p + \lambda a = 0\\
2y - 2y_p + \lambda b = 0\\
2ax + 2by +2c =0
\end{eqnarray}
Solving these leads to 
$x_p-x = a\frac{ax_p + by_p+c}{a^2+b^2}$ and $y_p-y = b\frac{ax_p + by_p+c}{a^2+b^2}$.
We can now substitute these expressions into $\sqrt{(x-x_p)^2+(y-y_p)^2}$ and we get (after some simplification) the desired result.
%%%%%
%%%%%
\end{document}
