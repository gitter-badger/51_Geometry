\documentclass[12pt]{article}
\usepackage{pmmeta}
\pmcanonicalname{Hypotenuse}
\pmcreated{2013-03-22 12:02:58}
\pmmodified{2013-03-22 12:02:58}
\pmowner{CWoo}{3771}
\pmmodifier{CWoo}{3771}
\pmtitle{hypotenuse}
\pmrecord{15}{31096}
\pmprivacy{1}
\pmauthor{CWoo}{3771}
\pmtype{Definition}
\pmcomment{trigger rebuild}
\pmclassification{msc}{51-00}
\pmsynonym{hypothenuse}{Hypotenuse}
\pmrelated{Triangle}
\pmrelated{RightTriangle}
\pmrelated{PythagorasTheorem}
\pmrelated{Sohcahtoa}

\endmetadata

\usepackage{amssymb}
\usepackage{amsmath}
\usepackage{amsfonts}
\usepackage{graphicx}
%%%\usepackage{xypic}
\begin{document}
Let $ABC$ a right triangle in a Euclidean geometry with right angle at $C$. Then $AB$ is called the \emph{hypotenuse} of $ABC$.

\begin{center}
\includegraphics{hyp}
\end{center}

The midpoint $P$ of the hypotenuse coincides with the circumcenter of the triangle, so it is equidistant from the three vertices. When the triangle is inscribed on his circumcircle, the hypotenuse becomes a diameter. So the distance from $P$ to the vertices is precisely the circumradius.

The hypotenuse's length can be calculated by means of the Pythagorean theorem:
$$c=\sqrt{a^2+b^2}$$

%\textbf{Remark}.  Sometimes, we call a side of an \emph{arbitrary} triangle a hypotenuse if its length is longer than the lengths of the other two sides.  As a result, an isosceles triangle has no hypotenuse if the length of its two congruent legs is greater than or equal to its third leg.  However, it is not hard to adjust the definition above so that any triangle has a hypotenuse.  This can be done by replacing ``longer than'' with ``longer than or equal to''.  However, with this definition, some isosceles triangles may end up with two or three hypotenuses (three in the case of an equilateral triangle).  As a result, this naming is rarely used.  
%%%%%
%%%%%
%%%%%
\end{document}
