\documentclass[12pt]{article}
\usepackage{pmmeta}
\pmcanonicalname{LinearPair}
\pmcreated{2013-03-22 17:35:38}
\pmmodified{2013-03-22 17:35:38}
\pmowner{Wkbj79}{1863}
\pmmodifier{Wkbj79}{1863}
\pmtitle{linear pair}
\pmrecord{7}{40007}
\pmprivacy{1}
\pmauthor{Wkbj79}{1863}
\pmtype{Definition}
\pmcomment{trigger rebuild}
\pmclassification{msc}{51-00}
\pmdefines{linear pair postulate}

\usepackage{amssymb}
\usepackage{amsmath}
\usepackage{amsfonts}
\usepackage{pstricks}
\usepackage{psfrag}
\usepackage{graphicx}
\usepackage{amsthm}
%%\usepackage{xypic}

\begin{document}
\PMlinkescapeword{adjacent}

Two angles are a \emph{linear pair} if the angles are adjacent and the two unshared rays form a line.  Below is an example of a linear pair:

\begin{center}
\begin{pspicture}(0,0)(5,3)
\psline{<->}(0,0)(5,0)
\psline{->}(2.5,0)(4,3)
\end{pspicture}
\end{center}

The \emph{linear pair postulate} states that two angles that form a linear pair are \PMlinkname{supplementary}{Supplementary2}.
%%%%%
%%%%%
\end{document}
