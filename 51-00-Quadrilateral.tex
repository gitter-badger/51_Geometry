\documentclass[12pt]{article}
\usepackage{pmmeta}
\pmcanonicalname{Quadrilateral}
\pmcreated{2013-03-22 12:02:18}
\pmmodified{2013-03-22 12:02:18}
\pmowner{drini}{3}
\pmmodifier{drini}{3}
\pmtitle{quadrilateral}
\pmrecord{8}{31072}
\pmprivacy{1}
\pmauthor{drini}{3}
\pmtype{Definition}
\pmcomment{trigger rebuild}
\pmclassification{msc}{51-00}
\pmrelated{CyclicQuadrilateral}
\pmrelated{Rhombus}
\pmrelated{ParallelogramLaw}
\pmrelated{Rectangle}
\pmrelated{Kite}
\pmrelated{Square}
\pmrelated{Parallelogram}

\usepackage{amssymb}
\usepackage{amsmath}
\usepackage{amsfonts}
\usepackage{graphicx}
%%%\usepackage{xypic}
\begin{document}
A \emph{quadrilateral} is a  four-sided polygon.


Very special kinds of quadrilaterals are parallelograms (squares, rhombuses, rectangles, etc.), although cyclic quadrilaterals are also interesting on their own.
Notice however, that there are quadrilaterals that are neither parallelograms nor cyclic quadrilaterals.

\begin{center}
\includegraphics{quads}
\end{center}
%%%%%
%%%%%
%%%%%
\end{document}
