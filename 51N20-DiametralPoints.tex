\documentclass[12pt]{article}
\usepackage{pmmeta}
\pmcanonicalname{DiametralPoints}
\pmcreated{2013-03-22 18:32:14}
\pmmodified{2013-03-22 18:32:14}
\pmowner{pahio}{2872}
\pmmodifier{pahio}{2872}
\pmtitle{diametral points}
\pmrecord{5}{41249}
\pmprivacy{1}
\pmauthor{pahio}{2872}
\pmtype{Definition}
\pmcomment{trigger rebuild}
\pmclassification{msc}{51N20}
\pmclassification{msc}{51M04}
\pmrelated{Antipodal}
\pmdefines{diametral}
\pmdefines{diametral circle}

% this is the default PlanetMath preamble.  as your knowledge
% of TeX increases, you will probably want to edit this, but
% it should be fine as is for beginners.

% almost certainly you want these
\usepackage{amssymb}
\usepackage{amsmath}
\usepackage{amsfonts}

% used for TeXing text within eps files
%\usepackage{psfrag}
% need this for including graphics (\includegraphics)
%\usepackage{graphicx}
% for neatly defining theorems and propositions
 \usepackage{amsthm}
% making logically defined graphics
%%%\usepackage{xypic}

% there are many more packages, add them here as you need them

% define commands here

\theoremstyle{definition}
\newtheorem*{thmplain}{Theorem}

\begin{document}
Two points $P_1$ and $P_2$ on the circumference of a circle (or on a sphere) are {\em diametral}, if the line segment $P_1P_2$ connecting them passes through the centre of the circle (resp. the sphere), i.e. is a \PMlinkname{diametre}{Diameter}. Equivalently, the shortest distance of the diametral points $P_1$ and $P_2$ on the circle is maximal on the circle (resp. on the sphere), namely a half of the \PMlinkname{perimetre}{Perimeter}.

It's easily justified that a point of a circle (resp. a sphere) has exactly one diametral point.\\

A circle $c$ is a {\em diametral circle} of a given circle $c_0$, if $c$ intersects $c_0$ {\em diametrically}, i.e. in two diametral points of $c_0$.

If the equation of $c_0$ is\, $(x-x_0)^2+(y-y_0)^2 = r^2$\, and\, $(a,\,b)$\, is inside $c_0$, then the equation of the diametral circle $c$ with centre\, $(a,\,b)$\, is given by
$$(x-a)^2+(y-b)^2 = r^2-(x_0-a)^2-(y_0-b)^2.$$ 
%%%%%
%%%%%
\end{document}
