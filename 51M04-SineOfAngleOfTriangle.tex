\documentclass[12pt]{article}
\usepackage{pmmeta}
\pmcanonicalname{SineOfAngleOfTriangle}
\pmcreated{2013-03-22 18:27:16}
\pmmodified{2013-03-22 18:27:16}
\pmowner{pahio}{2872}
\pmmodifier{pahio}{2872}
\pmtitle{sine of angle of triangle}
\pmrecord{5}{41117}
\pmprivacy{1}
\pmauthor{pahio}{2872}
\pmtype{Derivation}
\pmcomment{trigger rebuild}
\pmclassification{msc}{51M04}
\pmrelated{DifferenceOfSquares}

\endmetadata

% this is the default PlanetMath preamble.  as your knowledge
% of TeX increases, you will probably want to edit this, but
% it should be fine as is for beginners.

% almost certainly you want these
\usepackage{amssymb}
\usepackage{amsmath}
\usepackage{amsfonts}

% used for TeXing text within eps files
%\usepackage{psfrag}
% need this for including graphics (\includegraphics)
%\usepackage{graphicx}
% for neatly defining theorems and propositions
 \usepackage{amsthm}
% making logically defined graphics
%%%\usepackage{xypic}

% there are many more packages, add them here as you need them

% define commands here

\theoremstyle{definition}
\newtheorem*{thmplain}{Theorem}

\begin{document}
The cosines law allows to express the cosine of an angle of triangle through the sides:
\begin{align}
\cos\alpha = \frac{b^2+c^2-a^2}{2bc}.
\end{align}
Substituting this to the ``fundamental formula of trigonometry'',
$$\sin^2\alpha+\cos^2\alpha \;=\;1,$$
we can calculate as follows:
\begin{align*}
\sin\alpha & \;=\; +\sqrt{1-\left(\frac{b^2+c^2-a^2}{2bc}\right)^2}\\
& \;=\; \sqrt{\frac{(2bc)^2-(b^2+c^2-a^2)^2}{(2bc)^2}}\\
& \;=\; \frac{\sqrt{(2bc+b^2+c^2-a^2)(2bc-b^2-c^2+a^2)}}{2bc}\\
& \;=\; \frac{\sqrt{[(b+c)^2-a^2][a^2-(b-c)^2]}}{2bc}\\
& \;=\; \frac{\sqrt{(b+c+a)(b+c-a)(a+b-c)(a-b+c)}}{2bc}\\
\end{align*}
Thus we have the beautiful formula
$$\sin\alpha\;=\; \frac{\sqrt{(-a\!+\!b\!+\!c)(a\!-\!b\!+\!c)(a\!+\!b\!-\!c)(a\!+\!b\!+\!c)}}{2bc}.$$\\

Substituting (1) similarly to the general formula for the sine of \PMlinkname{half-angle}{GoniometricFormulae}
$$\sin\frac{\alpha}{2} = \pm\sqrt{\frac{1-\cos\alpha}{2}},$$
one can obtain the formula
$$\sin\frac{\alpha}{2} \;=\; \sqrt{\frac{(a\!-\!b\!+\!c)(a\!+\!b\!-\!c)}{4bc}}.$$

%%%%%
%%%%%
\end{document}
