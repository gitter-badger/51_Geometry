\documentclass[12pt]{article}
\usepackage{pmmeta}
\pmcanonicalname{SinesLaw}
\pmcreated{2013-03-22 11:42:40}
\pmmodified{2013-03-22 11:42:40}
\pmowner{drini}{3}
\pmmodifier{drini}{3}
\pmtitle{sines law}
\pmrecord{26}{30028}
\pmprivacy{1}
\pmauthor{drini}{3}
\pmtype{Theorem}
\pmcomment{trigger rebuild}
\pmclassification{msc}{51-00}
\pmclassification{msc}{97-01}
\pmsynonym{law of sines}{SinesLaw}
%\pmkeywords{Sine}
%\pmkeywords{Trigonometry}
%\pmkeywords{Circle}
%\pmkeywords{Circumcircle}
%\pmkeywords{Triangle}
\pmrelated{CosinesLaw}
\pmrelated{SinesLawProof}
\pmrelated{Triangle}

\usepackage{amssymb}
\usepackage{amsmath}
\usepackage{amsfonts}
\usepackage{graphicx}
%%%%%%%%%%%%%%%\usepackage{xypic}

\begin{document}
\textbf{Sines Law.}\\
Let $ABC$ be a triangle where $a,b,c$ are the sides opposite to $A,B,C$ respectively, and let $R$ be the radius of the circumcircle. Then the following relation holds:
$$\frac{a}{\sin A}=\frac{b}{\sin B}=\frac{c}{\sin C}=2R.$$
\medskip
\begin{center}
\includegraphics[scale=0.5]{SinesLaw}
\end{center}
%%%%%
%%%%%
%%%%%
%%%%%
%%%%%
%%%%%
%%%%%
%%%%%
%%%%%
%%%%%
%%%%%
%%%%%
%%%%%
%%%%%
%%%%%
\end{document}
