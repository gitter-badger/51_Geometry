\documentclass[12pt]{article}
\usepackage{pmmeta}
\pmcanonicalname{PerpendicularBisector}
\pmcreated{2013-03-22 16:29:03}
\pmmodified{2013-03-22 16:29:03}
\pmowner{CWoo}{3771}
\pmmodifier{CWoo}{3771}
\pmtitle{perpendicular bisector}
\pmrecord{18}{38652}
\pmprivacy{1}
\pmauthor{CWoo}{3771}
\pmtype{Definition}
\pmcomment{trigger rebuild}
\pmclassification{msc}{51M15}
\pmclassification{msc}{51N20}
\pmclassification{msc}{51N05}
\pmsynonym{center normal}{PerpendicularBisector}
\pmrelated{Circumcircle}
\pmdefines{bisector}

\usepackage{amssymb,amscd}
\usepackage{amsmath}
\usepackage{amsfonts}

% used for TeXing text within eps files
%\usepackage{psfrag}
% need this for including graphics (\includegraphics)
\usepackage{graphicx}
% for neatly defining theorems and propositions
%\usepackage{amsthm}
% making logically defined graphics
%%%\usepackage{xypic}
%\usepackage{pst-plot}
%\usepackage{psfrag}

% define commands here

\begin{document}
Let $\overline{AB}$ be a line segment in a plane (we are assuming the Euclidean plane).  A \emph{bisector} of $\overline{AB}$ is any line that passes through the midpoint of $\overline{AB}$.  A \emph{perpendicular bisector} of $\overline{AB}$ is a bisector that is perpendicular to $\overline{AB}$.

It is an easy exercise to show that a line $\ell$ is a perpendicular bisector of $\overline{AB}$ iff every point lying on $\ell$ is equidistant from $A$ and $B$.  From this, one concludes that \emph{the} perpendicular bisector of a line segment is always unique.

A basic way to construct the perpendicular bisector $\ell$ given a line segment $\overline{AB}$ using the standard ruler and compass construction is as follows:


\begin{enumerate}
\item use a compass to draw the circle $C_1$ centered at point $A$ with radius the length of $\overline{AB}$, by fixing one end of the compass at $A$ and the movable end at $B$,
\item similarly, draw the circle $C_2$ centered at $B$ with the same radius as above, with the compass fixed at $B$ and movable at $A$,
\item $C_1$ and $C_2$ intersect at two points, say $P,Q$ (why?)
\item with a ruler, draw the line $\overleftrightarrow{PQ}=\ell$,
\item then $\ell$ is the perpendicular bisector of $\overline{AB}$.
\end{enumerate}

\begin{center}
\begin{figure}[!htb]
\begin{center}
\includegraphics{construct.1.eps}
\caption{The construction of a perpendicular bisector}
\end{center}
\end{figure}
\end{center}

(Note: we assume that there is always an ample supply of compasses and rulers of varying sizes, so that given any positive real number $r$, we can find a compass that opens wider than $r$ and a ruler that is longer than $r$).

One of the most common use of perpendicular bisectors is to find the center of a circle constructed from three points in a Euclidean plane:
\begin{quote}
Given three non collinear points $X,Y,Z$ in a Euclidean plane, let $C$ be the unique circle determined by $X,Y,Z$.  Then the center of $C$ is located at the intersection of the perpendicular bisectors of $\overline{XY}$ and $\overline{YZ}$.
\end{quote}
%%%%%
%%%%%
\end{document}
