\documentclass[12pt]{article}
\usepackage{pmmeta}
\pmcanonicalname{CartesianCoordinates}
\pmcreated{2013-03-22 14:28:59}
\pmmodified{2013-03-22 14:28:59}
\pmowner{pahio}{2872}
\pmmodifier{pahio}{2872}
\pmtitle{Cartesian coordinates}
\pmrecord{9}{36016}
\pmprivacy{1}
\pmauthor{pahio}{2872}
\pmtype{Definition}
\pmcomment{trigger rebuild}
\pmclassification{msc}{51N20}
\pmsynonym{rectangular coordinates}{CartesianCoordinates}
%\pmkeywords{coordinate}
\pmrelated{AnalyticGeometry}
\pmdefines{abscissa}
\pmdefines{ordinate}
\pmdefines{applicate}

\endmetadata

% this is the default PlanetMath preamble.  as your knowledge
% of TeX increases, you will probably want to edit this, but
% it should be fine as is for beginners.

% almost certainly you want these
\usepackage{amssymb}
\usepackage{amsmath}
\usepackage{amsfonts}

% used for TeXing text within eps files
%\usepackage{psfrag}
% need this for including graphics (\includegraphics)
%\usepackage{graphicx}
% for neatly defining theorems and propositions
%\usepackage{amsthm}
% making logically defined graphics
%%%\usepackage{xypic}

% there are many more packages, add them here as you need them

% define commands here
\begin{document}
The {\em Cartesian coordinates} of a point in $\mathbb{R}^3$ for determining its \PMlinkescapetext{place} in three-dimensional space are the three real numbers $x$, $y$ and $z$, which are called 
\begin{itemize}
 \item $x$-coordinate or {\em abscissa},
 \item $y$-coordinate or {\em ordinate},
 \item $z$-coordinate or {\em applicate}.
\end{itemize}
The last name ``applicate'' is rare in English, but its \PMlinkescapetext{equivalents} in continental European \PMlinkescapetext{languages}, as ``die Applikate'' in German and ``aplikaat'' in Estonian, are more known.

Similarly, in $\mathbb{R}^n$ for all\, $n = 1,\,2,\,3,\,\ldots$\, one needs $n$ coordinates for specifying the location of a point.
%%%%%
%%%%%
\end{document}
