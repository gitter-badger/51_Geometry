\documentclass[12pt]{article}
\usepackage{pmmeta}
\pmcanonicalname{ProofOfPicksTheorem}
\pmcreated{2013-03-22 13:09:47}
\pmmodified{2013-03-22 13:09:47}
\pmowner{giri}{919}
\pmmodifier{giri}{919}
\pmtitle{proof of Pick's theorem}
\pmrecord{5}{33606}
\pmprivacy{1}
\pmauthor{giri}{919}
\pmtype{Proof}
\pmcomment{trigger rebuild}
\pmclassification{msc}{51A99}
\pmclassification{msc}{05B99}
\pmclassification{msc}{68U05}

% this is the default PlanetMath preamble.  as your knowledge
% of TeX increases, you will probably want to edit this, but
% it should be fine as is for beginners.

% almost certainly you want these
\usepackage{amssymb}
\usepackage{amsmath}
\usepackage{amsfonts}

% used for TeXing text within eps files
%\usepackage{psfrag}
% need this for including graphics (\includegraphics)
\usepackage{graphicx}
% for neatly defining theorems and propositions
%\usepackage{amsthm}
% making logically defined graphics
%%%\usepackage{xypic}

% there are many more packages, add them here as you need them

% define commands here

\newcommand{\Nats}{\mathbb{N}}
\newcommand{\Ints}{\mathbb{Z}}
\newcommand{\Reals}{\mathbb{R}}
\newcommand{\Complex}{\mathbb{C}}
\begin{document}
\PMlinkescapeword{completes}
\PMlinkescapeword{clearly}
\PMlinkescapeword{character}
Pick's theorem:

Let $P\subset\Reals^2$ be a polygon with all vertices on lattice points on the grid $\Ints^2$. Let $I$ be the number of lattice points that lie \emph{inside} $P$, and let
$O$ be the number of lattice points that lie on the boundary of $P$. Then the area of  $P$ is
                         $$A(P) = I + \frac{1}{2}O - 1. $$
\begin{center}
\includegraphics{pick.eps}
\end{center}

To prove, we shall first show that Pick's theorem has an additive character. Suppose our polygon has more than 3 vertices. Then we can divide the polygon $P$ into 2 polygons $P_1$ and $P_2$ such that their interiors do not meet. Both have fewer vertices than $P.$
We claim that the validity of Pick's theorem for $P$ is equivalent to the validity of Pick's theorem for $P_1$ and $P_2.$

Denote the area, number of interior lattice points and number of boundary lattice points for $P_k$ by $A_k, I_k$ and $O_k,$ respectively, for $k=1,2.$

Clearly $A = A_1 + A_2.$

Also, if we denote the number of lattice points on the edges common to $P_1$ and $P_2$ by $L,$ then
$$I = I_1 + I_2 + L - 2$$ and $$O = O_1 + O_2 -2L + 2$$

Hence $$I + \frac{1}{2}O - 1 =   I_1 + I_2 + L - 2 + \frac{1}{2}O_1 + \frac{1}{2}O_2 - L + 1 - 1$$
$$ = I_1 + \frac{1}{2}O_1 -1 + I_2 + \frac{1}{2}O_2 -1$$

This proves the claim. Therefore we can triangulate $P$ and it suffices to prove Pick's theorem for triangles. Moreover by further triangulations
we may assume that there are no lattice points on the boundary of the triangle other than the vertices.
To prove pick's theorem for such triangles, embed them into rectangles.

\begin{center}
\includegraphics{pick1.eps}
\end{center}

Again by additivity, it suffices to prove Pick's theorem for rectangles and rectangular triangles which have no lattice points on the 
hypotenuse and whose other two sides are parallel to the coordinate axes.
If these two sides have lengths $a$ and $b,$ respectively, we have $$A = \frac{1}{2}ab$$ and $$O = a+b+1.$$
Furthermore, by thinking of the triangle as half of a rectangle, we get $$I = \frac{1}{2}(a-1)(b-1).$$
(Note that here it is essential that no lattice points are on the hypotenuse)
From these equations for $A, I$ and $O,$ Pick's theorem is satisfied for these triangles.

Finally for a rectangle, whose sides have lengths $a$ and $b,$ we find that $$A = ab$$ 
$$I = (a-1)(b-1)$$ and 
$$O = 2a + 2b.$$
From these Pick's theorem follows for  rectangles too.
This completes our proof.
%%%%%
%%%%%
\end{document}
