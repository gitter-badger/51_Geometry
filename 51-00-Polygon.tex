\documentclass[12pt]{article}
\usepackage{pmmeta}
\pmcanonicalname{Polygon}
\pmcreated{2013-03-22 12:10:15}
\pmmodified{2013-03-22 12:10:15}
\pmowner{Mathprof}{13753}
\pmmodifier{Mathprof}{13753}
\pmtitle{polygon}
\pmrecord{43}{31384}
\pmprivacy{1}
\pmauthor{Mathprof}{13753}
\pmtype{Definition}
\pmcomment{trigger rebuild}
\pmclassification{msc}{51-00}
\pmclassification{msc}{51G05}
\pmrelated{RegularPolygon}
\pmrelated{Semiperimeter}
\pmrelated{EquilateralPolygon}
\pmrelated{EquiangularPolygon}
\pmrelated{Pentagon}
\pmrelated{BasicPolygon}
\pmrelated{Hexagon}
\pmrelated{GeneralizedPythagoreanTheorem}
\pmdefines{side}
\pmdefines{vertex}
\pmdefines{vertices}
\pmdefines{simple polygon}
\pmdefines{side-lines}
\pmdefines{ray}
\pmdefines{simple way}
\pmdefines{way}
\pmdefines{region}
\pmdefines{convex region}
\pmdefines{Jordan polygon}
\pmdefines{angles of a polygon}
\pmdefines{plane polygon}
\pmdefines{broken line}

\usepackage{graphicx}
%%%%\usepackage{xypic} 
\usepackage{amsthm}
\usepackage{bbm}
\newcommand{\Z}{\mathbb{Z}}
\newcommand{\C}{\mathbb{C}}
\newcommand{\R}{\mathbb{R}}
\newcommand{\Q}{\mathbb{Q}}
\newcommand{\mathbb}[1]{\mathbbmss{#1}}
\newcommand{\figura}[1]{\begin{center}\includegraphics{#1}\end{center}}
\newcommand{\figuraex}[2]{\begin{center}\includegraphics[#2]{#1}\end{center}}

%\theoremstyle{definition}
%\newtheorem{Definition}{Definition}
\begin{document}
\PMlinkescapeword{simple}
\PMlinkescapeword{angles}
\PMlinkescapeword{join}

\section{Definitions}
We follow Forder \cite{forder} for most of this entry.
The term polygon can be defined if  one has a definition of an interval. For this 
entry we use betweenness geometry. A betweenness geometry
is just one for which there is a set of points and a betweenness relation $B$ defined.
Rather than write $(a,b,c) \in B$ we write $a*b*c$.

\begin{enumerate}
\item If $a$ and $b$ are distinct points, the \emph{line $ab$} is the set of
all points $p$ such that $p*a*b$ or $a*p*b$ or $a*b*p$. It can be shown
that the line $ab$ and the line $ba$ are the same set of points. 
\item If $o$ and $a$ are distinct points, a \emph{ray $[oa$} is the set of all points $p$ such that 
$p=o$ or $o*p*a$ or $o*a*p$. 
\item If $a$ and $b$ are distinct points, the \emph{open interval} is the set of points
$p$ such that $a*p*b$. It is denoted by $(a,b).$
\item If $a$ and $b$ are distinct points, the \emph{closed interval} is 
$(a,b) \cup \{a\} \cup \{b\}$, and denoted by $[a,b].$
\item The \emph{way $a_1a_2\ldots a_n$} is the finite set of points $\{a_1, \ldots , a_n\}$
along with the open intervals $(a_1, a_2), (a_2,a_3), \ldots, (a_{n-1}, a_n)$.
The points $a_1, \ldots, a_n$ are called the \emph{vertices} of the way, and the
open intervals are called the \emph{sides} of the way. 
A way is also called a \emph{broken line}.
The closed intervals $[a_1,a_2], \ldots, [a_{n-1},a_n]$ are called the \emph{side-intervals} of
the way. The lines $a_1a_2, \ldots , a_{n-1}a_n$ are called the \emph{side-lines}
of the way. 
The way  $a_1a_2\ldots a_n$ is said to \emph{join} $a_1$ to $a_n$.
It is assumed that $a_{i-1}, a_i, a_{i+1}$ are not collinear. 
\item A way is said to be \emph{simple} if it does not meet itself. To be precise,
(i) no two side-intervals meet in any point which is not a vertex, and (ii) no three side-intervals
meet in any point. 
\item A \emph{polygon} is a way $a_1 a_2 \ldots a_n$ for which $a_1 = a_n$. Notice that there is
no assumption that the points are coplanar. 
\item A \emph{simple polygon} is polygon for which the way is simple.
\item A \emph{region} is a set of points not all collinear, any two of which can be joined by points of a way using
only points of the region. 
\item A region $R$ is \emph{convex} if for each pair of points $a,b \in R$  the open interval $(a,b)$ is 
contained in $R.$
\item Let $X$ and $Y$ be two sets of points. If there is a set of points $S$ such that every way 
joining a point of $X$ to a point of $Y$ meets $S$ then $S$ is said to \emph{separate}
$X$ from $Y$. 
\item If $a_1 a_2 \ldots a_n$ is a polygon, then the \emph{angles of the polygon} are
$\angle a_na_1a_2, \angle a_1a_2a_3$, and so on. 
\end{enumerate}

Now assume that all points of the geometry are in one plane. Let $P$ be a polygon. ($P$ is called
a \emph{plane polygon}.)
\begin{enumerate} 
\item A ray or line which does not go through a vertex of $P$ will be called \emph{suitable}.
\item An  \emph{inside point} $a$ of $P$ is one for which a  suitable ray from $a$
meets $P$ an odd number of times. Points that are not on or inside $P$ are said to be \emph{outside} 
$P$. 
\item Let $\{P_i\}$ be a set of polygons. We say that $\{P_i\}$ \emph{dissect} $P$ if the following
three conditions are satisfied: (i) $P_i$ and $P_j$ do not have a common inside point for $i \not = j$,
(ii) each inside point of $P$ is inside or on some $P_i$ and (iii) each inside point of $P_i$ is
inside $P$.
\item A \emph{convex polygon} is one whose inside points are all on the same side of any side-line 
of the polygon. 
\end{enumerate}

\section{Theorems}
Assume that all points are in one plane. Let $P$ be a polygon.
\begin{enumerate}
\item It can be shown that $P$ separates the other points of the plane into at least two regions and that
if $P$ is simple there are exactly two regions. Moise proves this directly in \cite{moise}, pp. 16-18.
\item It can be shown that $P$ can be dissected into triangles $\{T_i\}$ such that 
every vertex of a $T_i$ is a vertex of $P$. 
\item The following theorem of Euler can be shown: Suppose  $P$ is dissected into $f>1$ polygons
and that the total number of vertices of these polygons is $v$, and the number of open intervals
which are sides is $e$. Then 
$$
v-e+f = 1
$$.
\end{enumerate}
 

A plane simple polygon with $n$ sides is called an $n$-gon, although for small $n$
there are more traditional names:

\begin{center}
\begin{tabular}{||c|c||} \hline
Number of sides& Name of the polygon \\ \hline
3 & triangle \\
4 & quadrilateral \\
5 & pentagon\\
6 & hexagon \\
7 & heptagon\\
8 & octagon\\
10 & decagon\\
\hline 
\end{tabular}
\end{center}

A plane simple polygon is also called a \emph{Jordan polygon}. 















\begin{thebibliography}{9}
\bibitem{borsuk-szmielew}
K. Borsuk and W. Szmielew, \emph{Foundations of Geometry},
North-Holland Publishing Company, 1960.
\bibitem{forder}
H.G. Forder, \emph{The Foundations of Euclidean Geometry},
Dover Publications, 1958.
\bibitem{moise}
E.E. Moise, \emph{Geometric Topology in Dimensions 2 and 3}, 
Springer-Verlag, 1977.
 
\end{thebibliography}

\PMlinkescapeword{segments}
\PMlinkescapeword{maximal}
\PMlinkescapeword{name}
\PMlinkescapeword{names}
\PMlinkescapeword{meet}
\PMlinkescapeword{opens}
\PMlinkescapeword{properties}
\PMlinkescapeword{divides}
\PMlinkescapeword{bounded}
\PMlinkescapeword{unbounded}
\PMlinkescapeword{complex}

%%%%%
%%%%%
%%%%%
\end{document}
