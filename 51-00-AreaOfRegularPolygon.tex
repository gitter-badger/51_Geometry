\documentclass[12pt]{article}
\usepackage{pmmeta}
\pmcanonicalname{AreaOfRegularPolygon}
\pmcreated{2013-03-22 17:11:06}
\pmmodified{2013-03-22 17:11:06}
\pmowner{Wkbj79}{1863}
\pmmodifier{Wkbj79}{1863}
\pmtitle{area of regular polygon}
\pmrecord{6}{39501}
\pmprivacy{1}
\pmauthor{Wkbj79}{1863}
\pmtype{Theorem}
\pmcomment{trigger rebuild}
\pmclassification{msc}{51-00}

\usepackage{amssymb}
\usepackage{amsmath}
\usepackage{amsfonts}
\usepackage{pstricks}
\usepackage{psfrag}
\usepackage{graphicx}
\usepackage{amsthm}
%%\usepackage{xypic}
\newtheorem{thm*}{Theorem}

\begin{document}
\PMlinkescapeword{perimeter}
\PMlinkescapeword{regular}
\PMlinkescapeword{divides}

\begin{thm*}
Given a \PMlinkname{regular $n$-gon}{RegularPolygon} with apothem of length $a$ and \PMlinkname{perimeter}{Perimeter2} $P$, its area is

$$A=\frac{1}{2}aP.$$
\end{thm*}

\begin{proof}
Given a regular $n$-gon $R$, line segments can be drawn from its center to each of its vertices.  This divides $R$ into $n$ congruent triangles.  The area of each of these triangles is $\displaystyle \frac{1}{2}as$, where $s$ is the length of one of the sides of the triangle.  Also note that the perimeter of $R$ is $P=ns$.  Thus, the area $A$ of $R$ is

\begin{center}
$\begin{array}{rl}
A & \displaystyle =n\left( \frac{1}{2}as \right) \\
& \\
& \displaystyle =\frac{1}{2}a(ns) \\
& \\
& \displaystyle =\frac{1}{2}aP. \end{array}$
\end{center}
\end{proof}

To illustrate what is going on in the proof, a regular hexagon appears below with each line segment from its center to one of its vertices drawn in red and one of its apothems drawn in blue.

\begin{center}
\begin{pspicture}(-2,-2)(2,2)
\psline[linecolor=blue](0,0)(0,-1.732)
\psline[linecolor=red](-2,0)(2,0)
\psline[linecolor=red](-1,-1.732)(1,1.732)
\psline[linecolor=red](1,-1.732)(-1,1.732)
\pspolygon(2,0)(1,1.732)(-1,1.732)(-2,0)(-1,-1.732)(1,-1.732)
\end{pspicture}
\end{center}
%%%%%
%%%%%
\end{document}
