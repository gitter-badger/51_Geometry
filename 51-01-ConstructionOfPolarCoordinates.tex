\documentclass[12pt]{article}
\usepackage{pmmeta}
\pmcanonicalname{ConstructionOfPolarCoordinates}
\pmcreated{2013-03-22 15:12:28}
\pmmodified{2013-03-22 15:12:28}
\pmowner{CWoo}{3771}
\pmmodifier{CWoo}{3771}
\pmtitle{construction of polar coordinates}
\pmrecord{10}{36968}
\pmprivacy{1}
\pmauthor{CWoo}{3771}
\pmtype{Derivation}
\pmcomment{trigger rebuild}
\pmclassification{msc}{51-01}
\pmdefines{polar origin}
\pmdefines{polar axis}

\endmetadata

\usepackage{pstricks}
\usepackage{pst-plot}
\usepackage{amssymb}
\usepackage{amsmath}
\usepackage{amsfonts}
\usepackage{psfrag}
\usepackage{graphicx}
\usepackage{amsthm}
%%\usepackage{xypic}
\begin{document}
\section{Construction}
When a Euclidean plane $E$ is equipped with the usual Cartesian
coordinates, we can represent any point $P$ by a pair of real
numbers $(x,y)$ in a unique fashion. The Cartesian coordinates also
make $E$ into a vector space over the reals.

Polar coordinates are an alternative way of describing points in
$E$. Like the Cartesian coordinates, polar coordinates of a point
are also expressed by a pair of real numbers $(r,\theta)$. But this
is where the similarity ends.

To set up the polar coordinates in $E$, we first pick a point of
reference $O$ and call it the \emph{polar origin}. For any point $P$
in $E$, we can measure the distance $r$ between $P$ and the polar
origin $O$. In fact, along any ray emanating from $O$, we can
uniquely identify any point on that ray by its distance $r$ from
$O$. However, if more than one ray enters the picture, distance
alone would not be enough to uniquely identify points in $E$. If we
fix the distance $r$, we are looking at a collection of points
called a circle, with radius $r$ and center $O$.
\begin{center}
\includegraphics{polar_1.eps}
\end{center}
To ``distinguish'' one point from another on the circle, a ``second
coordinate'' is needed. To this end, we fix a ray, a ray of
reference, say $x$ and call it the \emph{polar axis}. With the polar
axis $x$, a point $P_x$ of distance $r$ to $O$ is located. Then any
point on the circle of radius $r$ can now be located by a
measurement of ``how far'' it is from $P_x$. This measurement
corresponds to the angle $\theta$ between the polar axis $x$ and the
ray in question. Furthermore, this angle uniquely identifies a ray.
With $O$ and $x$, it is now enough to locate any point on $E$
uniquely. Operationally, the construction can be broken down into
the following sequence of steps:
\begin{enumerate}
\item pick a point $O$ and a ray $x$ emanating from $O$,
\item for any given point $P$, draw a straight line segment connecting $O$ and $P$,
\item measure the length $r$ of the line segment $\overline{OP}$,
\item measure the angle $\theta$ between $x$ and $\overline{OP}$,
by sweeping $x$ counterclockwise until it first reaches
$\overline{OP}$,
\item then $(r,\theta)$ are the polar coordinates of $P$.
\end{enumerate}
\begin{center}
\includegraphics{polar_2.eps}
\end{center}
All of the above steps can be carried out in a Euclidean plane,
which, in this case, is $E$. Careful readers will, however, see a
potential problem in Step 2 above when $P=O$, since one point does
not determine a unique line segment in $E$. The quick remedy is to
set $(r,\theta)=(0,0)$ be the polar coordinates of the polar origin
$O$. This is consistent with the way polar coordinates are defined
for $P\neq O$.

The construction establishes a one-to-one correspondence $f$ between
$\lbrace (0,0)\rbrace \cup (0,\infty)\times[0,2\pi)$ and $E$. Later
we will extend $f$ to $\mathbb{R}^2$ and see that every point $E$
has infinitely many representations in polar coordinates, a property
not shared by the Cartesian coordinates.

Notice also that the choice for measuring angles using the
counterclockwise sweep of the polar axis is arbitrary. We could have
used the clockwise sweep instead. To switch from one choice of
angular measurement to another, we simply perform a reflection
$\rho$ about the polar axis (again, this is possible in a Euclidean
plane):
\begin{center}
\includegraphics{polar_3.eps}
\end{center}
We will follow the standard method of measuring angles by using the
counterclockwise sweep of the polar axis described above.

\section{Relations with Cartesian coordinates}
From the discussion above, we see that $E$ is now equipped with two
coordinate systems. We can now superimpose the two coordinate
systems to seek out any properties between the two systems. First,
identify the polar origin with the rectangular origin of the
Cartesian coordinates (by translation if necessary). Then, line up
the polar axis with the positive ray of the horizontal axis of the
Cartesian coordinates (by rotation if necessary).
\begin{center}
\includegraphics{polar_4.eps}
\end{center}
With this identification, the two sets of coordinates of a point $P$
can be related by the following equations:
\begin{eqnarray}
x=r\cos\theta\qquad \mbox{ and}\qquad y=r\sin\theta,
\end{eqnarray}
where $(x,y)$ and $(r,\theta)$ are respectively the Cartesian and
polar coordinates of $P$.

With the pair of equations, we can now show how to extend the
one-to-one correspondence $f$ (see Section 1) to a map
$\widetilde{f}$ that is $\infty$-to-one. First, note that if $r=0$,
then $x=y=0$ no matter what $\theta$ is. Since the Cartesian origin
\emph{is} the polar origin, we identify $(0,\theta)$ with the polar
origin, for any $\theta\in\mathbb{R}$. This means that the polar
origin has uncountably many polar coordinate representations.

Next, if $r>0$, we see that $$x=r\cos\theta=r\cos(\theta+2n\pi)\quad
\mbox{ and} \quad y=r\sin\theta=r\sin(\theta+2n\pi),$$ for all
$n\in\mathbb{Z}$. This suggests the identification of
$(r,\theta+2n\pi)$ with $(r,\theta)$. The construction so far
establishes a map from $[0,\infty)\times\mathbb{R}$ onto $E$, by
extending $f$ in the domain of its second polar coordinate.

To complete the rest of the construction, we need to extend the
domain of the first coordinate from the non-negative reals to all of
$\mathbb{R}$. Again using equations (1), we see that if $r<0$, then
$$x=r\cos\theta=-r\cos(\theta+\pi)\qquad \mbox{ and}
\qquad y=r\sin\theta=-r\sin(\theta+\pi).$$ So if we identify
$(r,\theta)$ with $(-r,\theta+\pi)$, we have extended $f$ to
$\widetilde{f}:\mathbb{R}^2\to E$, completing the construction. If
the metric topology is added to $E$, then $\widetilde{f}$ is a
covering map of $E$.

It's possible to define additions via polar coordinates. The usual
way to go about this is to convert the polar coordinates to
Cartesian coordinates using equations (1), add, and then convert the
result back to polar coordinates. The formula for additions in polar
coordinates is messy and do not follow any algebraic expressions
(involving transcendental functions).

Multiplications by a real scalar can be defined similarly. This
time, there is a simple formula: $t(r,\theta)=(tr,\theta)$.
%%%%%
%%%%%
\end{document}
