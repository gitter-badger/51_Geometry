\documentclass[12pt]{article}
\usepackage{pmmeta}
\pmcanonicalname{StrictBetweennessRelation}
\pmcreated{2013-03-22 17:18:56}
\pmmodified{2013-03-22 17:18:56}
\pmowner{Mathprof}{13753}
\pmmodifier{Mathprof}{13753}
\pmtitle{strict betweenness relation}
\pmrecord{7}{39665}
\pmprivacy{1}
\pmauthor{Mathprof}{13753}
\pmtype{Definition}
\pmcomment{trigger rebuild}
\pmclassification{msc}{51G05}
\pmrelated{SomeTheoremsOnStrictBetweennessRelations}

% this is the default PlanetMath preamble.  as your knowledge
% of TeX increases, you will probably want to edit this, but
% it should be fine as is for beginners.

% almost certainly you want these
\usepackage{amssymb}
\usepackage{amsmath}
\usepackage{amsfonts}

% used for TeXing text within eps files
%\usepackage{psfrag}
% need this for including graphics (\includegraphics)
%\usepackage{graphicx}
% for neatly defining theorems and propositions
%\usepackage{amsthm}
% making logically defined graphics
%%%\usepackage{xypic}

% there are many more packages, add them here as you need them

% define commands here
\newtheorem{thm}{Theorem}

\begin{document}
\section{Definition} A \emph{strict betweenness relation} is a betweenness relation that satisfies 
the following axioms:
\begin{itemize}
\item[$O2^{\prime}$] $(p,q,p)\notin B)$ for \emph{each} pair of points $p$ and $q$. 
\item[$O3^{\prime}$] for each $p,q\in A$ such that $p\ne q$, there is an $r\in A$ such that $(p,q,r)\in B$.
\item[$O4^{\prime}$] for each $p,q\in A$ such that $p\ne q$, there is an $r\in A$ such that $(p,r,q)\in B$.
\item[$O5^{\prime}$] if $(p,q,r)\in B$, then $(q,p,r)\notin B$.
\end{itemize}
\section{Remarks}
\begin{itemize}
\item A very simple  example of a strict betweenness relation is the empty set.  
In $\varnothing$, all the conditions  are vacuously satisfied.  
The empty set, in this context, is called the trivial strict betweenness relation.
\item Any strict betweenness relation can be enlarged to a betweenness
relation by including all triples of the forms $(p,p,q),(p,q,p),$ or
$(p,q,q)$.
\item Conversely, any betweenness relation can be reduced
to a strict betweenness relation by removing all triples of the
forms just listed.  However, it is possible that the ``derived''
strict betweenness relation is trivial.
\item From axiom $O2^{\prime}$ we have $(p,p,p) \notin B.$
\end{itemize}
%%%%%
%%%%%
\end{document}
