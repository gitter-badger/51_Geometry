\documentclass[12pt]{article}
\usepackage{pmmeta}
\pmcanonicalname{ProofOfTriangleIncenter}
\pmcreated{2013-03-22 17:12:26}
\pmmodified{2013-03-22 17:12:26}
\pmowner{rm50}{10146}
\pmmodifier{rm50}{10146}
\pmtitle{proof of triangle incenter}
\pmrecord{6}{39529}
\pmprivacy{1}
\pmauthor{rm50}{10146}
\pmtype{Proof}
\pmcomment{trigger rebuild}
\pmclassification{msc}{51M99}

% this is the default PlanetMath preamble.  as your knowledge
% of TeX increases, you will probably want to edit this, but
% it should be fine as is for beginners.

% almost certainly you want these
\usepackage{amssymb}
\usepackage{amsmath}
\usepackage{amsfonts}

% used for TeXing text within eps files
%\usepackage{psfrag}
% need this for including graphics (\includegraphics)
%\usepackage{graphicx}
% for neatly defining theorems and propositions
%\usepackage{amsthm}
% making logically defined graphics
%%%\usepackage{xypic}
\usepackage{pstricks}

% there are many more packages, add them here as you need them

% define commands here

\begin{document}
In $\triangle ABC$ and construct bisectors of the angles at $A$ and $C$, intersecting at $O$\footnote{Note that the angle bisectors must intersect by Euclid's Postulate 5, which states that ``if a straight line falling on two straight lines makes the interior angles on the same side less than two right angles, the two straight lines, if produced indefinitely, meet on that side on which are the angles less than the two right angles.'' They must meet inside the triangle by considering which side of $AB$ and $CB$ they fall on.} Draw $BO$. We show that $BO$ bisects the angle at $B$, and that $O$ is in fact the incenter of $\triangle ABC$.
% Generated by eukleides 1.0.3
\begin{center}
\begin{pspicture*}(-2.0000,-2.0000)(8.0000,6.0000)
\rput(-3,-2){.}
\rput(9,6){.}
\uput{0.3000}[0.0000](2.5292,1.4169){O}
\pspolygon(0.0000,0.0000)(6.0000,0.0000)(2.2500,3.6742)
\uput{0.3000}[-180.0000](0.0000,0.0000){A}
\uput{0.3000}[0.0000](6.0000,0.0000){B}
\uput{0.3000}[90.0000](2.2500,3.6742){\PMlinkescapetext{C}}
\psline[linestyle=dotted](-2.0000,-1.1205)(8.0000,4.4819)
\psline[linestyle=dotted](8.0000,-0.8165)(-2.0000,3.2660)
\psline[linestyle=dotted](2.9519,-2.0000)(1.9623,6.0000)
\psarc(0.0000,0.0000){0.5000}{29.2589}{58.5178}
\psline(0.3063,0.2946)(0.4144,0.3986)
\psarc(0.0000,0.0000){0.5000}{0.0000}{29.2589}
\psline(0.4112,0.1073)(0.5564,0.1452)
\psdots[dotstyle=*, dotscale=1.2](2.1701,3.3052)
\psarc(2.2500,3.6742){0.5000}{-121.4822}{-82.9487}
\psdots[dotstyle=*, dotscale=1.2](2.4174,3.3358)
\psarc(2.2500,3.6742){0.5000}{-82.9487}{-44.4153}
\pscircle(2.5292,1.4169){1.4169}
\uput{0.3000}[45.0000](3.5209,2.4290){D}
\uput{0.3000}[135.0000](1.3208,2.1569){E}
\uput{0.3000}[-90.0000](2.5292,0.0000){F}
\psline(2.5292,1.4169)(3.5209,2.4290)
\psline(2.5292,1.4169)(1.3208,2.1569)
\psline(2.5292,1.4169)(2.5292,0.0000)
\psline(3.6280,2.3241)(3.5230,2.2169)(3.4159,2.3219)
\psline(1.3992,2.2848)(1.5271,2.2065)(1.4488,2.0786)
\psline(2.3792,0.0000)(2.3792,0.1500)(2.5292,0.1500)
\end{pspicture*}
\end{center}
% End of figure
Drop perpendiculars from $O$ to each of the three sides, intersecting the sides in $D$, $E$, and $F$. Clearly, by AAS, $\triangle COD \cong \triangle COE$ and also $\triangle AOE\cong\triangle AOF$. Thus $FO=EO=DO$. It follows that $O$ is the incenter of $\triangle ABC$ since its distance from all three sides is equal.

Also, since $FO=DO$ we see that $\triangle BOF$ and $\triangle BOD$ are right triangles with two equal sides, so by SSA (which is applicable for right triangles), $\triangle BOF\cong\triangle BOD$. Thus $BO$ bisects $\angle ABC$.
\end{document}
%%%%%
%%%%%
\end{document}
