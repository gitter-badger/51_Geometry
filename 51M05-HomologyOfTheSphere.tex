\documentclass[12pt]{article}
\usepackage{pmmeta}
\pmcanonicalname{HomologyOfTheSphere}
\pmcreated{2013-03-22 13:46:49}
\pmmodified{2013-03-22 13:46:49}
\pmowner{mathcam}{2727}
\pmmodifier{mathcam}{2727}
\pmtitle{homology of the sphere}
\pmrecord{14}{34490}
\pmprivacy{1}
\pmauthor{mathcam}{2727}
\pmtype{Derivation}
\pmcomment{trigger rebuild}
\pmclassification{msc}{51M05}
%\pmkeywords{cohomology}
%\pmkeywords{sphere}
\pmrelated{sphere}
\pmrelated{HomologyTopologicalSpace}
\pmrelated{Sphere}

\endmetadata

% this is the default PlanetMath preamble.  as your knowledge
% of TeX increases, you will probably want to edit this, but
% it should be fine as is for beginners.

% almost certainly you want these
\usepackage{amssymb}
\usepackage{amsmath}
\usepackage{amsfonts}
\usepackage{amsthm}

% used for TeXing text within eps files
%\usepackage{psfrag}
% need this for including graphics (\includegraphics)
%\usepackage{graphicx}
% for neatly defining theorems and propositions
%\usepackage{amsthm}
% making logically defined graphics
%%%\usepackage{xypic}

% there are many more packages, add them here as you need them

% define commands here

\newcommand{\mc}{\mathcal}
\newcommand{\mb}{\mathbb}
\newcommand{\mf}{\mathfrak}
\newcommand{\ol}{\overline}
\newcommand{\ra}{\rightarrow}
\newcommand{\la}{\leftarrow}
\newcommand{\La}{\Leftarrow}
\newcommand{\Ra}{\Rightarrow}
\newcommand{\nor}{\vartriangleleft}
\newcommand{\Gal}{\text{Gal}}
\newcommand{\GL}{\text{GL}}
\newcommand{\Z}{\mb{Z}}
\newcommand{\R}{\mb{R}}
\newcommand{\Q}{\mb{Q}}
\newcommand{\C}{\mb{C}}
\newcommand{\<}{\langle}
\renewcommand{\>}{\rangle}
\begin{document}
Every loop on the sphere $S^2$ is contractible to a point, so its fundamental group, $\pi_1(S^2)$, is trivial.

Let $H_n(S^2,\Z)$ denote the $n$-th homology group of $S^2$.  We can compute all of these groups using the basic results from algebraic topology:
\begin{itemize} 
\item $S^2$ is a compact orientable smooth manifold, so $H_2(S^2,\Z)=\Z$;
\item $S^2$ is connected, so $H_0(S^2,\Z)=\Z$;
\item $H_1(S^2,\Z)$ is the abelianization of $\pi_1(S^2)$, so it is also trivial;
\item $S^2$ is two-dimensional, so for $k>2$, we have $H_k(S^2,\Z)=0$
\end{itemize}

In fact, this pattern generalizes nicely to higher-dimensional spheres:

\begin{align*}
H_k(S^n, \Z)=
\begin{cases}
\Z&k=0,n\\
0&{\rm else}
\end{cases}
\end{align*}

This also provides the proof that the hyperspheres $S^n$ and $S^m$ are non-homotopic for $n\neq m$, for this would imply an isomorphism between their homologies.
%%%%%
%%%%%
\end{document}
