\documentclass[12pt]{article}
\usepackage{pmmeta}
\pmcanonicalname{CircumferentialAngleIsHalfTheCorrespondingCentralAngle}
\pmcreated{2013-03-22 17:13:28}
\pmmodified{2013-03-22 17:13:28}
\pmowner{rm50}{10146}
\pmmodifier{rm50}{10146}
\pmtitle{circumferential angle is half the corresponding central angle}
\pmrecord{14}{39551}
\pmprivacy{1}
\pmauthor{rm50}{10146}
\pmtype{Theorem}
\pmcomment{trigger rebuild}
\pmclassification{msc}{51M04}
\pmrelated{AngleOfViewOfALineSegment}
\pmrelated{RiemannSphere}
\pmdefines{circumferential angle}
\pmdefines{central angle}

\endmetadata

% this is the default PlanetMath preamble.  as your knowledge
% of TeX increases, you will probably want to edit this, but
% it should be fine as is for beginners.

% almost certainly you want these
\usepackage{amssymb}
\usepackage{amsmath}
\usepackage{amsfonts}

% used for TeXing text within eps files
%\usepackage{psfrag}
% need this for including graphics (\includegraphics)
%\usepackage{graphicx}
% for neatly defining theorems and propositions
\usepackage{amsthm}
% making logically defined graphics
%%%\usepackage{xypic}
\usepackage{pstricks}
% there are many more packages, add them here as you need them

% define commands here
\newtheorem{thm}{Theorem}
\newtheorem{cor}[thm]{Corollary}
\newtheorem{lem}[thm]{Lemma}
\newtheorem{prop}[thm]{Proposition}
\newtheorem{ax}{Axiom}
\begin{document}
Consider a circle with center $O$ and two distinct points on the circle $A$ and $B$. If $C$ is a third point on the circle not equal to either $A$ or $B$, then the \emph{circumferential angle} at $C$ subtending the arc $AB$ is the angle $ACB$. Here, by arc $AB$, we mean the arc of the circle that does not contain the points $C$.

Similarly, the \emph{central angle} subtending arc $AB$ is the angle $AOB$. The central angle corresponds to the arc $AB$ measured on the same side of the circle as the angle itself. Note that if $AB$ is a diameter of the circle, then the central angle is $180^{\circ}$.

\begin{thm} \emph{[Euclid, Book III, Prop. 20]} In any circle, a circumferential angle is half the size of the central angle subtending the same arc.
\end{thm}

\begin{proof}
There are actually several distinct cases. Consider $\angle BAC$ in a circle with center $O$, and draw $AO, BO, CO$ as well as the chord containing both $A$ and $O$:
\begin{center}
%--eukleides
%frame(-1,-2,5,3.2)
%A B C triangle(4.2); draw(A,B,C)
%c=circle(A,B,C); draw(c)
%draw(A); draw("$A$",A,-180:)
%draw(B); draw("$B$",B,0:)
%draw(C); draw("$C$",C,90:)
%O=center(c); draw(O); draw("$O$",O,140:)
%F' F intersection(line(A,O),c)
%draw(segment(F',F))
%draw(F); draw("$F$",F,0:)
%a=segment(O,A)
%b=segment(O,B); draw(b)
%c=segment(O,C); draw(c)
%mark(a); mark(b); mark(c)
%--end
% Generated by eukleides 1.0.3
\begin{pspicture*}(-1.0000,-2.0000)(5.0000,3.2000)
\rput(-1.1,-2){.}
\rput(5,4){.}
\pspolygon(0.0000,0.0000)(4.2000,0.0000)(1.5750,2.5720)
\pscircle(2.1000,0.4822){2.1547}
\psdots[dotstyle=*, dotscale=1.0000](0.0000,0.0000)
\uput{0.3000}[-180.0000](0.0000,0.0000){$A$}
\psdots[dotstyle=*, dotscale=1.0000](4.2000,0.0000)
\uput{0.3000}[0.0000](4.2000,0.0000){$B$}
\psdots[dotstyle=*, dotscale=1.0000](1.5750,2.5720)
\uput{0.3000}[90.0000](1.5750,2.5720){$C$}
\psdots[dotstyle=*, dotscale=1.0000](2.1000,0.4822)
\uput{0.3000}[140.0000](2.1000,0.4822){$O$}
\psline(0.0000,0.0000)(4.2000,0.9645)
\psdots[dotstyle=*, dotscale=1.0000](4.2000,0.9645)
\uput{0.3000}[0.0000](4.2000,0.9645){$F$}
\psline(2.1000,0.4822)(4.2000,0.0000)
\psline(2.1000,0.4822)(1.5750,2.5720)
\psline(1.0836,0.0949)(1.0164,0.3873)
\psline(3.1836,0.3873)(3.1164,0.0949)
\psline(1.6920,1.4906)(1.9830,1.5637)
\end{pspicture*}
% End of figure
\end{center}

In this case, the center of the circle lies between the arms of the circumferential angle. Now, since $AO=OB$, $\triangle AOB$ is isosceles, and $\angle FOB$ is an exterior angle. Thus
\[\angle FOB=\angle OAB + \angle OBA = 2\angle OAB\]
Similarly, $\triangle AOC$ is isosceles, and
\[\angle FOC=\angle OAC + \angle OCA = 2\angle OAC\]
and it follows that
\[\angle BOC=\angle FOB + \angle FOC = 2\angle OAB + 2\angle OAC = 2\angle BAC\]
proving the result.

A second case is the case in which both arms of the angle lie to one side of the circle's center:
\begin{center}
% Generated by eukleides 1.0.3
%--eukleides
%frame(-1,-3,5,2)
%A B C triangle(4.2,30:,40:); draw(A,B,C)
%c=circle(A,B,C); draw(c)
%draw(A); draw("$A$",A,-180:)
%draw(B); draw("$B$",B,0:)
%draw(C); draw("$C$",C,90:)
%O=center(c); draw(O); draw("$O$",O,-45:)
%F' F intersection(line(A,O),c)
%draw(segment(F',F))
%draw(F); draw("$F$",F,0:)
%a=segment(O,A)
%b=segment(O,B); draw(b)
%c=segment(O,C); draw(c)
%mark(a); mark(b); mark(c)
%--end
\begin{pspicture*}(-1.0000,-3.0000)(5.0000,2.0000)
\rput(-1.1,-3){.}
\rput(5,2.1){.}
\pspolygon(0.0000,0.0000)(4.2000,0.0000)(2.4881,1.4365)
\pscircle(2.1000,-0.7643){2.2348}
\psdots[dotstyle=*, dotscale=1.0000](0.0000,0.0000)
\uput{0.3000}[-180.0000](0.0000,0.0000){$A$}
\psdots[dotstyle=*, dotscale=1.0000](4.2000,0.0000)
\uput{0.3000}[0.0000](4.2000,0.0000){$B$}
\psdots[dotstyle=*, dotscale=1.0000](2.4881,1.4365)
\uput{0.3000}[90.0000](2.4881,1.4365){$C$}
\psdots[dotstyle=*, dotscale=1.0000](2.1000,-0.7643)
\uput{0.3000}[-45.0000](2.1000,-0.7643){$O$}
\psline(0.0000,0.0000)(4.2000,-1.5287)
\psdots[dotstyle=*, dotscale=1.0000](4.2000,-1.5287)
\uput{0.3000}[0.0000](4.2000,-1.5287){$F$}
\psline(2.1000,-0.7643)(4.2000,0.0000)
\psline(2.1000,-0.7643)(2.4881,1.4365)
\psline(0.9987,-0.5231)(1.1013,-0.2412)
\psline(3.0987,-0.2412)(3.2013,-0.5231)
\psline(2.1463,0.3621)(2.4418,0.3100)
\end{pspicture*}
% End of figure
\end{center}
The proof is similar to the previous case, except that the angle in question is the difference rather than the sum of two known angles. Here we see that both $\triangle AOB$ and $\triangle AOC$ are isosceles, so that again
\begin{align*}
\angle COF &= 2\angle OAC \\
\angle BOF &= 2\angle OAB
\end{align*}
Subtracting, we get
\[\angle COB = \angle COF - \angle BOF = 2\angle OAC -2\angle OAB = 2\angle BAC\]
as desired.

The final case is the case in which one arm of the angle goes through the center of the circle. This is a degenerate form of the first case, and the same proof follows through except that one of the angles is zero.
\end{proof}
%%%%%
%%%%%
\end{document}
