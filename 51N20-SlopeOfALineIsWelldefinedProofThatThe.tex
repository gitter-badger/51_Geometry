\documentclass[12pt]{article}
\usepackage{pmmeta}
\pmcanonicalname{SlopeOfALineIsWelldefinedProofThatThe}
\pmcreated{2013-03-22 15:32:43}
\pmmodified{2013-03-22 15:32:43}
\pmowner{Dr_Absentius}{537}
\pmmodifier{Dr_Absentius}{537}
\pmtitle{slope of a line is well-defined, proof that the}
\pmrecord{8}{37442}
\pmprivacy{1}
\pmauthor{Dr_Absentius}{537}
\pmtype{Proof}
\pmcomment{trigger rebuild}
\pmclassification{msc}{51N20}

%\documentclass{amsart}
\usepackage{amsmath}
%\usepackage[all,poly,knot,dvips]{xy}
\usepackage{pstricks,pst-poly,pst-node}
\usepackage{amssymb}
\usepackage{amsthm}
\usepackage{eucal,latexsym}

% THEOREM Environments --------------------------------------------------

\newtheorem{thm}{Theorem}
 \newtheorem*{mainthm}{Main~Theorem}
 \newtheorem{cor}[thm]{Corollary}
 \newtheorem{lem}[thm]{Lemma}
 \newtheorem{prop}[thm]{Proposition}
 \newtheorem{claim}[thm]{Claim}
 \theoremstyle{definition}
 \newtheorem{defn}[thm]{Definition}
 \theoremstyle{remark}
 \newtheorem{rem}[thm]{Remark}
 %\numberwithin{equation}{subsection}


%---------------------  Greek letters, etc ------------------------- 

\newcommand{\CA}{\mathcal{A}}
\newcommand{\CC}{\mathcal{C}}
\newcommand{\CM}{\mathcal{M}}
\newcommand{\CP}{\mathcal{P}}
\newcommand{\CS}{\mathcal{S}}
\newcommand{\BC}{\mathbb{C}}
\newcommand{\BN}{\mathbb{N}}
\newcommand{\BR}{\mathbb{R}}
\newcommand{\BZ}{\mathbb{Z}}
\newcommand{\FF}{\mathfrak{F}}
\newcommand{\FL}{\mathfrak{L}}
\newcommand{\FM}{\mathfrak{M}}
\newcommand{\Ga}{\alpha}
\newcommand{\Gb}{\beta}
\newcommand{\Gg}{\gamma}
\newcommand{\GG}{\Gamma}
\newcommand{\Gd}{\delta}
\newcommand{\GD}{\Delta}
\newcommand{\Ge}{\varepsilon}
\newcommand{\Gz}{\zeta}
\newcommand{\Gh}{\eta}
\newcommand{\Gq}{\theta}
\newcommand{\GQ}{\Theta}
\newcommand{\Gi}{\iota}
\newcommand{\Gk}{\kappa}
\newcommand{\Gl}{\lambda}
\newcommand{\GL}{\Lamda}
\newcommand{\Gm}{\mu}
\newcommand{\Gn}{\nu}
\newcommand{\Gx}{\xi}
\newcommand{\GX}{\Xi}
\newcommand{\Gp}{\pi}
\newcommand{\GP}{\Pi}
\newcommand{\Gr}{\rho}
\newcommand{\Gs}{\sigma}
\newcommand{\GS}{\Sigma}
\newcommand{\Gt}{\tau}
\newcommand{\Gu}{\upsilon}
\newcommand{\GU}{\Upsilon}
\newcommand{\Gf}{\varphi}
\newcommand{\GF}{\Phi}
\newcommand{\Gc}{\chi}
\newcommand{\Gy}{\psi}
\newcommand{\GY}{\Psi}
\newcommand{\Gw}{\omega}
\newcommand{\GW}{\Omega}
\newcommand{\Gee}{\epsilon}
\newcommand{\Gpp}{\varpi}
\newcommand{\Grr}{\varrho}
\newcommand{\Gff}{\phi}
\newcommand{\Gss}{\varsigma}

\def\co{\colon\thinspace}
\begin{document}
The purpose of this article is to prove that for a non-vertical line $l$
in a the Cartesian plane its slope $m$  is a well-defined  real number.
Let $P_i=(x_i,y_i)$, $i=1,2$ be two points in the Cartesian plane
with the property that $x_2 \neq x_1$; denote by $m(P_1,P_2)$ the
following ratio 
 $$m(P_1,P_2) := \frac{y_2 - y_1}{x_2 - x_1}\,.$$  

\begin{thm}
  Let $l$ be a a non-vertical line in the Cartesian plane.
  Then for any two pairs of distinct points $P_1,P_2$  and
  $P_1',P_2'$ in $l$ we have 
       $$m(P_1,P_2) = m(P_1',P_2')\,.$$
\end{thm}

\begin{proof}
  Let $P_1 = (x_1,y_1)$, $P_2 = (x_2,y_2)$ and $P_1' = (x_1',y_1')$,
  $P_2' = (x_2',y_2')$ be two arbitrary pairs of points in $l$.
  Draw a vertical line $v$ (respectively $v'$) from $P_2$ (resp.
  $P_2'$) and a horizontal line $h$ (resp. $h'$) from $P_1$ (resp.
  $P_1'$) and let $Q$ (resp. $Q'$) denote the point where these two
  lines meet. If we temporally assume that $l$ is also not
  horizontal we have two nondegenerate right triangles $QP_1P_2$
  and $Q'P_1'P_2'$ (see Figure~\ref{fig:slopewelldefnd}).

  \begin{figure}[htbp]
    \centering
    \begin{pspicture}(-5,-5)(5.5,5.5)
      \psline[linewidth=.5pt]{->}(-5,0)(5,0)
      \psline[linewidth=.5pt]{->}(0,-5)(0,5)
\rput(5.3,0){$\mathbf{x}$}
\rput(0,5.3){$\mathbf{y}$}
      \psline[linecolor=blue](-4,-5)(3,5)
\psdots(-3,-3.571)(-1,-0.714)(1,2.143)(2,3.571)(2,2.143)(-1,-3.571)
\psline[linestyle=dotted,linecolor=green](-1,-5)(-1,5)
\psline[linestyle=dotted,linecolor=green](2,-5)(2,5)
\psline[linestyle=dotted,linecolor=red](-5,-3.571)(5,-3.571)
\psline[linestyle=dotted,linecolor=red](-5,2.143)(5,2.143)
 \psline[linecolor=green](-1,-0.714)(-1,-3.571)
 \psline[linecolor=green](2,3.571)(2,2.143)
 \psline[linecolor=red](-3,-3.571)(-1,-3.571)
 \psline[linecolor=red](1,2.143)(2,2.143)
\rput(-1.3,-0.514){$P_2$}
\rput(-3.3,-3.371){$P_1$}
\rput(-.7,-3.871){$Q$}
\rput(1.7,3.871){$P_2'$}
\rput(.7,2.443){$P_1'$}
\rput(2.3,1.943){$Q'$}
\rput(3.3,5){ $l$}
\rput(5,2.443){ $v'$}
\rput(5,-3.271){ $v$}
\rput(1.7,5){ $h'$}
\rput(-1.3,5){ $h$}
    \end{pspicture}
    \caption{The slope is well defined.}
    \label{fig:slopewelldefnd}
  \end{figure}

These two triangles are similar and therefore we have that 
$$\frac{QP_2}{QP_1} = \frac{Q'P_2'}{Q'P_1'} $$
or equivalently since $Q$ (resp. $Q'$) has coordinates $(x_2,y_1)$ 
(resp. $(x_2',y_1')$)
$$  \frac{\vert y_2-y_1 \vert }{\vert x_2-x_1 \vert} = \frac{\vert y_2'-y_1'\vert }{\vert x_2'-x_1'\vert}\,.$$
To conclude that 
\begin{equation}
  \label{eq:slopewelldfnd}
  \frac{y_2-y_1}{x_2-x_1} = \frac{y_2'-y_1'}{x_2'-x_1'}
\end{equation}
we argue as follows.  First notice that $m(P_1,P_2)$ is symmetric
with respect to $P_1$ and $P_2$ and therefore without loss of
generality (exchanging the roles of $P_1$ and $P_2$ if necessary) we
may assume that both denominators in~\eqref{eq:slopewelldfnd} are
positive, that is $Q$ (resp. $Q'$) is to the right of $P_1$ (resp.
$P_1'$).  We will show that then $P_2$ is above $Q$ exactly when
$P_2'$ is above $Q'$.  To see this notice that by the fifth
postulate of Euclid $P_2$ is above $Q$ exactly when the angle
$QP_1P_2$ is acute, but this happens exactly when $Q'P_1'P_2'$ is
acute, that is, exactly when $P_2'$ is above $Q'$.  Thus the numerators
of the two fractions in~\eqref{eq:slopewelldfnd} have the same sign.
Thus Equation~\eqref{eq:slopewelldfnd} holds.

We derived Equation~\eqref{eq:slopewelldfnd} by assuming that the
line $l$ is not horizontal (and therefore $P_1$ and $Q$ are
distinct).  However when $l$ is horizontal then the numerators of
both fractions in Equation~\eqref{eq:slopewelldfnd} are $0$ and the
equation still holds.  Thus the theorem is true for any non-vertical
line.
\end{proof}

We can dispense with the assumption that $l$ is not vertical by
letting the slope to take values not in an affine line but in a
\emph{projective} line.  In other words, we may extend the real line 
by adjoining one point at infinity: 
$$\hat{ \mathbb{R} } = \mathbb{R}\cup \{ \infty \}$$
and define the slope to take values in $\hat{\mathbb{R}}$, so that
the slope of a vertical line is $\infty$.
%%%%%
%%%%%
\end{document}
