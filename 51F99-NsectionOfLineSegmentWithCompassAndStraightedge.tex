\documentclass[12pt]{article}
\usepackage{pmmeta}
\pmcanonicalname{NsectionOfLineSegmentWithCompassAndStraightedge}
\pmcreated{2013-03-22 17:24:41}
\pmmodified{2013-03-22 17:24:41}
\pmowner{pahio}{2872}
\pmmodifier{pahio}{2872}
\pmtitle{$n$-section of line segment with compass and straightedge}
\pmrecord{12}{39784}
\pmprivacy{1}
\pmauthor{pahio}{2872}
\pmtype{Algorithm}
\pmcomment{trigger rebuild}
\pmclassification{msc}{51F99}
\pmclassification{msc}{51M05}
\pmclassification{msc}{51-00}
\pmrelated{CompassAndStraightedgeConstructionOfParallelLine}

\endmetadata

% this is the default PlanetMath preamble.  as your knowledge
% of TeX increases, you will probably want to edit this, but
% it should be fine as is for beginners.

% almost certainly you want these
\usepackage{amssymb}
\usepackage{amsmath}
\usepackage{amsfonts}
\usepackage{pstricks}

% used for TeXing text within eps files
%\usepackage{psfrag}
% need this for including graphics (\includegraphics)
%\usepackage{graphicx}
% for neatly defining theorems and propositions
 \usepackage{amsthm}
% making logically defined graphics
%%\usepackage{xypic}

% there are many more packages, add them here as you need them

% define commands here

\theoremstyle{definition}
\newtheorem*{thmplain}{Theorem}

\begin{document}
\textbf{Task.}  Let $AB$ be a given line segment and $n$ a positive integer $> 1$.  Divide $AB$ to $n$ equal parts.

\textbf{\PMlinkescapetext{Solution}.}  Draw a half-line $p$ beginning from $A$ but not parallel to $AB$.  From $p$ separate $n$ consecutive equally long segments $AA_1$, $A_1A_2$, $A_2A_3$, \ldots, $A_{n-1}A_n$.  Draw the line  $A_nB$ and denote by $B_1$, $B_2$, \ldots, $B_{n-1}$ the points of $AB$ such that
    $$A_1B_1\;\parallel\;A_2B_2\;\parallel\;\ldots\;\parallel\;A_{n-1}B_{n-1}\;\parallel\;A_nB$$
(see compass and straightedge construction of parallel line).
These points divide the line segment $AB$ in $n$ equal segments.

{\em Proof.}  For clarity, we prove the theorem only in the case\, $n = 3$.  
\begin{center}
\begin{pspicture}(-0.5,-0.5)(6.5,4)
\psline[linecolor=blue](0,0)(6,0)
\psline(0,0)(6.25,4)
\psline(1.25,0.8)(2,0)
\psline(2.5,1.6)(4,0)
\psline(3.75,2.4)(6,0)
\rput[l](6.33,3.9){$p$}
\rput[l](-0.15,-0.2){$A$}
\rput[l](6,-0.2){$B$}
\rput[l](2,-0.3){$B_1$}
\rput[l](4,-0.3){$B_2$}
\rput[l](0.9,1.05){$A_1$}
\rput[l](2.1,1.85){$A_2$}
\rput[l](3.4,2.7){$A_3$}
\end{pspicture}
\end{center}
The line $AB$ intersects the parallel lines $A_1B_1$, $A_2B_2$ and $A_3B$, and thus the \PMlinkname{corresponding angles}{CorrespondingAnglesInTransversalCutting} $A_1B_1A$, $A_2B_2A$ and $A_3BA$ are equal.  Similarly the angles $AA_1B_1$, $AA_2B_2$ and $AA_3B$ are equal.  Because of the equal angles, the triangle $AA_2B_2$ is similar to the triangle $AA_3B$ with the ratio of similarity $2\!:\!3$.  Therefore
$$AB_2 = \frac{2}{3}AB;\quad B_2B = \frac{1}{3}AB.$$
Also the triangle $AA_1B_1$ is similar to the triangle $AA_3B$ with the line ratio $1\!:\!3$, whence
$$AB_1 = \frac{1}{3}AB;\quad B_1B_2 = \frac{1}{3}AB.$$
The equations show that the points $B_1$ and $B_2$ divide the line segment $AB$ in 3 equal segments.
%%%%%
%%%%%
\end{document}
