\documentclass[12pt]{article}
\usepackage{pmmeta}
\pmcanonicalname{CylindricalCoordinates}
\pmcreated{2013-03-22 17:01:54}
\pmmodified{2013-03-22 17:01:54}
\pmowner{Wkbj79}{1863}
\pmmodifier{Wkbj79}{1863}
\pmtitle{cylindrical coordinates}
\pmrecord{6}{39317}
\pmprivacy{1}
\pmauthor{Wkbj79}{1863}
\pmtype{Definition}
\pmcomment{trigger rebuild}
\pmclassification{msc}{51M05}
\pmrelated{PolarCoordinates}
\pmrelated{SphericalCoordinates}

\usepackage{amssymb}
\usepackage{amsmath}
\usepackage{amsfonts}

\usepackage{psfrag}
\usepackage{graphicx}
\usepackage{amsthm}
%%\usepackage{xypic}

\begin{document}
\emph{Cylindrical coordinates} are a system of coordinates for $\mathbb{R}^3$.  Two of the coordinates correspond to the polar coordinates of $\mathbb{R}^2$, and the third coordinate corresponds with the $z$ axis.  Thus, the coordinates are given by

$$\left( \begin{array}{c}
x \\
y \\
z \end{array} \right)=\left( \begin{array}{c}
r \cos \theta \\
r \sin \theta \\
z \end{array} \right),$$

where $r$ is the distance from $(0,0,0)$ to $(x,y,0)$ and $\theta$ is the azimuthal angle defined for $\theta \in [0,2\pi )$.

Just as with polar coordinates, one can convert from Cartesian coordinates to cylindrical coordinates for any point not lying on the $z$ axis via

\begin{eqnarray*}
r(x,y) &=& \sqrt{x^2+ y^2}, \\
\theta(x,y) &=& \arctan (x,y),
\end{eqnarray*}

where $\arctan$ is defined \PMlinkname{here}{OperatornamearcTanWithTwoArguments}.
%%%%%
%%%%%
\end{document}
