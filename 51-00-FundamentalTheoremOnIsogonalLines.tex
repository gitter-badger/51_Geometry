\documentclass[12pt]{article}
\usepackage{pmmeta}
\pmcanonicalname{FundamentalTheoremOnIsogonalLines}
\pmcreated{2013-03-22 13:01:16}
\pmmodified{2013-03-22 13:01:16}
\pmowner{drini}{3}
\pmmodifier{drini}{3}
\pmtitle{fundamental theorem on isogonal lines}
\pmrecord{4}{33407}
\pmprivacy{1}
\pmauthor{drini}{3}
\pmtype{Theorem}
\pmcomment{trigger rebuild}
\pmclassification{msc}{51-00}
\pmrelated{Isogonal}
\pmrelated{IsogonalConjugate}
\pmrelated{LemoinePoint}
\pmrelated{Symmedian}
\pmrelated{Triangle}

\endmetadata

\usepackage{graphicx}
%%%\usepackage{xypic} 
\usepackage{bbm}
\newcommand{\Z}{\mathbbmss{Z}}
\newcommand{\C}{\mathbbmss{C}}
\newcommand{\R}{\mathbbmss{R}}
\newcommand{\Q}{\mathbbmss{Q}}
\newcommand{\mathbb}[1]{\mathbbmss{#1}}
\newcommand{\figura}[1]{\begin{center}\includegraphics{#1}\end{center}}
\newcommand{\figuraex}[2]{\begin{center}\includegraphics[#2]{#1}\end{center}}
\begin{document}
Let $\triangle ABC$ be a triangle and $AX, BY, CZ$ three concurrent lines at $P$.
If $AX',BY', CZ'$ are the respective isogonal conjugate lines for $AX,BY,CZ$, then $AX', BY',CZ'$ are also concurrent at some point $P'$.

An applications of this theorem proves the existence of Lemoine point (for it is the intersection point of the symmedians):
\figura{lemoinep}

This theorem is a direct consequence of Ceva's theorem (trigonometric version).
%%%%%
%%%%%
\end{document}
