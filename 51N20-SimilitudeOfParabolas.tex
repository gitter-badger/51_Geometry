\documentclass[12pt]{article}
\usepackage{pmmeta}
\pmcanonicalname{SimilitudeOfParabolas}
\pmcreated{2013-03-22 18:51:04}
\pmmodified{2013-03-22 18:51:04}
\pmowner{pahio}{2872}
\pmmodifier{pahio}{2872}
\pmtitle{similitude of parabolas}
\pmrecord{8}{41660}
\pmprivacy{1}
\pmauthor{pahio}{2872}
\pmtype{Theorem}
\pmcomment{trigger rebuild}
\pmclassification{msc}{51N20}
\pmclassification{msc}{51N10}
\pmsynonym{similarity of parabolas}{SimilitudeOfParabolas}
\pmrelated{Homothety}

\endmetadata

% this is the default PlanetMath preamble.  as your knowledge
% of TeX increases, you will probably want to edit this, but
% it should be fine as is for beginners.

% almost certainly you want these
\usepackage{amssymb}
\usepackage{amsmath}
\usepackage{amsfonts}

% used for TeXing text within eps files
%\usepackage{psfrag}
% need this for including graphics (\includegraphics)
%\usepackage{graphicx}
% for neatly defining theorems and propositions
 \usepackage{amsthm}
% making logically defined graphics
%%%\usepackage{xypic}

% there are many more packages, add them here as you need them

% define commands here

\theoremstyle{definition}
\newtheorem*{thmplain}{Theorem}

\begin{document}
Two parabolas need not be congruent, but they are always similar.\, Without the definition of parabola by focus and directrix, the fact turns out of the simplest equation \,$y = ax^2$\, of parabola.

Let us take two parabolas
$$y \;=\; ax^2 \quad \mbox{and} \quad y \;=\; bx^2$$
which have the origin as common vertex and the $y$-axis as common axis.\, Cut the parabolas with the line \,$y = mx$\, through the vertex.\, The first parabola gives
$$ax^2 = mx,$$
whence the abscissa of the other point of intersection is $\frac{m}{a}$; the corresponding ordinate is thus
$\frac{m^2}{a}$.\, So, this point has the position vector
$$\vec{u} \;=\; \left(\!\begin{array}{c}\frac{m}{a}\\\frac{m^2}{a}\end{array}\!\right)
\;=\; \frac{m}{a}\left(\!\begin{array}{c}1\\m\end{array}\!\right)$$
Similarly, the cutting point of the line and the second parabola has the position vector
$$\vec{v} \;=\; \left(\!\begin{array}{c}\frac{m}{b}\\\frac{m^2}{b}\end{array}\!\right)
\;=\; \frac{m}{b}\left(\!\begin{array}{c}1\\m\end{array}\!\right)$$
Accordingly, those position vectors have the \PMlinkid{linear depencence}{848}
$$a\vec{u} \;=\; b\vec{v}$$
for all values of the slope $m$ of the cutting line.\, This means that both parabolas are homothetic with respect to the origin and therefore also similar. 


%%%%%
%%%%%
\end{document}
