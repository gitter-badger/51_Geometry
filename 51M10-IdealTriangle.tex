\documentclass[12pt]{article}
\usepackage{pmmeta}
\pmcanonicalname{IdealTriangle}
\pmcreated{2013-03-22 17:08:26}
\pmmodified{2013-03-22 17:08:26}
\pmowner{Wkbj79}{1863}
\pmmodifier{Wkbj79}{1863}
\pmtitle{ideal triangle}
\pmrecord{5}{39447}
\pmprivacy{1}
\pmauthor{Wkbj79}{1863}
\pmtype{Definition}
\pmcomment{trigger rebuild}
\pmclassification{msc}{51M10}
\pmclassification{msc}{51-00}
\pmrelated{LimitingTriangle}

\endmetadata

\usepackage{amssymb}
\usepackage{amsmath}
\usepackage{amsfonts}
\usepackage{pstricks}
\usepackage{psfrag}
\usepackage{graphicx}
\usepackage{amsthm}
%%\usepackage{xypic}

\begin{document}
In hyperbolic geometry, an \emph{ideal triangle} is a set of three lines which connect three distinct points on the boundary of the model of hyperbolic geometry.

Below is an example of an ideal triangle in the Beltrami-Klein model:

\begin{center}
\begin{pspicture}(-2,-2)(2,2)
\pscircle[linestyle=dashed](0,0){2}
\psline{o-o}(-1.732,-1)(0,2)
\psline{o-o}(0,2)(1.732,-1)
\psline{o-o}(-1.732,-1)(1.732,-1)
\end{pspicture}
\end{center}

Below is an example of an ideal triangle in the Poincar\'e disc model:

\begin{center}
\begin{pspicture}(-2,-2)(2,2)
\pscircle[linestyle=dashed](0,0){2}
\psarc{o-o}(0,-4){3.4641}{60}{120}
\psarc{o-o}(-3.4641,2){3.4641}{300}{360}
\psarc{o-o}(3.4641,2){3.4641}{180}{240}
\end{pspicture}
\end{center}

Below are some examples of ideal triangles in the upper half plane model:

\begin{center}
\begin{pspicture}(-2,-0.1)(4,4)
\psline[linestyle=dashed]{<->}(-2,0)(4,0)
\psline{o->}(-1,0)(-1,4)
\psline{o->}(3,0)(3,4)
\psarc{o-o}(1,0){2}{0}{180}
\end{pspicture}
\end{center}

\begin{center}
\begin{pspicture}(-5,-0.1)(5,4)
\psline[linestyle=dashed]{<->}(-5,0)(5,0)
\psarc{o-o}(-2,0){2}{0}{180}
\psarc{o-o}(2,0){2}{0}{180}
\psarc{o-o}(0,0){4}{0}{180}
\end{pspicture}
\end{center}

\PMlinkescapetext{Strictly} speaking, none of these figures are triangles in hyperbolic geometry; however, ideal triangles are useful for proving that, given $r \in \mathbb{R}$ with $0<r<\pi$, there is a triangle in hyperbolic geometry whose angle sum in radians is equal to $r$.
%%%%%
%%%%%
\end{document}
