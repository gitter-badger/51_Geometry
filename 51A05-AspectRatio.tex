\documentclass[12pt]{article}
\usepackage{pmmeta}
\pmcanonicalname{AspectRatio}
\pmcreated{2013-03-22 18:30:19}
\pmmodified{2013-03-22 18:30:19}
\pmowner{PrimeFan}{13766}
\pmmodifier{PrimeFan}{13766}
\pmtitle{aspect ratio}
\pmrecord{5}{41189}
\pmprivacy{1}
\pmauthor{PrimeFan}{13766}
\pmtype{Definition}
\pmcomment{trigger rebuild}
\pmclassification{msc}{51A05}
\pmclassification{msc}{00A06}

% this is the default PlanetMath preamble.  as your knowledge
% of TeX increases, you will probably want to edit this, but
% it should be fine as is for beginners.

% almost certainly you want these
\usepackage{amssymb}
\usepackage{amsmath}
\usepackage{amsfonts}

% used for TeXing text within eps files
%\usepackage{psfrag}

% need this for including graphics (\includegraphics)
\usepackage{graphicx}

% for neatly defining theorems and propositions
%\usepackage{amsthm}
% making logically defined graphics
%%%\usepackage{xypic}

% there are many more packages, add them here as you need them

% define commands here

\begin{document}
The {\em aspect ratio} of a rectangular image display area is the ratio of the width of the rectangle to its height, expressed either as a ratio of coprime integers or a real number to 1. The two most commonly used aspect ratios are ``fullscreen'' (4:3 or 1.66:1) and ``widescreen'' (16:9 OR 2.35:1).

For example, a detail from Leonardo da Vinci's {\it Mona Lisa} in a fullscreen aspect ration:

\begin{center}
\includegraphics[scale=0.5]{MonaLisa_FullScreen}
\end{center}

And in a widescreen aspect ratio:

\begin{center}
\includegraphics[scale=0.25]{MonaLisa_WideScreen}
\end{center}


%%%%%
%%%%%
\end{document}
