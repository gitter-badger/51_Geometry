\documentclass[12pt]{article}
\usepackage{pmmeta}
\pmcanonicalname{LinearFunction}
\pmcreated{2013-03-22 19:14:46}
\pmmodified{2013-03-22 19:14:46}
\pmowner{CWoo}{3771}
\pmmodifier{CWoo}{3771}
\pmtitle{linear function}
\pmrecord{12}{42171}
\pmprivacy{1}
\pmauthor{CWoo}{3771}
\pmtype{Definition}
\pmcomment{trigger rebuild}
\pmclassification{msc}{51A45}
\pmclassification{msc}{51A05}
\pmclassification{msc}{05C65}

\endmetadata

\usepackage{amssymb,amscd}
\usepackage{amsmath}
\usepackage{amsfonts}
\usepackage{mathrsfs}

% used for TeXing text within eps files
%\usepackage{psfrag}
% need this for including graphics (\includegraphics)
%\usepackage{graphicx}
% for neatly defining theorems and propositions
\usepackage{amsthm}
% making logically defined graphics
%%\usepackage{xypic}
\usepackage{pst-plot}

% define commands here
\newcommand*{\abs}[1]{\left\lvert #1\right\rvert}
\newtheorem{prop}{Proposition}
\newtheorem{thm}{Theorem}
\newtheorem{ex}{Example}
\newcommand{\real}{\mathbb{R}}
\newcommand{\pdiff}[2]{\frac{\partial #1}{\partial #2}}
\newcommand{\mpdiff}[3]{\frac{\partial^#1 #2}{\partial #3^#1}}
\begin{document}
Let $\mathscr{S}_1=(\mathcal{P}_1,\mathcal{L}_1)$ and $\mathscr{S}_2=(\mathcal{P}_2,\mathcal{L}_2)$ be two near-linear spaces.  

\textbf{Definition}.  A \emph{linear function} from $\mathscr{S}_1$ to $\mathscr{S}_2$ is a mapping on the points that sends lines of $\mathscr{S}_1$ to lines of $\mathscr{S}_2$.  In other words, a linear function is a function $\sigma: \mathcal{P}_1 \to \mathcal{P}_2$ such that $$\sigma(\ell)\in \mathcal{L}_2\mbox{ for every }\ell \in \mathcal{L}_1.$$  Here, $\sigma(\ell)$ is the set $\lbrace \sigma(P)\mid P\in \ell \rbrace$.  A linear function is also called a homomorphism.

When both $\mathscr{S}_1$ and $\mathscr{S}_2$ are linear spaces, then $\sigma$ being a linear function is equvalent to saying that $P,Q$ are collinear iff $\sigma(P),\sigma(Q)$ are collinear.

If $\mathscr{S}_1$ is a linear space, then so is $(\sigma(\mathcal{P}_1),\sigma(\mathcal{L}_1))$.  This shows that if $\sigma: \mathscr{S}_1\to \mathscr{S}_1$ is onto, $\mathscr{S}_2$ is a linear space if $\mathscr{S}_1$ is.

Let $\sigma: \mathscr{S}_1\to \mathscr{S}_2$ be a one-to-one linear function.  If points $P_1\ne P_2$ lie on line $\ell$, then $\sigma(P_1)\ne \sigma(P_2)$ lie on $\sigma(\ell)$.  This also shows that three collinear points in $\mathscr{S}_1$ are mapped to three collinear points in $\mathscr{S}_2$.  In addition, we have 
\begin{center}
$|\ell|=|\sigma(\ell)|$ for any line $\ell$ in $\mathscr{S}_1$.
\end{center}

\textbf{Definition}.  When $\sigma:\mathscr{S}_1\to \mathscr{S}_2$ is a bijection whose inverse $\sigma^{-1}$ is also linear, we say that $\sigma$ is an isomorphism.  When $\mathscr{S}_1=\mathscr{S}_2=\mathscr{S}$, we call $\sigma$ an automorphism, or more commonly among geometers, a \emph{collineation}, of the space $\mathscr{S}$.

Suppose $\sigma:\mathscr{S}_1\to \mathscr{S}_2$ is an isomorphism.  For every point $P$, let $P^*$ be the set of all lines passing through $P$.  Then
\begin{center}
$|P^*|=|\sigma(P)^*|$ for any point $P$ in $\mathscr{S}_1$.
\end{center}

It is possible to have a bijective linear function whose inverse is not linear.  For example, let $\mathscr{S}_1$ be the space with two points $P,Q$ with no lines, and $\mathscr{S}_2$ the space with the same two points with line $\lbrace P,Q\rbrace$.  Then the identity function on $\lbrace P,Q\rbrace$ is a bijective linear function whose inverse is not linear.  On the other hand, if the both spaces are linear, then the inverse is always linear.

\textbf{Remark}.  The usage of the term ``linear function'' differs from its more usual meaning as a linear transformation between vector spaces in the study of linear algebra.

\begin{thebibliography}{7}
\bibitem{LB} L. M. Batten, {\it Combinatorics of Finite Geometries}, 2nd edition, Cambridge University Press (1997)
\end{thebibliography}
%%%%%
%%%%%
\end{document}
