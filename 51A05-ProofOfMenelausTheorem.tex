\documentclass[12pt]{article}
\usepackage{pmmeta}
\pmcanonicalname{ProofOfMenelausTheorem}
\pmcreated{2013-03-22 12:46:46}
\pmmodified{2013-03-22 12:46:46}
\pmowner{mathwizard}{128}
\pmmodifier{mathwizard}{128}
\pmtitle{proof of Menelaus' theorem}
\pmrecord{4}{33092}
\pmprivacy{1}
\pmauthor{mathwizard}{128}
\pmtype{Proof}
\pmcomment{trigger rebuild}
\pmclassification{msc}{51A05}

\endmetadata

% this is the default PlanetMath preamble.  as your knowledge
% of TeX increases, you will probably want to edit this, but
% it should be fine as is for beginners.

% almost certainly you want these
\usepackage{amssymb}
\usepackage{amsmath}
\usepackage{amsfonts}

% used for TeXing text within eps files
%\usepackage{psfrag}
% need this for including graphics (\includegraphics)
\usepackage{graphicx}
% for neatly defining theorems and propositions
%\usepackage{amsthm}
% making logically defined graphics
%%%\usepackage{xypic}

% there are many more packages, add them here as you need them

% define commands here
\begin{document}
First we note that there are two different cases: Either the line connecting $X$, $Y$ and $Z$ intersects two sides of the triangle or none of them. So in the first case that it intersects two of the triangle's sides we get the following picture:
\begin{center}
\includegraphics{menelaus.eps}
\end{center}
From this we follow ($h_1$, $h_2$ and $h_3$ being undircted):
\begin{eqnarray*}
\frac{AZ}{ZB}& = &-\frac{h_1}{h_2}\\
\frac{BY}{YC} & = & \frac{h_2}{h_3}\\
\frac{CX}{XA} & = & \frac{h_3}{h_1}.
\end{eqnarray*}
Mulitplying all this we get:
$$\frac{AZ}{ZB}\cdot\frac{BY}{YC}\cdot\frac{CX}{XA} = -\frac{h_1h_2h_3}{h_2h_3h_1} = -1.$$

The second case is that the line connecting $X$, $Y$ and $Z$ does not intersect any of the triangle's sides:
\begin{center}
\includegraphics{menelaus2.eps}
\end{center}
In this case we get:
\begin{eqnarray*}
\frac{AZ}{ZB}&=&-\frac{h_1}{h_2}\\
\frac{BY}{YC}&=&-\frac{h_2}{h_3}\\
\frac{CX}{XA}&=&-\frac{h_3}{h_1}.
\end{eqnarray*}
So multiplication again yields Menelaus' theorem.
%%%%%
%%%%%
\end{document}
