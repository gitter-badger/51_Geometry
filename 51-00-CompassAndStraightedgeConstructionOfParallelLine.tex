\documentclass[12pt]{article}
\usepackage{pmmeta}
\pmcanonicalname{CompassAndStraightedgeConstructionOfParallelLine}
\pmcreated{2013-03-22 17:11:18}
\pmmodified{2013-03-22 17:11:18}
\pmowner{pahio}{2872}
\pmmodifier{pahio}{2872}
\pmtitle{compass and straightedge construction of parallel line}
\pmrecord{24}{39506}
\pmprivacy{1}
\pmauthor{pahio}{2872}
\pmtype{Algorithm}
\pmcomment{trigger rebuild}
\pmclassification{msc}{51-00}
\pmclassification{msc}{51M15}
\pmsynonym{construction of parallel}{CompassAndStraightedgeConstructionOfParallelLine}
\pmsynonym{construction of parallel line}{CompassAndStraightedgeConstructionOfParallelLine}
%\pmkeywords{Euclidean geometry}
\pmrelated{ParallelPostulate}
\pmrelated{NSectionOfLineSegmentWithCompassAndStraightedge}

\endmetadata

% this is the default PlanetMath preamble.  as your knowledge
% of TeX increases, you will probably want to edit this, but
% it should be fine as is for beginners.

% almost certainly you want these
\usepackage{amssymb}
\usepackage{amsmath}
\usepackage{amsfonts}
\usepackage{pstricks}
% used for TeXing text within eps files
%\usepackage{psfrag}
% need this for including graphics (\includegraphics)
%\usepackage{graphicx}
% for neatly defining theorems and propositions
 \usepackage{amsthm}
% making logically defined graphics
%%%\usepackage{xypic}

% there are many more packages, add them here as you need them

% define commands here

\theoremstyle{definition}
\newtheorem*{thmplain}{Theorem}

\begin{document}
\PMlinkescapeword{solution}
\PMlinkescapeword{center}

\textbf{Task.}\; Construct the line parallel to a given line $\ell$ and passing through a given point $P$ which is not on $\ell$.

\begin{center}
\begin{pspicture}(-3,-2)(7,1)
\rput[l](-3,-2){.}
\psline{-}(-3,-2)(7,-2)
\rput[a](7,-1.8){$\ell$}
\psdot(0,0)
\rput[a](0,0.25){$P$}
\end{pspicture}
\end{center}

\textbf{Solution.}

\begin{enumerate}
\item Draw a circle $c_1$ with \PMlinkname{center}{Center8} $P$ and intersecting $\ell$ at two points, one of which is $A$.

\begin{center}
\begin{pspicture}(-3,-2)(7,3)
\rput[l](-3,-2){.}
\rput[a](0,2.5){.}
\rput[b](0,-2.5){.}
\psline{-}(-3,-2)(7,-2)
\rput[a](7,-1.8){$\ell$}
\pscircle[linecolor=blue](0,0){2.5}
\psdots(0,0)(1.5,-2)
\rput[a](0,0.2){$P$}
\rput[r](1.3,-1.8){$A$}
\end{pspicture}
\end{center}

\item Draw a second circle $c_2$ with center $A$ and the same radius $r$ as $c_1$.  This circle also intersects $\ell$ at two points, one of which is $B$.

\begin{center}
\begin{pspicture}(-3,-5)(7,3)
\rput[l](-3,-2){.}
\rput[a](0,2.5){.}
\rput[b](1.5,-4.5){.}
\psline{-}(-3,-2)(7,-2)
\rput[a](7,-1.8){$\ell$}
\pscircle(0,0){2.5}
\pscircle[linecolor=blue](1.5,-2){2.5}
\psdots(0,0)(1.5,-2)(4,-2)
\rput[a](0,0.2){$P$}
\rput[a](1.3,-1.8){$A$}
\rput[a](4.2,-2.2){$B$}
\end{pspicture}
\end{center}

\item Draw a third circle $c_3$ with center $B$ and radius $r$.\, Let $C$ be the intersection point of $c_3$ (drawn below in red) with $c_1$ (drawn below in green) which lies on the same side of $\ell$ as $P$ does.\, The line $PC$ (drawn below in blue) is the required parallel to $\ell$.

\begin{center}
\begin{pspicture}(-3,-5)(7,3)
\rput[l](-3,-2){.}
\rput[a](0,2.5){.}
\rput[b](1.5,-4.5){.}
\psline{-}(-3,-2)(7,-2)
\rput[a](7,-1.8){$\ell$}
\pscircle[linecolor=green](0,0){2.5}
\pscircle(1.5,-2){2.5}
\pscircle[linecolor=red](4,-2){2.5}
\psline[linecolor=blue]{-}(-3,0)(7,0)
\psdots(0,0)(1.5,-2)(4,-2)(2.5,0)
\rput[a](0,0.2){$P$}
\rput[a](1.3,-1.8){$A$}
\rput[a](4.2,-2.2){$B$}
\rput[a](2.7,-0.2){$C$}
\end{pspicture}
\end{center}

\end{enumerate}

\textbf{Note 1.}\; The construction is based on the fact that the quadrilateral $PABC$ is a parallelogram.  In fact, $PABC$ is a rhombus.  The reasoning is as follows:
\begin{itemize}
\item The green circle shows that $\overline{PC}$ and $\overline{PA}$ are congruent.
\item The black circle shows that $\overline{PA}$ and $\overline{AB}$ are congruent.
\item The red circle shows that $\overline{AB}$ and $\overline{BC}$ are congruent.
\item Since\, $PABC$\, is a quadrilateral with all sides congruent, it is a rhombus (and therefore a parallelogram).
\end{itemize}
\textbf{Note 2.}\; It is clear that the construction only needs the compass, not a straightedge:  In determining the point $C$, the straightedge is totally superfluous, and the points $P$ and $C$ determine the desired line (which thus is not necessary to actually draw!).  It may be proved that all constructions with compass and straightedge are possible using only the compass.

\textbf{Note 3.}\; Another construction of the parallel uses the fact that the endpoints of two congruent chords (red) in a circle determine two parallel chords:

\begin{center}
\begin{pspicture}(-5,0)(5,5)
\rput[l](-5.5,2){.}
\rput[a](0,5){.}
\rput[b](-0.05,0){.}
\psdots(0,0)(-3,4)(4.58,2)
\rput[l](-3.15,4.3){$P$}
\psarc[linecolor=blue](-4.58,2){2.55}{40}{80}
\psarc[linecolor=blue](4.58,2){2.55}{100}{150}
\psline[linecolor=red](-3,4)(-4.58,2)
\psline[linecolor=red](3,4)(4.58,2)
\psline[linecolor=blue](-5,4)(5,4)
\psline(-5.5,2)(5.5,2)
\psline[linestyle=dashed](0,0)(4.89,1)
\rput[r](5.8,2){$l$}
\psarc(0,0){5}{5}{170}
\end{pspicture}
\end{center}

If you are interested in seeing the rules for compass and straightedge constructions, click on the \PMlinkescapetext{link} provided.
%%%%%
%%%%%
\end{document}
