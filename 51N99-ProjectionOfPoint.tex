\documentclass[12pt]{article}
\usepackage{pmmeta}
\pmcanonicalname{ProjectionOfPoint}
\pmcreated{2013-03-22 17:09:50}
\pmmodified{2013-03-22 17:09:50}
\pmowner{pahio}{2872}
\pmmodifier{pahio}{2872}
\pmtitle{projection of point}
\pmrecord{21}{39475}
\pmprivacy{1}
\pmauthor{pahio}{2872}
\pmtype{Definition}
\pmcomment{trigger rebuild}
\pmclassification{msc}{51N99}
\pmsynonym{orthogonal projection}{ProjectionOfPoint}
%\pmkeywords{orthogonal projection}
\pmrelated{Projection}
\pmrelated{CompassAndStraightedgeConstructionOfPerpendicular}
\pmrelated{MeusniersTheorem}
\pmdefines{project}
\pmdefines{projection of line segment}

\endmetadata

% this is the default PlanetMath preamble.  as your knowledge
% of TeX increases, you will probably want to edit this, but
% it should be fine as is for beginners.

% almost certainly you want these
\usepackage{amssymb}
\usepackage{amsmath}
\usepackage{amsfonts}
\usepackage{pstricks}

% used for TeXing text within eps files
%\usepackage{psfrag}
% need this for including graphics (\includegraphics)
%\usepackage{graphicx}
% for neatly defining theorems and propositions
 \usepackage{amsthm}
% making logically defined graphics
%%\usepackage{xypic}

% there are many more packages, add them here as you need them

% define commands here

\theoremstyle{definition}
\newtheorem*{thmplain}{Theorem}

\begin{document}
\PMlinkescapeword{onto} \PMlinkescapeword{projection}

Let a \PMlinkescapetext{straight} line $l$ be given in a Euclidean plane or space.  The ({\em orthogonal}) {\em projection of a \PMlinkescapetext{point}} $P$ onto the line $l$ is the point $P'$ of $l$ at which the normal line of $l$ passing through $P$ intersects $l$.  One says that $P$ has been ({\em orthogonally}) {\em projected} onto the line $l$.

\begin{center}
\begin{pspicture}(-3,-3)(3,3)
\rput[b](-3,-3){.}
\rput[a](3,3){.}
\psline(-3,-3)(3,3)
\psline[linestyle=dashed](-2,2)(0,0)
\psline(-0.3,0.3)(0,0.6)
\psline(0,0.6)(0.3,0.3)
\psdots(-2,2)(0,0)
\rput[r](-2.2,2){$P$}
\rput[l](0.1,-0.1){$P'$}
\rput[r](2.8,3){$l$}
\end{pspicture}
\end{center}

The {\em projection of a set} $S$ of points onto the line $l$ is defined to be the set of projection points of all points of $S$ on $l$.

Especially, the {\em projection of a \PMlinkescapetext{line segment}} $PQ$ onto $l$ is the line segment $P'Q'$ determined by the projection points $P'$ and $Q'$ of $P$ and $Q$.  If the length of $PQ$ is $a$ and the \PMlinkname{angle between the lines}{AngleBetweenTwoLines} $PQ$ and $l$ is $\alpha$, then the length $p$ of its projection is
\begin{align}
p \;=\; a\,\cos\alpha.
\end{align}


\begin{center}
\begin{pspicture}(-7,-7)(3,3)
\rput[b](-7,-7){.}
\rput[a](3,3){.}
\psline(-7,-7)(3,3)
\psline[linestyle=dashed](-3,3)(0,0)
\psline(-0.3,0.3)(0,0.6)
\psline(0,0.6)(0.3,0.3)
\psline[linestyle=dashed](-4,0)(-2,-2)
\psline(-2.3,-1.7)(-2,-1.4)
\psline(-2,-1.4)(-1.7,-1.7)
\psline[linecolor=red](-3,3)(-4,0)
\psline[linecolor=red,linestyle=dashed](-4,0)(-6,-6)
\psarc(-6,-6){0.5}{45}{71.565}
\rput[b](-5.5,-5.4){$\alpha$}
\psline[linecolor=blue](0,0)(-2,-2)
\psdots(-3,3)(0,0)(-4,0)(-2,-2)
\rput[r](-3.2,3){$P$}
\rput[l](0.1,-0.1){$P'$}
\rput[r](-4.2,0){$Q$}
\rput[l](-1.9,-2.1){$Q'$}
\rput[r](2.8,3){$l$}
\end{pspicture}
\end{center}

\textbf{Remark.}\, As one speaks of the projections onto a line $l$, one can speak in the Euclidean space also of {\em projections onto a plane} $\tau$.
%%%%%
%%%%%
\end{document}
