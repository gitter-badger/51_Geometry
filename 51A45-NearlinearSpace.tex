\documentclass[12pt]{article}
\usepackage{pmmeta}
\pmcanonicalname{NearlinearSpace}
\pmcreated{2013-03-22 13:05:32}
\pmmodified{2013-03-22 13:05:32}
\pmowner{kshum}{5987}
\pmmodifier{kshum}{5987}
\pmtitle{near-linear space}
\pmrecord{29}{33509}
\pmprivacy{1}
\pmauthor{kshum}{5987}
\pmtype{Definition}
\pmcomment{trigger rebuild}
\pmclassification{msc}{51A45}
\pmclassification{msc}{51A05}
\pmclassification{msc}{05C65}
\pmsynonym{partial plane}{NearlinearSpace}
\pmrelated{FinitePlane}
\pmrelated{ProjectivePlane2}
\pmrelated{DeBruijnErdHosTheorem}

\usepackage{amssymb}
\usepackage{amsmath}
\usepackage{mathrsfs}
\usepackage{amsfonts}
\usepackage{pst-plot}

\usepackage{bbm}
\newcommand{\Z}{\mathbbmss{Z}}
\newcommand{\C}{\mathbbmss{C}}
\newcommand{\R}{\mathbbmss{R}}
\newcommand{\Q}{\mathbbmss{Q}}
\begin{document}
A {\em near-linear space} $\mathscr{S}=(\mathcal{P},\mathcal{L},\mathcal{I})$ is an incidence structure, where blocks (elements of $\mathcal{L}$) are called lines, that satisfies the following conditions:
\begin{enumerate}
\item any line is incident with at least two points, and
\item any two points are incident with at most one line.
\end{enumerate}
A near-linear space is also called a partial plane.

Given points $P,Q$, by condition 2, we may write $PQ$ to mean the unique line (if it exists) incident with both $P$ and $Q$.  Note that $PQ=QP$.

It is easy to see that a near-linear space is a simple incidence structure, so that lines can be identified with the set of points incident with it, which we shall do from now on, and we will write $(\mathcal{P},\mathcal{L})$ for the near-linear space whose point set and line set are $\mathcal{P}$ and $\mathcal{L}$ respectively.

\paragraph{Examples:}
\begin{enumerate}
\item If we take all the vertices in a graph as points, and edges as lines, it is then a near-linear space in which every line contains two points.

\item
A linear space is a near-linear space where any two points lie on exactly one line.

\item Let us enumerate all (non-isomorphic) near-linear spaces with four points:
\begin{itemize}
\item no lines
\begin{center}
\begin{pspicture}(-1,0)(1,1)
\psset{unit=20pt}
\psdots[linecolor=blue,dotsize=5pt](-1,0)
\psdots[linecolor=blue,dotsize=5pt](0,0)
\psdots[linecolor=blue,dotsize=5pt](-1,1)
\psdots[linecolor=blue,dotsize=5pt](0,1)
\end{pspicture}
\end{center}
\item one line
\begin{center}
\begin{pspicture}(-1,0)(1,1)
\psset{unit=20pt}
\psdots[linecolor=blue,dotsize=5pt](-1,0)
\psdots[linecolor=blue,dotsize=5pt](0,0)
\psdots[linecolor=blue,dotsize=5pt](-1,1)
\psdots[linecolor=blue,dotsize=5pt](0,1)
\psline(-1,0)(-1,1)
\end{pspicture}
\begin{pspicture}(-1,0)(1,1)
\psset{unit=20pt}
\psdots[linecolor=blue,dotsize=5pt](0,1)
\psdots[linecolor=blue,dotsize=5pt](-1,0)
\psdots[linecolor=blue,dotsize=5pt](0,0)
\psdots[linecolor=blue,dotsize=5pt](1,0)
\psline(-1,0)(1,0)
\end{pspicture}
\begin{pspicture}(-2,0)(2,1)
\psset{unit=20pt}
\psdots[linecolor=blue,dotsize=5pt](-1,0)
\psdots[linecolor=blue,dotsize=5pt](0,0)
\psdots[linecolor=blue,dotsize=5pt](1,0)
\psdots[linecolor=blue,dotsize=5pt](2,0)
\psline(-1,0)(2,0)
\end{pspicture}
\end{center}
\item two lines
\begin{center}
\begin{pspicture}(-1,0)(1,1)
\psset{unit=20pt}
\psdots[linecolor=blue,dotsize=5pt](-1,0)
\psdots[linecolor=blue,dotsize=5pt](0,0)
\psdots[linecolor=blue,dotsize=5pt](-1,1)
\psdots[linecolor=blue,dotsize=5pt](0,1)
\psline(-1,0)(-1,1)
\psline(-1,1)(0,1)
\end{pspicture}
\begin{pspicture}(-1,0)(1,1)
\psset{unit=20pt}
\psdots[linecolor=blue,dotsize=5pt](-1,0)
\psdots[linecolor=blue,dotsize=5pt](0,0)
\psdots[linecolor=blue,dotsize=5pt](-1,1)
\psdots[linecolor=blue,dotsize=5pt](0,1)
\psline(-1,0)(-1,1)
\psline(0,0)(0,1)
\end{pspicture}
\begin{pspicture}(-1,0)(1,1)
\psset{unit=20pt}
\psdots[linecolor=blue,dotsize=5pt](0,1)
\psdots[linecolor=blue,dotsize=5pt](-1,0)
\psdots[linecolor=blue,dotsize=5pt](0,0)
\psdots[linecolor=blue,dotsize=5pt](1,0)
\psline(-1,0)(1,0)
\psline(0,1)(1,0)
\end{pspicture}
\end{center}
\item three lines
\begin{center}
\begin{pspicture}(-1,0)(1,1)
\psset{unit=20pt}
\psdots[linecolor=blue,dotsize=5pt](-1,0)
\psdots[linecolor=blue,dotsize=5pt](0,0)
\psdots[linecolor=blue,dotsize=5pt](-1,1)
\psdots[linecolor=blue,dotsize=5pt](0,1)
\psline(-1,0)(-1,1)
\psline(-1,1)(0,1)
\psline(0,1)(0,0)
\end{pspicture}
\begin{pspicture}(-1,0)(1,1)
\psset{unit=20pt}
\psdots[linecolor=blue,dotsize=5pt](-1,0)
\psdots[linecolor=blue,dotsize=5pt](0,0)
\psdots[linecolor=blue,dotsize=5pt](-1,1)
\psdots[linecolor=blue,dotsize=5pt](0,1)
\psline(-1,0)(-1,1)
\psline(-1,1)(0,1)
\psline(0,1)(-1,0)
\end{pspicture}
\begin{pspicture}(-1,0)(1,1)
\psset{unit=20pt}
\psdots[linecolor=blue,dotsize=5pt](0,1)
\psdots[linecolor=blue,dotsize=5pt](-1,0)
\psdots[linecolor=blue,dotsize=5pt](0,0)
\psdots[linecolor=blue,dotsize=5pt](1,0)
\psline(-1,0)(1,0)
\psline(0,1)(1,0)
\psline(0,1)(-1,0)
\end{pspicture}
\end{center}
\item four lines
\begin{center}
\begin{pspicture}(-1,0)(1,1)
\psset{unit=20pt}
\psdots[linecolor=blue,dotsize=5pt](-1,0)
\psdots[linecolor=blue,dotsize=5pt](0,0)
\psdots[linecolor=blue,dotsize=5pt](-1,1)
\psdots[linecolor=blue,dotsize=5pt](0,1)
\psline(-1,0)(-1,1)
\psline(-1,1)(0,1)
\psline(0,1)(0,0)
\psline(0,0)(-1,0)
\end{pspicture}
\begin{pspicture}(-1,0)(1,1)
\psset{unit=20pt}
\psdots[linecolor=blue,dotsize=5pt](-1,0)
\psdots[linecolor=blue,dotsize=5pt](0,0)
\psdots[linecolor=blue,dotsize=5pt](-1,1)
\psdots[linecolor=blue,dotsize=5pt](0,1)
\psline(-1,0)(-1,1)
\psline(-1,1)(0,1)
\psline(0,1)(0,0)
\psline(0,1)(-1,0)
\end{pspicture}
\begin{pspicture}(-1,0)(1,1)
\psset{unit=20pt}
\psdots[linecolor=blue,dotsize=5pt](0,1)
\psdots[linecolor=blue,dotsize=5pt](-1,0)
\psdots[linecolor=blue,dotsize=5pt](0,0)
\psdots[linecolor=blue,dotsize=5pt](1,0)
\psline(-1,0)(1,0)
\psline(0,1)(1,0)
\psline(0,1)(0,0)
\psline(0,1)(-1,0)
\end{pspicture}
\end{center}
\item five lines
\begin{center}
\begin{pspicture}(-1,-1)(1,1)
\psset{unit=10pt}
\psdots[linecolor=blue,dotsize=5pt](-1,-1)
\psdots[linecolor=blue,dotsize=5pt](1,-1)
\psdots[linecolor=blue,dotsize=5pt](-1,1)
\psdots[linecolor=blue,dotsize=5pt](1,1)
\psline(-1,-1)(-1,1)
\psline(-1,1)(1,1)
\psline(1,1)(1,-1)
\psline(1,-1)(-1,-1)
\psline(-1,-1)(1,1)
\end{pspicture}
\end{center}
\item six lines
\begin{center}
\begin{pspicture}(-1,-1)(1,1)
\psset{unit=10pt}
\psdots[linecolor=blue,dotsize=5pt](-1,-1)
\psdots[linecolor=blue,dotsize=5pt](1,-1)
\psdots[linecolor=blue,dotsize=5pt](-1,1)
\psdots[linecolor=blue,dotsize=5pt](1,1)
\psline(-1,-1)(-1,1)
\psline(-1,1)(1,1)
\psline(1,1)(1,-1)
\psline(1,-1)(-1,-1)
\psline(-1,-1)(1,1)
\psline(-1,1)(1,-1)
\end{pspicture}
\end{center}
\end{itemize}
Out of these, the linear spaces are the last ones of the one-line and four-line spaces, as well as the six-line space, which also happens to be the smallest affine plane.
\item Examples of near-linear spaces may be constructed from existing ones.  For more detail, see here.
\end{enumerate}

\paragraph{Some properties:}
\begin{enumerate}
\item In a near-linear space, if two distinct lines intersect, they intersect in one point.

\item There is no proper inclusion of lines in a near-linear space, i.e., if $\ell_1$ and $\ell_2$ are two lines such that $\ell_1\subseteq \ell_2$, then $\ell_1=\ell_2$.

\item In a near-linear space $\mathscr{S}=(\mathcal{P},\mathcal{L})$, 
$$\sum_{\ell \in \mathcal{L}} \binom{|\ell|}{2} \leq \binom{|\mathcal{P}|}{2}$$
with equality holds if and only if $\mathscr{S}$ is a linear space.
\end{enumerate}

\begin{thebibliography}{7}
\bibitem{LB} L. M. Batten, {\it Combinatorics of Finite Geometries}, 2nd edition, Cambridge University Press (1997)
\end{thebibliography}
%%%%%
%%%%%
\end{document}
