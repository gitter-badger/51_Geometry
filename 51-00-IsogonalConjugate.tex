\documentclass[12pt]{article}
\usepackage{pmmeta}
\pmcanonicalname{IsogonalConjugate}
\pmcreated{2013-03-22 13:01:13}
\pmmodified{2013-03-22 13:01:13}
\pmowner{drini}{3}
\pmmodifier{drini}{3}
\pmtitle{isogonal conjugate}
\pmrecord{7}{33406}
\pmprivacy{1}
\pmauthor{drini}{3}
\pmtype{Definition}
\pmcomment{trigger rebuild}
\pmclassification{msc}{51-00}
\pmrelated{Symmedian}
\pmrelated{LemoinePoint}
\pmrelated{FundamentalTheoremOnIsogonalLines}
\pmdefines{isogonal conjugate pair}
\pmdefines{isogonal}

\usepackage{graphicx}
%%%\usepackage{xypic} 
\usepackage{bbm}
\newcommand{\Z}{\mathbbmss{Z}}
\newcommand{\C}{\mathbbmss{C}}
\newcommand{\R}{\mathbbmss{R}}
\newcommand{\Q}{\mathbbmss{Q}}
\newcommand{\mathbb}[1]{\mathbbmss{#1}}
\newcommand{\figura}[1]{\begin{center}\includegraphics{#1}\end{center}}
\newcommand{\figuraex}[2]{\begin{center}\includegraphics[#2]{#1}\end{center}}
\begin{document}
\PMlinkescapeword{inner}
Let $\triangle ABC$ be a triangle, $AL$ the angle bisector of $\angle BAC$ and $AX$ any line passing through $A$. The isogonal conjugate line to $AX$ is the line $AY$ obtained by reflecting the line $AX$ on the angle bisector $AL$.

\figura{isogonal.eps}
In the picture $\angle YAL = \angle LAX$. This is the reason why $AX$ and $AY$ are called isogonal conjugates, since they form the same angle with $AL$. (iso= equal, gonal = angle).

Let $P$ be a point on the plane. The lines $AP,BP,CP$ are concurrent by construction. Consider now their isogonals conjugates (reflections on the inner angle bisectors). The isogonals conjugates will also concurr by the   fundamental theorem on isogonal lines, and their intersection point $Q$ is called the isogonal conjugate of $P$. 

If $Q$ is the isogonal conjugate of $P$, then $P$ is the isogonal conjugate of $Q$ so both are often referred as an isogonal conjugate pair.

An example of isogonal conjugate pair is found by looking at the centroid of the triangle and the Lemoine point.
%%%%%
%%%%%
\end{document}
