\documentclass[12pt]{article}
\usepackage{pmmeta}
\pmcanonicalname{AreaOfTheNsphere}
\pmcreated{2013-03-22 13:47:06}
\pmmodified{2013-03-22 13:47:06}
\pmowner{CWoo}{3771}
\pmmodifier{CWoo}{3771}
\pmtitle{area of the $n$-sphere}
\pmrecord{14}{34495}
\pmprivacy{1}
\pmauthor{CWoo}{3771}
\pmtype{Derivation}
\pmcomment{trigger rebuild}
\pmclassification{msc}{51M05}
\pmrelated{VolumeOfTheNSphere}
\pmrelated{AreaOfASphericalTriangle}
\pmrelated{AreaOfSphericalZone}

\endmetadata

% this is the default PlanetMath preamble.  as your knowledge
% of TeX increases, you will probably want to edit this, but
% it should be fine as is for beginners.

% almost certainly you want these
\usepackage{amssymb}
\usepackage{amsmath}
\usepackage{amsfonts}

% used for TeXing text within eps files
%\usepackage{psfrag}
% need this for including graphics (\includegraphics)
%\usepackage{graphicx}
% for neatly defining theorems and propositions
%\usepackage{amsthm}
% making logically defined graphics
%%%\usepackage{xypic}

% there are many more packages, add them here as you need them

% define commands here
\def\sse{\subseteq}
\def\bigtimes{\mathop{\mbox{\Huge $\times$}}}
\def\impl{\Rightarrow}
\def\R{\mathbb{R}}
\begin{document}
The area of $S^n$ the unit $n$-sphere (or hypersphere) is the same as the total
solid angle it subtends at the origin. To calculate it, consider the following
integral
\[
  I(n) = \int_{\R^{n+1}} e^{-\sum_{i=1}^{n+1} x_i^2}\, d^{n+1} x.
\]
Switching to polar coordinates we let $r^2=\sum_{i=1}^{n+1} x_i^2$ and the
integral becomes
\[
  I(n) = \int_{S^n} d\Omega \int_{0}^{\infty} r^{n} e^{-r^2}\, dr.
\]
The first integral is the integral over all solid angles and is exactly what we
want to evaluate. Let us denote it by $A(n)$. With the change of variable
$t=r^2$, the second integral can be evaluated in terms of the gamma function
$\Gamma(x)$:
\[
  I(n)/A(n) = \frac{1}{2}\int_0^\infty t^{\frac{n-1}{2}} e^{-t}\, dt
    = \frac{1}{2}\Gamma\left(\frac{n+1}{2}\right).
\]
We can also evaluate $I(n)$ directly in Cartesian coordinates:
\[
  I(n) = \left[ \int_{-\infty}^\infty e^{-x^2}\, dx \right]^{n+1}
    = \pi^{\frac{n+1}{2}},
\]
where we have used the standard Gaussian integral $\int_{-\infty}^\infty
e^{-x^2}\, dx = \sqrt{\pi}$.

Finally, we can solve for the area
\[ A(n) = \frac{2\pi^{\frac{n+1}{2}}}{\Gamma\left(\frac{n+1}{2}\right)}. \]
If the radius of the sphere is $R$ and not $1$, the correct area is
$A(n)R^{n}$.

Note that this formula works only for $n\ge0$. The first few special cases
are
\begin{itemize}
  \item[$n=0$] $\Gamma(1/2)=\sqrt{\pi}$, hence $A(0)=2$ (in this case, the
    area just counts the number of points in $S^0=\{+1,-1\}$);
  \item[$n=1$] $\Gamma(1)=1$, hence $A(1)=2\pi$ (this is the familiar result
    for the circumference of the unit circle);
  \item[$n=2$] $\Gamma(3/2)=\sqrt{\pi}/2$, hence $A(2)=4\pi$ (this is the
    familiar result for the area of the unit sphere);
  \item[$n=3$] $\Gamma(2)=1$, hence $A(3)=2\pi^2$;
  \item[$n=4$] $\Gamma(5/2)=3\sqrt{\pi}/4$, hence $A(4)=8\pi^2/3$.
\end{itemize}
%%%%%
%%%%%
\end{document}
