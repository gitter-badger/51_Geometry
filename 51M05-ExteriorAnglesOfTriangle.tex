\documentclass[12pt]{article}
\usepackage{pmmeta}
\pmcanonicalname{ExteriorAnglesOfTriangle}
\pmcreated{2013-05-05 8:44:43}
\pmmodified{2013-05-05 8:44:43}
\pmowner{pahio}{2872}
\pmmodifier{pahio}{2872}
\pmtitle{exterior angles of triangle}
\pmrecord{2}{87387}
\pmprivacy{1}
\pmauthor{pahio}{2872}
\pmtype{Theorem}
\pmclassification{msc}{51M05}

\endmetadata

% this is the default PlanetMath preamble.  as your knowledge
% of TeX increases, you will probably want to edit this, but
% it should be fine as is for beginners.

% almost certainly you want these
\usepackage{amssymb}
\usepackage{amsmath}
\usepackage{amsfonts}

% need this for including graphics (\includegraphics)
\usepackage{graphicx}
% for neatly defining theorems and propositions
\usepackage{amsthm}

% making logically defined graphics
%\usepackage{xypic}
% used for TeXing text within eps files
%\usepackage{psfrag}

% there are many more packages, add them here as you need them

% define commands here

\begin{document}
The exterior angle of an angle of triangle is greater than both other 
angles of the triangle.

{\it Proof.}\, Let us study in an arbitrary triangle $ABC$ for example the exterior angle $\wedge ACD$ where $D$ is point on the lengthening of the side  $BC$ nearer to $C$ than to $B$.\, Let $E$ be the midpoint of $AC$.\, Let $BE$ be the median of the triangle.\, We find on its lengthening the point $F$ such that\, $EF = EB$.\, Then the triangles $ABE$ and $CEF$ are congruent (SAS).\, Consequently, we have\; 
$\wedge ECF \;=\;\, \wedge BAE$\;
and therefore\; $\wedge ACD \;>\; \wedge BAC$.\;\, Analogically one shows that\; 
$\wedge ACD \;>\; \wedge ABC$.\;\; $\Box$

\begin{thebibliography}{8}
\bibitem{ariva}{\sc Karl Ariva}: {\it Lobatsevski geomeetria}.\, Kirjastus ``Valgus'', Tallinn (1992).
\end{thebibliography} 
\end{document}
