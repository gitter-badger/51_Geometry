\documentclass[12pt]{article}
\usepackage{pmmeta}
\pmcanonicalname{Prismatoid}
\pmcreated{2013-03-22 17:12:03}
\pmmodified{2013-03-22 17:12:03}
\pmowner{Mathprof}{13753}
\pmmodifier{Mathprof}{13753}
\pmtitle{prismatoid}
\pmrecord{10}{39520}
\pmprivacy{1}
\pmauthor{Mathprof}{13753}
\pmtype{Definition}
\pmcomment{trigger rebuild}
\pmclassification{msc}{51-00}
\pmrelated{SimpsonsRule}
\pmrelated{Volume2}
\pmrelated{TruncatedCone}
\pmdefines{altitude}
\pmdefines{bases}
\pmdefines{prismoidal formula}

% this is the default PlanetMath preamble.  as your knowledge
% of TeX increases, you will probably want to edit this, but
% it should be fine as is for beginners.

% almost certainly you want these
\usepackage{amssymb}
\usepackage{amsmath}
\usepackage{amsfonts}

% used for TeXing text within eps files
%\usepackage{psfrag}
% need this for including graphics (\includegraphics)
%\usepackage{graphicx}
% for neatly defining theorems and propositions
%\usepackage{amsthm}
% making logically defined graphics
%%%\usepackage{xypic}

% there are many more packages, add them here as you need them

% define commands here

\begin{document}
A \emph{prismatoid} is a polyhedron, possibly not convex, whose vertices all lie in one or the other 
of two parallel planes.
The perpendicular distance between the two planes is called the \emph{altitude}
of the prismatoid.
The faces that lie in the parallel planes are called the \emph{bases}
of the prismatoid.
The \emph{midsection} is the polygon formed by cutting the prismatoid by
 a plane parallel to the bases halfway between them.

The volume of a prismatoid is given by the \emph{prismoidal formula}:

$$
V = \frac{1}{6} h(B_1 + B_2 + 4M)
$$
where $h$ is the altitude, $B_1$ and $B_2$ are the areas of the bases and $M$
is the area of the midsection. 

An alternate formula is :

$$
V = \frac{1}{4}h ( B_1 + 3S)
$$ 
where $S$ is the area of the polygon that is formed by cutting the prismatoid
by a plane parallel to the bases but 2/3 of the distance from $B_1$ to $B_2$.

A proof of the prismoidal formula for the case where
the prismatoid is convex is in \cite{Br}. It is also proved in \cite{Ha} for any prismatoid.
The alternate formula is proved in \cite{Ha}.

Some authors impose the condition that the lateral faces must be triangles
or trapezoids. However, this condition is unnecessary since it is easily shown
to hold. 

\begin{thebibliography}{99}
\bibitem{Br}
A. Day Bradley, Prismatoid, Prismoid, Generalized Prismoid, \emph{The American Math. Monthly,}
\textbf{86}, (1979), 486-490.
\bibitem{Ha}
G.B. Halsted, \emph{Rational Geometry: A textbook for the Science of Space. Based on
Hilbert's Foundations}, second edition, John Wiley and Sons, New York, 1907
\end{thebibliography}



%%%%%
%%%%%
\end{document}
