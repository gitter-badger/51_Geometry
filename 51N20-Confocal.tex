\documentclass[12pt]{article}
\usepackage{pmmeta}
\pmcanonicalname{Confocal}
\pmcreated{2013-03-22 14:44:58}
\pmmodified{2013-03-22 14:44:58}
\pmowner{Mathprof}{13753}
\pmmodifier{Mathprof}{13753}
\pmtitle{confocal}
\pmrecord{6}{36388}
\pmprivacy{1}
\pmauthor{Mathprof}{13753}
\pmtype{Definition}
\pmcomment{trigger rebuild}
\pmclassification{msc}{51N20}

\endmetadata

% this is the default PlanetMath preamble.  as your knowledge
% of TeX increases, you will probably want to edit this, but
% it should be fine as is for beginners.

% almost certainly you want these
\usepackage{amssymb}
\usepackage{amsmath}
\usepackage{amsfonts}

% used for TeXing text within eps files
%\usepackage{psfrag}
% need this for including graphics (\includegraphics)
%\usepackage{graphicx}
% for neatly defining theorems and propositions
%\usepackage{amsthm}
% making logically defined graphics
%%%\usepackage{xypic}

% there are many more packages, add them here as you need them

% define commands here
\begin{document}
Two conics are {\em confocal} if they have coincident foci. 

\textbf{Examples}
\begin{enumerate}
\item The family of ellipses
$$\frac{x^2}{a^2+s}+\frac{y^2}{b^2+s} = 1,$$
where \,$a^2 > b^2$\, and the parameter $s$ is  $> -b^2$, is confocal.
\item The family of hyperbolas
$$\frac{x^2}{a^2-t}-\frac{y^2}{t-b^2} = 1,$$
where \,$a^2 > b^2$\, and the parameter $t$ is between $a^2$ and $b^2$, is confocal.
\end{enumerate}
%%%%%
%%%%%
\end{document}
