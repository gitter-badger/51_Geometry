\documentclass[12pt]{article}
\usepackage{pmmeta}
\pmcanonicalname{CorollaryOfMorleysTheorem}
\pmcreated{2013-03-22 13:46:22}
\pmmodified{2013-03-22 13:46:22}
\pmowner{drini}{3}
\pmmodifier{drini}{3}
\pmtitle{corollary of Morley's theorem}
\pmrecord{8}{34479}
\pmprivacy{1}
\pmauthor{drini}{3}
\pmtype{Corollary}
\pmcomment{trigger rebuild}
\pmclassification{msc}{51M04}

\endmetadata

\usepackage{amssymb}
\usepackage{amsmath}
\usepackage{amsfonts}
\usepackage{graphicx}
\begin{document}

We describe here, informally, a limiting case of Morley's theorem.
\includegraphics{morley2}

One of the vertices of the triangle $ABC$, namely $C$, has been pushed
off to infinity. Instead of two segments $BC$ and $CA$, plus
two trisectors between them, we now have four parallel and equally
spaced lines. The triangle $PQR$ is still equilateral, and the three
triangles adjacent to it are still isosceles, but one of those has become
equilateral. We have
$$AQ\cdot BR = AR\cdot BP\;.$$
%%%%%
%%%%%
\end{document}
