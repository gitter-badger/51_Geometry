\documentclass[12pt]{article}
\usepackage{pmmeta}
\pmcanonicalname{TriangleSolving}
\pmcreated{2013-03-22 15:44:02}
\pmmodified{2013-03-22 15:44:02}
\pmowner{stevecheng}{10074}
\pmmodifier{stevecheng}{10074}
\pmtitle{triangle solving}
\pmrecord{11}{37684}
\pmprivacy{1}
\pmauthor{stevecheng}{10074}
\pmtype{Definition}
\pmcomment{trigger rebuild}
\pmclassification{msc}{51-00}
\pmrelated{Congruence}

% this is the default PlanetMath preamble.  as your knowledge
% of TeX increases, you will probably want to edit this, but
% it should be fine as is for beginners.

% almost certainly you want these
\usepackage{amssymb}
\usepackage{amsmath}
\usepackage{amsfonts}
\usepackage{graphicx}
%%\usepackage{xypic}

% used for TeXing text within eps files
%\usepackage{psfrag}
% need this for including graphics (\includegraphics)
\usepackage{graphicx}
% for neatly defining theorems and propositions
 \usepackage{amsthm}
% making logically defined graphics
%%%\usepackage{xypic}

% there are many more packages, add them here as you need them

% define commands here

\theoremstyle{definition}
\newtheorem*{thmplain}{Theorem}
\begin{document}
Let us consider skew-angled triangles.\, If one knows three parts of a triangle, among which at least one side, then the other parts may be calculated by using the law of sines and the law of cosines.\, We distinguish four cases:
\begin{enumerate}

\item \textbf{ASA.}\, Known two angles and one side, e.g. $\alpha$, $\beta$, $c$.\, Other parts:
$$\gamma = 180\sp\circ\!-\!(\alpha\!+\!\beta), \quad
a = \frac{c\sin\alpha}{\sin\gamma},\quad b = \frac{c\sin\beta}{\sin\gamma}$$

\begin{figure}[!htb]
\begin{center}
\includegraphics{triangle.1.eps}
\end{center}
\caption{ASA (angle-side-angle)}
\end{figure}

\item \textbf{SSS.}\, Known all sides $a$, $b$, $c$.\, The angles are obtained from
$$\cos\alpha = \frac{b^2\!+\!c^2\!-\!a^2}{2bc},
 \quad\cos\beta = \frac{c^2\!+\!a^2\!-\!b^2}{2ca}, 
 \quad\cos\gamma = \frac{a^2\!+\!b^2\!-\!c^2}{2ab}.$$

\begin{figure}[!htb]
\begin{center}
\includegraphics{triangle.2.eps}
\end{center}
\caption{SSS (side-side-side)}
\end{figure}

\item \textbf{SAS.}\, Known two sides and the angle between them, e.g. $b$, $c$, $\alpha$.\, Other parts from
$$a^2 = b^2\!+\!c^2\!-\!2bc\cos\alpha, \quad \sin\beta = \frac{b\sin\alpha}{a}, 
\quad \sin\gamma = \frac{c\sin\alpha}{a}$$

\begin{figure}[!htb]
\begin{center}
\includegraphics{triangle.3.eps}
\end{center}
\caption{SAS (side-angle-side)}
\end{figure}


\item \textbf{SSA.}\, Known two sides and the angle \PMlinkescapetext{opposite} of one of them, e.g. $a$, $b$, $\alpha$.\, Other parts are gotten from
$$\sin\beta = \frac{b\sin\alpha}{a},
 \quad \gamma = 180\sp\circ\!-\!(\alpha\!+\!\beta), 
 \quad c = \frac{a\sin\gamma}{\sin\alpha}.$$

\begin{figure}[!htb]
\begin{center}
\includegraphics{triangle.4.eps}
\end{center}
\caption{SSA (side-side-angle)}
\end{figure}

Since the SSA criterion alone does not prove congruence,
it is not surprising that there may not always be a single solution 
for $\beta$ here.
In fact, if the first equation gives\, $\sin\beta > 1$,\, then the situation is impossible and the triangle does not exist.\, If the equation gives\, $\sin\beta < 1$,\, one gets two distinct values of $\beta$; an acute $\beta_1$ and an obtuse\, $\beta_2 = 180\sp\circ-\beta_1$.\, If in this case\, $\beta_1 > \alpha$,\, then there are two different triangles as \PMlinkescapetext{solutions}, but if\, 
$\beta_1 \le \alpha$,\, then there is only one triangle.

\end{enumerate}

\begin{itemize}
\item
\PMlinkexternal{MetaPost source code for the above diagrams}{http://svn.gold-saucer.org/repos/PlanetMath/TriangleSolving/triangle.mp}
\end{itemize}

%%%%%
%%%%%
\end{document}
