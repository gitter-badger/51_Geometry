\documentclass[12pt]{article}
\usepackage{pmmeta}
\pmcanonicalname{Ray}
\pmcreated{2013-03-22 15:28:36}
\pmmodified{2013-03-22 15:28:36}
\pmowner{CWoo}{3771}
\pmmodifier{CWoo}{3771}
\pmtitle{ray}
\pmrecord{8}{37331}
\pmprivacy{1}
\pmauthor{CWoo}{3771}
\pmtype{Definition}
\pmcomment{trigger rebuild}
\pmclassification{msc}{51G05}
\pmsynonym{half line}{Ray}
\pmsynonym{open ray}{Ray}
\pmrelated{BetweennessInRays}
\pmdefines{closed ray}
\pmdefines{opposite ray}

\endmetadata

\usepackage{amssymb,amscd}
\usepackage{amsmath}
\usepackage{amsfonts}

% used for TeXing text within eps files
%\usepackage{psfrag}
% need this for including graphics (\includegraphics)
%\usepackage{graphicx}
% for neatly defining theorems and propositions
%\usepackage{amsthm}
% making logically defined graphics
%%%\usepackage{xypic}

% define commands here

\renewcommand{\line}[1]{\overleftrightarrow{#1}}
\newcommand{\ray}[1]{\overrightarrow{#1}}
\begin{document}
\PMlinkescapeword{between}

\textbf{Rays on the real line}.  A \emph{ray} on the real line $\mathbb{R}$ is just an open set of the form $(p,\infty)$, or $(-\infty,p)$.  A ray is also called a \emph{half line}, or an \emph{open ray}, to distinguish the notion of a \emph{closed ray}, which includes its endpoint.

\textbf{Properties}  Suppose $p,q\in\mathbb{R}$ and $p\le q$.
\begin{itemize}
\item $(p,\infty)\cap(q,\infty)=(q,\infty)$.
\item $(p,\infty)\cup(q,\infty)=(p,\infty)$.
\item $(p,\infty)\cap(-\infty,q)=(p,q)$ if $p\ne q$, and $\varnothing$ if $p=q$.
\item $(-\infty,p)\cap(q,\infty)=\varnothing$.
\item $(p,\infty)\cup(-\infty,q)=\mathbb{R}$ if $p\ne q$, and $\mathbb{R}-\lbrace p\rbrace$ if $p=q$.
\item $(-\infty,p)\cup(q,\infty)=\mathbb{R}-[p,q]$ if $p\ne q$, and $\mathbb{R}-\lbrace p\rbrace$ if $p=q$.
\end{itemize}
\textbf{Rays in a general Euclidean space}.  Let $\ell$ be a line in $\mathbb{R}^n$ and let $p$ be a point lying on the $\ell$.  We may parameterize $\ell=\ell(t)$ (parameter $t\in \mathbb{R}$) so that $\ell(0)=p$. An (\emph{open}) \emph{ray} $\rho$ lying on $\ell$ with \emph{endpoint} $p$ is the set of points 
$$\rho=\lbrace r\mid r=\ell(t), t>0 \rbrace.$$
If the inequality $t>0$ is relaxed to $t\ge 0$ in the above expression, then we have a \emph{closed ray}.  Note that if the inequality above were changed to $t<0$ instead, we end up again with a ray lying on $\ell$ and endpoint $p$.  It is a ray because we can reparameterize $\ell$ by using the parameter $s=-t$ instead, so that 
$$\lbrace r\mid r=\ell(t), t<0\rbrace = \lbrace r \mid r=\ell(s),s>0\rbrace.$$ 
The difference between the two rays is that they point in the opposite directions.  Therefore, in general, a ray can be
characterized by 
\begin{itemize}
\item a line,
\item a point lying on the line, and
\item a direction on the line.
\end{itemize}
\textbf{Rays in an ordered geometry}: Given two distinct points
$p,q$ in an ordered geometry $(A,B)$ ($A$ is the underlying \PMlinkname{incidence geometry}{IncidenceGeometry} and $B$ is the strict betweenness relation defined on the points of $A$).  The set 
$$\overline{pq}\cup\lbrace q\rbrace\cup\lbrace r\mid q\in\overline{pr} \rbrace,$$ 
where $\overline{st}$ denotes the open line segment with endpoints $s$ and $t$, is called the \emph{(open) ray generated by} $p$ and $q$ emanating from $p$. It is denoted by $\ray{pq}$. $p$ in $\ray{pq}$ is called the \emph{source} or the
\emph{end point} of the ray.  A \emph{closed ray generated by} $p$ and $q$ with endpoint $p$ is the set $\ray{pq}\cup\lbrace p\rbrace$.

\textbf{Properties.}
\begin{itemize}
\item for any point $r\in\ray{pq}$, $\ray{pr}=\ray{pq}$.
\item $\ray{pq}\cup\ray{qp}=\line{pq}$ and $\ray{pq}\cap\ray{qp}=\overline{pq}$.  We say that a ray lies on a line if all of the points in the ray are incident with the line.  Also, a line segment lies on a ray if it is a subset of the ray.
\item The \emph{opposite ray} of $\rho=\ray{pq}$ is defined to be $$\ray{qp}-\overline{pq}-\lbrace q\rbrace.$$  It is denoted by $-\rho$.
\item The opposite ray $-\rho$ of a ray $\rho$ is ray.  Suppose $\rho=\ray{pq}$.  Then $\rho$ has the property that
\begin{enumerate}
\item $\rho\cap(-\rho)=\varnothing$ and
\item $\rho\cup(-\rho)=\line{pq}-\lbrace p\rbrace$.
\end{enumerate}
Conversely, given a ray $\rho=\ray{pq}$, any ray $\rho^{\prime}$ satisfying the above two properties (replacing $-\rho$ by $\rho^{\prime}$) is the opposite ray of $\rho$.
\item Given any point $p$ on a line $\ell$, there are exactly two rays lying on $\ell$ with endpoint $p$.  Furthermore, $p$ is between $q$ and $r$ in $\ell$ iff $q$ and $r$ lie on opposite rays on $\ell$.
\item Given any two rays $\rho$ and $\varrho$, exactly one of the following holds:
\begin{enumerate}
\item $\rho\cap\varrho=\varnothing$,
\item $\rho\cap\varrho=$ a line segment, or
\item $\rho\cap\varrho=$ a ray.
\end{enumerate}
It is not hard to see that in the last case, one ray is included in the other, and their intersection is the ``smaller'' of the two rays.  In the first two cases, the two rays are said to be (pointing) in the opposite direction. In the last case, the two are said to be in the same direction. Opposite rays are clearly pointing in the opposite direction.
\item An equivalence relation can be defined on the set of all rays lying on a line $\ell$ by whether they are pointing in the same direction or not.  Thus, the set of all rays lying on $\ell$ can be partitioned into two subsets $R$ and $R^{\prime}$, so that if $\rho,\varrho\in R$ (or $R^{\prime}$), then they are pointing in the same direction; and if $\rho\in R$ and $\varrho\in R^{\prime}$ are pointing in the opposite direction.
\item Pick one of the two subsets from above, say $R$.  Define $\le$ on $R$ by $\rho\le\varrho$ if $\rho\subseteq\varrho$.  Then $\le$ is a linear order on $R$.  This $\le$ induces a linear order $\le_{\ell}$ on the line $\ell$ in the following way: $p\le_{\ell}q$ if the corresponding rays $\rho,\varrho\in R$, with endpoints $p$
and $q$ respectively, we have $\rho\le\varrho$.  This is one way to define a linear ordering on a line $\ell$.  An alternative, but equivalent way of defining a linear ordering on a line in an ordered geometry can be found in the entry under ordered geometry.
\item Note that in defining $\le$, we could have used $R^{\prime}$ instead of $R$.  This is an example of the duality of linear ordering.
\end{itemize}

\begin{thebibliography}{6}
\bibitem{dh} D. Hilbert, {\it Foundations of Geometry}, Open Court Publishing Co. (1971)
\bibitem{bs} K. Borsuk and W. Szmielew, {\it Foundations of Geometry}, North-Holland Publishing Co. Amsterdam (1960)
\bibitem{mg} M. J. Greenberg, {\it Euclidean and Non-Euclidean Geometries, Development and History}, W. H. Freeman and Company, San Francisco (1974)
\end{thebibliography}
%%%%%
%%%%%
\end{document}
