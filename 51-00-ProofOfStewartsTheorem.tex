\documentclass[12pt]{article}
\usepackage{pmmeta}
\pmcanonicalname{ProofOfStewartsTheorem}
\pmcreated{2013-03-22 12:38:37}
\pmmodified{2013-03-22 12:38:37}
\pmowner{Mathprof}{13753}
\pmmodifier{Mathprof}{13753}
\pmtitle{proof of Stewart's theorem}
\pmrecord{7}{32908}
\pmprivacy{1}
\pmauthor{Mathprof}{13753}
\pmtype{Proof}
\pmcomment{trigger rebuild}
\pmclassification{msc}{51-00}
\pmrelated{StewartsTheorem}
\pmrelated{ApolloniusTheorem}
\pmrelated{CosinesLaw}
\pmrelated{ProofOfApolloniusTheorem2}

\endmetadata

\usepackage{graphicx}
%%%\usepackage{xypic} 
\usepackage{bbm}
\newcommand{\Z}{\mathbbmss{Z}}
\newcommand{\C}{\mathbbmss{C}}
\newcommand{\R}{\mathbbmss{R}}
\newcommand{\Q}{\mathbbmss{Q}}
\newcommand{\mathbb}[1]{\mathbbmss{#1}}
\newcommand{\figura}[1]{\begin{center}\includegraphics{#1}\end{center}}
\newcommand{\figuraex}[2]{\begin{center}\includegraphics[#2]{#1}\end{center}}
\begin{document}
Let $\theta$ be the angle $\angle AXB$.
\figura{stewart}

Cosines law on $\triangle AXB$ says 
$c^2=m^2 + p^2-2pm\cos \theta$ and thus
$$\cos \theta =\frac{m^2+p^2-c^2}{2pm}$$

Using cosines law on $\triangle AXC$ and noting that $\psi=\angle AXC=180^\circ-\theta$ and thus $\cos \theta=-\cos\psi$ we get
$$\cos \theta =\frac{b^2-n^2-p^2}{2pn}.$$

From the expressions above we obtain
$$2pn(m^2+p^2-c^2)=2pm(b^2-n^2-p^2).$$
By cancelling $2p$ on both sides and collecting we are led to
$$m^2n +mn^2 +p^2n +p^2m = b^2m + c^2 n$$
and from there
$mn(m+n)+p^2(m+n)=b^2m+c^2n$. Finally, we note that $a=m+n$ so we conclude that
$$a(mn+p^2)=b^2m+c^2n.$$
QED
%%%%%
%%%%%
\end{document}
