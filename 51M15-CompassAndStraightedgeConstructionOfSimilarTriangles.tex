\documentclass[12pt]{article}
\usepackage{pmmeta}
\pmcanonicalname{CompassAndStraightedgeConstructionOfSimilarTriangles}
\pmcreated{2013-03-22 17:16:02}
\pmmodified{2013-03-22 17:16:02}
\pmowner{Wkbj79}{1863}
\pmmodifier{Wkbj79}{1863}
\pmtitle{compass and straightedge construction of similar triangles}
\pmrecord{7}{39606}
\pmprivacy{1}
\pmauthor{Wkbj79}{1863}
\pmtype{Algorithm}
\pmcomment{trigger rebuild}
\pmclassification{msc}{51M15}
\pmclassification{msc}{51F99}

\usepackage{amssymb}
\usepackage{amsmath}
\usepackage{amsfonts}
\usepackage{pstricks}
\usepackage{psfrag}
\usepackage{graphicx}
\usepackage{amsthm}
%%\usepackage{xypic}

\begin{document}
\PMlinkescapeword{label}
\PMlinkescapeword{perpendicular}

Let $a>0$ and $b>0$.  If line segments of lengths $a$ and $b$ are constructible, one can construct a line segment of length $ab$ using compass and straightedge as follows:

\begin{enumerate}
\item Draw a line segment of length $a$.  Label its endpoints as $C$ and $D$.

\begin{center}
\begin{pspicture}(-1,-1)(6,1)
\rput[a](0,0.08){.}
\psline[linecolor=blue](0,0)(4,0)
\psdots(0,0)(4,0)
\rput[a](-0.4,-0.3){$C$}
\rput[a](4,-0.3){$D$}
\end{pspicture}
\end{center}

\item Extend the line segment past both $C$ and $D$

\begin{center}
\begin{pspicture}(-1,-1)(6,1)
\rput[a](0,0.08){.}
\rput[l](-1,0){.}
\rput[r](6,0){.}
\psline[linecolor=blue]{<->}(-1,0)(6,0)
\psline(0,0)(4,0)
\psdots(0,0)(4,0)
\rput[a](-0.4,-0.3){$C$}
\rput[a](4,-0.3){$D$}
\end{pspicture}
\end{center}

\item Erect the perpendicular to $\overleftrightarrow{CD}$ at $C$.

\begin{center}
\begin{pspicture}(-1,-1)(6,3)
\rput[l](-1,0){.}
\rput[r](6,0){.}
\rput[a](0,3){.}
\rput[b](0,-1){.}
\psline{<->}(-1,0)(6,0)
\psarc[linecolor=blue](-0.5,0){0.6}{-50}{50}
\psarc[linecolor=blue](0.5,0){0.6}{130}{230}
\psline[linecolor=blue]{<->}(0,-1)(0,3)
\psdots(0,0)(4,0)
\rput[a](-0.4,-0.3){$C$}
\rput[a](4,-0.3){$D$}
\end{pspicture}
\end{center}

\item Use the compass to determine a point $E$ on the erected perpendicular such that $CE=1$.

\begin{center}
\begin{pspicture}(-1,-1)(6,3)
\rput[l](-1,0){.}
\rput[r](6,0){.}
\rput[a](0,3){.}
\rput[b](0,-1){.}
\psline{<->}(-1,0)(6,0)
\psarc(-0.5,0){0.6}{-50}{50}
\psarc(0.5,0){0.6}{130}{230}
\psline{<->}(0,-1)(0,3)
\psarc[linecolor=blue](0,0){2}{80}{100}
\psline[linecolor=blue](0,0)(0,2)
\psdots(0,0)(4,0)(0,2)
\rput[a](-0.4,-0.3){$C$}
\rput[a](4,-0.3){$D$}
\rput[r](-0.4,1.8){$E$}
\end{pspicture}
\end{center}

\item Use the compass to determine a point $F$ on $\overrightarrow{CE}$ such that $CF=b$.

Note that the pictures indicate that $b>1$, but the exact same procedure works if $0<b\le 1$.

\begin{center}
\begin{pspicture}(-1,-1)(6,3)
\rput[l](-1,0){.}
\rput[r](6,0){.}
\rput[a](0,3){.}
\rput[b](0,-1){.}
\psline{<->}(-1,0)(6,0)
\psarc(-0.5,0){0.6}{-50}{50}
\psarc(0.5,0){0.6}{130}{230}
\psline{<->}(0,-1)(0,3)
\psarc(0,0){2}{80}{100}
\psarc[linecolor=blue](0,0){2.5}{85}{95}
\psline[linecolor=blue](0,0)(0,2.5)
\psdots(0,0)(4,0)(0,2)(0,2.5)
\rput[a](-0.4,-0.3){$C$}
\rput[a](4,-0.3){$D$}
\rput[r](-0.4,1.8){$E$}
\rput[r](-0.4,2.3){$F$}
\end{pspicture}
\end{center}

\item Connect the points $D$ and $E$.

\begin{center}
\begin{pspicture}(-1,-1)(6,3)
\rput[l](-1,0){.}
\rput[r](6,0){.}
\rput[a](0,3){.}
\rput[b](0,-1){.}
\psline{<->}(-1,0)(6,0)
\psarc(-0.5,0){0.6}{-50}{50}
\psarc(0.5,0){0.6}{130}{230}
\psline{<->}(0,-1)(0,3)
\psarc(0,0){2}{80}{100}
\psarc(0,0){2.5}{85}{95}
\psline[linecolor=blue](0,2)(4,0)
\psdots(0,0)(4,0)(0,2)(0,2.5)
\rput[a](-0.4,-0.3){$C$}
\rput[a](4,-0.3){$D$}
\rput[r](-0.4,1.8){$E$}
\rput[r](-0.4,2.3){$F$}
\end{pspicture}
\end{center}

\item Copy the angle $\angle CDE$ at $F$ to form similar triangles.  Label the intersection of the constructed ray and $\overleftrightarrow{CD}$ as $G$.

Note that, if $0<b<1$, then $F$ will be between $C$ and $E$, and $G$ will be between $C$ and $D$.  Also, if $b=1$, then $F=E$ and $G=D$.

\begin{center}
\begin{pspicture}(-1,-1)(6,3)
\rput[l](-1,0){.}
\rput[r](6,0){.}
\rput[a](0,3){.}
\rput[b](0,-1){.}
\psline{<->}(-1,0)(6,0)
\psarc(-0.5,0){0.6}{-50}{50}
\psarc(0.5,0){0.6}{130}{230}
\psline{<->}(0,-1)(0,3)
\psarc(0,0){2}{80}{100}
\psarc(0,0){2.5}{85}{95}
\psline(0,2)(4,0)
\psarc[linecolor=blue](0,2){0.4}{250}{350}
\psarc[linecolor=blue](0,2.5){0.4}{250}{350}
\psarc[linecolor=blue](0,2.1){0.4}{0}{60}
\psline[linecolor=blue]{->}(0,2.5)(6,-0.5)
\psdots(0,0)(4,0)(0,2)(0,2.5)(5,0)
\rput[a](-0.4,-0.3){$C$}
\rput[a](4,-0.3){$D$}
\rput[r](-0.4,1.8){$E$}
\rput[r](-0.4,2.3){$F$}
\rput[a](5,-0.3){$G$}
\end{pspicture}
\end{center}
\end{enumerate}

This construction is justified by the following:

\begin{itemize}
\item Since the angle $\angle CFG$ was copied from the angle $\angle CED$ and the two triangles share the angle $\angle DCE$, then the two triangles $\triangle CED$ and $\triangle CFG$ are similar;
\item Since $\triangle CED \sim \triangle CFG$, we have that $\displaystyle \frac{CE}{CF}=\frac{CD}{CG}$;
\item Plugging in $CD=a$, $CE=1$, and $CF=b$ yields that $CG=ab$.
\end{itemize}

If you are interested in seeing the rules for compass and straightedge constructions, click on the \PMlinkescapetext{link} provided.
%%%%%
%%%%%
\end{document}
