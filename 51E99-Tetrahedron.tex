\documentclass[12pt]{article}
\usepackage{pmmeta}
\pmcanonicalname{Tetrahedron}
\pmcreated{2013-03-22 14:26:32}
\pmmodified{2013-03-22 14:26:32}
\pmowner{rspuzio}{6075}
\pmmodifier{rspuzio}{6075}
\pmtitle{tetrahedron}
\pmrecord{15}{35956}
\pmprivacy{1}
\pmauthor{rspuzio}{6075}
\pmtype{Definition}
\pmcomment{trigger rebuild}
\pmclassification{msc}{51E99}
%\pmkeywords{Polyhedron}
\pmrelated{StateOnTheTetrahedron}
\pmrelated{RegularTetrahedron3}
\pmrelated{Grafix}
\pmrelated{Triangle}
\pmdefines{regular tetrahedron}

\endmetadata

% this is the default PlanetMath preamble.  as your knowledge
% of TeX increases, you will probably want to edit this, but
% it should be fine as is for beginners.

% almost certainly you want these
\usepackage{amssymb}
\usepackage{amsmath}
\usepackage{amsfonts}

% used for TeXing text within eps files
%\usepackage{psfrag}
% need this for including graphics (\includegraphics)
\usepackage{graphicx}
% for neatly defining theorems and propositions
%\usepackage{amsthm}
% making logically defined graphics
%%%\usepackage{xypic}

% there are many more packages, add them here as you need them

% define commands here
\begin{document}
\section{Definition}

A \emph{tetrahedron} is a polyhedron with four faces, which are
triangles.  A tetrahedron is called non-degenerate if the four
vertices do not lie in the same plane.  For the remainder of this
entry, we shall assume that all tetrahedra are non-degenerate.

If all six edges of a tetrahedron are equal, it is called a
\emph{regular tetrahedron}.  The faces of a regular tetrahedron are
equilateral triangles.

\section{Basic properties}

A tetrahedron has four vertices and six edges.  These six edges can be
arranged in three pairs such that the edges of a pair do not
intersect.  A tetrahedron is always convex.

In many ways, the geometry of a tetrahedron is the three-dimensional
analogue of the geometry of the triangle in two dimensions.  In
particular, the special points associated to a triangle have their
three-dimensional analogues.

Just as a triangle always can be inscribed in a unique circle, so too
a tetrahedron can be inscribed in a unique sphere.  To find the centre
of this sphere, we may construct the perpendicular bisectors of the
edges of the tetrahedron.  These six planes will meet in the centre of
the sphere which passes through the vertices of the tetrahedron.

The six planes which connect an edge with the midpoint of the opposite
edge (see what was said about edges coming in pairs above) meet in the
barycentre (a.k.a. centroid, centre of mass, centre of gravity) of the
tetrahedron.

\section{Mensuration}

Formulas for volumes, areas and lengths associated to a terahedron are
best obtained and expressed using the method of determinants.  If the
vertices of the tetrahedron are located at the points $(a_x, a_y,
a_z)$, $(b_x, b_y, b_z)$, $(c_x, c_y, c_z)$, and $(d_x, d_y, d_z)$,
then the volume of the tetrahedron is given by the following
determinant:

\[ \pm\frac{1}{6} \left| \begin{matrix} a_x & a_y & a_z & 1 \\ b_x & b_y
& b_z & 1 \\ c_x & c_y & c_z & 1 \\ d_x & d_y & d_z & 1 \\
\end{matrix} \right|. \]

\begin{figure}
\begin{center}
\includegraphics{tetrahedron.eps}
\end{center}
\caption{A regular tetrahedron}
\end{figure}

%%%%%
%%%%%
\end{document}
