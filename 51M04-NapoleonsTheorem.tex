\documentclass[12pt]{article}
\usepackage{pmmeta}
\pmcanonicalname{NapoleonsTheorem}
\pmcreated{2013-03-22 13:48:50}
\pmmodified{2013-03-22 13:48:50}
\pmowner{drini}{3}
\pmmodifier{drini}{3}
\pmtitle{Napoleon's theorem}
\pmrecord{7}{34538}
\pmprivacy{1}
\pmauthor{drini}{3}
\pmtype{Theorem}
\pmcomment{trigger rebuild}
\pmclassification{msc}{51M04}

% this is the default PlanetMath preamble.  as your 
% of TeX increases, you will probably want to edit this, but
% it should be fine as is for beginners.

% almost certainly you want these
\usepackage{amssymb}
\usepackage{amsmath}
\usepackage{amsthm}
\usepackage{amsfonts}
\usepackage{graphics}

% used for TeXing text within eps files
%\usepackage{psfrag}
% need this for including graphics (\includegraphics)
%\usepackage{graphicx}
% for neatly defining theorems and propositions
%\usepackage{amsthm}
% making logically defined graphics
%%%\usepackage{xypic}

% there are many more packages, add them here as you need them

% define commands here

\newtheorem*{thm}{Theorem}
\newtheorem*{defn}{Definition}
\newtheorem*{prop}{Proposition}
\newtheorem{lemma}{Lemma}
\newtheorem{cor}{Corollary}

\theoremstyle{definition}
\newtheorem{exa}{Example}

% Some sets
\newcommand{\Nats}{\mathbb{N}}
\newcommand{\Ints}{\mathbb{Z}}
\newcommand{\Reals}{\mathbb{R}}
\newcommand{\Complex}{\mathbb{C}}
\newcommand{\Rats}{\mathbb{Q}}
\newcommand{\Gal}{\operatorname{Gal}}
\newcommand{\Cl}{\operatorname{Cl}}
\newcommand{\ind}{\operatorname{ind}}
\begin{document}
%51M04
\PMlinkescapeword{relation}
\PMlinkescapeword{vertices}

\begin{thm} 
If equilateral triangles
are erected externally on the three sides of any given triangle, then
their centres are the vertices of an equilateral triangle.
\end{thm}
\begin{center}
\includegraphics{napoleon}
\end{center}
If we embed the statement in the complex plane, the proof
is a mere calculation. In the notation
of the figure, we can assume that $A=0$, $B=1$, and
$C$ is in the upper half plane. The hypotheses are
\begin{equation} \label{eq:hyp}
\frac{1-0}{Z-0}=\frac{C-1}{X-1}=\frac{0-C}{Y-C}=\alpha
\end{equation}
where $\alpha=\exp{\pi i/3}$, and the conclusion we want is
\begin{equation} \label{eq:conc}
\frac{N-L}{M-L}=\alpha
\end{equation}
where
$$L=\frac{1+X+C}{3}\qquad M=\frac{C+Y+0}{3}\qquad N=\frac{0+1+Z}{3}\;.$$
From (\ref{eq:hyp}) and the relation $\alpha^2=\alpha-1$, we get $X,Y,Z$:
$$X=\frac{C-1}{\alpha}+1=(1-\alpha)C+\alpha$$
$$Y=-\frac{C}{\alpha}+C=\alpha C$$
$$Z=1/{\alpha}=1-\alpha$$
and so
\begin{eqnarray*}
3(M-L)&=&Y-1-X\\
&=&(2\alpha-1)C-1-\alpha\\
3(N-L)&=&Z-X-C\\
&=&(\alpha-2)C+1-2\alpha\\
&=&(2\alpha-2-\alpha)C-\alpha+1-\alpha\\
&=&(2\alpha^2-\alpha)C-\alpha-\alpha^2\\
&=&3(M-L)\alpha
\end{eqnarray*}
proving (\ref{eq:conc}).

\textbf{Remarks:} The attribution to Napol\'eon Bonaparte (1769-1821) is
traditional, but dubious. For more on the story, see
\PMlinkexternal{MathPages}{http://www.mathpages.com/home/kmath270/kmath270.htm}.
%%%%%
%%%%%
\end{document}
