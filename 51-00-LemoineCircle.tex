\documentclass[12pt]{article}
\usepackage{pmmeta}
\pmcanonicalname{LemoineCircle}
\pmcreated{2013-03-22 12:11:05}
\pmmodified{2013-03-22 12:11:05}
\pmowner{drini}{3}
\pmmodifier{drini}{3}
\pmtitle{Lemoine circle}
\pmrecord{6}{31449}
\pmprivacy{1}
\pmauthor{drini}{3}
\pmtype{Definition}
\pmcomment{trigger rebuild}
\pmclassification{msc}{51-00}
\pmrelated{Triangle}
\pmrelated{Symmedian}
\pmrelated{LemoinePoint}
\pmrelated{Incircle}


\begin{document}
If through the Lemoine point of a triangle are drawn parallels to the sides, the six points where these intersect the circle lie all on a same circle. This circle is called the \emph{Lemoine circle} of the triangle
%%%%%
%%%%%
%%%%%
\end{document}
