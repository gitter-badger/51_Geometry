\documentclass[12pt]{article}
\usepackage{pmmeta}
\pmcanonicalname{GreatCircle}
\pmcreated{2013-03-22 16:06:02}
\pmmodified{2013-03-22 16:06:02}
\pmowner{Wkbj79}{1863}
\pmmodifier{Wkbj79}{1863}
\pmtitle{great circle}
\pmrecord{10}{38164}
\pmprivacy{1}
\pmauthor{Wkbj79}{1863}
\pmtype{Definition}
\pmcomment{trigger rebuild}
\pmclassification{msc}{51-00}
\pmrelated{VolumeOfSphericalCapAndSphericalSector}
\pmdefines{great arc}

% this is the default PlanetMath preamble.  as your knowledge
% of TeX increases, you will probably want to edit this, but
% it should be fine as is for beginners.

% almost certainly you want these
\usepackage{amssymb}
\usepackage{amsmath}
\usepackage{amsfonts}

% used for TeXing text within eps files
%\usepackage{psfrag}
% need this for including graphics (\includegraphics)
%\usepackage{graphicx}
% for neatly defining theorems and propositions
%\usepackage{amsthm}
% making logically defined graphics
%%%\usepackage{xypic}

% there are many more packages, add them here as you need them

% define commands here

\begin{document}
The intersection of a sphere with a plane that passes through the center of the sphere is called a \emph{great circle}.  Note that it is equivalent to say that a great circle of a sphere is any circle that lies on the surface of the sphere and has maximum circumference.  Geographically speaking, longitudes are examples of great circles; however, with the exception of the equator, \emph{no} latitude is a great circle.

Infinitely many great circles pass through two antipodal points of a sphere.  Otherwise, two distinct points on a sphere determine a unique great circle.

An arc of a great circle is called a \emph{great arc}.

Note that great circles and great arcs are geodesics of the surface of the sphere on which they lie.  Thus, in spherical geometry, if a sphere is serving as the model, then \PMlinkescapetext{lines} are defined to be great circles of the sphere, and \PMlinkescapetext{line segments} are defined to be great arcs of the sphere.
%%%%%
%%%%%
\end{document}
