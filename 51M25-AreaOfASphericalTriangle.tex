\documentclass[12pt]{article}
\usepackage{pmmeta}
\pmcanonicalname{AreaOfASphericalTriangle}
\pmcreated{2013-03-22 14:21:38}
\pmmodified{2013-03-22 14:21:38}
\pmowner{Mathprof}{13753}
\pmmodifier{Mathprof}{13753}
\pmtitle{area of a spherical triangle}
\pmrecord{9}{35841}
\pmprivacy{1}
\pmauthor{Mathprof}{13753}
\pmtype{Theorem}
\pmcomment{trigger rebuild}
\pmclassification{msc}{51M25}
\pmclassification{msc}{51M04}
\pmrelated{AreaOfTheNSphere}
\pmrelated{Defect}
\pmrelated{SolidAngle}
\pmrelated{LimitingTriangle}
\pmrelated{SphericalTrigonometry}

\endmetadata

% this is the default PlanetMath preamble.  as your knowledge
% of TeX increases, you will probably want to edit this, but
% it should be fine as is for beginners.

% almost certainly you want these
\usepackage{amssymb}
\usepackage{amsmath}
\usepackage{amsfonts}

% used for TeXing text within eps files
%\usepackage{psfrag}
% need this for including graphics (\includegraphics)
\usepackage{graphicx}
% for neatly defining theorems and propositions
\usepackage{amsthm}
% making logically defined graphics
%%%\usepackage{xypic}

% there are many more packages, add them here as you need them

% define commands here
\def\sse{\subseteq}
\def\bigtimes{\mathop{\mbox{\Huge $\times$}}}
\def\impl{\Rightarrow}
\newtheorem*{thm}{Theorem}
\begin{document}
A spherical triangle is formed by connecting three points on the surface
of a sphere with great arcs; these three points do not lie on a great circle of the sphere. The measurement of an angle of a spherical
triangle is intuitively obvious, since on a small scale the surface of
a sphere looks flat. More precisely, the angle at each vertex is measured as the
angle between the tangents to the incident sides in the vertex tangent plane.

Theorem.
The area of a spherical triangle $ABC$ on a sphere of radius $R$ is
\begin{equation}\label{sph-tri-area}
  S_{ABC} = (\angle A + \angle B + \angle C - \pi) R^2.
\end{equation}

Incidentally, this formula shows that the sum of the angles of a spherical
triangle must be greater than or equal to $\pi$, with equality holding
in case the triangle has zero area.

Since the sphere is compact, there might be some ambiguity as to whether
the area of the triangle or its complement is being considered. For
the purposes of the above formula, we only consider triangles with
each angle smaller than $\pi$.

An illustration of a spherical triangle
formed by points $A$, $B$, and $C$ is shown below.
\begin{center}
  \includegraphics{sph-tri.1}
\end{center}
Note that by continuing the sides of the original triangle into full
great circles, another spherical triangle is formed. The triangle $A'B'C'$
is antipodal to $ABC$ since it can be obtained by reflecting the original
one through the center of the sphere. By symmetry, both triangles must
have the same area.

\begin{proof}
For the proof of the above formula, the notion of a \emph{spherical diangle}
is helpful. As its name suggests, a diangle is formed by two great arcs
that intersect in two points, which must lie on a diameter. Two diangles
with vertices on the diameter $AA'$ are shown below.
\begin{center}
  \includegraphics{sph-tri.2}
\end{center}
At each vertex, these diangles form an angle of $\angle A$. Similarly,
we can form diangles with vertices on the diameters $BB'$ and $CC'$
respectively.
\begin{center}
  \includegraphics{sph-tri.3}
  \includegraphics{sph-tri.4}
\end{center}
Note that these diangles cover the entire sphere while overlapping
only on the triangles $ABC$ and $A'B'C'$. Hence, the total area of
the sphere can be written as
\begin{equation}\label{sph-cover}
  S_{\mathrm{sphere}} = 2S_{AA'} + 2S_{BB'} + 2S_{CC'} - 4S_{ABC}.
\end{equation}

Clearly, a diangle occupies an area that is proportional to the angle
it forms. Since the \PMlinkname{area of the sphere}{AreaOfTheNSphere}
is $4\pi R^2$, the area of a diangle of angle $\alpha$ must be $2\alpha R^2$.

Hence, we can rewrite equation \eqref{sph-cover} as
\begin{gather*}
  4\pi R^2 = 2 R^2 (2\angle A + 2\angle B + 2\angle C) - 4S_{ABC}, \\
  \therefore ~ S_{ABC} = (\angle A + \angle B + \angle C - \pi) R^2,
\end{gather*}
which is the same as equation \eqref{sph-tri-area}.
\end{proof}
%%%%%
%%%%%
\end{document}
