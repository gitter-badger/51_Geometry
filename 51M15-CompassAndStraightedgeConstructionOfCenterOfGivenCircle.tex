\documentclass[12pt]{article}
\usepackage{pmmeta}
\pmcanonicalname{CompassAndStraightedgeConstructionOfCenterOfGivenCircle}
\pmcreated{2013-03-22 17:13:44}
\pmmodified{2013-03-22 17:13:44}
\pmowner{Wkbj79}{1863}
\pmmodifier{Wkbj79}{1863}
\pmtitle{compass and straightedge construction of center of given circle}
\pmrecord{10}{39556}
\pmprivacy{1}
\pmauthor{Wkbj79}{1863}
\pmtype{Algorithm}
\pmcomment{trigger rebuild}
\pmclassification{msc}{51M15}
\pmclassification{msc}{51-00}
\pmrelated{ConstructTheCenterOfAGivenCircle}

\usepackage{amssymb}
\usepackage{amsmath}
\usepackage{amsfonts}
\usepackage{pstricks}
\usepackage{psfrag}
\usepackage{graphicx}
\usepackage{amsthm}
%%\usepackage{xypic}

\begin{document}
\PMlinkescapeword{center}
\PMlinkescapeword{label}
\PMlinkescapeword{order}

Given a circle in the Euclidean plane, one can construct its \PMlinkname{center}{Center8} using compass and straightedge as follows:

\begin{enumerate}

\item Draw a chord.  Label its endpoints as $A$ and $B$.

\begin{center}
\begin{pspicture}(-4,-4)(4,4)
\rput[r](3,0){.}
\rput[a](0,3){.}
\rput[l](-3,0){.}
\rput[b](0,-3){.}
\pscircle(0,0){3}
\psline[linecolor=blue](-2.5981,-1.5)(2.5981,-1.5)
\rput[a](-2.5981,-1.8){$A$}
\rput[a](2.5981,-1.8){$B$}
\psdots(-2.5981,-1.5)(2.5981,-1.5)
\end{pspicture}
\end{center}

\item Construct the perpendicular bisector of $\overline{AB}$ in order to find the two points $C$ and $D$ where it intersects the circle.

\begin{center}
\begin{pspicture}(-4,-4)(4,4)
\rput[r](3,0){.}
\rput[a](0,4){.}
\rput[l](-3,0){.}
\rput[b](0,-4){.}
\pscircle(0,0){3}
\psline(-2.5981,-1.5)(2.5981,-1.5)
\rput[a](-2.5981,-1.8){$A$}
\rput[a](2.5981,-1.8){$B$}
\psarc[linecolor=blue](-2.5981,-1.5){2.8}{-45}{45}
\psarc[linecolor=blue](2.5981,-1.5){2.8}{135}{225}
\psline[linecolor=blue]{<->}(0,-4)(0,4)
\rput[r](-0.1,-3.4){$C$}
\rput[r](-0.1,3.2){$D$}
\psdots(-2.5981,-1.5)(2.5981,-1.5)(0,-3)(0,3)
\end{pspicture}
\end{center}

\item Construct the perpendicular bisector of $\overline{CD}$ to determine the midpoint $O$ of $\overline{CD}$.  $O$ is the center of the circle.

\begin{center}
\begin{pspicture}(-4,-4)(4,4)
\rput[r](4,0){.}
\rput[a](0,4){.}
\rput[l](-4,0){.}
\rput[b](0,-4){.}
\pscircle(0,0){3}
\psline(-2.5981,-1.5)(2.5981,-1.5)
\rput[a](-2.5981,-1.8){$A$}
\rput[a](2.5981,-1.8){$B$}
\psarc(-2.5981,-1.5){2.8}{-45}{45}
\psarc(2.5981,-1.5){2.8}{135}{225}
\psline{<->}(0,-4)(0,4)
\rput[r](-0.1,-3.4){$C$}
\rput[r](-0.1,3.2){$D$}
\psarc[linecolor=blue](0,-3){3.2}{60}{120}
\psarc[linecolor=blue](0,3){3.2}{240}{300}
\psline[linecolor=blue]{<->}(-4,0)(4,0)
\rput[r](-0.1,0){$O$}
\psdots(-2.5981,-1.5)(2.5981,-1.5)(0,-3)(0,3)(0,0)
\end{pspicture}
\end{center}

\end{enumerate}

A justification for these constructions is supplied in the entry construct the center of a given circle.

If you are interested in seeing the rules for compass and straightedge constructions, click on the \PMlinkescapetext{link} provided.
%%%%%
%%%%%
\end{document}
