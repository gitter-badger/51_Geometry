\documentclass[12pt]{article}
\usepackage{pmmeta}
\pmcanonicalname{DeterminingRotationsAndReflectionsInmathbbR2}
\pmcreated{2013-03-22 17:14:34}
\pmmodified{2013-03-22 17:14:34}
\pmowner{Wkbj79}{1863}
\pmmodifier{Wkbj79}{1863}
\pmtitle{determining rotations and reflections in $\mathbb{R}^2$}
\pmrecord{4}{39572}
\pmprivacy{1}
\pmauthor{Wkbj79}{1863}
\pmtype{Derivation}
\pmcomment{trigger rebuild}
\pmclassification{msc}{51A15}
\pmclassification{msc}{51A10}
\pmclassification{msc}{15A04}
\pmrelated{Symmetry2}

\usepackage{amssymb}
\usepackage{amsmath}
\usepackage{amsfonts}
\usepackage{pstricks}
\usepackage{psfrag}
\usepackage{graphicx}
\usepackage{amsthm}
%%\usepackage{xypic}

\begin{document}
\PMlinkescapeword{formula}

Let $E \colon \mathbb{R}^2 \to \mathbb{R}^2$ be a rotation about some point $(x_0,y_0)$, and let $\theta$ the angle of rotation for $E$.  A formula for $E$ can be determined as follows.

First, translate the point $(x_0,y_0)$ to $(0,0)$.  The map of this translation is $T(x,y)=(x-x_0,y-y_0)$.

Next, rotate by $\theta$ about the origin.  The map of this rotation is $R(x,y)=(x\cos\theta-y\sin\theta,x\sin\theta+y\cos\theta)$.

Finally, translate the point $(0,0)$ back to $(x_0,y_0)$.  The map of this translation is $T^{-1}(x,y)=(x+x_0,y+y_0)$.

The fact that $E=T^{-1} \circ R \circ T$ can be used to obtain a formula for $E$:

\begin{center}
$\begin{array}{rl}
E(x,y) & =(T^{-1} \circ R \circ T)(x,y) \\
& =(T^{-1} \circ R)(x-x_1,y-y_0) \\
& =T^{-1}((x-x_0)\cos\theta-(y-y_0)\sin\theta,(x-x_0)\sin\theta+(y-y_0)\cos\theta) \\
& =((x-x_0)\cos\theta-(y-y_0)\sin\theta+x_0,(x-x_0)\sin\theta+(y-y_0)\cos\theta+y_0). \end{array}$
\end{center}

Let $E \colon \mathbb{R}^2 \to \mathbb{R}^2$ be a reflection about some line $y=mx+b$.  Let $\theta=2\arctan m$.  A formula for $E$ can be determined as follows.

First, translate the $y$ intercept $(0,b)$ to $(0,0)$.  The map of this translation is $B(x,y)=(x,y-b)$.

Next, reflect about the line $y=mx$.  The map of this reflection is $F(x,y)=(x\cos\theta+y\sin\theta,x\sin\theta-y\cos\theta)$.

Finally, translate the point $(0,0)$ back to $(0,b)$.  The map of this translation is $B^{-1}(x,y)=(x,y+b)$.

The fact that $E=B^{-1} \circ F \circ B$ can be used to obtain a formula for $E$:

\begin{center}
$\begin{array}{rl}
E(x,y) & =(B^{-1} \circ F \circ B)(x,y) \\
& =(B^{-1} \circ F)(x,y-b) \\
& =B^{-1}(x\cos\theta+(y-b)\sin\theta,x\sin\theta-(y-b)\cos\theta) \\
& =(x\cos\theta+(y-b)\sin\theta,x\sin\theta-(y-b)\cos\theta+b). \end{array}$
\end{center}

Let $E \colon \mathbb{R}^2 \to \mathbb{R}^2$ be a reflection about some line $x=c$.  A formula for $E$ can be determined as follows.

First, translate the $x$ intercept $(c,0)$ to $(0,0)$.  The map of this translation is $C(x,y)=(x-c,y)$.

Next, reflect about the line $x=0$.  The map of this reflection is $M(x,y)=(-x,y)$.

Finally, translate the point $(0,0)$ back to $(c,0)$.  The map of this translation is $C^{-1}(x,y)=(x+c,y)$.

The fact that $E=C^{-1} \circ M \circ C$ can be used to obtain a formula for $E$:

\begin{center}
$\begin{array}{rl}
E(x,y) & =(C^{-1} \circ M \circ C)(x,y) \\
& =(C^{-1} \circ M)(x-c,y) \\
& =C^{-1}(c-x,y) \\
& =(2c-x,y). \end{array}$
\end{center}
%%%%%
%%%%%
\end{document}
