\documentclass[12pt]{article}
\usepackage{pmmeta}
\pmcanonicalname{ProofOfHeronsFormula}
\pmcreated{2013-03-22 12:41:38}
\pmmodified{2013-03-22 12:41:38}
\pmowner{mathwizard}{128}
\pmmodifier{mathwizard}{128}
\pmtitle{proof of Heron's formula}
\pmrecord{5}{32974}
\pmprivacy{1}
\pmauthor{mathwizard}{128}
\pmtype{Proof}
\pmcomment{trigger rebuild}
\pmclassification{msc}{51-00}

% this is the default PlanetMath preamble.  as your knowledge
% of TeX increases, you will probably want to edit this, but
% it should be fine as is for beginners.

% almost certainly you want these
\usepackage{amssymb}
\usepackage{amsmath}
\usepackage{amsfonts}

% used for TeXing text within eps files
%\usepackage{psfrag}
% need this for including graphics (\includegraphics)
%\usepackage{graphicx}
% for neatly defining theorems and propositions
%\usepackage{amsthm}
% making logically defined graphics
%%%\usepackage{xypic}

% there are many more packages, add them here as you need them

% define commands here
\begin{document}
Let $\alpha$ be the angle between the sides $b$ and $c$, then we get from the cosines law:
$$\cos\alpha =\frac{b^2+c^2-a^2}{2bc}.$$
Using the equation
$$\sin\alpha=\sqrt{1-\cos^2\alpha}$$
we get:
$$\sin\alpha=\frac{\sqrt{-a^4-b^4-c^4+2b^2c^2+2a^2b^2+2a^2c^2}}{2bc}.$$
Now we know:
$$\Delta=\frac{1}{2}bc\sin\alpha.$$
So we get:
\begin{eqnarray*}
\Delta & = & \frac{1}{4}\sqrt{-a^4-b^4-c^4+2b^2c^2+2a^2b^2+2a^2c^2}\\
& = & \frac{1}{4}\sqrt{(a+b+c)(b+c-a)(a+c-b)(a+b-c)}\\
& = & \sqrt{s(s-a)(s-b)(s-c)}.
\end{eqnarray*}
This is Heron's formula. $\Box$
%%%%%
%%%%%
\end{document}
