\documentclass[12pt]{article}
\usepackage{pmmeta}
\pmcanonicalname{ExampleNeedingTwoLagrangeMultipliers}
\pmcreated{2013-03-22 18:48:18}
\pmmodified{2013-03-22 18:48:18}
\pmowner{pahio}{2872}
\pmmodifier{pahio}{2872}
\pmtitle{example needing two Lagrange multipliers}
\pmrecord{7}{41606}
\pmprivacy{1}
\pmauthor{pahio}{2872}
\pmtype{Example}
\pmcomment{trigger rebuild}
\pmclassification{msc}{51N20}
\pmclassification{msc}{26B10}
\pmsynonym{using Lagrange multipliers to find semi-axes}{ExampleNeedingTwoLagrangeMultipliers}
%\pmkeywords{Lagrange multiplier}
\pmrelated{ExampleOfLagrangeMultipliers}
\pmrelated{ExampleOfUsingLagrangeMultipliers}

\endmetadata

% this is the default PlanetMath preamble.  as your knowledge
% of TeX increases, you will probably want to edit this, but
% it should be fine as is for beginners.

% almost certainly you want these
\usepackage{amssymb}
\usepackage{amsmath}
\usepackage{amsfonts}

% used for TeXing text within eps files
%\usepackage{psfrag}
% need this for including graphics (\includegraphics)
%\usepackage{graphicx}
% for neatly defining theorems and propositions
 \usepackage{amsthm}
% making logically defined graphics
%%%\usepackage{xypic}

% there are many more packages, add them here as you need them

% define commands here

\theoremstyle{definition}
\newtheorem*{thmplain}{Theorem}

\begin{document}
Find the semi-axes of the ellipse of intersection, formed when the plane \,$z = x\!+\!y$\, intersects the ellipsoid 
$$\frac{x^2}{4}+\frac{y^2}{5}+\frac{z^2}{25} = 1.$$\\

Let \,$(x,\,y,\,z)$\, be any point of the ellipsoid.\, The \PMlinkname{square}{SquareOfNumber} $x^2\!+\!y^2\!+\!z^2$ of the distance of this point from the \PMlinkname{midpoint}{Midpoint3} \,$(0,\,0,\,0)$\, has under the constraints
\begin{align}
\begin{cases}
g \;:=\; \frac{x^2}{4}+\frac{y^2}{5}+\frac{z^2}{25}-1 \;=\;0,\\
h \;:=\; x+y-z \;=\; 0
\end{cases}
\end{align}
the minimum and maximum values at the end points of the semi-axes of the ellipse.\, Since we have two constraints, we must take equally many Lagrange multipliers, $\lambda$ and $\mu$.\, A necessary condition of the extremums of
$$f \;:=\, x^2\!+\!y^2\!+\!z^2$$
is that in \PMlinkescapetext{addition} to (1), also the equations
\begin{align}
\begin{cases}
\frac{\partial f}{\partial x}+\lambda\frac{\partial g}{\partial x}+\mu\frac{\partial h}{\partial x}
\;=\; 2x+\frac{1}{2}x\lambda+\mu \;=\; 0,\\
\frac{\partial f}{\partial y}+\lambda\frac{\partial g}{\partial y}+\mu\frac{\partial h}{\partial y}
\;=\; 2y+\frac{2}{5}y\lambda+\mu \;=\; 0,\\
\frac{\partial f}{\partial z}+\lambda\frac{\partial g}{\partial z}+\mu\frac{\partial h}{\partial z}
\;=\; 2z+\frac{2}{25}z\lambda-\mu \;=\; 0,
\end{cases}
\end{align}
are satisfied.\, I.e., we have five equations (1), (2) and five unknowns $\lambda$, $\mu$, $x$, $y$, $z$.

The equations (2) give
$$x \;=\; -\frac{2\mu}{\lambda\!+\!4}, \quad y \;=\; -\frac{5\mu}{2\lambda\!+\!10}, 
\quad z \;=\; \frac{25\mu}{2\lambda\!+\!50},$$
which expressions may be put into the equation\, $h = 0$, and so on.\, One obtains the values 
$$\lambda_1 = -10, \quad \lambda_2 = -\frac{75}{17}, \quad \mu_1 = \pm6\sqrt{\frac{5}{19}}, 
\quad \mu_2 = \pm\frac{140}{17\sqrt{646}}$$
with which the extremum points \,$(x,\,y,\,z)$ can be evaluated.\, The corresponding values of $f$are 10 and $\frac{75}{17}$, whence the major semi-axis is $\sqrt{10}$ and the minor semi-axis $\frac{5\sqrt{255}}{17}$.

%%%%%
%%%%%
\end{document}
