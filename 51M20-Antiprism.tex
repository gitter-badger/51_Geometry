\documentclass[12pt]{article}
\usepackage{pmmeta}
\pmcanonicalname{Antiprism}
\pmcreated{2013-03-22 16:57:02}
\pmmodified{2013-03-22 16:57:02}
\pmowner{Wkbj79}{1863}
\pmmodifier{Wkbj79}{1863}
\pmtitle{antiprism}
\pmrecord{9}{39219}
\pmprivacy{1}
\pmauthor{Wkbj79}{1863}
\pmtype{Definition}
\pmcomment{trigger rebuild}
\pmclassification{msc}{51M20}
\pmclassification{msc}{51M04}

\endmetadata

\usepackage{amssymb}
\usepackage{amsmath}
\usepackage{amsfonts}

\usepackage{psfrag}
\usepackage{graphicx}
\usepackage{amsthm}
%%\usepackage{xypic}

\begin{document}
\PMlinkescapeword{bases}
\PMlinkescapeword{base}
\PMlinkescapeword{opposite}

An \emph{antiprism} is a prismatoid whose bases are congruent and whose lateral faces are congruent triangles.

An antiprism can be used to construct an icosahedron:  Let $A$ be an antiprism with regular pentagons as its bases and equilateral triangles as its lateral faces.  To each base of $A$, attach a \PMlinkescapetext{right} pentagonal pyramid whose base is congruent to the bases of $A$ and whose lateral faces are equilateral triangles.

For any integer $n \ge 3$, the following are equivalent statements about an antiprism $A$:

\begin{enumerate}
\item $A$ has a base that is an $n$-gon;
\item $A$ has $2n+2$ faces;
\item $A$ has $2n$ vertices;
\item $A$ has $4n$ edges.
\end{enumerate}
%%%%%
%%%%%
\end{document}
