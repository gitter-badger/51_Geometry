\documentclass[12pt]{article}
\usepackage{pmmeta}
\pmcanonicalname{ApolloniusCircle}
\pmcreated{2013-03-22 11:44:22}
\pmmodified{2013-03-22 11:44:22}
\pmowner{drini}{3}
\pmmodifier{drini}{3}
\pmtitle{Apollonius' circle}
\pmrecord{11}{30154}
\pmprivacy{1}
\pmauthor{drini}{3}
\pmtype{Definition}
\pmcomment{trigger rebuild}
\pmclassification{msc}{51-00}
\pmclassification{msc}{35-01}
%\pmkeywords{locus}
%\pmkeywords{homothety}
\pmrelated{HarmonicDivision}

\endmetadata

\usepackage{amssymb}
\usepackage{amsmath}
\usepackage{amsfonts}
\usepackage{graphicx}
%%%%%%%\usepackage{xypic}
\begin{document}
\textbf{Apollonius' circle}.
The locus of a point moving so that the ratio of its distances from two fixed points is fixed, is a circle.
\medskip


If two circles $C_1$ and $C_2$ are fixed with radii $r_1$ and $r_2$, then the circle of Apollonius of the two centers with ratio $r_1/r_2$ is the circle whose diameter is the segment that \PMlinkescapetext{joins} the two homothety centers of the circles.
%%%%%
%%%%%
%%%%%
%%%%%
%%%%%
%%%%%
%%%%%
\end{document}
