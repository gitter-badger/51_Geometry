\documentclass[12pt]{article}
\usepackage{pmmeta}
\pmcanonicalname{SumOfAnglesOfTriangleInEuclideanGeometry}
\pmcreated{2013-09-26 10:00:16}
\pmmodified{2013-09-26 10:00:16}
\pmowner{pahio}{2872}
\pmmodifier{pahio}{2872}
\pmtitle{sum of angles of triangle in Euclidean geometry}
\pmrecord{4}{87388}
\pmprivacy{1}
\pmauthor{pahio}{2872}
\pmtype{Theorem}
\pmcomment{Intuitive proof}
\pmclassification{msc}{51M05}

\endmetadata

% this is the default PlanetMath preamble.  as your knowledge
% of TeX increases, you will probably want to edit this, but
% it should be fine as is for beginners.

% almost certainly you want these
\usepackage{amssymb}
\usepackage{amsmath}
\usepackage{amsfonts}

% need this for including graphics (\includegraphics)
\usepackage{graphicx}
% for neatly defining theorems and propositions
\usepackage{amsthm}

% making logically defined graphics
%\usepackage{xypic}
% used for TeXing text within eps files
%\usepackage{psfrag}

% there are many more packages, add them here as you need them

% define commands here

\begin{document}
The parallel postulate (in the form given in the \PMlinkname{parent entry}{ParallelPostulate}) allows to prove the important fact about the triangles in the Euclidean geometry:

\textbf{Theorem.}\, The sum of the interior angles of any triangle equals the straight angle.

{\it Proof.}\, Let $ABC$ be an arbitrary triangle with the interior angles $\alpha$, $\beta$, $\gamma$.\, In the plane of the triangle we set the lines $AD$ and $AE$ such that\, $\wedge BAD = \beta$\, and\, $\wedge CAE = \gamma$.\, Then the lines do not intersect the line $BC$.\, In fact, if e.g. $AD$ would intersect $BC$ in a point $P$, then there would exist a triangle $ABP$ where an \PMlinkname{exterior angle}{ExteriorAnglesOfTriangle} of an angle would equal to an interior angle of another angle which is impossible.\, Thus $AD$ and $AE$ are both parallel to $BC$.\, By the parallel postulate, these lines have to coincide.\, This means that the addition of the triangle angles $\alpha$, $\beta$, $\gamma$ gives a straight angle.

See also \PMlinkexternal{this intuitive proof}{http://www.cs.bham.ac.uk/research/projects/cogaff/misc/triangle-sum.html#pardoe-proof}!

\begin{thebibliography}{8}
\bibitem{ariva}{\sc Karl Ariva}: {\it Lobatsevski geomeetria}.\, Kirjastus ``Valgus'', Tallinn (1992).
\end{thebibliography} 
\end{document}
