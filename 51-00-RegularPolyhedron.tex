\documentclass[12pt]{article}
\usepackage{pmmeta}
\pmcanonicalname{RegularPolyhedron}
\pmcreated{2013-03-22 12:24:17}
\pmmodified{2013-03-22 12:24:17}
\pmowner{mathwizard}{128}
\pmmodifier{mathwizard}{128}
\pmtitle{regular polyhedron}
\pmrecord{20}{32254}
\pmprivacy{1}
\pmauthor{mathwizard}{128}
\pmtype{Definition}
\pmcomment{trigger rebuild}
\pmclassification{msc}{51-00}
\pmsynonym{Platonic solid}{RegularPolyhedron}
\pmsynonym{regular polyhedra}{RegularPolyhedron}
\pmsynonym{regular}{RegularPolyhedron}
\pmrelated{RegularPolygon}
\pmrelated{Grafix}
\pmdefines{tetrahedron}
\pmdefines{octahedron}
\pmdefines{dodecahedron}
\pmdefines{icosahedron}
\pmdefines{regular tetrahedron}
\pmdefines{regular octahedron}
\pmdefines{regular dodecahedron}
\pmdefines{regular icosahedron}

\endmetadata

\usepackage{graphicx}
%%%\usepackage{xypic} 
%\usepackage{bbm}
%\newcommand{\Z}{\mathbbmss{Z}}
%\newcommand{\C}{\mathbbmss{C}}
%\newcommand{\R}{\mathbbmss{R}}
%\newcommand{\Q}{\mathbbmss{Q}}
%\newcommand{\mathbb}[1]{\mathbbmss{#1}}
\begin{document}
A \emph{regular polyhedron} is a polyhedron such that
\begin{itemize}
\item Every face is a regular polygon.
\item On each vertex, the same number of edges concur.
\item The dihedral angle between any two faces is always the same.
\end{itemize}

These polyhedra are also known as Platonic solids, since Plato described them in his work. There are only 5 regular polyhedra, as was first shown by Theaetetus, one of Plato's students. Some sources ascribe to Theaetetus also the discovery of the dodecahedron. 

The five solids are:
\begin{description}
\item[Regular Tetrahedron] It has 6 edges and 4 vertices and 4 faces, each one being an equilateral triangle. Its symmetry group is $S_4$.
\item[Regular Hexahedron] Also known as cube. It has 8 vertices, 12 edges and 6 faces each one being a square. Its symmetry group is $S_4\times C_2$.
\item[Regular Octahedron] It has 6 vertices, 12 edges and 8 faces, each one being an equilateral triangle Its symmetry group is $S_4\times C_2$.
\item[Regular Dodecahedron] It has 20 vertices, 30 edges and 12 faces, each one being a regular pentagon. Its symmetry group is $A_5\times C_2$.
\item[Regular Icosahedron] It has 12 vertices, 30 edges and 20 faces, each one being an equilateral triangle. Its symmetry group is $A_5\times C_2$.
\end{description}
\begin{figure}[h]
\begin{centering}
\includegraphics[scale=0.7]{platonic.ps}
\caption{The five Platonic solids -- created in Blender 2.36. (Download the \htmladdnormallink{Blender file}{http://aux.planetmath.org/files/objects/2254/platonic.blend} for this picture.)}
\end{centering}
\end{figure}

\small Note: $A_n$ is the alternating group of order $n$, $S_n$ is the symmetric group of order $n$ and $C_n$ is the cyclic group with order $n$.
%%%%%
%%%%%
\end{document}
