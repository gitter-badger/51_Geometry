\documentclass[12pt]{article}
\usepackage{pmmeta}
\pmcanonicalname{PaschsTheorem}
\pmcreated{2013-03-22 15:32:09}
\pmmodified{2013-03-22 15:32:09}
\pmowner{CWoo}{3771}
\pmmodifier{CWoo}{3771}
\pmtitle{Pasch's theorem}
\pmrecord{13}{37429}
\pmprivacy{1}
\pmauthor{CWoo}{3771}
\pmtype{Theorem}
\pmcomment{trigger rebuild}
\pmclassification{msc}{51G05}
\pmrelated{Angle}
\pmrelated{OrderedGeometry}

\endmetadata

\usepackage{amssymb,amscd}
\usepackage{amsmath}
\usepackage{amsfonts}
\usepackage{amsthm}

% used for TeXing text within eps files
%\usepackage{psfrag}
% need this for including graphics (\includegraphics)
%\usepackage{graphicx}
% for neatly defining theorems and propositions
%\usepackage{amsthm}
% making logically defined graphics
%%%\usepackage{xypic}

% define commands here
\newtheorem*{thm}{Theorem}
\begin{document}
\begin{thm}{(Pasch)}  Let $\triangle abc$ be a triangle with
non-collinear vertices $a,b,c$ in a {linear} ordered geometry.
Suppose a line $\ell$ intersects one side, say open line segment $\overline{ab}$, at a
point strictly between $a$ and $b$, then $\ell$ also intersects exactly one of the following:
$$\overline{bc}\mbox{, }\qquad\qquad\overline{ac}\mbox{, }\qquad\qquad c.$$
\end{thm}
\begin{proof}
First, note that vertices $a$ and $b$ are on opposite sides of line $\ell$.  Then either $c$ lies on $\ell$, or $c$ does not. if $c$ does not, then it must lie on either side (half plane) of $\ell$.  In other words, $c$ and $a$ must be on the opposite sides of $\ell$, or $c$ and $b$ must be on the opposite sides of $\ell$.
If $c$ and $a$ are on the opposite sides, $\ell$ has a non-empty intersection with $\overline{ac}$.  But if $c$ and $a$ are on the opposite sides, then $c$ and $b$ are on the same side, which means that $\overline{bc}$ does not intersect $\ell$.
\end{proof}
\textbf{Remark}
 A companion property states that if line $\ell$ passes through one vertex $a$ of a triangle $\triangle abc$ and at least one other point on $\triangle abc$, then it must intersect exactly one of the following:
$$b\mbox{, }\qquad\qquad c\mbox{, }\qquad\qquad\overline{bc}.$$
Of course, if $\ell$ passes through $b$, $\overline{ab}$ must lie on $\ell$.  Similarly, $\overline{ac}$ lies on $\ell$ if $\ell$ passes through $c$.
%%%%%
%%%%%
\end{document}
