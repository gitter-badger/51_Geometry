\documentclass[12pt]{article}
\usepackage{pmmeta}
\pmcanonicalname{ConstructionOfFourthProportional}
\pmcreated{2013-03-22 18:49:59}
\pmmodified{2013-03-22 18:49:59}
\pmowner{pahio}{2872}
\pmmodifier{pahio}{2872}
\pmtitle{construction of fourth proportional}
\pmrecord{7}{41637}
\pmprivacy{1}
\pmauthor{pahio}{2872}
\pmtype{Application}
\pmcomment{trigger rebuild}
\pmclassification{msc}{51M15}
\pmclassification{msc}{51M04}
\pmrelated{ConstructionOfCentralProportion}
\pmrelated{ProportionEquation}

% this is the default PlanetMath preamble.  as your knowledge
% of TeX increases, you will probably want to edit this, but
% it should be fine as is for beginners.

% almost certainly you want these
\usepackage{amssymb}
\usepackage{amsmath}
\usepackage{amsfonts}

% used for TeXing text within eps files
%\usepackage{psfrag}
% need this for including graphics (\includegraphics)
%\usepackage{graphicx}
% for neatly defining theorems and propositions
 \usepackage{amsthm}
% making logically defined graphics
%%%\usepackage{xypic}
\usepackage{pstricks}
\usepackage{pst-plot}

% there are many more packages, add them here as you need them

% define commands here

\theoremstyle{definition}
\newtheorem*{thmplain}{Theorem}

\begin{document}
\PMlinkescapeword{solution} \PMlinkescapeword{side}
\textbf{Task.}\, Given three line segments $a$, $b$ and $c$.\, Using compass and straightedge, construct the fourth proportional of the line segments.\\

\emph{Solution.}\, Draw an angle ($\alpha$) and denote its \PMlinkname{vertex}{Angle} by $P$.\, Separate from one \PMlinkname{side}{Angle} of the angle the line segments \,$PA = a$\, and\, $AB = b$, and from the other side of the angle the line segment \,$PC = c$.\, Draw the line $AC$ and another line parallel to it passing through $B$.\, If the last line intersects the other side of the angle in the point $D$, then the line segment \,$CD = x$\, is the required fourth proportional:
$$a:b \;=\; c:x$$
Justification: the intercept theorem.

The below picture illustrates this solution:

\begin{center}
\begin{pspicture}(-1.5,-0.5)(5,4)
\rput(-1.5,-0.5){.}
\rput(5,4){.}
\psline(4.2,2.5)(0,0)(4.7,0)
\psline(2,0)(1.5,2)
\psline(4.2,0)(3.43,3)
\rput(0.6,0.15){$\alpha$}
\rput(-0.3,0){$P$}
\rput(2,-0.2){$A$}
\rput(4.2,-0.2){$B$}
\rput(2,0.95){$C$}
\rput(4,2.1){$D$}
\rput(1.1,-0.2){$a$}
\rput(3.1,-0.2){$b$}
\rput(1,0.8){$c$}
\rput(2.7,1.85){$x$}
\end{pspicture}
\end{center}

\textbf{Note.}\, The special case \,$c = b$\, gives the third proportional $x$ of $a$ and $b$:
$$a:b \;=\; b:x$$
%%%%%
%%%%%
\end{document}
