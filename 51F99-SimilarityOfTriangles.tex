\documentclass[12pt]{article}
\usepackage{pmmeta}
\pmcanonicalname{SimilarityOfTriangles}
\pmcreated{2013-03-22 17:49:41}
\pmmodified{2013-03-22 17:49:41}
\pmowner{pahio}{2872}
\pmmodifier{pahio}{2872}
\pmtitle{similarity of triangles}
\pmrecord{12}{40292}
\pmprivacy{1}
\pmauthor{pahio}{2872}
\pmtype{Theorem}
\pmcomment{trigger rebuild}
\pmclassification{msc}{51F99}
\pmclassification{msc}{51M05}
\pmclassification{msc}{51-00}
\pmsynonym{similar triangles}{SimilarityOfTriangles}
%\pmkeywords{equal angles}
%\pmkeywords{proportional sides}
\pmrelated{HarmonicMeanInTrapezoid}
\pmrelated{AreaOfSphericalCalotteByMeansOfChord}
\pmrelated{InterceptTheorem}
\pmdefines{AA}
\pmdefines{AA postulate}
\pmdefines{AA theorem}

\endmetadata

% this is the default PlanetMath preamble.  as your knowledge
% of TeX increases, you will probably want to edit this, but
% it should be fine as is for beginners.

% almost certainly you want these
\usepackage{amssymb}
\usepackage{amsmath}
\usepackage{amsfonts}

% used for TeXing text within eps files
%\usepackage{psfrag}
% need this for including graphics (\includegraphics)
%\usepackage{graphicx}
% for neatly defining theorems and propositions
 \usepackage{amsthm}
% making logically defined graphics
%%%\usepackage{xypic}

% there are many more packages, add them here as you need them

% define commands here

\theoremstyle{definition}
\newtheorem*{thmplain}{Theorem}

\begin{document}
The following theorems are valid in Euclidean geometry:

\textbf{Theorem AA.}\, If one triangle has a pair of angles that are congruent to a pair of angles in another triangle, then the two triangles are similar.

\textbf{Theorem \PMlinkescapetext{SAS}.}\, If a pair of sides of a triangle are proportional to a pair of sides in another triangle and if the angles included by the side-pairs are congruent, then the triangles are similar.

\textbf{Theorem \PMlinkescapetext{SSS}.}\, If the sides of a triangle are proportional to the sides of another triangle, then the triangles are similar.\\

The AA theorem may be regarded as the definition of the similarity of triangles.  In some texts, the AA theorem is assumed as a postulate.  The other two theorems may be proved by using the law of cosines for determining the the ratios other sides (for \PMlinkescapetext{SAS}) and the angles.

In hyperbolic geometry and spherical geometry, similar triangles are congruent.  (See the AAA theorem for more details.)  Thus, the \PMlinkescapetext{SAS} theorem and \PMlinkescapetext{SSS} theorem are invalid in these \PMlinkescapetext{geometries}.
%%%%%
%%%%%
\end{document}
