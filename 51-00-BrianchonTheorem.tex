\documentclass[12pt]{article}
\usepackage{pmmeta}
\pmcanonicalname{BrianchonTheorem}
\pmcreated{2013-03-22 14:04:29}
\pmmodified{2013-03-22 14:04:29}
\pmowner{vmoraru}{1243}
\pmmodifier{vmoraru}{1243}
\pmtitle{Brianchon theorem}
\pmrecord{7}{35434}
\pmprivacy{1}
\pmauthor{vmoraru}{1243}
\pmtype{Theorem}
\pmcomment{trigger rebuild}
\pmclassification{msc}{51-00}

\endmetadata

\usepackage{amssymb}
\usepackage{amsmath}
\usepackage{amsfonts}
\usepackage{graphicx}
\usepackage{amsthm}
%%\usepackage{xypic}
\begin{document}
If an hexagon $ABCDEF$ (not necessarily convex) is inscribed into
a conic (in particular into a circle), then the three diagonals
$AD, BE, CF$ are concurrent. This theorem is the dual of Pascal
line theorem. (C. Brianchon, 1806)
\begin{center}
\includegraphics{brianchon}
\end{center}

\textbf{Proof:}It follows by duality.
%%%%%
%%%%%
\end{document}
