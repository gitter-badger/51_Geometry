\documentclass[12pt]{article}
\usepackage{pmmeta}
\pmcanonicalname{BisectorsTheorem}
\pmcreated{2013-03-22 14:49:23}
\pmmodified{2013-03-22 14:49:23}
\pmowner{drini}{3}
\pmmodifier{drini}{3}
\pmtitle{bisectors theorem}
\pmrecord{9}{36485}
\pmprivacy{1}
\pmauthor{drini}{3}
\pmtype{Theorem}
\pmcomment{trigger rebuild}
\pmclassification{msc}{51A05}
\pmsynonym{angle bisector theorem}{BisectorsTheorem}
\pmsynonym{generalization of bisectors theorem}{BisectorsTheorem}
\pmrelated{CevasTheorem}
\pmrelated{HarmonicDivision}

\usepackage{graphicx}
%%%\usepackage{xypic} 
\usepackage{bbm}
\newcommand{\Z}{\mathbbmss{Z}}
\newcommand{\C}{\mathbbmss{C}}
\newcommand{\R}{\mathbbmss{R}}
\newcommand{\Q}{\mathbbmss{Q}}
\newcommand{\mathbb}[1]{\mathbbmss{#1}}
\newcommand{\figura}[1]{\begin{center}\includegraphics{#1}\end{center}}
\newcommand{\figuraex}[2]{\begin{center}\includegraphics[#2]{#1}\end{center}}
\newtheorem{dfn}{Definition}
\begin{document}
\textbf{Bisectors theorem}.\\
Let $ABC$ a triangle, and $AP$ the angle bisector of $\angle CAB$. Then
\[
\frac{BP}{PC}=\frac{BA}{CA}.
\]
\figuraex{bisector}{scale=0.75}
\bigskip

\textbf{Generalization of bisectors theorem}.\\
Let $ABC$ a triangle, and $P$ any point on the side $BC$. Then
\[
\frac{BP}{PC} = \frac{BA \sin BAP}{CA \sin PAC}.
\]
\figuraex{genbisector}{scale=0.75}

The latest theorem even holds when $P$ is not on the segment $BC$ but anywhere on the line (possibly at the infnity), but in this case, the ratio $BP/PC$ must be done with directed segments and the angles considered as directed angles.
%%%%%
%%%%%
\end{document}
