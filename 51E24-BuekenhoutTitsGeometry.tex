\documentclass[12pt]{article}
\usepackage{pmmeta}
\pmcanonicalname{BuekenhoutTitsGeometry}
\pmcreated{2013-03-22 19:13:50}
\pmmodified{2013-03-22 19:13:50}
\pmowner{CWoo}{3771}
\pmmodifier{CWoo}{3771}
\pmtitle{Buekenhout-Tits geometry}
\pmrecord{10}{42154}
\pmprivacy{1}
\pmauthor{CWoo}{3771}
\pmtype{Definition}
\pmcomment{trigger rebuild}
\pmclassification{msc}{51E24}
\pmclassification{msc}{20C33}
\pmclassification{msc}{05B25}
\pmsynonym{Buekenhout geometry}{BuekenhoutTitsGeometry}
\pmsynonym{Tits geometry}{BuekenhoutTitsGeometry}
\pmsynonym{diagram geometry}{BuekenhoutTitsGeometry}
\pmrelated{IncidenceStructures}
\pmdefines{rank}
\pmdefines{graph of a geometry}

\usepackage{amssymb,amscd}
\usepackage{amsmath}
\usepackage{amsfonts}
\usepackage{mathrsfs}

% used for TeXing text within eps files
%\usepackage{psfrag}
% need this for including graphics (\includegraphics)
%\usepackage{graphicx}
% for neatly defining theorems and propositions
\usepackage{amsthm}
% making logically defined graphics
%%\usepackage{xypic}
\usepackage{pst-plot}

% define commands here
\newcommand*{\abs}[1]{\left\lvert #1\right\rvert}
\newtheorem{prop}{Proposition}
\newtheorem{thm}{Theorem}
\newtheorem{ex}{Example}
\newcommand{\real}{\mathbb{R}}
\newcommand{\pdiff}[2]{\frac{\partial #1}{\partial #2}}
\newcommand{\mpdiff}[3]{\frac{\partial^#1 #2}{\partial #3^#1}}
\begin{document}
A \emph{Buekenhout-Tits geoemtry}, or simply \emph{geometry}, is a quadruple $(\Gamma,\#,T,\tau)$, where
\begin{enumerate}
\item $\Gamma$ is a set whose elements are called objects or varieties
\item $T$ is a set whose elements are called types
\item $\tau$ is a function from $\Gamma$ onto $T$, and we say that two objects $a,b\in \Gamma$ are of the same type if $\tau(a)=\tau(b)$
\item $\#$ is a reflexive symmetric relation on $\Gamma$ (called the incidence relation) such that if objects $a,b$ are of the same type and $a \# b$, then $a=b$.
\end{enumerate}
We usually write $\Gamma$ for the geometry.  The cardinality of $T$ is called the \emph{rank} of $\Gamma$.  $T$ is usually assumed to be finite.

Here's a simple example of a Buekenhout-Tits geometry.  Let $T$ be the set consisting of four types: vertex, edge, face, and cube.  Let $X=\lbrace A,B,C,D,E,F,G,H\rbrace$, and $\Gamma$ be the set of subsets of $X$ consisting of the following:
\begin{itemize}
\item all of the singletons
\item the following doubletons: $AB, BC, CD, DA, EF, FG, GH, HE, AE, BF, CG$, and $DH$, where a two-letter string represents the set containing the letters
\item the following 4-sets: $ABCD, EFGH, BCFG, AEDH, ABFE$, and $CDHG$, where the four-letter string represents the set containing the letters, and
\item $X$ itself
\end{itemize}
The map $\tau$ is defined as follows: all the singletons are mapped to vertex, all the doubletons to edge, the 4-sets to face, and $X$ to cube.  The following diagram is useful as a visual reference.

\begin{center}
\begin{pspicture}(-14,-3)(2,4)
\psset{unit=2cm}
\pspolygon[linestyle=dashed, dash=4pt 4pt](-4,-0.5)(-2,-0.5)(-2,1.5)(-4,1.5)
\pspolygon[linestyle=dashed, dash=4pt 4pt](-5,1)(-4,1.5)(-4,-0.5)(-5,-1)
\pspolygon(-5,1)(-4,1.5)(-2,1.5)(-3,1)
\pspolygon(-2,-0.5)(-2,1.5)(-3,1)(-3,-1)
\pspolygon(-5,-1)(-3,-1)(-3,1)(-5,1)
\uput[l](-5,1){B}
\uput[d](-3,-1){D}
\uput[u](-3,1){C}
\uput[l](-5,-1){A}
\uput[d](-4,-0.5){E}
\uput[r](-2,-0.5){H}
\uput[u](-2,1.5){G}
\uput[u](-4,1.5){F}
\end{pspicture}
\end{center}

Finally, $\#$ is defined as the inclusion relation: $P\# Q$ iff $P\subseteq Q$ or $Q\subseteq P$.  Then $(\Gamma,\#,T,\tau)$ is a Buekenhout-Tits geometry.

Buekenhout-Tits geometries are generalizations of \PMlinkname{projective}{ProjectiveGeometry} and affine geometries, and indeed incidence geometries in general.  They also include examples where geometric meanings may not be apparent at first sight.  Below are two such examples:
\begin{itemize}
\item A bipartite graph can be thought of as a Buekenhout-Tits geometry: we can take $\Gamma$ as the set of vertices, and two vertices are of the same type if they have the same chromatic number, and are incident if they are either the same, or there is an edge connecting them.
\item Let $G$ be a group and $\lbrace G_i\mid i\in I\rbrace$ be a set of subgroups of $G$, indexed by some set $I$.  Let $\Gamma$ be the set of all (left) cosets of $G_i$, for all $i\in I$.  Note that $xG_i=yG_j$ iff $i=j$ and $y^{-1}x\in G_i$.  Call $xG_i$ and $yG_j$ to be of the same type if $i=j$, and $xG_i \# yG_j$ if $xG_i\cap yG_j\ne \varnothing$.  Then the geometry defined is a Buekenhout-Tits geometry.
\end{itemize}

An incidence structure can be thought of as a Buekenhout-Tits geometry of rank 2. 

\textbf{Remark}.  The \emph{graph} of a geometry $(\Gamma,\#,T,\tau)$ is the pair $(\Gamma,\#)$, where objects are vertices of the graph, and $(a,b)$ is an edge of the graph iff $a\#b$.  Properly speaking, the associated graph is a pseudograph since $\#$ is reflexive, so that there is a loop for each vertex.

\begin{thebibliography}{6}
\bibitem{fb} {\it Handbook of Incidence Geometry}, edited by Francis Buekenhout,
Elsevier Science Publishing Co. (1995)
\bibitem{ma} M. Aschbacher, {\it Finite Group Theory}, Cambridge University Press (2000)
\bibitem{pc} P. J. Cameron, {\it \PMlinkexternal{Projective and Polar Spaces}{http://www.maths.qmul.ac.uk/~pjc/pps/}}, QMW Maths Notes, 13, London: Queen Mary and Westfield College School of Mathematical Sciences (2000)
\end{thebibliography}
%%%%%
%%%%%
\end{document}
