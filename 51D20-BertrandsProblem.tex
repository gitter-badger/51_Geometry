\documentclass[12pt]{article}
\usepackage{pmmeta}
\pmcanonicalname{BertrandsProblem}
\pmcreated{2013-03-22 16:24:18}
\pmmodified{2013-03-22 16:24:18}
\pmowner{PrimeFan}{13766}
\pmmodifier{PrimeFan}{13766}
\pmtitle{Bertrand's problem}
\pmrecord{14}{38553}
\pmprivacy{1}
\pmauthor{PrimeFan}{13766}
\pmtype{Definition}
\pmcomment{trigger rebuild}
\pmclassification{msc}{51D20}
\pmsynonym{Bertrand's paradox}{BertrandsProblem}
\pmrelated{CircleLinePicking}

\endmetadata

% this is the default PlanetMath preamble.  as your knowledge
% of TeX increases, you will probably want to edit this, but
% it should be fine as is for beginners.

% almost certainly you want these
\usepackage{amssymb}
\usepackage{amsmath}
\usepackage{amsfonts}

% used for TeXing text within eps files
%\usepackage{psfrag}
% need this for including graphics (\includegraphics)
\usepackage{graphicx}
% for neatly defining theorems and propositions
%\usepackage{amsthm}
% making logically defined graphics
%%%\usepackage{xypic}

% there are many more packages, add them here as you need them

% define commands here

\begin{document}
Given an equilateral triangle inscribed on a circle, draw a chord on the circle at random. \emph{Bertrand's problem}, first posed by Joseph Bertrand in 1888, asks: what is the probability that that chord will be longer than a side of the triangle? Put another way, what is the probability that the length of a chord from a random single point along the circumference of the circle to a vertex of the triangle will form an angle from the intersection of the angle along the circle?

There are at least three different ways to randomly choose the chords: by choosing midpoints, by choosing endpoints, or by choosing chords. Thus the following solutions are obtained:

1. Randomly place a dot anywhere in the circle. Then draw the chord that has that dot as its midpoint. If the distance from the chord's midpoint to the center of the circle is less than half the radius of the circle, then the chord will definitely be longer than a side of the triangle. The probability is then 25\%.

\begin{center}
\includegraphics[scale = 0.5]{Bertrand3}
\end{center}

2. Randomly place a dot anywhere on the circumference of the circle. Then draw a chord to connect that dot to any vertex of the triangle. It becomes visibly obvious that the midpoint of that chord has to fall somewhere between the two vertices other than the vertex we chose to connect our random dot to in order for the resulting chord to be longer than a side of the triangle. The probability is then approximately 33\%.

\begin{center}
\includegraphics[scale = 0.5]{Bertrand1}
\end{center}

3. Randomly draw a chord that is parallel to one of the sides of the triangle. If the chord falls entirely on what is not part of the triangle, then it is clear that it will be shorter than that side. The same is true for that reflected part on the other side of the triangle. The probability is then 50\%.

\begin{center}
\includegraphics[scale = 0.5]{Bertrand2}
\end{center}


Because three different results to this problem are obtained depending on the selection method, the problem is sometimes known as \emph{Bertrand's paradox}.
%%%%%
%%%%%
\end{document}
