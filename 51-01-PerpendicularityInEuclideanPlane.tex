\documentclass[12pt]{article}
\usepackage{pmmeta}
\pmcanonicalname{PerpendicularityInEuclideanPlane}
\pmcreated{2013-04-19 15:00:12}
\pmmodified{2013-04-19 15:00:12}
\pmowner{pahio}{2872}
\pmmodifier{pahio}{2872}
\pmtitle{perpendicularity in Euclidean plane}
\pmrecord{7}{39899}
\pmprivacy{1}
\pmauthor{pahio}{2872}
\pmtype{Definition}
\pmcomment{trigger rebuild}
\pmclassification{msc}{51-01}
\pmrelated{ConditionOfOrthogonality}
\pmrelated{MutualPositionsOfVectors}
\pmrelated{AngleBetweenTwoLines}
\pmrelated{ParallellismInEuclideanPlane}
\pmrelated{OrthogonalCircles}
\pmrelated{DihedralAngle}
\pmdefines{perpendicularity}
\pmdefines{perpendicular}
\pmdefines{orthogonality}
\pmdefines{orthogonal}

% this is the default PlanetMath preamble.  as your knowledge
% of TeX increases, you will probably want to edit this, but
% it should be fine as is for beginners.

% almost certainly you want these
\usepackage{amssymb}
\usepackage{amsmath}
\usepackage{amsfonts}

% used for TeXing text within eps files
%\usepackage{psfrag}
% need this for including graphics (\includegraphics)
%\usepackage{graphicx}
% for neatly defining theorems and propositions
 \usepackage{amsthm}
% making logically defined graphics
%%%\usepackage{xypic}

% there are many more packages, add them here as you need them

% define commands here

\theoremstyle{definition}
\newtheorem*{thmplain}{Theorem}

\begin{document}
Two lines in the Euclidean plane are {\em perpendicular} to each other if and only if they intersect and two of the angles they form are congruent. 

This definition \PMlinkescapetext{bases} on the one in Hilbert's {\em Grundlagen der Geometrie} (``Ein Winkel, welcher einem seiner Nebenwinkel kongruent ist, hei\ss t ein {\em rechter Winkel}'').

The {\em perpendicularity} of $l$ and $m$ is denoted 
                         $$l \bot m.$$

\begin{thebibliography}{8}
\bibitem{Grundlagen}{\sc D. Hilbert}: {\em Grundlagen der Geometrie}. Neunte Auflage, revidiert und erg\"anzt von Paul Bernays.\;  B. G. Teubner Verlagsgesellschaft, Stuttgart (1962).
\end{thebibliography} 


%%%%%
%%%%%
\end{document}
