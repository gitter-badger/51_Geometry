\documentclass[12pt]{article}
\usepackage{pmmeta}
\pmcanonicalname{MedialTriangle}
\pmcreated{2013-03-22 13:11:17}
\pmmodified{2013-03-22 13:11:17}
\pmowner{yark}{2760}
\pmmodifier{yark}{2760}
\pmtitle{medial triangle}
\pmrecord{8}{33640}
\pmprivacy{1}
\pmauthor{yark}{2760}
\pmtype{Definition}
\pmcomment{trigger rebuild}
\pmclassification{msc}{51-00}
\pmsynonym{auxiliary triangle}{MedialTriangle}
\pmdefines{Spieker center}
\pmdefines{Spieker circle}
\pmdefines{medial circle}

\endmetadata

\usepackage{amssymb}
\usepackage{amsmath}
\usepackage{amsfonts}
\usepackage{graphicx}

\begin{document}
\PMlinkescapeword{property}
\PMlinkescapeword{similar}

The \emph{medial triangle} of a triangle $\triangle ABC$ is the triangle formed by joining the midpoints of the sides of the triangle $\triangle ABC.$

\begin{center}
\includegraphics{med.eps}
\end{center}

Here, $\triangle A'B'C'$ is the medial triangle.
The incircle of the medial triangle is called the \emph{Spieker circle} and the incenter is called the \emph{Spieker center}.
The circumcircle of the medial triangle is called the \emph{medial circle}.

An important property of the medial triangle is that the medial triangle $\triangle A'B'C'$ of the medial triangle $\triangle DEF$ of  $\triangle ABC$ is similar to  $\triangle ABC.$

\begin{center}
\includegraphics{med1.eps}
\end{center}
%%%%%
%%%%%
\end{document}
