\documentclass[12pt]{article}
\usepackage{pmmeta}
\pmcanonicalname{ConstructionOfTangent}
\pmcreated{2013-03-22 17:36:04}
\pmmodified{2013-03-22 17:36:04}
\pmowner{pahio}{2872}
\pmmodifier{pahio}{2872}
\pmtitle{construction of tangent}
\pmrecord{15}{40015}
\pmprivacy{1}
\pmauthor{pahio}{2872}
\pmtype{Algorithm}
\pmcomment{trigger rebuild}
\pmclassification{msc}{51M15}
\pmclassification{msc}{51-00}
%\pmkeywords{tangent of circle}
\pmrelated{Incircle}
\pmrelated{AngleBisectorAsLocus}
\pmdefines{tangent angle}
\pmdefines{tangent-tangent angle}
\pmdefines{tangent chord}

% this is the default PlanetMath preamble.  as your knowledge
% of TeX increases, you will probably want to edit this, but
% it should be fine as is for beginners.

% almost certainly you want these
\usepackage{amssymb}
\usepackage{amsmath}
\usepackage{amsfonts}

% used for TeXing text within eps files
%\usepackage{psfrag}
% need this for including graphics (\includegraphics)
%\usepackage{graphicx}
% for neatly defining theorems and propositions
 \usepackage{amsthm}
% making logically defined graphics
%%%\usepackage{xypic}
\usepackage{pstricks}
\usepackage{pst-plot}

% there are many more packages, add them here as you need them

% define commands here

\theoremstyle{definition}
\newtheorem*{thmplain}{Theorem}

\begin{document}
\textbf{Task.}  Using the compass and straightedge, construct to a given circle the tangent lines through a given point outside the circle.

\PMlinkescapetext{{\em Solution.}}  Let $O$ be the centre of the given circle and $P$ the given point.  With $OP$ as diameter, draw the circle (see midpoint).  If $A$ and $B$ are the points where this circle intersects the given circle, then by Thales' theorem, the angles $OAP$ and $OBP$ are right angles.  According to the definition of the tangent of circle, the lines $AP$ and $BP$ are required tangents.\\

\begin{center}
\begin{pspicture}(-5.5,-3)(5.5,3)
\rput(-2.85,-0.1){$O$}
\rput[linecolor=blue](+2.85,-0.1){$P$}
\psdot(-2.6,0)
\psdot(+2.6,0)
\psline(-2.6,0)(2.6,0)
\pscircle[linecolor=blue](-2.6,0){2}
\psdot(0,0)
\pscircle(0,0){2.6}
\psdots(-1.8308,1.8462)(-1.8308,-1.8462)
\psline[linestyle=dashed](-2.6,0)(-1.8308,+1.8462)
\psline[linestyle=dashed](-2.6,0)(-1.8308,-1.8462)
\psline[linestyle=dotted](-1.8308,+1.8462)(-1.8308,-1.8462)
\rput(-1.8308,+2.2){$A$}
\rput(-1.8308,-2.2){$B$}
\psline[linecolor=blue](2.6,0)(-3,+2.3335)
\psline[linecolor=blue](2.6,0)(-3,-2.3335)
\rput(-5.5,-3){.}
\rput(5.5,3){.}
\end{pspicture}
\end{center}

The line segment $AB$ is 


The convex angle $APB$ is called a {\em tangent angle} (or {\em tangent-tangent angle}) of the given circle and the convex angle $AOB$ the {\em corresponding central angle}.  It is apparent that a tangent angle and the corresponding central angle are supplementary.\, The chord $AB$ is the {\em tangent chord} corresponding the tangent angle and the point $P$ (see \PMlinkname{equation of tangent chord}{EquationOfTangentOfCircle}!).

The tangent angle is the angle of view of the line segment $AB$ from the point $P$.

Note that if a circle is inscribed in a polygon, then the angles of the polygon are tangent angles of the circle and the centre of the circle is the common intersection point of the angle bisectors.
%%%%%
%%%%%
\end{document}
