\documentclass[12pt]{article}
\usepackage{pmmeta}
\pmcanonicalname{Geometry}
\pmcreated{2013-03-22 13:19:01}
\pmmodified{2013-03-22 13:19:01}
\pmowner{rspuzio}{6075}
\pmmodifier{rspuzio}{6075}
\pmtitle{geometry}
\pmrecord{46}{33824}
\pmprivacy{1}
\pmauthor{rspuzio}{6075}
\pmtype{Topic}
\pmcomment{trigger rebuild}
\pmclassification{msc}{51-00}
\pmclassification{msc}{51-01}
\pmsynonym{Egyptian geometry}{Geometry}
%\pmkeywords{calculus}
\pmrelated{FiniteProjectivePlane4}
\pmrelated{ProjectivePlane2}
\pmrelated{PointFreeGeometry}
\pmrelated{ComparisonOfCommonGeometries}
\pmrelated{DeBruijnErdHosTheorem}
\pmrelated{MulberryFactoryShopACrowdOfPeopleWaitingForTheirTurnToBecomeModish}
\pmrelated{MulberryFactoryShopACrowdOfPeopleWaitingForTheirTurnToBecomeModish2}
\pmdefines{Greek geometry}
\pmdefines{Euclidean geometry}

% this is the default PlanetMath preamble.  as your knowledge
% of TeX increases, you will probably want to edit this, but
% it should be fine as is for beginners.

% almost certainly you want these
\usepackage{amssymb}
\usepackage{amsmath}
\usepackage{amsfonts}

% used for TeXing text within eps files
%\usepackage{psfrag}
% need this for including graphics (\includegraphics)
%\usepackage{graphicx}
% for neatly defining theorems and propositions
%\usepackage{amsthm}
% making logically defined graphics
%%%\usepackage{xypic}

% there are many more packages, add them here as you need them

% define commands here
\begin{document}
\PMlinkescapeword{algebraic}
\PMlinkescapeword{free}
\PMlinkescapeword{ideal} 
\PMlinkescapeword{meet}
\PMlinkescapeword{moment} 
\PMlinkescapeword{natural} 
\PMlinkescapeword{one way}
\PMlinkescapeword{regular} 
\PMlinkescapeword{bound}
\PMlinkescapeword{foundations}

{\bf Note:} This entry is very rough at the moment, and requires
work. I mainly wrote it to help motivate other entries and to let
others work on this entry, if it is at all feasible. Please feel free
to help out, including making suggestions, deleting things, adding
things, etc.

Geometry, or literally, the measurement of land, is among the oldest
and largest areas of mathematics.  It is as old as civilization itself
--- even when texts and traditions have been lost, such monuments as
Stonehenge and the pyramids of Egypt and South America stand as mute
\PMlinkescapetext{witnesses} to the geometrical knowledge of the
ancients.  Over the centuries, geometry has grown from its humble
origins in land measurement to a study of the properties of space in
the widest sense of the \PMlinkescapetext{term}.  In addition to the
familiar three-dimensional space in which we move and breathe, modern
geometers routinely consider spaces of more than three dimensions,
even infinite-dimensional and fractional dimensional spaces, curved
spaces, discrete spaces, non-commutative spaces, infinitesimal spaces,
and many other \PMlinkescapetext{types} of spaces.

For this reason, it is quite difficult to provide a precise definition
of geometry. In this survey of geometry, we shall indicate several
approaches to the subject.  We start with the synthetic (or axiomatic)
approach to Euclidean geometry not only because that is historically
the oldest, but because it is the approach one is most likely to
encounter first.  After this, we move on to other approaches in
roughly an \PMlinkescapetext{order} of increasing mathematical
sophistication.

In this survey, our goal is to give the reader an overview of the
different \PMlinkescapetext{subfields} of geometry, the concepts and
techniques used, and the sort of results which are proven.  In order
to make this accessible to a wide audience, we have assumed the
minimum of knowledge on the part of the reader necessary to understand
and appreciate the topics presented in a meaningful way.  Since our
goal is to present the substance and flavor of the subjects discussed
as opposed to giving a comprehensive and detailed account, we
sometimes omit technical details in the interest of clarity.  To
compensate for this shortcoming, we have included
\PMlinkescapetext{links} to entries in which the interested reader may
find more detailed and rigorous treatments of the topics discussed
here as well as related topics which had to be omitted to keep the
\PMlinkescapetext{size} of this entry within reasonable
\PMlinkescapetext{bounds}.

\subsection{\PMlinkname{Axiomatic method}{AxiomaticGeometry}}

\subsection{Analytic and Descriptive Geometry}
\begin{enumerate}
\item Euclidean geometry of plane
\item Euclidean geometry of space
\item \PMlinkid{Coordinate systems}{6977}
\item Topics on vectors
\item Index of entries on compass and straightedge constructions
\end{enumerate}

\subsection{\PMlinkname{Geometry as the study of invariants under certain transformations}{GeometryAsTheStudyOfInvariantsUnderCertainTransformations}}

\subsection{Differential geometry}

Differential geometry studies geometrical objects using techniques of
calculus.  In fact, its early history is indistiguishable from that of
calculus --- it is a matter of personal taste whether one chooses to
regard Fermat's method of drawing tangents and finding extrema as a
contribution to calculus or differential geometry; the pioneering work
of Barrow and Newton on calculus was presented in a geometrical
language; Halley's 1696 paper in which he announces his discovery that
$\displaystyle \int \frac{dx}{x} = \log x + C$ is entitled quadrature of the hyperbola.

It is only later on, when calculus became more algebraic in outlook
that one can begin to make a meaningful separation between the
subjects of calculus and differential geometry.

Below are some main topic entries on PlanetMath on differential geometry:
\begin{enumerate}
\item Euclidean geometry of plane
\item Euclidean geometry of space
\item \PMlinkid{Coordinate systems}{6977}
\item Topics on vectors
\item Classical differential geometry 
\item Bibliography for differential geometry
\item Fundamental concepts in differential geometry
\item Concepts in symplectic geometry
\end{enumerate}


\subsection{Algebraic geometry}

\begin{thebibliography}{8}
\bibitem{Grundlagen}{\sc D. Hilbert}: {\em Grundlagen der Geometrie}. Neunte Auflage, revidiert und erg\"anzt von Paul Bernays.\;  B. G. Teubner Verlagsgesellschaft, Stuttgart (1962).
\end{thebibliography} 
%%%%%
%%%%%
\end{document}
