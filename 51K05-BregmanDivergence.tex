\documentclass[12pt]{article}
\usepackage{pmmeta}
\pmcanonicalname{BregmanDivergence}
\pmcreated{2013-03-22 19:11:38}
\pmmodified{2013-03-22 19:11:38}
\pmowner{FrankTokyo}{25936}
\pmmodifier{FrankTokyo}{25936}
\pmtitle{Bregman divergence}
\pmrecord{6}{42106}
\pmprivacy{1}
\pmauthor{FrankTokyo}{25936}
\pmtype{Definition}
\pmcomment{trigger rebuild}
\pmclassification{msc}{51K05}
\pmsynonym{Bregman distance}{BregmanDivergence}

\endmetadata

% this is the default PlanetMath preamble.  as your knowledge
% of TeX increases, you will probably want to edit this, but
% it should be fine as is for beginners.

% almost certainly you want these
\usepackage{amssymb}
\usepackage{amsmath}
\usepackage{amsfonts}

% used for TeXing text within eps files
%\usepackage{psfrag}
% need this for including graphics (\includegraphics)
%\usepackage{graphicx}
% for neatly defining theorems and propositions
%\usepackage{amsthm}
% making logically defined graphics
%%%\usepackage{xypic}

% there are many more packages, add them here as you need them

% define commands here
\def\Innerproduct#1#2{ \left\langle {#1},{#2} \right\rangle }
\def\innerproduct#1#2{ {\langle #1,#2 \rangle} }
\begin{document}
A \emph{Bregman divergence}, or \emph{Bregman distance}, $B_F$ on a space $\mathcal{X}\subseteq \mathbb{R}^d$ is defined for a strictly convex and differentiable function $F: \mathcal{X} \to \mathbb{R}$ as

\begin{equation}
B_F(p,q)=F(p)-F(q)-\innerproduct{p-q}{\nabla F(q)},
\end{equation} 

where $$\innerproduct{p}{q}=p^T q$$ denotes the inner product, and $$\nabla F(x)=[\frac{\partial F}{\partial x_1}, \cdots , \frac{\partial F}{\partial x_d}]^T$$ the partial derivatives.

Choosing $F(x)=\sum_{i=1}^d x_i^2$ yields the squared Euclidean distance $B_{x^2}(p,q)=||p-q||^2$, and choosing $F(x)=\sum_{i=1}^d x_i\log x_i$ yields the relative entropy, called the Kullback-Leibler divergence.
%%%%%
%%%%%
\end{document}
