\documentclass[12pt]{article}
\usepackage{pmmeta}
\pmcanonicalname{AreaOfAPolygonalRegion}
\pmcreated{2013-03-22 17:00:29}
\pmmodified{2013-03-22 17:00:29}
\pmowner{Mathprof}{13753}
\pmmodifier{Mathprof}{13753}
\pmtitle{area of a polygonal region}
\pmrecord{12}{39290}
\pmprivacy{1}
\pmauthor{Mathprof}{13753}
\pmtype{Definition}
\pmcomment{trigger rebuild}
\pmclassification{msc}{51M25}
\pmrelated{Area2}
\pmrelated{InterceptTheorem}
\pmrelated{BaseAndHeightOfTriangle}
\pmdefines{triangular region}
\pmdefines{polygonal region}
\pmdefines{area of a triangle}

% this is the default PlanetMath preamble.  as your knowledge
% of TeX increases, you will probably want to edit this, but
% it should be fine as is for beginners.

% almost certainly you want these
\usepackage{amssymb}
\usepackage{amsmath}
\usepackage{amsfonts}

% used for TeXing text within eps files
%\usepackage{psfrag}
% need this for including graphics (\includegraphics)
%\usepackage{graphicx}
% for neatly defining theorems and propositions
\usepackage{amsthm}
% making logically defined graphics
%%%\usepackage{xypic}
\usepackage{mathrsfs}

% there are many more packages, add them here as you need them

% define commands here
\newtheorem{thm}{Theorem}

\begin{document}
We consider only Euclidean geometry.

A \emph{triangular region} is a triangle together with its interior.
The sides of the triangle are called the \emph{edges} of the region and the vertices of the
triangle are called the \emph{vertices} of the region.

A \emph{polygonal region} is a plane figure that can be written as a union of a finite 
number of triangular regions, subject to a constraint. The constraint is that
the triangular regions are nonoverlapping. That means that if two triangular regions
intersect, then their intersection is either a vertex or edge of each of the triangular regions.

A given polygonal region can of course be written as such a union in infinitely many ways.

We wish to assign a number to each polygonal region that corresponds to our intuitive idea of
"area". To do that we first write down a set of postulates that "area" should satisfy and then
show that the postulates  can be satisfied. We let $\mathscr{R}$ denote the set of polygonal regions.
\begin{enumerate}
\item There is a function $\alpha: \mathscr{R} \to \mathbb{R}$. 
\item $\alpha > 0$.
\item If two triangular regions are congruent, then they have the same area.
\item If two polygonal regions $R_1$ and $R_2$ are nonoverlapping then 
$\alpha ( R_1 \cup R_2) = \alpha (R_1) + \alpha (R_2)$. 
\item If a square  has edges of length 1, then its area is 1.
\end{enumerate}

The fifth postulate is needed because if $\alpha$ satisfies the first 4 postulates, then so does 
$2\alpha$. So, this postulate serves to make $\alpha$ a unique function. 

\begin{thm}
If a square has sides of length $s$ then its area is $s^2$.
\end{thm}
\textbf{Proof.} We first show that if a square has sides of length $\frac{1}{q}$ then its area is
$\frac{1}{q^2}$. A unit square can be decomposed into $q^2$ squares, each having side of length
$\frac{1}{q}$. Each of the smaller squares has the same area, $A$,  by postulates 3 and 4. 
Hence,
$$
1 = q^2 A,
$$
so that $A = \frac{1}{q^2}$. Next, if a square has sides of length $\frac{p}{q}$ then its area
is $\frac{p^2}{q^2}$. This is because the square can be written as $p^2$ squares each having a side of length
$\frac{1}{q}$.  If $A$ is the area then
$$
A = p^2 \frac{1}{q^2} = \frac{p^2}{q^2}.
$$
Finally, let $T$ be a square of side $s$ and $T_{p/q}$ denote a square of side $\frac{p}{q}$ having
one angle in common with $T$. In the following sequence of statements, each statement 
is equivalent to the next one:
\begin{enumerate}
\item $\frac{p}{q} < s.$
\item $T_{p/q} \subset T.$ 
\item $\alpha(T_{p/q}) < \alpha(T).$
\item $\frac{p^2}{q^2} < \alpha(T).$
\item $\frac{p}{q} < \sqrt{\alpha(T)}.$
\end{enumerate}
 
Hence, statements 1 and 5 are equivalent so that 
$$
s = \sqrt{\alpha(T)},
$$
so that $s^2 = \alpha(T)$.
This proves the theorem.

One can then proceed on the basis of the postulates to show the following sequence of results.
\begin{thm}
The area of a rectangular region is the product of its base and its altitude.
\end{thm}

Rather than speak of the area of a triangular region or the area of a rectangle region,
we shall refer to the area of a triangle, or area of a rectangle, or other polygonal
region, as is usually done. 

\begin{thm} The area of a right triangle is 1/2 product of the lengths of its legs.
\end{thm}
\begin{thm}
The area of a triangle is 1/2 the product of any base and the corresponding altitude.
\end{thm}
\begin{thm}
The area of a parallelogram is the product of a base and the corresponding altitude.
\end{thm}
\begin{thm}
The area of a trapezoid is 1/2 the product of the altitude and the sum of the bases.
\end{thm}
\begin{thm}
If two triangles are similar then the ratio of their areas is the square of the ratio
of any two corresponding sides. 
\end{thm}

What remains to do is to show that there is a function $\alpha$ that satisfies postulates
1 to 5. 
It might seem reasonable to just define the area of a rectangle, but rectangles are not 
useful for dissecting polygonal regions. The easier way to use triangles.
So one \emph{defines} the area of a triangle to be 
1/2 the product of the base and its corresponding altitude. For this to be well-defined
one has to show first that the product of a base and altitude (for a given triangle)
does not depend on which base is chosen.  Next, one defines the \emph{area of a polygonal region}
as the sum of the areas of the triangular regions in some decomposition. Of course one has to 
show that this sum does not depend on  the particular decomposition that is used.
Given this definition for $\alpha$, it is then a simple matter to show that
postulates 1 to 5 are satisfied.

The function $\alpha$ can be defined on a larger domain, which includes circular sectors and disks.
But to do this rigorously ones needs the concept of a limit.
In the hyperbolic geometry case, one can show the following theorem.

\begin{thm}
If $\alpha$ is a function that satisfies postulates 1 to 4 then there is a $k>0$ such
that 
$$
\alpha(R) = k \delta(R)
$$
for every polygonal region, where $\delta(R)$ is the defect of $R$.
\end{thm}


\begin{thebibliography}{99}
\bibitem{EEM} Edwin E. Moise, \emph{Elementary Geometry from an Advanced Standpoint, Third edition}, 
Addison-Wesley, 1990
\end{thebibliography}






%%%%%
%%%%%
\end{document}
