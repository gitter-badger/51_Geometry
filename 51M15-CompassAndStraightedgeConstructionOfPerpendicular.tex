\documentclass[12pt]{article}
\usepackage{pmmeta}
\pmcanonicalname{CompassAndStraightedgeConstructionOfPerpendicular}
\pmcreated{2013-03-22 17:14:01}
\pmmodified{2013-03-22 17:14:01}
\pmowner{Wkbj79}{1863}
\pmmodifier{Wkbj79}{1863}
\pmtitle{compass and straightedge construction of perpendicular}
\pmrecord{5}{39562}
\pmprivacy{1}
\pmauthor{Wkbj79}{1863}
\pmtype{Algorithm}
\pmcomment{trigger rebuild}
\pmclassification{msc}{51M15}
\pmclassification{msc}{51-00}
\pmrelated{ProjectionOfPoint}
\pmdefines{drop the perpendicular}
\pmdefines{dropping the perpendicular}
\pmdefines{erect the perpendicular}
\pmdefines{erecting the perpendicular}

\endmetadata

\usepackage{amssymb}
\usepackage{amsmath}
\usepackage{amsfonts}
\usepackage{pstricks}
\usepackage{psfrag}
\usepackage{graphicx}
\usepackage{amsthm}
%%\usepackage{xypic}

\begin{document}
\PMlinkescapeword{label}
\PMlinkescapeword{right}

Let $P$ be a point and $\ell$ be a line in the Euclidean plane.  One can construct a line $m$ perpendicular to $\ell$ and passing through $P$.  The construction given here yields $m$ in any circumstance:  Whether $P \in \ell$ or $P \notin \ell$ does not matter.  On the other hand, the construction looks quite different in these two cases.  Thus, the sequence of pictures on the left (in which $\ell$ is in red) is for the case that $P \notin \ell$, and the sequence of pictures on the right (in which $\ell$ is in green) is for the case that $P \in \ell$.

\begin{enumerate}

\item With one point of the compass on $P$, draw an arc that intersects $\ell$ at two points.  Label these as $Q$ and $R$.

\begin{center}
\begin{pspicture}(-8,-2)(8,3)
\rput[l](-8,0){.}
\rput[r](8,0){.}
\rput[b](-5,-0.5){.}
\rput[a](5,2.5){.}
\psline[linecolor=red]{<->}(-8,0)(-2,0)
\rput[a](-2.1,-0.3){$\ell$}
\psarc[linecolor=blue](-5,2){2.5}{200}{340}
\psdots(-5,2)(-6.5,0)(-3.5,0)
\rput[l](-4.8,2){$P$}
\rput[a](-6.5,-0.3){$Q$}
\rput[a](-3.5,-0.3){$R$}
\psline[linecolor=green]{<->}(2,0)(8,0)
\rput[a](7.9,-0.3){$\ell$}
\psarc[linecolor=blue](5,0){2.5}{-10}{190}
\psdots(2.5,0)(5,0)(7.5,0)
\rput[a](2.8,-0.3){$Q$}
\rput[a](5,-0.3){$P$}
\rput[a](7.2,-0.3){$R$}
\end{pspicture}
\end{center}

\item Construct the perpendicular bisector of $\overline{QR}$.  This line is $m$.

\begin{center}
\begin{pspicture}(-8,-2)(8,3)
\rput[l](-8,0){.}
\rput[r](8,0){.}
\rput[b](-5,-2){.}
\rput[a](-5,3){.}
\psline[linecolor=red]{<->}(-8,0)(-2,0)
\rput[a](-2.1,-0.3){$\ell$}
\psarc(-5,2){2.5}{200}{340}
\psarc[linecolor=blue](-6.5,0){2}{-60}{60}
\psarc[linecolor=blue](-3.5,0){2}{120}{240}
\psline[linecolor=blue]{<->}(-5,-2)(-5,3)
\psdots(-5,2)(-6.5,0)(-3.5,0)
\rput[l](-4.8,2){$P$}
\rput[a](-6.5,-0.3){$Q$}
\rput[a](-3.5,-0.3){$R$}
\psline[linecolor=green]{<->}(2,0)(8,0)
\rput[a](7.9,-0.3){$\ell$}
\psarc(5,0){2.5}{-10}{190}
\psarc[linecolor=blue](2.5,0){3}{-45}{45}
\psarc[linecolor=blue](7.5,0){3}{135}{225}
\psline[linecolor=blue]{<->}(5,-2)(5,3)
\psdots(2.5,0)(5,0)(7.5,0)
\rput[a](2.8,-0.3){$Q$}
\rput[a](5,-0.3){$P$}
\rput[a](7.2,-0.3){$R$}
\end{pspicture}
\end{center}

\end{enumerate}

This construction is justified because $Q$ and $R$ are constructed so that $P$ is equidistant from them and thus lies on the perpendicular bisector of $\overline{QR}$.

In the case that $P \notin \ell$, this construction is referred to as \emph{dropping the perpendicular} from $P$ to $\ell$.  In the case that $P \in \ell$, this construction is referred to as \emph{erecting the perpendicular} to $\ell$ at $P$.

If you are interested in seeing the rules for compass and straightedge constructions, click on the \PMlinkescapetext{link} provided.
%%%%%
%%%%%
\end{document}
