\documentclass[12pt]{article}
\usepackage{pmmeta}
\pmcanonicalname{CombinatorialUniquenessOfHesseConfiguration}
\pmcreated{2014-02-28 16:18:25}
\pmmodified{2014-02-28 16:18:25}
\pmowner{rspuzio}{6075}
\pmmodifier{rspuzio}{6075}
\pmtitle{combinatorial uniqueness of Hesse Configuration}
\pmrecord{22}{88021}
\pmprivacy{1}
\pmauthor{rspuzio}{6075}
\pmtype{Definition}
\pmclassification{msc}{51E20}
\pmclassification{msc}{51A45}

\endmetadata

% this is the default PlanetMath preamble.  as your knowledge
% of TeX increases, you will probably want to edit this, but
% it should be fine as is for beginners.

% almost certainly you want these
\usepackage{amssymb}
\usepackage{amsmath}
\usepackage{amsfonts}

% need this for including graphics (\includegraphics)
\usepackage{graphicx}
% for neatly defining theorems and propositions
\usepackage{amsthm}

% making logically defined graphics
%\usepackage{xypic}
% used for TeXing text within eps files
%\usepackage{psfrag}

% there are many more packages, add them here as you need them

% define commands here
\newtheorem{def}{Definition}
\newtheorem{thm}{Theorem}
\begin{document}
In this article, we will show that a collection 
of objects which has the incidence structure of
a Hesse configuration is unique up to relabeling.

\begin{def}
An \emph{abstract Hesse configuration} is a pair of
sets $(P,L)$ which satisfies the following conditions:
\begin{enumerate}
\item $P$ has nine elements
\item Every element of $L$ is a subset of $P$ with three elements.
\item For any two distinct elements $x,y$ of $P$, there
exists exactly one element of $L$ which contains both $x$ and $y$.
\end{enumerate}
\end{def}

\begin{thm}
If $(P,L)$ is an abstract Hesse configuration,
then $L$ has 12 elements.
\end{thm}

\begin{proof}
Let $D$ be the set of all two-element subsets of $P$.
Then $D$ has ${9 \choose 2} = 36$ elements.  Each
element of $L$ is a subset of $P$ with three elements,
hence has ${3 \choose 2} = 3$ subsets of cardinality 2.
By the definition above, every element of $D$ must
be a subset of exactly one element of $L$.  For this
to be possible, $L$ must have cardinality $36/3 = 12$.
\end{proof}

\begin{thm}
If $(P,L)$ is an abstract Hesse configuration then, for
every $p \in P$, there exist exactly four elements
$\ell \in L$ such that $p \in \ell$.
\end{thm}

\begin{proof}
To every $q \in P$ such that $q \neq p$, there exists
exactly one $\ell \in L$ such that $p \in \ell$ and 
$q \in \ell$.  Furthermore, for every $\ell \in L$ such
that $p \in \ell$, there will be exactly two elements
of $\ell$ other than $p$.  Hence, there exist $(9-1)/2 = 4$
elements $\ell \in L$ such that $p \in \ell$.
\end{proof}

\begin{thm}
If $(P,L)$ is an abstract Hesse configuration and $\ell \in L$,
then there exist $m,n \in L$ such that $\ell \cap m =
\ell \cap n = m \cap n = \emptyset$.
\end{thm}

\begin{proof}
By the foregoing result, given $p \in \ell$, there are four
elements of $L$ to which $p$ belongs.  One of these, of course,
is $\ell$ itself, and the other three are distinct from $\ell$.
Since $\ell$ has three elements, this means that there are at
most $3 \cdot 3 + 1 = 10$ elements $k \in L$ such that 
$P \cap L \neq \emptyset$.  Because $L$ has 12 elements. there
must exist $m,n \in L$ such that $\ell \cap m = \ell \cap n = 
\emptyset$.

It remains to show that $m \cap n = \emptyset$.  Suppose to the
contrary that there exists a $p$ such that $p \in m$ and 
$p \in n$.  Since $m \cap \ell = \emptyset$, it follows that 
$p \notin \ell$, hence there will exist three distinct elements
of $L$ containing $p$ and an element of $\ell$.  Because 
$\ell \cap m = \ell \cap n = \emptyset$, these three elements 
are distinct from $m$ and $n$.  That makes for a total of
five distinct elements of $L$ containing $p$, which
contradicts the previous theorem, hence $m \cap n = \emptyset$.
\end{proof}

\begin{thm}
If $m,n,k$ are elements of $L$ such that $m \cap n = n \cap k 
= k \cap m = \emptyset$ and $\ell \in L \setminus \{m,n,k\}$, 
then $\ell$ has exactly one element in common with each of $m,n,k$.
\end{thm}

\begin{proof}
Since each element of $L$ is a subset of $P$ with three elements
and $m,n,k$ are pairwise disjoint but $P$ only has nine
elements, it follows that every element of $P$ must belong
to exactly one of $m,n,k$.  In particular, this means that 
every element of $\ell$ must belong to one of $m,n,k$.  Were
two elements of $\ell$ to belong to the same element of 
$\{m,n,k\}$ then, by the third defining property, that element
would have to equal $\ell$, contrary to its definition.  Hence,
each element of $\ell$ must belong to a distinct element of 
$\{m,n,k\}$.
\end{proof}

\begin{thm}
If $(P,L)$ is an abstract Hesse configuration, then we can
label the elements of $P$ as A,B,C,D,E,F,G,H,I in such a way
that the elements of $L$ are 
\[\{A,B,C\}, \{D,E,F\}, \{G,H,I\}, \]
\[\{A,D,G\}, \{B,E,H\}, \{C,F,I\}, \]
\[\{D,H,C\}, \{A,E,I\}, \{B,F,G\}, \]
\[\{B,D,I\}, \{C,E,G\}, \{A,F,H\}. \]
\end{thm}

\begin{proof}
By theorem 3, there exist $a,b,c \in L$ such that
$a \cap b = b \cap c = c \cap a = \emptyset$.  Since $L$ has
twelve elements, there must exist an a elemetn of $L$ distinct
from $a,b,c$.  Pick such an element and call it $d$.  By 
another application of theorem 3, there must exist $e,f \in L$ 
such that $d \cap e = e \cap f = f \cap d = \emptyset$.

By theorem 4, $a$ must have exactly one element in common with 
each of $d,e,f$; let $A$ the element it has in common with $d$,
$B$ be the element it has in common with $e$ and $C$ be the 
element it has in common with $f$.  Likewise, $b$ must have 
exactly one element in common with each of $d,e,f$, as must $c$.
Let $D$ be the element $b$ has in common with $d$, $E$ be the
element $b$ has in common with $e$, $F$ be the element $b$
has in common with $f$, $G$ be the element $c$ has in common 
with $d$, $H$ be the element $c$ has in common with $e$ and $I$
be the element $c$ has in common with $f$.

Summarizing what we just said another way, we have assigned
labels $A,B,C,D,E,F,G,H,I$ to the elements of $P$ in such
a way that 
\[ a = \{A,B,C\}, ~ b = \{D,E,F\}, ~ c = \{G,H,I\}, \]
\[ d = \{A,D,G\}, ~ e = \{B,E,H\}, ~ f = \{C,F,I\}. \]
That is half of what we set out to do; we must still label
the remaining six elements of $L$.

By theorem 4, if $\ell \in L \setminus \{a,b,c,d,e,f\}$, then
$\ell$ must have exactly one element in common with each of 
$a,b,c$ and exactly one element in common with each of $d,e,f$.

Suppose that $A \in \ell$.  It could not be the case that 
$D \in \ell$ because then $\ell$ would have two elements in 
common with $d$.  Since $\ell$ must have one element in common
with $b$, that means that either $E \in \ell$ or $F \in \ell$.
If $A,E \in \ell$, then the element $\ell$ has in common with 
$c$ cannot be $G$ because $\ell$ would have both $A$ and $G$
in common with $c$ and it cannot be $H$ because $\ell$ and
$e$ would have both $E$ and $H$ in common, hence the only
possibility is to have $I \in \ell$, i.e. $\ell = \{A,E,I\}$.
Likewise, if $A,F \in \ell$, it follows that $H \in \ell$.

Summarrizing the last few sentences, if $A \in \ell$, then
either $\ell = \{A,E,I\}$ or $\ell = \{A,F,H\}$.  By a similar
line of reasoning, if $B \in \ell$, then either $\ell = 
\{B,D,I\}$ or $\ell = \{B,F,G\}$ and, if $B \in \ell$, then 
either $\ell = \{C,D,H\}$ or $\ell = \{C,E,G\}$.  Since $\ell$
must contain one of $A,B,C$, it follows that there are omnly
the following six possibilities for $\ell$:
\[\{D,H,C\}, \{A,E,I\}, \{B,F,G\}, \]
\[\{B,D,I\}, \{C,E,G\}, \{A,F,H\}. \]
However, since $L \setminus \{a,b,c,d,e,f\}$ has cardinality
six, all these possibilities must be actual members of the set.
\end{proof}

\end{document}
