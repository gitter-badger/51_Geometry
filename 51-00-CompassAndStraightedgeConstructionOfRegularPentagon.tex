\documentclass[12pt]{article}
\usepackage{pmmeta}
\pmcanonicalname{CompassAndStraightedgeConstructionOfRegularPentagon}
\pmcreated{2013-03-22 17:11:12}
\pmmodified{2013-03-22 17:11:12}
\pmowner{Wkbj79}{1863}
\pmmodifier{Wkbj79}{1863}
\pmtitle{compass and straightedge construction of regular pentagon}
\pmrecord{27}{39503}
\pmprivacy{1}
\pmauthor{Wkbj79}{1863}
\pmtype{Algorithm}
\pmcomment{trigger rebuild}
\pmclassification{msc}{51-00}
\pmclassification{msc}{51M15}
\pmsynonym{construction of regular pentagon}{CompassAndStraightedgeConstructionOfRegularPentagon}
%\pmkeywords{Euclidean geometry}
\pmrelated{RegularPolygon}

\endmetadata

\usepackage{amssymb}
\usepackage{amsmath}
\usepackage{amsfonts}
\usepackage{pstricks}
\usepackage{psfrag}
\usepackage{graphicx}
\usepackage{amsthm}
%%\usepackage{xypic}

\begin{document}
\PMlinkescapeword{label}
\PMlinkescapeword{extension}
\PMlinkescapeword{regular}

One can construct a \PMlinkname{regular}{RegularPolygon} pentagon with sides of a given length $s$ using compass and straightedge as follows:

\begin{enumerate}
\item Draw a line segment of length $s$.  Label its endpoints $P$ and $Q$.

\begin{center}
\begin{pspicture}(-1,-1)(5,1)
\psline[linecolor=blue](0,0)(2,0)
\psdots(0,0)(2,0)
\rput[b](0,-0.4){$P$}
\rput[a](2.3,0.2){$Q$}
\end{pspicture}
\end{center}

\item Extend the line segment past $Q$.

\begin{center}
\begin{pspicture}(-1,-1)(5,1)
\rput[r](5,0){.}
\psline(0,0)(2,0)
\psline[linecolor=blue]{o->}(2,0)(5,0)
\psdots(0,0)(2,0)
\rput[b](0,-0.4){$P$}
\rput[a](2.3,0.2){$Q$}
\end{pspicture}
\end{center}

\item Erect the perpendicular to $\overrightarrow{PQ}$ at $Q$.

\begin{center}
\begin{pspicture}(-2,-3)(5,5)
\psline{o->}(0,0)(5,0)
\rput[r](5,0){.}
\rput[b](2,-3){.}
\rput[a](2,5){.}
\psarc[linecolor=blue](2,0){2}{0}{180}
\psarc[linecolor=blue](0,0){2.5}{-65}{65}
\psarc[linecolor=blue](4,0){2.5}{115}{245}
\psline[linecolor=blue]{<->}(2,-3)(2,5)
\psdots(0,0)(2,0)(4,0)
\rput[b](0,-0.4){$P$}
\rput[a](2.3,0.2){$Q$}
\end{pspicture}
\end{center}

\item Using the line drawn in the previous step, mark off a line segment of length $2s$ such that one of its endpoints is $Q$.  Label the other endpoint as $R$.

\begin{center}
\begin{pspicture}(-2,-3)(5,5)
\psline{o->}(0,0)(5,0)
\rput[r](5,0){.}
\rput[b](2,-3){.}
\rput[a](2,5){.}
\psarc(2,0){2}{0}{180}
\psarc(0,0){2.5}{-65}{65}
\psarc(4,0){2.5}{115}{245}
\psline{<->}(2,-3)(2,5)
\psline[linecolor=blue](2,0)(2,4)
\psdots(0,0)(2,0)(4,0)(2,4)
\rput[b](0,-0.4){$P$}
\rput[a](2.3,0.2){$Q$}
\rput[r](1.9,4){$R$}
\end{pspicture}
\end{center}

\item Connect $P$ and $R$.

\begin{center}
\begin{pspicture}(-2,-3)(5,5)
\psline{o->}(0,0)(5,0)
\rput[r](5,0){.}
\rput[b](2,-3){.}
\rput[a](2,5){.}
\psarc(2,0){2}{0}{180}
\psarc(0,0){2.5}{-65}{65}
\psarc(4,0){2.5}{115}{245}
\psline{<->}(2,-3)(2,5)
\psline[linecolor=blue](0,0)(2,4)
\psdots(0,0)(2,0)(4,0)(2,4)
\rput[b](0,-0.4){$P$}
\rput[a](2.3,0.2){$Q$}
\rput[r](1.9,4){$R$}
\end{pspicture}
\end{center}

\item Extend the line segment $\overline{PR}$ past $P$.

\begin{center}
\begin{pspicture}(-2,-3)(5,5)
\psline{o->}(0,0)(5,0)
\rput[r](5,0){.}
\rput[a](2,5){.}
\rput[b](-1.5,-3){.}
\psarc(2,0){2}{0}{180}
\psarc(0,0){2.5}{-65}{65}
\psarc(4,0){2.5}{115}{245}
\psline{<->}(2,-3)(2,5)
\psline(0,0)(2,4)
\psline[linecolor=blue]{o->}(0,0)(-1.5,-3)
\psdots(0,0)(2,0)(4,0)(2,4)
\rput[b](0,-0.4){$P$}
\rput[a](2.3,0.2){$Q$}
\rput[r](1.9,4){$R$}
\end{pspicture}
\end{center}

\item On the extension, mark off another line segment of length $s$ such that one of its endpoints is $P$.  Label the other endpoint as $S$.

\begin{center}
\begin{pspicture}(-2,-3)(5,5)
\psline{o->}(0,0)(5,0)
\rput[r](5,0){.}
\rput[a](2,5){.}
\rput[b](-1.5,-3){.}
\psarc(2,0){2}{0}{180}
\psarc(0,0){2.5}{-65}{65}
\psarc(4,0){2.5}{115}{245}
\psline{<->}(2,-3)(2,5)
\psline{o->}(2,4)(-1.5,-3)
\psline[linecolor=blue](0,0)(-0.89443,-1.78885)
\psdots(0,0)(2,0)(4,0)(2,4)(-0.89443,-1.78885)
\rput[b](0,-0.4){$P$}
\rput[a](2.3,0.2){$Q$}
\rput[r](1.9,4){$R$}
\rput[l](-0.8,-1.78885){$S$}
\end{pspicture}
\end{center}

\item Construct the midpoint of the line segment $\overline{RS}$.  Label it as $M$.  (Below, $\overline{PS}$ is drawn in red, and $\overline{MR}$ is drawn in green.)

\begin{center}
\begin{pspicture}(-2,-3)(5,5)
\psline{o->}(0,0)(5,0)
\rput[r](5,0){.}
\rput[b](2,-3){.}
\rput[a](2,5){.}
\rput[a](-2,2.382){.}
\psarc(2,0){2}{0}{180}
\psarc(0,0){2.5}{-65}{65}
\psarc(4,0){2.5}{115}{245}
\psline{<->}(2,-3)(2,5)
\psline{o->}(2,4)(-1.5,-3)
\psarc[linecolor=blue](2,4){3.5}{210}{275}
\psarc[linecolor=blue](-0.89443,-1.78885){3.5}{30}{95}
\psline[linecolor=blue]{<->}(-2,2.382)(2.5,0.132)
\psline[linecolor=red](0,0)(-0.89443,-1.78885)
\psline[linecolor=green](0.5528,1.1056)(2,4)
\psdots(0,0)(2,0)(4,0)(2,4)(-0.89443,-1.78885)(0.5528,1.1056)
\rput[b](0,-0.4){$P$}
\rput[a](2.3,0.2){$Q$}
\rput[r](1.9,4){$R$}
\rput[l](-0.8,-1.78885){$S$}
\rput[l](0.5528,1.1056){$M$}
\end{pspicture}
\end{center}

Note that the length of the line segment $\overline{MR}$ is $\displaystyle \frac{1+\sqrt{5}}{2}s$, which is the length of each diagonal of a regular pentagon with sides of length $s$.

\item Separately from the drawing from the previous steps, draw a line segment of length $s$.

\begin{center}
\begin{pspicture}(-2,-2)(2,-1)
\psline[linecolor=red](-1,-1.377)(1,-1.377)
\psdots(-1,-1.377)(1,-1.377)
\end{pspicture}
\end{center}

\item Adjust the compass to the length of $\overline{MR}$ and draw an arc from each endpoint of the line segment from the previous step so that the arcs intersect.

\begin{center}
\begin{pspicture}(-2,-2)(2,2)
\psline(-1,-1.377)(1,-1.377)
\psarc[linecolor=green](-1,-1.377){3.236}{55}{90}
\psarc[linecolor=green](1,-1.377){3.236}{90}{125}
\psdots(-1,-1.377)(1,-1.377)(0,1.702)
\end{pspicture}
\end{center}

\item Adjust the compass to the length of $\overline{PS}$ and draw arcs from each of the three points to determine the other two points of the regular pentagon.

\begin{center}
\begin{pspicture}(-2,-2)(2,2)
\psline(-1,-1.377)(1,-1.377)
\psarc(-1,-1.377){3.236}{55}{90}
\psarc(1,-1.377){3.236}{90}{125}
\psarc[linecolor=red](-1,-1.377){2}{90}{130}
\psarc[linecolor=red](1,-1.377){2}{50}{90}
\psarc[linecolor=red](0,1.702){2}{200}{240}
\psarc[linecolor=red](0,1.702){2}{300}{340}
\psdots(0,1.702)(-1.619,0.526)(-1,-1.377)(1,-1.377)(1.619,0.526)
\end{pspicture}
\end{center}

\item Draw the regular pentagon.

\begin{center}
\begin{pspicture}(-2,-2)(2,2)
\psarc(-1,-1.377){3.236}{55}{90}
\psarc(1,-1.377){3.236}{90}{125}
\psarc(-1,-1.377){2}{90}{130}
\psarc(1,-1.377){2}{50}{90}
\psarc(0,1.702){2}{200}{240}
\psarc(0,1.702){2}{300}{340}
\pspolygon[linecolor=blue](0,1.702)(-1.619,0.526)(-1,-1.377)(1,-1.377)(1.619,0.526)
\psdots(0,1.702)(-1.619,0.526)(-1,-1.377)(1,-1.377)(1.619,0.526)
\end{pspicture}
\end{center}

\end{enumerate}

The law of cosines can be used to justify this construction.  Note that, in the picture below, the lengths of the line segments drawn in red are $s$ and the lengths of the line segments drawn in green are $\displaystyle \frac{1+\sqrt{5}}{2}s$.  The color of these line segments is based off of how the pentagon above was constructed.

\begin{center}
\begin{pspicture}(-2,-2)(2,2)
\pspolygon[linecolor=red](0,1.702)(-1.619,0.526)(-1,-1.377)(1,-1.377)(1.619,0.526)
\psline[linecolor=green](-1,-1.377)(0,1.702)(1,-1.377)
\psdots(0,1.702)(-1.619,0.526)(-1,-1.377)(1,-1.377)(1.619,0.526)
\end{pspicture}
\end{center}

If you are interested in seeing the rules for compass and straightedge constructions, click on the \PMlinkescapetext{link} provided.
%%%%%
%%%%%
\end{document}
