\documentclass[12pt]{article}
\usepackage{pmmeta}
\pmcanonicalname{PtolemysTheorem}
\pmcreated{2013-03-22 11:43:13}
\pmmodified{2013-03-22 11:43:13}
\pmowner{drini}{3}
\pmmodifier{drini}{3}
\pmtitle{Ptolemy's theorem}
\pmrecord{18}{30100}
\pmprivacy{1}
\pmauthor{drini}{3}
\pmtype{Theorem}
\pmcomment{trigger rebuild}
\pmclassification{msc}{51-00}
\pmclassification{msc}{60K25}
\pmclassification{msc}{18-00}
\pmclassification{msc}{68Q70}
\pmclassification{msc}{37B15}
\pmclassification{msc}{18-02}
\pmclassification{msc}{18B20}
%\pmkeywords{Quadrilateral}
%\pmkeywords{Circle}
%\pmkeywords{Cyclic}
%\pmkeywords{Ptolemy}
\pmrelated{CyclicQuadrilateral}
\pmrelated{ProofOfPtolemysTheorem}
\pmrelated{PtolemysTheorem}
\pmrelated{PythagorasTheorem}
\pmrelated{CrossedQuadrilateral}

\usepackage{amssymb}
\usepackage{amsmath}
\usepackage{amsfonts}
\usepackage{graphicx}
%%%%%%%%%%%\usepackage{xypic}
\begin{document}
If $ABCD$ is a cyclic quadrilateral, then the product of the two diagonals is equal to the sum of the products of opposite sides.
\begin{center}
\includegraphics{ptolemy}
\end{center}

\[AC\cdot BD = AB\cdot CD + AD \cdot BC.\]

When the quadrilateral is not cyclic we have the following inequality
\[AB\cdot CD+BC\cdot AD>AC\cdot BD\]

An interesting particular case is when both $AC$ and $BD$ are diameters, since we get another proof of Pythagoras' theorem.
%%%%%
%%%%%
%%%%%
%%%%%
%%%%%
%%%%%
%%%%%
%%%%%
%%%%%
%%%%%
%%%%%
\end{document}
