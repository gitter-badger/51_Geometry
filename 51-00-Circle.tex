\documentclass[12pt]{article}
\usepackage{pmmeta}
\pmcanonicalname{Circle}
\pmcreated{2013-03-22 13:36:23}
\pmmodified{2013-03-22 13:36:23}
\pmowner{PrimeFan}{13766}
\pmmodifier{PrimeFan}{13766}
\pmtitle{circle}
\pmrecord{14}{34236}
\pmprivacy{1}
\pmauthor{PrimeFan}{13766}
\pmtype{Definition}
\pmcomment{trigger rebuild}
\pmclassification{msc}{51-00}
\pmsynonym{circular}{Circle}
\pmrelated{SqueezingMathbbRn}
\pmrelated{CurvatureOfACircle}
\pmdefines{unit circle}
\pmdefines{radius}
\pmdefines{radii}
\pmdefines{perimeter}
\pmdefines{circumference}
\pmdefines{three point formula for the circle}
\pmdefines{center}

\endmetadata

% this is the default PlanetMath preamble.  as your knowledge
% of TeX increases, you will probably want to edit this, but
% it should be fine as is for beginners.

% almost certainly you want these
\usepackage{amssymb}
\usepackage{amsmath}
\usepackage{amsfonts}

\newcommand{\sR}[0]{\mathbb{R}}

\makeatletter
\@ifundefined{bibname}{}{\renewcommand{\bibname}{References}}
\makeatother
\begin{document}
A \emph{circle} is the locus of points which are equidistant from some fixed point. It is in the plane is determined by a \emph{center} and a \emph{radius}. The center is a point in the plane,
and the length of the radius is a positive real number, the radius being a line segment from the center to the circumference. The circle consists of all points whose distance from the center equals the radius.

Another way of defining the circle is thus: Given $A$ and $P$ as two points and $O$ as another point, the circle with center $O$ is the locus of points $X$ with $OX$ congruent to $AP$. (Hilbert, 1927)

Notice there is no definition of distance needed to make that definition
and so it works in many geometries, even ones with no distance function.
Hilbert uses it in his Foundations of Geometry book
Also used by Forder in his Foundations of Euclidean geometry book, c 1927

In this entry, we only work with the  standard Euclidean 
norm in the plane. A circle has one center only.

A circle determines a closed curve in the plane, and this curve is called the \emph{perimeter} or \emph{circumference}
of the circle. If the radius of a circle is $r$, then the length of the 
perimeter  is $2\pi r$. Also, the area of the  circle is $\pi r^2$. More precisely, the interior of the perimeter has 
area $\pi r^2$. The \emph{diameter} of a circle is defined as $d = 2r$. 

The circle is a special case of an ellipse. Also, in three dimensions, the analogous geometric object to a circle 
is a sphere. 

\section{\PMlinkescapetext{The circle in analytic geometry}} 

Let us next derive an analytic equation for a circle in Cartesian coordinates $(x,y)$. If the circle has center $(a,b)$ and radius $r > 0$, we obtain the following condition for the points of the sphere, 
\begin{equation}
\label{firsteq}
(x - a)^2 + (y - b)^2 = r^2.
\end{equation}
In other words, the circle is the set of all points $(x,y)$ that satisfy the above equation. In the special case that 
$a = b = 0$, the equation is simply $x^2 + y^2 = r^2$. The \emph{unit circle} is the circle $x^2 + y^2 = 1$. 

It is clear that equation \ref{firsteq} can always be reduced to the form
\begin{equation}
\label{secondeq}
x^2 + y^2 + Dx + Ey + F = 0,
\end{equation}
where $D,E,F$ are real numbers. Conversely, suppose that we are given an equation of the above form where $D,E,F$ are
arbitrary real numbers. Next we derive conditions for these constants, so that equation~\eqref{secondeq} determines 
a circle \cite{kindle}. Completing the squares yields
\begin{equation*}
x^2 + Dx + \frac{D^2}{4} + y^2 + Ey + \frac{E^2}{4} =  - F + \frac{D^2}{4} +  \frac{E^2}{4},
\end{equation*}
whence
\begin{equation*}
\left(x + \frac{D}{2}\right)^2 + \left(y + \frac{E}{2}\right)^2 =  \frac{D^2 - 4F + E^2}{4}.
\end{equation*}
There are three cases:
\begin{enumerate}
\item If $D^2 - 4F + E^2 > 0$, then equation~\eqref{secondeq} determines
a circle with center $( - \frac{D}{2}, - \frac{E}{2})$ and radius 
$\frac{1}{2}\sqrt{D^2 - 4F + E^2}$.
\item If $D^2 - 4F + E^2 = 0$, then equation~\eqref{secondeq} determines
the point $( - \frac{D}{2}, - \frac{E}{2})$.
\item If $D^2 - 4F + E^2<0$, then equation~\eqref{secondeq} has no (real)
solution in the $(x,y)$ - plane.
\end{enumerate}

\section{The circle in polar coordinates}

Using polar coordinates for the plane, we can parameterize the circle. Consider the circle with center $(a,b)$ and radius $r > 0$ in the plane $\sR^2$. It is then natural to introduce polar coordinates $(\rho, \phi)$ for $\sR^2\setminus\{(a,b)\}$ by 
\begin{align*}
x(\rho, \phi) & =  a + \rho \cos \phi, \\
y(\rho, \phi) & =  b + \rho \sin \phi,
\end{align*}
with $\rho > 0$ and $\phi\in [0,2\pi)$. Since we wish to parameterize the circle, the point $(a,b)$ does not pose a problem; it is not part of the circle. Plugging these expressions for $x,y$ into 
equation~\eqref{firsteq} yields the condition $\rho = r$. 
The given circle is thus parameterization by $\phi\mapsto (a + \rho \cos \phi, b + \rho \sin \phi)$, $\phi\in[0,2\pi)$. It follows that a circle is a closed curve in the plane. 

\section{Three point formula for the circle}

Suppose we are given three points on a circle, say $(x_1, y_1)$, $(x_2, y_2)$, $(x_3, y_3)$. 
We next derive expressions for the parameters $D,E,F$ in terms of these points. 
We also derive equation~\eqref{determ}, which gives an equation for a circle in terms 
of a determinant. 

First, from equation~\eqref{secondeq}, we have
\begin{align*}
x_1^2 + y_1^2 + Dx_1 + Ey_1 + F & =  0, \\
x_2^2 + y_2^2 + Dx_2 + Ey_2 + F & =  0, \\
x_3^2 + y_3^2 + Dx_3 + Ey_3 + F & =  0.
\end{align*} 
These equations form a linear set of equations for $D,E,F$, i.e., 
\begin{equation*}
\begin{pmatrix}
 x_1 & y_1 & 1 \\
 x_2 & y_2 & 1 \\
 x_3 & y_3 & 1 
\end{pmatrix}
\cdot
\begin{pmatrix}
 D  \\E \\F 
\end{pmatrix} 
 =  
 - \begin{pmatrix}
x_1^2 + y_1^2   \\
x_2^2 + y_2^2  \\
x_3^2 + y_3^2  \\
 \end{pmatrix}.
\end{equation*}
Let us denote the matrix on the left hand side by $\Lambda$. Also, let us assume 
that $\det \Lambda \neq 0$. Then, using Cramer's rule, we obtain
\begin{align*}
D & =   - \frac{1}{\det \Lambda} \det \begin{pmatrix}
 x_1^2 + y_1^2  & y_1 & 1 \\
 x_2^2 + y_2^2  & y_2 & 1 \\
 x_3^2 + y_3^2 & y_3 & 1 \\
 \end{pmatrix}, \\
E & =   - \frac{1}{\det \Lambda} \det \begin{pmatrix}
 x_1 & x_1^2 + y_1^2  & 1 \\
 x_2 & x_2^2 + y_2^2  & 1 \\
 x_3 & x_3^2 + y_3^2  & 1 \\
 \end{pmatrix}, \\
F & =   - \frac{1}{\det \Lambda} \det \begin{pmatrix}
 x_1 & y_1 & x_1^2 + y_1^2 \\
 x_2 & y_2 & x_2^2 + y_2^2 \\
 x_3 & y_3 & x_3^2 + y_3^2 \\
 \end{pmatrix}.
\end{align*}
These equations give the parameters $D,E,F$ as functions of the 
three given points. 
Substituting these equations into equation~\eqref{secondeq} yields
\begin{align*}
(x^2 + y^2)\det \begin{pmatrix}
 x_1 & y_1 & 1 \\
 x_2 & y_2 & 1 \\
 x_3 & y_3 & 1 \\
 \end{pmatrix}  &\mathbin{ - } x\det \begin{pmatrix}
 x_1^2 + y_1^2  & y_1 & 1 \\
 x_2^2 + y_2^2  & y_2 & 1 \\
 x_3^2 + y_3^2 & y_3 & 1 \\
 \end{pmatrix} \\
&\mathbin{ - } y\det \begin{pmatrix}
 x_1 & x_1^2 + y_1^2  & 1 \\
 x_2 & x_2^2 + y_2^2  & 1 \\
 x_3 & x_3^2 + y_3^2  & 1 \\
 \end{pmatrix} \\
 &\mathbin{ - } \det \begin{pmatrix}
 x_1 & y_1 & x_1^2 + y_1^2 \\
 x_2 & y_2 & x_2^2 + y_2^2 \\
 x_3 & y_3 & x_3^2 + y_3^2 \\
 \end{pmatrix} = 0.
\end{align*}
Using the cofactor expansion, we can now write the equation for the circle passing through $(x_1,y_1), (x_2,y_2), (x_3,y_3)$ as \cite{weissteincirc, beta}
\begin{equation}
\label{determ}
\det \begin{pmatrix}
 x^2 + y^2 & x & y & 1 \\
 x_1^2 + y_1^2 & x_1 & y_1 & 1 \\
 x_2^2 + y_2^2 & x_2 & y_2 & 1 \\
 x_3^2 + y_3^2 & x_3 & y_3 & 1 \\
 \end{pmatrix} = 0.
\end{equation}

\begin{thebibliography}{9}
\bibitem{hilbert} D. Hilbert, {\it Foundations of Geometry} Chicago: The Open Court Publishing Co. (1921): 163
\bibitem {kindle} J. H. Kindle,
        \emph{Schaum's Outline Series: Theory and problems of plane of Solid Analytic Geometry},
        Schaum Publishing Co., 1950.
\bibitem{weissteincirc} E. Weisstein, Eric W. Weisstein's world of mathematics, 
\PMlinkexternal{entry on the circle}{http://mathworld.wolfram.com/Circle.html}.
\bibitem{beta} L. R\r{a}de, B. Westergren,
 \emph{Mathematics Handbook for Science and Engineering},
 Studentlitteratur, 1995.
\end{thebibliography}
%%%%%
%%%%%
\end{document}
