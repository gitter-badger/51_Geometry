\documentclass[12pt]{article}
\usepackage{pmmeta}
\pmcanonicalname{TransitionToSkewangledCoordinates}
\pmcreated{2013-03-22 17:09:39}
\pmmodified{2013-03-22 17:09:39}
\pmowner{pahio}{2872}
\pmmodifier{pahio}{2872}
\pmtitle{transition to skew-angled coordinates}
\pmrecord{15}{39472}
\pmprivacy{1}
\pmauthor{pahio}{2872}
\pmtype{Topic}
\pmcomment{trigger rebuild}
\pmclassification{msc}{51N20}
\pmrelated{RotationMatrix}
\pmrelated{Hyperbola2}
\pmrelated{ConjugateDiametersOfEllipse}
\pmdefines{skew-angled coordinate}

\endmetadata

% this is the default PlanetMath preamble.  as your knowledge
% of TeX increases, you will probably want to edit this, but
% it should be fine as is for beginners.

% almost certainly you want these
\usepackage{amssymb}
\usepackage{amsmath}
\usepackage{amsfonts}

% used for TeXing text within eps files
%\usepackage{psfrag}
% need this for including graphics (\includegraphics)
%\usepackage{graphicx}
% for neatly defining theorems and propositions
 \usepackage{amsthm}
% making logically defined graphics
%%%\usepackage{xypic}

% there are many more packages, add them here as you need them

% define commands here

\theoremstyle{definition}
\newtheorem*{thmplain}{Theorem}

\begin{document}
Let the Euclidean plane $\mathbb{R}$ be equipped with the rectangular coordinate system with the $x$ and $y$ coordinate axes.\, We choose new coordinate axes through the old origin and \PMlinkname{project}{Projection} the new coordinates $\xi$, $\eta$ of a point orthogonally on the $x$ and $y$ axes getting the old coordinates expressed as
\begin{align}
\begin{cases}
  x = \xi\cos\alpha+\eta\cos\beta,\\
  y = \xi\sin\alpha+\eta\sin\beta,
\end{cases}
\end{align}
where $\alpha$ and $\beta$ are the angles which the $\xi$-axis and $\eta$-axis, respectively, form with the $x$-axis (positive if $x$-axis may be rotated anticlocwise to $\xi$-axis, else negative; similarly for rotating the $x$-axis to the $\eta$-axis).

The \PMlinkescapetext{inverse formulas} of (1) are got by solving from it for $\xi$ and $\eta$, getting
$$\xi = \frac{x\sin\beta-y\cos\beta}{\sin(\beta\!-\!\alpha)},\quad
 \eta = \frac{-x\sin\alpha+y\cos\alpha}{\sin(\beta\!-\!\alpha)}.$$

\textbf{Example.}\; Let us consider the \PMlinkname{hyperbola}{Hyperbola2}
\begin{align}
   \frac{x^2}{a^2}-\frac{y^2}{b^2} = 1
\end{align}
and take its asymptote \,$y = -\frac{b}{a}x$\, for the $\xi$-axis and the asymptote\, $y = +\frac{b}{a}c$\, for the $\eta$-axis.\, If $\omega$ is the angle formed by the latter asymptote with the $x$-axis, then\, $\alpha = -\omega$,\, $\beta = \omega$.\, By (1) we get first
\begin{align*}
\begin{cases}
  x = \xi\cos\omega+\eta\cos\omega = (\eta\!+\!\xi)\cos\omega,\\
  y = -\xi\sin\omega+\eta\sin\omega = (\eta\!-\!\xi)\sin\omega.
\end{cases}
\end{align*}
Since\, $\displaystyle\tan\omega = \frac{b}{a}$,\, we see that\, $\displaystyle\cos\omega = \frac{a}{c}$,\, $\displaystyle\sin\omega = \frac{b}{c}$,\, where\, $c^2 = a^2+c^2$,\, 
and accordingly
$$\frac{x}{a} = (\eta\!+\!\xi)\frac{a}{c}:a = \frac{\eta\!+\!\xi}{c},\quad
  \frac{y}{b} = (\eta\!-\!\xi)\frac{b}{c}:b = \frac{\eta\!-\!\xi}{c}.$$
Substituting these quotients in the equation of the hyperbola we obtain
           $$(\eta\!+\!\xi)^2-(\eta\!-\!\xi)^2 = c^2,$$
and after simplifying,
\begin{align}
      \xi\eta = \frac{c^2}{4}.
\end{align}
This is the equation of the hyperbola (2) in the coordinate system of its asymptotes.\, Here, $c$ is the distance of the \PMlinkname{focus}{Hyperbola2} from the nearer \PMlinkname{apex}{Hyperbola2} of the hyperbola.\\

If we, conversely, have in the rectangular coordinate system ($x,\,y$) an equation of the form (3), e.g.
\begin{align}           
       xy = \mbox{\,constant},
\end{align}
we can infer that it \PMlinkescapetext{represents} a hyperbola with asymptotes the coordinate axes.  Since these are perpendicular to each other, it's clear that the hyperbola (4) is a \PMlinkname{rectangular}{Hyperbola2} one.

\begin{thebibliography}{8}
\bibitem{LL}{\sc L. Lindel\"of}: {\em Analyyttisen geometrian oppikirja}.\, Kolmas painos.\, Suomalaisen Kirjallisuuden Seura, Helsinki (1924).
\end{thebibliography}


%%%%%
%%%%%
\end{document}
