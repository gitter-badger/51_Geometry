\documentclass[12pt]{article}
\usepackage{pmmeta}
\pmcanonicalname{SolidOfRevolution}
\pmcreated{2013-03-22 17:19:57}
\pmmodified{2013-03-22 17:19:57}
\pmowner{nkirby}{11104}
\pmmodifier{nkirby}{11104}
\pmtitle{solid of revolution}
\pmrecord{10}{39685}
\pmprivacy{1}
\pmauthor{nkirby}{11104}
\pmtype{Definition}
\pmcomment{trigger rebuild}
\pmclassification{msc}{51M25}
\pmrelated{SurfaceOfRevolution2}

\endmetadata

% this is the default PlanetMath preamble.  as your knowledge
% of TeX increases, you will probably want to edit this, but
% it should be fine as is for beginners.

% almost certainly you want these
\usepackage{amssymb}
\usepackage{amsmath}
\usepackage{amsfonts}

% used for TeXing text within eps files
%\usepackage{psfrag}
% need this for including graphics (\includegraphics)
%\usepackage{graphicx}
% for neatly defining theorems and propositions
%\usepackage{amsthm}
% making logically defined graphics
%%%\usepackage{xypic}

% there are many more packages, add them here as you need them

% define commands here

\begin{document}
Let $y=f(x)$ be a curve for $x$ in an interval $[a,b]$ satisfying $f(x)> 0$ for $x$ in $(a,b)$. We may construct a corresponding solid of revolution, say $\mathcal{V}=\left \{(x,y,z): x \in [a,b] \mbox{ and } y^2+z^2\leq f\left(x\right)^2\right \}$. Intuitively, it is the solid generated by rotating the surface $y\leq f(x)$ about the $x$-axis.

The interior of a surface of revolution is always a solid of revolution. These include
\begin{itemize}
\item
the interior of a cylinder of radius $c>0$ and height $h$ with $f(x)=c$ for $0\leq x\leq h$,
\item
the interior of a sphere of radius $R>0$ with $f(x)=\sqrt{R^2-x^2}$ for $-R\leq x \leq R$, and
\item
the interior of a (right, circular) cone of base radius $R>0$ and height $h$ with $f(x)=Rx/h$ for $0\leq x\leq h$.
\end{itemize}

Let $\Gamma$ be a simple closed curve with parametrization $\alpha\left(t\right)=\left(X\left(t\right),Y\left(t\right)\right)$ 
for $t$ in an interval $[a,b]$ satisfying $Y\left(t\right)\geq 0$ for $t$ in $[a,b]$. 
By the Jordan curve theorem, we may choose the set of points, $\mathcal{S}$, "inside" the curve, 
i.e. let $\mathcal{S}$ be the bounded connected component of the two connected components
 found in $\mathbb{R}^2\setminus \Gamma$. 
Another sort of solid of revolution is given by 
$\mathcal{V}=\left \{ (x,y,z): x=X(t) \mbox{ for some } t \mbox{ in } [a,b] \mbox{ and } y^2+z^2=s^2 \mbox{ for some } s \mbox{ such that } (x,s)\in \mathcal{S} \cup \Gamma \right \}$. 
Intuitively, it is the solid generated by rotating $\mathcal{S}\cup \Gamma$ about the $x$-axis.

Some examples of this sort of solid of revolution include
\begin{itemize}
\item
the interior of a torus of minor radius $r>0$ and major radius $R>r$ with $\alpha\left(t\right)=\left(r\cos t,r \sin t+R\right)$ for $0\leq t\leq 2\pi$,
\item
a shell of a sphere with inner radius $r>0$ and outer radius $R>r$ with 
\[
\alpha\left(t\right)=\begin{cases} \left(R\cos \pi t, R\sin \pi t\right) & \mbox{ if } t \in [0,1] \\
\left(r\left(1-t\right)+R\left(t-2\right),0\right) & \mbox{ if } t \in [1,2]\\
\left(-r\cos \pi t,r\sin \pi t\right) & \mbox{ if } t \in [2,3]\\
\left(r\left(4-t\right)+R\left(t-3\right),0\right) & \mbox{ if } t \in [3,4].    \end{cases}
\]


\end{itemize}
%%%%%
%%%%%
\end{document}
