\documentclass[12pt]{article}
\usepackage{pmmeta}
\pmcanonicalname{AngleBetweenTwoPlanes}
\pmcreated{2013-03-22 16:19:41}
\pmmodified{2013-03-22 16:19:41}
\pmowner{CWoo}{3771}
\pmmodifier{CWoo}{3771}
\pmtitle{angle between two planes}
\pmrecord{9}{38457}
\pmprivacy{1}
\pmauthor{CWoo}{3771}
\pmtype{Definition}
\pmcomment{trigger rebuild}
\pmclassification{msc}{51N20}
\pmsynonym{angle between planes}{AngleBetweenTwoPlanes}
\pmrelated{AngleBetweenTwoLines}
\pmrelated{AngleBetweenLineAndPlane}
\pmdefines{angle between differentiable surfaces}

\endmetadata

\usepackage{amssymb,amscd}
\usepackage{amsmath}
\usepackage{amsfonts}

% used for TeXing text within eps files
%\usepackage{psfrag}
% need this for including graphics (\includegraphics)
\usepackage{graphicx}
% for neatly defining theorems and propositions
%\usepackage{amsthm}
% making logically defined graphics
%%%\usepackage{xypic}
%\usepackage{pst-plot}
%\usepackage{psfrag}

% define commands here

\begin{document}
Let $\pi_1$ and $\pi_2$ be two planes in the three-dimensional Euclidean space $\mathbb{R}^3$.\, The angle $\theta$ between these planes is defined by means of the normal vectors $\boldsymbol{n}_1$ and $\boldsymbol{n}_2$ of $\pi_1$ and $\pi_2$ through the relationship
$$\cos\theta = \Big| \frac{\langle \boldsymbol{n}_1,\boldsymbol{n}_2\rangle }{\| \boldsymbol{n}_1 \| \| \boldsymbol{n}_2 \|}\Big|,$$
where the numerator is the inner product of $\boldsymbol{n}_1$ and $\boldsymbol{n}_2$ and the denominator is product of the lengths of $\boldsymbol{n}_1$ and $\boldsymbol{n}_2$.\, The formula implies that the angle $\theta$ satisfies
$$0 \le \theta \le \frac{\pi}{2}.$$

The quotient in the formula remains unchanged as one multiplies the normal vectors by some non-zero real numbers, so that the cosine is independent of the lengths of the chosen vectors.\, Therefore, there is no ambiguity in this definition.

\begin{figure}
\begin{center}
\includegraphics{angle.eps}
\caption{Angle between two planes}
\end{center}
\end{figure}

\textbf{Generalization}.\, The above definition can be generalized, at least locally, to a pair of intersecting differentiable surfaces in $\mathbb{R}^3$.\, Given two differentiable surfaces $S_1$ and $S_2$ and a point $p\in S_1\cap S_2$, the angle between $S_1$ and $S_2$ at $p$ is defined to be the angle between the tangent planes $T_p(S_1)$ and $T_p(S_2)$.

%%%%%
%%%%%
\end{document}
