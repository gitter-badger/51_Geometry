\documentclass[12pt]{article}
\usepackage{pmmeta}
\pmcanonicalname{AntipodalMapOnSnIsHomotopicToTheIdentityIfAndOnlyIfNIsOdd}
\pmcreated{2013-03-22 15:47:33}
\pmmodified{2013-03-22 15:47:33}
\pmowner{mps}{409}
\pmmodifier{mps}{409}
\pmtitle{antipodal map on $S^n$ is homotopic to the identity if and only if $n$ is odd}
\pmrecord{5}{37753}
\pmprivacy{1}
\pmauthor{mps}{409}
\pmtype{Derivation}
\pmcomment{trigger rebuild}
\pmclassification{msc}{51M05}
\pmclassification{msc}{15-00}

% this is the default PlanetMath preamble.  as your knowledge
% of TeX increases, you will probably want to edit this, but
% it should be fine as is for beginners.

% almost certainly you want these
\usepackage{amssymb}
\usepackage{amsmath}
\usepackage{amsfonts}

% used for TeXing text within eps files
%\usepackage{psfrag}
% need this for including graphics (\includegraphics)
%\usepackage{graphicx}
% for neatly defining theorems and propositions
\usepackage{amsthm}
% making logically defined graphics
%%%\usepackage{xypic}

% there are many more packages, add them here as you need them

% define commands here
\newtheorem*{lemma*}{Lemma}
\newtheorem*{proposition*}{Proposition}
\begin{document}
\begin{lemma*}
If $X\colon S^n\to S^n$ is a unit vector field, then
there is a homotopy between the antipodal map on $S^{n}$
and the identity map.
\end{lemma*}

\begin{proof}
Regard $S^n$ as a subspace of $R^{n+1}$ and define
$H\colon S^n\times[0,1]\to R^{n+1}$ by 
$H(v,t)=(\cos\pi t)v+(\sin\pi t)X(v)$.  Since $X$ is a unit
vector field, $X(v)\perp v$ for any $v\in S^n$.  Hence
$\|H(v,t)\|=1$, so $H$ is into $S^n$.  Finally observe that
$H(v,0)=v$ and $H(v,1)=-v$.  Thus $H$ is a homotopy between
the antipodal map and the identity map.
\end{proof}

\begin{proposition*}
The antipodal map $A\colon S^n\to S^n$ is homotopic
to the identity if and only if $n$ is odd.
\end{proposition*}

\begin{proof}
If $n$ is even, then the antipodal map $A$ is the composition 
of an odd \PMlinkescapetext{number} of reflections.  It 
therefore has degree $-1$.  Since the degree of the identity
map is $+1$, the two maps are not homotopic.

Now suppose $n$ is odd, say $n=2k-1$.  Regard $S^n$ has a 
subspace of $\mathbb{R}^{2k}$.  So each point of $S^n$ has
coordinates $(x_1,\dots,x_{2k})$ with $\sum_i x_i^2=1$.  Define
a map $X\colon\mathbb{R}^{2k}\to\mathbb{R}^{2k}$ by
$X(x_1,x_2,\dots,x_{2k-1},x_{2k})=(-x_2,x_1,\dots,-x_{2k},x_{2k-1})$,
pairwise swapping coordinates and negating the even coordinates.
By construction, for any $v\in S^n$, we have that $\|X(v)\|=1$
and $X(v)\perp v$.  Hence $X$ is a unit vector field.  Applying the
lemma, we conclude that the antipodal map is homotopic to the identity.
\end{proof}

\begin{thebibliography}{9}
\bibitem{H}
Hatcher, A. {\em Algebraic topology}, Cambridge University Press, 2002.
\bibitem{M}
Munkres, J. {\em Elements of algebraic topology}, Addison-Wesley, 1984.
\end{thebibliography}
%%%%%
%%%%%
\end{document}
