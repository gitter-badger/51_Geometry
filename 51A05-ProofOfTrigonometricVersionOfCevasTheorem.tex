\documentclass[12pt]{article}
\usepackage{pmmeta}
\pmcanonicalname{ProofOfTrigonometricVersionOfCevasTheorem}
\pmcreated{2013-03-22 14:49:20}
\pmmodified{2013-03-22 14:49:20}
\pmowner{drini}{3}
\pmmodifier{drini}{3}
\pmtitle{proof of trigonometric version of Ceva's theorem}
\pmrecord{7}{36484}
\pmprivacy{1}
\pmauthor{drini}{3}
\pmtype{Proof}
\pmcomment{trigger rebuild}
\pmclassification{msc}{51A05}

\endmetadata

\usepackage{graphicx}
%%%\usepackage{xypic} 
\usepackage{bbm}
\newcommand{\Z}{\mathbbmss{Z}}
\newcommand{\C}{\mathbbmss{C}}
\newcommand{\R}{\mathbbmss{R}}
\newcommand{\Q}{\mathbbmss{Q}}
\newcommand{\mathbb}[1]{\mathbbmss{#1}}
\newcommand{\figura}[1]{\begin{center}\includegraphics{#1}\end{center}}
\newcommand{\figuraex}[2]{\begin{center}\includegraphics[#2]{#1}\end{center}}
\newtheorem{dfn}{Definition}
\usepackage{amsmath}
\begin{document}
The proof goes by proving the condition imposed on the sines, is equivalent to the  one imposed on the sides on the normal version of Ceva's theorem.
\figuraex{ceva}{scale=0.75}

We want to prove
\[
\frac{\sin ACZ}{\sin ZCB}\cdot\frac{\sin BAX}{\sin XAC}\cdot\frac{\sin CBY}{\sin YBA}=1
\]
if and only if
\[
\frac{AZ}{ZB}\cdot\frac{BX}{XC}\cdot\frac{CY}{YA}=1.
\]

Now, the generalization of bisectors theorem states that
\begin{align*}
\frac{AZ}{ZB} &= \frac{CA \sin ACZ}{BC \sin ZCB},\\
\frac{BX}{XC} &= \frac{AB \sin BAX}{CA \sin XAC},\\
\frac{CY}{YA} &= \frac{BC \sin CBY}{AB \sin YBA}.
\end{align*}

Multiplying the three equations, and cancelling all segments on the right side gives the desired equivalence.
%%%%%
%%%%%
\end{document}
