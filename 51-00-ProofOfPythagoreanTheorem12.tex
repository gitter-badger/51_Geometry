\documentclass[12pt]{article}
\usepackage{pmmeta}
\pmcanonicalname{ProofOfPythagoreanTheorem12}
\pmcreated{2013-03-22 17:09:35}
\pmmodified{2013-03-22 17:09:35}
\pmowner{rspuzio}{6075}
\pmmodifier{rspuzio}{6075}
\pmtitle{proof of Pythagorean theorem}
\pmrecord{22}{39471}
\pmprivacy{1}
\pmauthor{rspuzio}{6075}
\pmtype{Proof}
\pmcomment{trigger rebuild}
\pmclassification{msc}{51-00}
\pmrelated{GeneralizedPythagoreanTheorem}

% this is the default PlanetMath preamble.  as your knowledge
% of TeX increases, you will probably want to edit this, but
% it should be fine as is for beginners.

% almost certainly you want these
\usepackage{amssymb}
\usepackage{amsmath}
\usepackage{amsfonts}

% used for TeXing text within eps files
%\usepackage{psfrag}
% need this for including graphics (\includegraphics)
%\usepackage{graphicx}
% for neatly defining theorems and propositions
%\usepackage{amsthm}
% making logically defined graphics
%%\usepackage{xypic}

% there are many more packages, add them here as you need them

% define commands here

\begin{document}
Let $a$ and $b$ be two lengths and let $c$ denote the length of the
hypotenuse of a right triangle whose legs have lengths $a$ and $b$.
\[
\begin{xy}
,(0,0)
;(0,40)**@{-}
;(30,0)**@{-}
;(0,0)**@{-}
,(-2,15)*{a}
,(20,-2)*{b}
,(21,16)*{c}
\end{xy}
\]
Behold the following two ways of dissecting a square of length $a+b$:
\[
\begin{xy}
,(0,30)
;(0,0)**@{-}
;(70,0)**@{-}
;(70,70)**@{-}
;(0,70)**@{-}
;(0,30)**@{-}
;(40,0)**@{-}
;(70,40)**@{-}
;(30,70)**@{-}
;(0,30)**@{-}
,(-2,15)*{a}
,(-2,50)*{b}
,(72,20)*{b}
,(72,55)*{a}
,(20,-2)*{b}
,(55,-2)*{a}
,(15,72)*{a}
,(50,72)*{b}
,(21,16)*{c}
,(54,21)*{c}
,(49,54)*{c}
,(16,49)*{c}
,(-2,72)*{A}
,(-2,-2)*{B}
,(72,-2)*{C}
,(72,72)*{D}
,(30,72)*{H}
,(72,40)*{G}
,(40,-2)*{F}
,(-2,30)*{E}
\end{xy}
\]
\centerline{Figure 1}
\[
\begin{xy}
,(0,0)
;(0,70)**@{-}
;(70,70)**@{-}
;(70,0)**@{-}
;(0,0)**@{-}
,(0,40)
;(70,40)**@{-}
,(30,0)
;(30,70)**@{-}
,(0,40)
;(30,0)**@{-}
,(30,70)
;(70,40)**@{-}
,(15,72)*{a}
,(50,72)*{b}
,(15,-2)*{a}
,(50,-2)*{b}
,(-2,20)*{b}
,(-2,55)*{a}
,(72,20)*{b}
,(72,55)*{a}
,(28,55)*{a}
,(32,20)*{b}
,(15,42)*{a}
,(50,38)*{b}
,(14,19)*{c}
,(51,56)*{c}
,(-2,72)*{A'}
,(-2,-2)*{B'}
,(72,-2)*{C'}
,(72,72)*{D'}
,(30,72)*{K}
,(72,40)*{L}
,(30,-2)*{M}
,(-2,40)*{N}
,(32,38)*{O}
\end{xy}
\]
\centerline{Figure 2}
We now discuss the construction of these diagrams and note some facts about them,
starting with the first one.  To construct figure 1, we proceed as follows:
\begin{itemize}
\item Construct a square $ABCD$ whose sides have length $a+b$.
\item Lay point $E$ on the line segment $AB$ at a distance $a$ from $B$.
Likewise, lay the point $F$ on the line segment $BC$ at a distance $a$
from $C$, lay the point $G$ on the line segment $CD$ at a distance $a$
from $D$, and lay the point $H$ on the line segment $DA$ at a distance
$a$ from $A$.
\item  Connect the line segments $EF$, $FG$, $GH$, and $HE$.
\end{itemize}
We now note the following facts about figure 1:
\begin{itemize}
\item  Since the lengths of the four sides of the square $ABCD$ all equal $a+b$
and the line segments $EB$, $FC$, $GD$, and $HA$ were constructed to have length
$a$, it follows that the lengths of the line segments $AE$, $BF$, $CG$, $HD$ all
equal $b$, as indicated upon the figure.
\item  Since $ABCD$ is a square, the angles $ABC$, $BCD$, $CDA$, and $DAB$ are all
right angles, and hence equal each other.
\item  By the side-angle-side theorem, the triangles $HAE$, $EBF$, $FCG$, and
$GDH$ are all congruent to each other.
\item  By definition of $c$ as the length of a hypotenuse, it follows that the
line segments $EF$, $FG$, $GH$, and $HE$ all have length $c$, as indicated in the
figure.
\item  Because the sum of the angles of a triangle equals two right angles and
the angle $EAH$ is a right angle, it follows that the sum of the angles $AEH$ and
$AHE$ equals a right angle.  Since the triangle $AEH$ is congruent to the triangle
$BFE$, the angles $AHE$ and $BEF$ are equal.  Hence the sum of the angles $AEH$
and $BEF$ equals a right angle.  Thus, we may conclude that $HEF$ is a right angle.
By similar reasoning, we conclude that $EFG$, $FGH$, and $GHE$ are also right angles.
\item  Since its sides are of equal length and its angles are all right angles, the
quadrilateral $EFGH$ is a square.
\end{itemize}
To construct figure 2, we proceed as follows:
\begin{itemize}
\item Construct a square $A'B'C'D'$ whose sides have length $a+b$.
\item Lay point $N$ on the line segment $A'B'$ at a distance $a$ from $A'$.
Likewise, lay the point $M$ on the line segment $B'C'$ at a distance $a$
from $B'$, lay the point $L$ on the line segment $C'D'$ at a distance $a$
from $D'$, and lay the point $K$ on the line segment $D'A'$ at a distance
$a$ from $A'$.
\item  Connect the line segments $KM$, $NL$, $MN$, and $KL$.
\end{itemize}
We now note the following facts about figure 2:
\begin{itemize}
\item  Since the lengths of the four sides of the square $A'B'C'D'$ all equal $a+b$
and the line segments $A'N$, $A'K$, $B'M$, and $D'L$ were constructed to have length
$a$, it follows that the lengths of the line segments $B'N$, $C'L$, $C'M$, $D'K$ all
equal $b$, as indicated upon the figure.
\item  Since $A'B'C'D'$ is a square, the angles $A'B'C'$, $'B'C'D'$, $C'D'A'$, and 
$D'A'B'$ are all right angles, and hence equal each other.
\item Since $A'B'$ and $C'D'$, as opposite sides of a square, are parallel and their
subsegments $A'N$ and $D'L$ have equal length, it follows that $NL$ is parallel to
$A'D'$ and $B'C'$.  Likewise, since $A'D'$ and $B'C'$, as opposite sides of the
same square, are parallel and their subsegments $A'K$ and $B'M$ have equal length,
it follows that $KM$ is parallel to $A'B'$ and $C'D'$.
\item Since $NL$ is parallel to $B'C'$ and $A'B'C'$ is a right angle, it follows
that $A'NL$ is a right angle; moreover, since $A'D'C'$ is a right angle, $C'LN$ is
also a right angle.  Likewise, since $MK$ is parallel to $C'D'$ and $A'D'C'$ is a 
right angle, it follows that $A'KM$ is a right angle; moreover, since $A'B'C'$ is 
a right angle, $C'MK$ is also a right angle.  Since $A'B'$ is parallel to $KM$ and
$A'NL$ is a right angle, it follows that $KOL$, $LOM$, $MON$, and $NOK$ are all
right angles.
\item  Since all the angles of the quadrilateral $A'KON$ are right angles and two
of its adjacent sides, $A'K$ and $A'N$, have the same length $a$, this figure is a
square, hence the remaining sides, $OK$ and $ON$, also have length $a$.  Likewise,
Since all the angles of the quadrilateral $C'LOM$ are right angles and two
of its adjacent sides, $C'L$ and $C'M$, have the same length $b$, this figure is a
square, hence the remaining sides, $OL$ and $OM$, also have length $b$.
\item By the side-angle-side theorem, the traingles $KOL$, $LD'K$, $MB'N$, and
$NOM$ are congruent to each other and to the triangles $HAE$, $EBF$, $FCG$, and
$GDH$.
\end{itemize}

Hence, we have shown that a square with sides of length $a+b$ may be dissected into
either
\begin{itemize}
\item a square $A'KON$ with sides of length $a$,
\item a square $C'LOM$ with sides of length $b$,
\item four right triangles, $KOL$, $LD'K$, $MB'N$, and
$NOM$ with sides of length $a$, $b$, $c$
\end{itemize}
or
\begin{itemize}
\item a square $EFGH$ with sides of length $c$,
\item four right triangles, $HAE$, $EBF$, $FCG$, and
$GDH$,  with sides of length $a$, $b$, $c$.
\end{itemize}
Since the area of the whole equals the sum of the areas of the parts and the squares
$ABCD$ and $A'B'C'D'$ are congruent, hence have equal areas, it follows that the
sum of the areas of the figures comprising the former dissection equals the sum of the
areas of the figures comprising the latter dissection.  Since the triangles involved in
these dissections all are congruent, hence have equal areas, we may cancel their areas
to conclude that the area of $EFGH$ equals the sum of the areas of $A'KON$ and $C'LOM$,
or $a^2 + b^2 = c^2$.
%%%%%
%%%%%
\end{document}
