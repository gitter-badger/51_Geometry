\documentclass[12pt]{article}
\usepackage{pmmeta}
\pmcanonicalname{HeronsPrinciple}
\pmcreated{2014-09-15 15:38:36}
\pmmodified{2014-09-15 15:38:36}
\pmowner{pahio}{2872}
\pmmodifier{pahio}{2872}
\pmtitle{Heron's principle}
\pmrecord{13}{41619}
\pmprivacy{1}
\pmauthor{pahio}{2872}
\pmtype{Theorem}
\pmcomment{trigger rebuild}
\pmclassification{msc}{51M04}
\pmrelated{Catacaustic}
\pmrelated{PropertiesOfEllipse}
\pmrelated{HeronianMeanIsBetweenGeometricAndArithmeticMean}

\endmetadata

% this is the default PlanetMath preamble.  as your knowledge
% of TeX increases, you will probably want to edit this, but
% it should be fine as is for beginners.

% almost certainly you want these
\usepackage{amssymb}
\usepackage{amsmath}
\usepackage{amsfonts}
\usepackage{pstricks}
\usepackage{amsthm}

% used for TeXing text within eps files
%\usepackage{psfrag}
% need this for including graphics (\includegraphics)
%\usepackage{graphicx}
% for neatly defining theorems and propositions
 \usepackage{amsthm}
% making logically defined graphics
%%%\usepackage{xypic}
\usepackage{pstricks}
\usepackage{pst-plot}

% there are many more packages, add them here as you need them

% define commands here

\theoremstyle{definition}
\newtheorem*{thmplain}{Theorem}

\begin{document}
\PMlinkescapeword{side} \PMlinkescapeword{sides}

\textbf{Theorem.}\, In the Euclidean plane, let $l$ be a line 
and $A$ and $B$ two points not on $l$.\, If $X$ is a point of 
$l$ such that the sum $AX\!+\!XB$ is the least possible, 
then the lines $AX$ and $BX$ form equal angles with the 
line $l$.

This {\em Heron's principle}, concerning the reflection of 
light, is a special case of {\em Fermat's principle} in optics.\\

{\em Proof.}\, If $A$ and $B$ are on different sides of $l$, then $X$ must be on the line $AB$, and the assertion is trivial since the vertical angles are equal.\, Thus, let the points $A$ and $B$ be on the same side of $l$.\, Denote by $P$ and $Q$ the points of the line $l$ where the normals of $l$ set through $A$ and $B$ intersect $l$, respectively.\, Let $C$ be the intersection point of the lines $AQ$ and $BP$.\, Then, $X$ is the point of $l$ where the normal line of $l$ set through $C$ intersects $l$.
\begin{center}
\begin{pspicture}(-3,-1)(3,3)
\psline(-2.6,0)(2.6,0)
\psdots[linecolor=blue](-2,2.5)(2,1.6)
\psline[linestyle=dashed](-2,2.5)(-2,0)
\psline[linestyle=dashed](2,1.6)(2,0)
\psline(-2,2.5)(2,0)
\psline(2,1.6)(-2,0)
\psline(0.439,0.976)(0.439,0)
\psdot[linecolor=red](0.439,0)
\rput(-2.2,2.75){$A$}
\rput(2,1.83){$B$}
\rput(-2,-0.25){$P$}
\rput(2,-0.25){$Q$}
\rput(0.44,1.3){$C$}
\rput(0.44,-0.25){$X$}
\rput(2.8,0){$l$}
\end{pspicture}
\end{center}
Justification:\, From two pairs of similar right triangles we get the proportion equations
$$AP:CX \;=\; PQ:XQ, \qquad BQ:CX \;=\; PQ:PX,$$
which imply the equation
$$AP:PX \;=\; BQ:XQ.$$
From this we can infer that also
$$\Delta AXP \sim \Delta BXQ.$$
Thus the corresponding angles $AXP$ and $BXQ$ are equal.
\begin{center}
\begin{pspicture}(-3,-3)(3,3)
\psline(-2.6,0)(2.6,0)
\psdots[linecolor=blue](-2,2.5)(2,1.6)
\psline[linestyle=dashed](-2,2.5)(-2,-2.5)(0.439,0)
\psline[linecolor=blue](-2,2.5)(0.439,0)
\psline[linecolor=blue](0.439,0)(2,1.6)
\psdot[linecolor=red](0.439,0)
\rput(-2,2.75){$A$}
\rput(2,1.83){$B$}
\rput(-2.2,-0.25){$P$}
\rput(0.44,-0.27){$X$}
\rput(2.8,0){$l$}
\psline[linestyle=dotted](-2,2.5)(-0.7,0)(2,1.6)
\psdots(-0.7,0)(-2,-2.5)
\rput(-0.7,-0.29){$X_1$}
\rput(-2.3,-2.5){$A'$}
\psline(-2.15,1.15)(-1.85,1.15)
\psline(-2.15,1.05)(-1.85,1.05)
\psline(-2.15,-1.2)(-1.85,-1.2)
\psline(-2.15,-1.1)(-1.85,-1.1)
\end{pspicture}
\end{center}
We still state that the route $AXB$ is the shortest.\, If $X_1$ is another point of the line $l$, then\, $AX_1\,=\,A'X_1$,\, and thus we obtain
$$AX_1B \;=\; A'X_1B \;=\; A'X_1+X_1B \;\geqq\; A'B \;=\; A'XB \;=\; AXB.$$

\begin{thebibliography}{8}
\bibitem{th}{\sc Tero Harju}: {\em Geometria. Lyhyt kurssi}.\, 
Matematiikan laitos. Turun yliopisto (University of Turku), Turku (2007).
\end{thebibliography}

%%%%%
%%%%%
\end{document}
