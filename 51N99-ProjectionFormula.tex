\documentclass[12pt]{article}
\usepackage{pmmeta}
\pmcanonicalname{ProjectionFormula}
\pmcreated{2013-03-22 18:27:11}
\pmmodified{2013-03-22 18:27:11}
\pmowner{pahio}{2872}
\pmmodifier{pahio}{2872}
\pmtitle{projection formula}
\pmrecord{9}{41115}
\pmprivacy{1}
\pmauthor{pahio}{2872}
\pmtype{Theorem}
\pmcomment{trigger rebuild}
\pmclassification{msc}{51N99}
\pmsynonym{projection formula for triangles}{ProjectionFormula}
\pmrelated{BaseAndHeightOfTriangle}

% this is the default PlanetMath preamble.  as your knowledge
% of TeX increases, you will probably want to edit this, but
% it should be fine as is for beginners.

% almost certainly you want these
\usepackage{amssymb}
\usepackage{amsmath}
\usepackage{amsfonts}

% used for TeXing text within eps files
%\usepackage{psfrag}
% need this for including graphics (\includegraphics)
%\usepackage{graphicx}
% for neatly defining theorems and propositions
 \usepackage{amsthm}
% making logically defined graphics
%%%\usepackage{xypic}
\usepackage{pstricks}
\usepackage{pst-plot}

% there are many more packages, add them here as you need them

% define commands here

\theoremstyle{definition}
\newtheorem*{thmplain}{Theorem}

\begin{document}
\textbf{Theorem.}\; Let $a$, $b$, $c$ be the sides of a triangle and $\alpha$, $\beta$ the angles opposing $a$, $b$, respectively.\, Then one has
$$c = a\cos\beta+b\cos\alpha,$$
independently whether the angles are acute, right or obtuse.\\

Knowing the way to determine the length of the \PMlinkname{projection}{ProjectionOfPoint} of a line segment, the truth of the theorem is apparent; the below \PMlinkescapetext{diagrams} illustrate the cases where $\beta$ is acute and obtuse (cosine of an obtuse angle is negative). 


\begin{center}
\begin{pspicture}(-3,-1)(8.3,4)
\rput(-3,-1){.}
\rput(8.3,3){.}
\pspolygon(-2.5,0)(1,0)(0,3)
\psline[linestyle=dashed](0,3)(0,0)
\rput(-1.45,1.6){$a$}
\rput(0.7,1.6){$b$}
\rput(-2.1,0.2){$\beta$}
\rput(0.7,0.2){$\alpha$}
\rput(-1.2,-0.3){$a\cos\beta$}
\rput(0.6,-0.3){$b\cos\alpha$}

\pspolygon(4.7,0)(8,0)(3,3)
\psline[linestyle=dashed](3,3)(3,0)
\psline[linestyle=dashed](4.7,0)(3,0)
\rput(3.74,1.3){$a$}
\rput(5.8,1.7){$b$}
\rput(4.9,0.2){$\beta$}
\rput(7.4,0.16){$\alpha$}
\rput(3.8,-0.33){$|a\cos\beta|$}
\rput(4.9,-0.8){$b\cos\alpha$}
\psline{<-}(2.9,-0.8)(4.35,-0.8)
\psline{->}(5.5,-0.8)(8,-0.8)
\end{pspicture}
\end{center}


\textbf{Note.}\; Especially, if neither of $\alpha$ and $\beta$ is right angle, the formula of the theorem may be written
$$\frac{a}{\cos\alpha}+\frac{b}{\cos\beta} = \frac{c}{\cos\alpha\,\cos\beta}.$$

%%%%%
%%%%%
\end{document}
