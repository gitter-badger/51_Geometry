\documentclass[12pt]{article}
\usepackage{pmmeta}
\pmcanonicalname{MenelausTheorem}
\pmcreated{2013-03-22 12:46:20}
\pmmodified{2013-03-22 12:46:20}
\pmowner{mathwizard}{128}
\pmmodifier{mathwizard}{128}
\pmtitle{Menelaus' theorem}
\pmrecord{7}{33083}
\pmprivacy{1}
\pmauthor{mathwizard}{128}
\pmtype{Theorem}
\pmcomment{trigger rebuild}
\pmclassification{msc}{51A05}
\pmrelated{CevasTheorem}
\pmrelated{TrigonometricVersionOfCevasTheorem}
\pmrelated{Collinear}

\endmetadata

\usepackage{amssymb}
\usepackage{amsmath}
\usepackage{amsfonts}

% used for TeXing text within eps files
%\usepackage{psfrag}
% need this for including graphics (\includegraphics)
\usepackage{graphicx}
% for neatly defining theorems and propositions
%\usepackage{amsthm}
% making logically defined graphics
%%%\usepackage{xypic}

% there are many more packages, add them here as you need them

% define commands here
\begin{document}
If the points $X$, $Y$ and $Z$ are on the sides of a triangle $ABC$ (including their prolongations), collinear and do not coincide with any of the points $A$, $B$ and $C$, then the equation
$$\frac{AZ}{ZB}\cdot\frac{BY}{YC}\cdot\frac{CX}{XA} = -1$$
holds (all segments are directed line segments). The converse of this theorem also holds (thus: three points on the prolongations of the triangle's sides are collinear if the above equation holds).
\begin{center}
\includegraphics{menelaus.eps}
\end{center}
%%%%%
%%%%%
\end{document}
