\documentclass[12pt]{article}
\usepackage{pmmeta}
\pmcanonicalname{TangentOfConicSection}
\pmcreated{2013-03-22 14:28:40}
\pmmodified{2013-03-22 14:28:40}
\pmowner{pahio}{2872}
\pmmodifier{pahio}{2872}
\pmtitle{tangent of conic section}
\pmrecord{16}{36008}
\pmprivacy{1}
\pmauthor{pahio}{2872}
\pmtype{Definition}
\pmcomment{trigger rebuild}
\pmclassification{msc}{51N20}
\pmsynonym{tangent of quadratic curve}{TangentOfConicSection}
\pmrelated{TangentLine}
\pmrelated{TangentOfCircle}
\pmrelated{TangentPlaneOfQuadraticSurface}
\pmrelated{QuadraticInequality}
\pmrelated{ConjugateDiametersOfEllipse}
\pmrelated{ConjugateHyperbola}
\pmrelated{QuadraticCurves}
\pmrelated{EquationOfTangentOfCircle}
\pmrelated{TangentOfHyperbola}
\pmdefines{polarising}
\pmdefines{polarizing}
\pmdefines{polarize}
\pmdefines{mixed term}

\endmetadata

% this is the default PlanetMath preamble.  as your knowledge
% of TeX increases, you will probably want to edit this, but
% it should be fine as is for beginners.

% almost certainly you want these
\usepackage{amssymb}
\usepackage{amsmath}
\usepackage{amsfonts}

% used for TeXing text within eps files
%\usepackage{psfrag}
% need this for including graphics (\includegraphics)
%\usepackage{graphicx}
% for neatly defining theorems and propositions
%\usepackage{amsthm}
% making logically defined graphics
%%%\usepackage{xypic}

% there are many more packages, add them here as you need them

% define commands here
\begin{document}
The equation of every conic section (and the degenerate cases) in the rectangular 
$(x,\,y)$-coordinate system may be written in the form
            $$Ax^2+By^2 +2Cxy+2Dx+2Ey+F = 0,$$
where $A$, $B$, $C$, $D$, $E$ and $F$ are constants and \,$A^2+B^2+C^2 > 0.$\footnote{This is true also in any skew-angled coordinate system.} \, (The \PMlinkescapetext{{\em mixed term}} $2Cxy$ is present only if the \PMlinkescapetext{principal} axes are not parallel to the coordinate axes.)

The equation of the {\em tangent line} of an ordinary conic section (i.e., circle, ellipse, hyperbola and parabola) in the point $(x_0,\,y_0)$ of the curve is
  $$Ax_0x+By_0y+C(y_0x+x_0y)+D(x+x_0)+E(y+y_0)+F = 0.$$
Thus, the equation of the tangent line can be obtained from the equation of the curve by {\em polarizing} it, i.e. by replacing 

\qquad $x^2$ with $x_0x$, \,$y^2$ with $y_0y$, \,$2xy$ with $y_0x+x_0y$, \,$2x$ with $x+x_0$, \,$2y$ with $y+y_0$.

\textbf{Examples:} \,The \PMlinkescapetext{tangent} of the ellipse \,$\frac{x^2}{a^2}+\frac{y^2}{b^2} = 1$ \, is \,$\frac{x_0x}{a^2}+\frac{y_0y}{b^2} = 1$, the \PMlinkescapetext{tangent} of the hyperbola \,$xy = \frac{1}{2}$ \, is \,$y_0x+x_0y = 1$.
%%%%%
%%%%%
\end{document}
