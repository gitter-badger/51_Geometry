\documentclass[12pt]{article}
\usepackage{pmmeta}
\pmcanonicalname{CentralCollineations}
\pmcreated{2013-03-22 16:03:13}
\pmmodified{2013-03-22 16:03:13}
\pmowner{Algeboy}{12884}
\pmmodifier{Algeboy}{12884}
\pmtitle{central collineations}
\pmrecord{8}{38105}
\pmprivacy{1}
\pmauthor{Algeboy}{12884}
\pmtype{Definition}
\pmcomment{trigger rebuild}
\pmclassification{msc}{51A10}
\pmclassification{msc}{51A05}
\pmrelated{Perspectivity}
\pmdefines{transvection}
\pmdefines{center}
\pmdefines{axis}
\pmdefines{central collineation}

\endmetadata

\usepackage{latexsym}
\usepackage{amssymb}
\usepackage{amsmath}
\usepackage{amsfonts}
\usepackage{amsthm}

%%\usepackage{xypic}

%-----------------------------------------------------

%       Standard theoremlike environments.

%       Stolen directly from AMSLaTeX sample

%-----------------------------------------------------

%% \theoremstyle{plain} %% This is the default

\newtheorem{thm}{Theorem}

\newtheorem{coro}[thm]{Corollary}

\newtheorem{lem}[thm]{Lemma}

\newtheorem{lemma}[thm]{Lemma}

\newtheorem{prop}[thm]{Proposition}

\newtheorem{conjecture}[thm]{Conjecture}

\newtheorem{conj}[thm]{Conjecture}

\newtheorem{defn}[thm]{Definition}

\newtheorem{remark}[thm]{Remark}

\newtheorem{ex}[thm]{Example}



%\countstyle[equation]{thm}



%--------------------------------------------------

%       Item references.

%--------------------------------------------------


\newcommand{\exref}[1]{Example-\ref{#1}}

\newcommand{\thmref}[1]{Theorem-\ref{#1}}

\newcommand{\defref}[1]{Definition-\ref{#1}}

\newcommand{\eqnref}[1]{(\ref{#1})}

\newcommand{\secref}[1]{Section-\ref{#1}}

\newcommand{\lemref}[1]{Lemma-\ref{#1}}

\newcommand{\propref}[1]{Prop\-o\-si\-tion-\ref{#1}}

\newcommand{\corref}[1]{Cor\-ol\-lary-\ref{#1}}

\newcommand{\figref}[1]{Fig\-ure-\ref{#1}}

\newcommand{\conjref}[1]{Conjecture-\ref{#1}}


% Normal subgroup or equal.

\providecommand{\normaleq}{\unlhd}

% Normal subgroup.

\providecommand{\normal}{\lhd}

\providecommand{\rnormal}{\rhd}
% Divides, does not divide.

\providecommand{\divides}{\mid}

\providecommand{\ndivides}{\nmid}


\providecommand{\union}{\cup}

\providecommand{\bigunion}{\bigcup}

\providecommand{\intersect}{\cap}

\providecommand{\bigintersect}{\bigcap}

\DeclareMathOperator{\GL}{GL}


\begin{document}
\section*{Definitions and general properties}

\begin{defn}
A collineation of a finite dimensional projective geometry is a \emph{central collineation} 
if there is a hyperplane of points fixed by the collineation.
\end{defn}

Recall that collineations send any three collinear points to three collinear points.  Thus
if a collineation fixes more than a hyperplane of points then it in fact fixes all the points
of the geometry and so it is the identity map.  Therefore a central collineation can be 
viewed the simplest of the non-identity collineations. 

\begin{thm}
Every collineation of a finite dimensional projective geometry of dimension $n>1$ is
a product of at most $n$ central collineations.  In particular, the automorphism group
of a projective geometry of dimension $n>1$ is generated by central collineations.
\end{thm}

Suppose that a central collineation is not the identity.  Then the hyperplane of fixed points
is unique and receives the title of \emph{the axis of the central collineation}.  There is
one further important result which justifies the name ``central''.

\begin{prop}\label{prop:center}
Given a non-identity central collineation $f$, there is a 
unique point $C$ such that for all other points $P$, it follows that $C$, $P$ and $Pf$ are
collinear.
\end{prop}

The point $C$ determined by Proposition \ref{prop:center} is called the \emph{center}
of the non-identity central collineation.  It is possible for the center to lie on the axis.

\section*{Central collineations in coordinates}

Suppose we have a projective geometry of dimension $n>2$, that is, we exclude now
the case of projective lines and planes.  The the geometry can be coordinatized through
so that we may regard the projective geometry as the lattice
of subspaces of a vector space $V$ of dimension $n+1$ over a division ring $\Delta$.
Following the fundamental theorem of projective geometry we further know that 
every collineation is induced by a semi-linear transformation of $V$.  So it is possible
to explore central collineations as semi-linear transformations.  

Every hyperplane is a kernel of some linear functional, so we let $\varphi:V\to \Delta$ be a linear functional of $V$ with $H\varphi=0$.  Furthermore, we fix $v\in V$ so that
$v\varphi=1$ (which implieas also that $v\notin H$).     Hence, for each $u\in V$, 
$u=(u-(u\varphi)v)+(u\varphi)v$ where $u-(u\varphi) v\in H$ and $(u\varphi)v\in\langle v\rangle$.


Let $f\in \GL_{\Delta}(V)$ such that $f$ induces a central collineation $\tilde{f}$ on $PG(V)$ with
axis $H\leq V$.   As every scalar multiple of $f$ induces the same collineation of $PG(V)$, we 
may assume that $f$ is the identity on $H$.  Using the decomposition given by $\varphi$ we have
\[uf = ((u-(u\varphi)v)+(u\varphi)v)f = (u-(u\varphi)v) + (u\varphi) vf,\qquad u\in V.\]
Hence
\[uf = u + (u\varphi)\hat{v},\qquad \hat{v}:=vf-v.\]

Suppose instead that $\varphi$ is any linear functional of $V$.  Then select some $\hat{v}\in V$
such that $\hat{v}\varphi\neq -1$.  Then 
\[ug := u+(u\varphi)\hat{v}\]
fixes all the points of $\ker \varphi$ so $g$ induces a central collineation.

If we wish to do the same without appealing to linear functionals, we may select a basis 
$\{v_1,\dots, v_{n+1}\}$ such that $H=\langle v_1,\dots, v_n\rangle$ and 
$v_{n+1}\varphi=1$.  As $f$ is selected to be the identity on $H$ we have so far specified
$f$ by the matrix:
\[
\begin{bmatrix}
1 & \cdots & 0 & 0 \\
\vdots  & \ddots & & \\
0 & & 1 & \\
a_1 & a_2 &\cdots  & a_{n+1}
\end{bmatrix}
\]
in the basis $\{v_1,\dots, v_n,v_{n+1}\}$. 

%%%%%
%%%%%
\end{document}
