\documentclass[12pt]{article}
\usepackage{pmmeta}
\pmcanonicalname{IsogonalTrajectory}
\pmcreated{2013-03-22 18:59:20}
\pmmodified{2013-03-22 18:59:20}
\pmowner{pahio}{2872}
\pmmodifier{pahio}{2872}
\pmtitle{isogonal trajectory}
\pmrecord{7}{41855}
\pmprivacy{1}
\pmauthor{pahio}{2872}
\pmtype{Derivation}
\pmcomment{trigger rebuild}
\pmclassification{msc}{51N20}
\pmclassification{msc}{34A26}
\pmclassification{msc}{34A09}
%\pmkeywords{family of curves}
\pmrelated{AngleBetweenTwoCurves}
\pmrelated{OrthogonalCurves}
\pmrelated{ExampleOfIsogonalTrajectory}
\pmrelated{AngleBetweenTwoLines}
\pmdefines{isogonal trajectory}

\endmetadata

% this is the default PlanetMath preamble.  as your knowledge
% of TeX increases, you will probably want to edit this, but
% it should be fine as is for beginners.

% almost certainly you want these
\usepackage{amssymb}
\usepackage{amsmath}
\usepackage{amsfonts}

% used for TeXing text within eps files
%\usepackage{psfrag}
% need this for including graphics (\includegraphics)
%\usepackage{graphicx}
% for neatly defining theorems and propositions
 \usepackage{amsthm}
% making logically defined graphics
%%%\usepackage{xypic}

% there are many more packages, add them here as you need them

% define commands here

\theoremstyle{definition}
\newtheorem*{thmplain}{Theorem}

\begin{document}
Let a one-parametric family of plane curves \,$\gamma$\, have the differential equation
\begin{align}
F(x,\,y,\,\frac{dy}{dx}) \;=\; 0.
\end{align}
We want to determine the \emph{isogonal trajectories} of this family, i.e. the curves \,$\iota$\, intersecting all members of the family under a given angle, which is denoted by $\omega$.
For this purpose, we denote the slope angle of any curve $\gamma$ at such an intersection point by $\alpha$ and the slope angle of $\iota$ at the same point by $\beta$.\, Then
$$\beta-\alpha \;=\; \omega  \quad(\mbox{or alternatively\;\;} -\omega),$$
and accordingly
$$\frac{dy}{dx} \;=\; \tan\alpha \;=\; \frac{\tan\beta-\tan\omega}{1+\tan\beta\tan\omega} 
\;=\; \frac{y'-\tan\omega}{1+y'\tan\omega},$$
where $y'$ means the slope of $\iota$.\, Thus the equation
\begin{align}
F(x,\,y,\,\frac{y'-\tan\omega}{1+y'\tan\omega}) \;=\; 0
\end{align}
is satisfied by the derivative $y'$ of the ordinate of $\iota$.\, In other \PMlinkescapetext{words}, (2) is the differential equation of all isogonal trajectories of the given family of curves.\\

\textbf{Note.}\, In the special case \,$\omega = \frac{\pi}{2}$,\, it's a question of orthogonal trajectories.

%%%%%
%%%%%
\end{document}
