\documentclass[12pt]{article}
\usepackage{pmmeta}
\pmcanonicalname{PencilOfLines}
\pmcreated{2013-03-22 18:09:03}
\pmmodified{2013-03-22 18:09:03}
\pmowner{pahio}{2872}
\pmmodifier{pahio}{2872}
\pmtitle{pencil of lines}
\pmrecord{8}{40707}
\pmprivacy{1}
\pmauthor{pahio}{2872}
\pmtype{Theorem}
\pmcomment{trigger rebuild}
\pmclassification{msc}{51N20}
\pmrelated{LineInThePlane}
\pmrelated{Determinant2}
\pmrelated{HomogeneousLinearProblem}
\pmrelated{Pencil2}

% this is the default PlanetMath preamble.  as your knowledge
% of TeX increases, you will probably want to edit this, but
% it should be fine as is for beginners.

% almost certainly you want these
\usepackage{amssymb}
\usepackage{amsmath}
\usepackage{amsfonts}

% used for TeXing text within eps files
%\usepackage{psfrag}
% need this for including graphics (\includegraphics)
%\usepackage{graphicx}
% for neatly defining theorems and propositions
 \usepackage{amsthm}
% making logically defined graphics
%%%\usepackage{xypic}

% there are many more packages, add them here as you need them

% define commands here

\theoremstyle{definition}
\newtheorem*{thmplain}{Theorem}

\begin{document}
Let 
\begin{align}
A_ix+B_iy+C_i = 0
\end{align}
be equations of some lines.\, Use the short notations\, $A_ix+B_iy+C_i \,:=\, L_i$.

If the lines\, $L_1 = 0$\, and\, $L_2 = 0$\, have an intersection point $P$, then, by the \PMlinkname{parent entry}{LineThroughAnIntersectionPoint}, the equation
\begin{align}
k_1L_1+k_2L_2 = 0
\end{align}
with various real values of $k_1$ and $k_2$ can \PMlinkescapetext{represent} any line passing through the point $P$; this set of lines is called a {\em pencil of lines}.\\
 

\textbf{Theorem.}\, A necessary and sufficient condition in \PMlinkescapetext{order} to three lines
$$L_1 = 0, \quad L_2 = 0, \quad L_3 = 0$$
pass through a same point, is that the determinant formed by the coefficients of their equations (1) vanishes:
$$
\left|\begin{matrix}
A_1 & B_1 & C_1\\
A_2 & B_2 & C_2\\
A_3 & B_3 & C_3
\end{matrix}\right| = 
\left|\begin{matrix}
A_1 & A_2 & A_3\\
B_1 & B_2 & B_3\\
C_1 & C_2 & C_3
\end{matrix}\right| = 0.
$$

{\em Proof.}\, If the line\, $L_3 = 0$\, belongs to the fan of lines determined by the lines\, $L_1 = 0$\, and\, $L_2 = 0$,\, i.e. all the three lines have a common point, there must be the identity
$$L_3 \equiv L_1+L_2,$$
i.e. there exist three real numbers $k_1$, $k_2$, $k_3$, which are not all zeroes, such that the equation
\begin{align}
k_1L_1+k_2L_2+k_3L_3 \equiv 0
\end{align}
is satisfied identically by all real values of $x$ and $y$.
This means that the group of homogeneous linear equations
\begin{align*}
\begin{cases}
k_1A_1+k_2A_2+k_3A_3 = 0,\\
k_1B_1+k_2B_2+k_3B_3 = 0,\\  
k_1C_1+k_2C_2+k_3C_3 = 0
\end{cases}
\end{align*}
has nontrivial solutions $k_1,\,k_2,\,k_3$.
By linear algebra, it follows that the determinant of this group of equations has to vanish.

Suppose conversely that the determinant vanishes.\, This implies that the above group of equations has a nontrivial solution $k_1,\,k_2,\,k_3$.\, Thus we can write the identic equation (3).\, Let e.g.\, $k_1 \neq 0$.\, Solving (3) for $L_1$ yields
$$L_1 \equiv -\frac{k_2L_2+k_3L_3}{k_1},$$
which shows that the line\, $L_1 = 0$\, belongs to the fan determined by the lines\, $L_2 = 0$\, and\, $L_3 = 0$; so the lines pass through a common point.

\begin{thebibliography}{8}
\bibitem{LP}{\sc Lauri Pimi\"a}: {\em Analyyttinen geometria}.\, Werner S\"oderstr\"om Osakeyhti\"o, Porvoo and Helsinki (1958).
\end{thebibliography} 

%%%%%
%%%%%
\end{document}
