\documentclass[12pt]{article}
\usepackage{pmmeta}
\pmcanonicalname{PolarCoordinates}
\pmcreated{2013-03-22 15:12:16}
\pmmodified{2013-03-22 15:12:16}
\pmowner{CWoo}{3771}
\pmmodifier{CWoo}{3771}
\pmtitle{polar coordinates}
\pmrecord{14}{36963}
\pmprivacy{1}
\pmauthor{CWoo}{3771}
\pmtype{Definition}
\pmcomment{trigger rebuild}
\pmclassification{msc}{51-01}
\pmrelated{DerivationOfRotationMatrixUsingPolarCoordinates}
\pmrelated{CylindricalCoordinates}
\pmrelated{ArgumentOfProductAndQuotient}

% this is the default PlanetMath preamble.  as your knowledge
% of TeX increases, you will probably want to edit this, but
% it should be fine as is for beginners.

% almost certainly you want these
\usepackage{amssymb}
\usepackage{amsmath}
\usepackage{amsfonts}
\usepackage{amsthm}

\usepackage{mathrsfs}

% used for TeXing text within eps files
%\usepackage{psfrag}
% need this for including graphics (\includegraphics)
\usepackage{graphicx}
% for neatly defining theorems and propositions
%
% making logically defined graphics
%%%\usepackage{xypic}

% there are many more packages, add them here as you need them

% define commands here

\newcommand{\sR}[0]{\mathbb{R}}
\newcommand{\sC}[0]{\mathbb{C}}
\newcommand{\sN}[0]{\mathbb{N}}
\newcommand{\sZ}[0]{\mathbb{Z}}

 \usepackage{bbm}
 \newcommand{\Z}{\mathbbmss{Z}}
 \newcommand{\C}{\mathbbmss{C}}
 \newcommand{\R}{\mathbbmss{R}}
 \newcommand{\Q}{\mathbbmss{Q}}



\newcommand*{\norm}[1]{\lVert #1 \rVert}
\newcommand*{\abs}[1]{| #1 |}



\newtheorem{thm}{Theorem}
\newtheorem{defn}{Definition}
\newtheorem{prop}{Proposition}
\newtheorem{lemma}{Lemma}
\newtheorem{cor}{Corollary}
\begin{document}
\PMlinkescapeword{relations}

Let $x,y$ be Cartesian coordinates for $\R^2$. 

Then $r\ge 0$, $\theta\in [0,2\pi)$ related to $(x,y)$ by
\begin{eqnarray*}
  x(r,\theta) &=& r \cos \theta, \\
  y(r,\theta) &=& r \sin \theta, 
\end{eqnarray*}
are the \emph{polar coordinates} for $(x,y)$.  It is simply written $(r,\theta)$. 

\begin{center}
\includegraphics{polar_4.eps}
\end{center}

The polar coordinates of Cartesian coordinates 
$(x,y) \in \R^2\setminus\{0\}$ are
\begin{eqnarray*}
   r(x,y) &=& \sqrt{x^2+ y^2}, \\
   \theta(x,y) &=& \arctan (x,y),
\end{eqnarray*}
where $\arctan$ is defined \PMlinkname{here}{OperatornamearcTanWithTwoArguments}.

\textbf{Polar basis.}
Polar coordinates are equipped with an orthonormal base $\{\mathbf{e_r,e_\theta}\}$, which can be defined in terms of the standard cartesian base $\{\mathbf{i,j}\}$ in $\mathbbmss{R}^2$ as follows.
\begin{align*}
\begin{bmatrix}\mathbf{e_r}\\ \mathbf{e_\theta}\end{bmatrix}=
\begin{bmatrix}\hphantom{-}\cos\theta\mathbf{i}+\sin\theta\mathbf{j}\\ 
-\sin\theta\mathbf{i}+\cos\theta\mathbf{j}\end{bmatrix},
\end{align*}
where $\mathbf{e_r,e_\theta}$ are so-called {\em radial} and {\em traverse} or {\em angular} vector, respectively. Since these vectors are variable in direction, they are differentiable. In fact,
\begin{align*}
\begin{bmatrix}\frac{d\mathbf{e_r}}{d\theta}\\ \frac{d\mathbf{e_\theta}}{d\theta}\end{bmatrix}=
\begin{bmatrix}\hphantom{-}\mathbf{e_\theta}\\ -\mathbf{e_r}\end{bmatrix}.
\end{align*}
The geometrical action of the derivative operator $d/d\theta$ is like a rotation operator that rotates each base vector a counter-clockwise angle equals to $\pi/2$. 

\textbf{Position vector.} For an arbitrary point of polar coordinates $(r,\theta)$, its position vector comes given by the single equation
\begin{align*}
\mathbf{r}=r\mathbf{e_r}.
\end{align*}

\textbf{Relations with complex numbers.}
When the Euclidean plane $\R^2$ is identified with $\C$ by
$$(x,y)\leftrightarrow x+yi,$$ it is possible to define multiplications on $\R^2$.\, Via polar coordinates, the formula for this multiplication becomes very simple, thanks to \PMlinkname{Euler's formula}{EulerRelation}
$$\cos{\theta}+i\sin{\theta}=e^{i\theta}.$$
Thus, we have the following identification:
$$(r,\theta)\leftrightarrow(x,y)\leftrightarrow x+yi=
r\cos{\theta}+(r\sin{\theta})i=re^{i\theta}.$$ If $P=(r_1,\theta_1)$
and $Q=(r_2,\theta_2)$, the product of $P$ and $Q$ works out to be
$(r_1r_2,\theta_1+\theta_2)$.
(Even if one is not familiar with the complex exponential, this assertion may be checked 
directly using the angle sum identities for $\cos$ and $\sin$.)

Multiplications of polar coordinates have some simple geometric
interpretations. For example, if $R=(1,\alpha)$ and $Q=(r,\beta)$,
then $Q\rightarrow RQ$ given by
$(1,\alpha)(r,\beta)=(r,\alpha+\beta)$ is the rotation of $Q$ by
angle $\alpha$. If $S=(t,0)$, then $(t,0)(r,\beta)=(tr,\beta)$ can
be viewed as the scaling of $Q$ along the ray $\overrightarrow{OQ}$
by $t$. Note also that multiplication by $(t,0)$ has the same effect
as multiplication by the scalar $t$.
\begin{center}
\includegraphics{polar_5.eps}
\end{center}
For more on polar coordinates, including their construction and extensions on domain of polar coordinates $r$ and $\theta$, see \PMlinkname{here}{ConstructionOfPolarCoordinates}.

\textbf{Calculus in polar coordiantes.}
For reference, here are some formulae for computing integrals and derivatives
in polar coordinates.  The Jacobian for transforming from rectangular to
polar cordinates is
\[
 {\partial (x, y) \over \partial (r, \theta)} = r 
\]
so we may compute the integral of a scalar field $f$ as
\[
 \int f (r, \theta) r \, dr \, d\theta .
\]

Partial derivative operators transform as follows:
\begin{align*}
 {\partial \over \partial x} &= \cos \theta {\partial \over \partial r} -
                                {1 \over r} \sin \theta {\partial \over \partial \theta} \\
 {\partial \over \partial y} &= \sin \theta {\partial \over \partial r} + 
                                {1 \over r} \cos \theta {\partial \over \partial \theta} \\
 {\partial \over \partial r} &= \cos \theta {\partial \over \partial x} +
                                \sin \theta {\partial \over \partial y} \\
 {\partial \over \partial \theta} &= - r \sin \theta {\partial \over \partial x} +
                                     r \cos \theta {\partial \over \partial y}
\end{align*}
 
%%%%%
%%%%%
\end{document}
