\documentclass[12pt]{article}
\usepackage{pmmeta}
\pmcanonicalname{HyperbolasOrthogonalToEllipses}
\pmcreated{2013-03-22 18:08:53}
\pmmodified{2013-03-22 18:08:53}
\pmowner{pahio}{2872}
\pmmodifier{pahio}{2872}
\pmtitle{hyperbolas orthogonal to ellipses}
\pmrecord{11}{40704}
\pmprivacy{1}
\pmauthor{pahio}{2872}
\pmtype{Example}
\pmcomment{trigger rebuild}
\pmclassification{msc}{51N20}
\pmsynonym{ellipses orthogonal to hyperbolas}{HyperbolasOrthogonalToEllipses}
%\pmkeywords{confocal ellipses}
%\pmkeywords{confocal hyperbolas}
\pmrelated{OrthogonalCurves}
\pmrelated{ZeroRuleOfProduct}
\pmrelated{Pencil2}

\endmetadata

% this is the default PlanetMath preamble.  as your knowledge
% of TeX increases, you will probably want to edit this, but
% it should be fine as is for beginners.

% almost certainly you want these
\usepackage{amssymb}
\usepackage{amsmath}
\usepackage{amsfonts}

% used for TeXing text within eps files
%\usepackage{psfrag}
% need this for including graphics (\includegraphics)
%\usepackage{graphicx}
% for neatly defining theorems and propositions
 \usepackage{amsthm}
% making logically defined graphics
%%%\usepackage{xypic}
\usepackage{pstricks}
\usepackage{pst-plot}

% there are many more packages, add them here as you need them

% define commands here

\theoremstyle{definition}
\newtheorem*{thmplain}{Theorem}

\begin{document}
Let\, $a^2 > b^2$,\; $s > -b^2$\, and\, $a^2 > t > b^2$.\, Show that each of the \PMlinkname{ellipses}{Ellipse2}
\begin{align}
\frac{x^2}{a^2+s}+\frac{y^2}{b^2+s} = 1
\end{align}
is an orthogonal curve of every \PMlinkname{hyperbola}{Hyperbola2}
\begin{align}
\frac{x^2}{a^2-t}-\frac{y^2}{t-b^2} = 1.
\end{align}


Let\, $(x_0,\,y_0)$\, be an intersection point of an ellipse (1) and a hyperbola (2).\, By polarizing both equations in the point \,$(x_0,\,y_0)$\, we get the equations of the tangents of the curves in this point:
$$\frac{x_0x}{a^2+s}+\frac{y_0y}{b^2+s} = 1, \quad \frac{x_0x}{a^2-t}-\frac{y_0y}{t-b^2} = 1$$
Solving these equations for $y$ shows that the slopes of the tangents are
$$m_1 = -\frac{b^2+s}{a^2+s}\!\cdot\!\frac{x_0}{y_0}, \quad m_2 = -\frac{t-b^2}{a^2-t}\!\cdot\!\frac{x_0}{y_0},$$
and thus their product is
\begin{align}
m_1m_2 = -\frac{(b^2+s)(t-b^2)}{(a^2+s)(a^2-t)}\!\cdot\!\frac{x_0^2}{y_0^2}.
\end{align}
On the other hand, the point\, $(x_0,\,y_0)$\, satisfies the equation gotten from (1) and (2) via subtraction:
$$0 = \left(\frac{1}{a^2+s}-\frac{1}{a^2-t}\right)x_0^2+\left(\frac{1}{b^2+s}-\frac{1}{t-b^2}\right)y_0^2
= (s+t)\left(\frac{-x_0^2}{(a^2+s)(a^2-t)}+\frac{y_0^2}{(b^2+s)(t-b^2)}\right)$$
Since $s\!+\!t$ cannot be 0, the second \PMlinkname{factor}{Product} of the abobe product must vanish, which implies the proportion equation
$$\frac{x_0^2}{(a^2+s)(a^2-t)} = \frac{y_0^2}{(b^2+s)(t-b^2)}.$$
Utilising this in the equation (3) yields the condition of orthogonality
$$m_1m_2 = -1,$$ 
for the tangents, which means that the ellipse and the hyperbola intersect orthogonally.\\

\begin{center}
\begin{pspicture}(-5,-4)(5,4)
\psaxes[Dx=9,Dy=9]{->}(0,0)(-4.5,-3.5)(4.5,3.5)
\rput(0.3,3.3){$y$}
\rput(4.4,-0.3){$x$}
\rput(-5,-4){.}
\rput(5,4){.}
\psplot[linecolor=blue]{1}{4}{x x mul 1 sub sqrt 2 div}
\psplot[linecolor=blue]{-4}{-1}{x x mul 1 sub sqrt 2 div}
\psplot[linecolor=blue]{1}{4}{0 x x mul 1 sub sqrt 2 div sub}
\psplot[linecolor=blue]{-4}{-1}{0 x x mul 1 sub sqrt 2 div sub}

\psplot[linecolor=blue]{0.86603}{4}{x x mul 2 mul 3 div 0.5 sub sqrt}
\psplot[linecolor=blue]{-4}{-0.86603}{x x mul 2 mul 3 div 0.5 sub sqrt}
\psplot[linecolor=blue]{0.86603}{4}{0 x x mul 2 mul 3 div 0.5 sub sqrt sub}
\psplot[linecolor=blue]{-4}{-0.86603}{0 x x mul 2 mul 3 div 0.5 sub sqrt sub}

\psplot[linecolor=cyan]{-1.5}{1.5}{1 -4 x mul x mul 9 div add sqrt}
\psplot[linecolor=cyan]{-1.5}{1.5}{0 1 -4 x mul x mul 9 div add sqrt sub}

\psplot[linecolor=cyan]{-1.80277}{1.80277}{1 -4 x mul x mul 13 div add 2 mul sqrt}
\psplot[linecolor=cyan]{-1.80277}{1.80277}{0 1 -4 x mul x mul 13 div add 2 mul sqrt sub}

\psdots[linecolor=red](+1.118,0)(-1.118,0)

\rput(0,-4.5){$ $}
\end{pspicture}
\end{center}

\textbf{Note.}\, Both the ellipses (1) and the hyperbolas (2) have the common foci  \,$(\pm\sqrt{a^2\!-\!b^2},\,0)$,\, being thus confocal.

%%%%%
%%%%%
\end{document}
