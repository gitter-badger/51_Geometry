\documentclass[12pt]{article}
\usepackage{pmmeta}
\pmcanonicalname{PropertiesOfRegularTetrahedron}
\pmcreated{2013-03-22 18:29:39}
\pmmodified{2013-03-22 18:29:39}
\pmowner{pahio}{2872}
\pmmodifier{pahio}{2872}
\pmtitle{properties of regular tetrahedron}
\pmrecord{15}{41178}
\pmprivacy{1}
\pmauthor{pahio}{2872}
\pmtype{Topic}
\pmcomment{trigger rebuild}
\pmclassification{msc}{51E99}
\pmsynonym{regular tetrahedron}{PropertiesOfRegularTetrahedron}
%\pmkeywords{carbon atom}
\pmrelated{Grafix}
\pmrelated{DehnsTheorem}
\pmrelated{Tetrahedron}

% this is the default PlanetMath preamble.  as your knowledge
% of TeX increases, you will probably want to edit this, but
% it should be fine as is for beginners.

% almost certainly you want these
\usepackage{amssymb}
\usepackage{amsmath}
\usepackage{amsfonts}

% used for TeXing text within eps files
%\usepackage{psfrag}
% need this for including graphics (\includegraphics)
%\usepackage{graphicx}
% for neatly defining theorems and propositions
 \usepackage{amsthm}
% making logically defined graphics
%%%\usepackage{xypic}

% there are many more packages, add them here as you need them

\usepackage{pstricks}

% define commands here

\theoremstyle{definition}
\newtheorem*{thmplain}{Theorem}

\begin{document}
\PMlinkescapeword{base}

A regular tetrahedron may be formed such that each of its edges is a diagonal of a face of a cube; then the tetrahedron has been inscribed in the cube.
\begin{center}
\begin{pspicture}(-3,-3)(3,3)
\psdots(-2,0)(2,0)(3.5,1)
\psline[linecolor=blue,linestyle=dashed](3.5,3)(-0.5,-1)
\pspolygon[linecolor=cyan](-2,-2)(2,-2)(3.5,-1)(3.5,3)(-0.5,3)(-2,2)
\psline[linecolor=cyan](2,2)(3.5,3)
\psline[linecolor=cyan](-2,2)(2,2)(2,-2)
\psline[linecolor=cyan,linestyle=dotted](-2,-2)(-0.5,-1)(3.5,-1)
\psline[linecolor=cyan,linestyle=dotted](-0.5,-1)(-0.5,3)
\pspolygon[linecolor=blue,linewidth=0.04](2,-2)(3.5,3)(-2,2)
\psline[linecolor=blue](-2,2)(-0.5,-1)(2,-2)
\end{pspicture}
\end{center}

It's apparent that a plane passing through the midpoints of three parallel edges of the cube cuts the regular tetrahedron into two congruent pentahedrons and that the intersection figure is a square, the midpoint $M$ of which is the centroid of the tetrahedron.\\


The angles between the four half-lines from the centroid $M$ of the regular tetrahedron to the \PMlinkname{vertices}{Polyhedron} are $2\arctan\!{\sqrt{2}}$ ($\approx 109^\circ$), which is equal the angle between the four covalent bonds of a carbon \PMlinkescapetext{atom}.\, A half of this angle, $\alpha$, can be found from the right triangle in the below figure, where the catheti are $\frac{s}{\sqrt{2}}$ and 
$\frac{s}{2}$.

\begin{center}
\begin{pspicture}(-3,-3.5)(3.9,3)
\psdot(2.75,0.5)
\psdot[linecolor=red](0.75,0.5)
\psline[linecolor=red](3.5,3)(0.75,0.5)(2.75,0.5)
\psline(2.8,0.68)(2.6,0.68)(2.53,0.5)
\psline[linecolor=blue,linestyle=dashed](0.98,0.47)(-0.5,-1)
\pspolygon[linecolor=cyan](-2,-2)(2,-2)(3.5,-1)(3.5,3)(-0.5,3)(-2,2)
\psline[linecolor=cyan,linewidth=0.05](2,2)(3.5,3)
\psline[linecolor=cyan,linewidth=0.05](-2,2)(2,2)(2,-2)
\psline[linecolor=cyan,linestyle=dotted](-2,-2)(-0.5,-1)(3.5,-1)
\psline[linecolor=cyan,linestyle=dotted](-0.5,-1)(-0.5,3)
\pspolygon[linecolor=blue,linewidth=0.04](2,-2)(3.5,3)(-2,2)
\psline[linecolor=blue](-2,2)(-0.5,-1)(2,-2)
\rput(0.47,0.56){$M$}
\rput(1.2,0.63){$\alpha$}
\rput(-0.2,-2.2){$s$}
\rput(2.9,-1.6){$s$}
\rput(3.7,0.8){$s$}
\rput(2.8,1.55){$\frac{s}{\sqrt{2}}$}
\rput(-3,-3.5){.}
\rput(3.9,3){.}
\end{pspicture}
\end{center}


One can consider the regular tetrahedron as a cone.\, Let its edge be $a$ and its height $h$.\, Because of symmetry, a height line intersects the corresponding base triangle in the centroid of this equilateral triangle.\, Thus we have (see the below \PMlinkescapetext{diagram}) the rectangular triangle with hypotenuse $a$, one cathetus $h$ and the other \PMlinkname{cathetus}{Cathetus} \,$\frac{2}{3}\!\cdot\!\frac{a\sqrt{3}}{2} = \frac{a}{\sqrt{3}}$\, (i.e. $\frac{2}{3}$ of the \PMlinkname{median}{Median} $\frac{a\sqrt{3}}{2}$ of the equilateral triangle --- see the common point of triangle medians).\, The Pythagorean theorem then gives
$$h \;=\; \sqrt{a^2-\left(\frac{a}{\sqrt{3}}\right)^2} \;=\; \frac{a\sqrt{6}}{3}.$$

\begin{center}
\begin{pspicture}(-3,-1.5)(3,4)
\pspolygon[linecolor=blue](-2,1)(-1,-1)(3,0.5)(0,3.5)
\psline[linecolor=blue,linestyle=dotted](-2,1)(3,0.5)
\psline[linestyle=dashed](-2,1)(1,-0.25)
\psline[linestyle=dashed](0,0.15)(0,3.5)
\psline[linecolor=blue](-1,-1)(0,3.5)
\psline(0,0.32)(-0.15,0.37)(-0.15,0.2)
\rput(0.2,1.5){$h$}
\rput(-1.2,2.3){$a$}
\rput(0,-1){$\frac{a}{2}$}
\rput(2.2,-0.2){$\frac{a}{2}$}
\rput(-3,-1.5){.}
\rput(3,4){.}
\end{pspicture}
\end{center}

Consequently, the height of the regular tetrahedron is $\displaystyle\frac{a\sqrt{6}}{3}$.

Since the area of the \PMlinkname{base triangle}{EquilateralTriangle} is $\frac{a^2\sqrt{3}}{4}$, the volume (one third of the product of the base and the height) of the regular tetrahedron is $\displaystyle\frac{a^3\sqrt{2}}{12}$.\\



%%%%%
%%%%%
\end{document}
