\documentclass[12pt]{article}
\usepackage{pmmeta}
\pmcanonicalname{ParallelLinesInHyperbolicGeometry}
\pmcreated{2013-03-22 17:06:43}
\pmmodified{2013-03-22 17:06:43}
\pmowner{Wkbj79}{1863}
\pmmodifier{Wkbj79}{1863}
\pmtitle{parallel lines in hyperbolic geometry}
\pmrecord{11}{39411}
\pmprivacy{1}
\pmauthor{Wkbj79}{1863}
\pmtype{Topic}
\pmcomment{trigger rebuild}
\pmclassification{msc}{51-00}
\pmclassification{msc}{51M10}
\pmdefines{asymptotically parallel}
\pmdefines{asymptotically parallel lines}
\pmdefines{hyperparallel}
\pmdefines{hyperparallel lines}
\pmdefines{disjointly parallel}
\pmdefines{disjointly parallel lines}
\pmdefines{ultraparallel}
\pmdefines{ultraparallel lines}

\usepackage{amssymb}
\usepackage{amsmath}
\usepackage{amsfonts}
\usepackage{pstricks}
\usepackage{psfrag}
\usepackage{graphicx}
\usepackage{amsthm}
%%\usepackage{xypic}

\begin{document}
In hyperbolic geometry, there are two kinds of parallel lines.  If two lines do not intersect within a model of hyperbolic geometry but they do intersect on its boundary, then the lines are called \emph{asymptotically parallel} or \emph{hyperparallel}.  (Note that, in the upper half plane model, any two vertical rays are asymptotically parallel.  Thus, for consistency, $\infty$ is considered to be part of the boundary.)  Any other set of parallel lines is called \emph{disjointly parallel} or \emph{ultraparallel}.

Below is an example of asymptotically parallel lines in the Beltrami-Klein model:

\begin{center}
\begin{pspicture}(-2,-2)(2,2)
\pscircle[linestyle=dashed](0,0){2}
\psline{o-o}(-2,0)(2,0)
\psline{o-o}(-1.6,1.2)(2,0)
\end{pspicture}
\end{center}

Below are some examples of asymptotically parallel lines in the Poincar\'e disc model:

\begin{center}
\begin{pspicture}(-2,-2)(2,2)
\pscircle[linestyle=dashed](0,0){2}
\psline{o-o}(-2,0)(2,0)
\psarc{o-o}(2,4.667){4.667}{223.92}{270}
\end{pspicture}
\end{center}

\begin{center}
\begin{pspicture}(-2,-2)(2,2)
\pscircle[linestyle=dashed](0,0){2}
\psarc{o-o}(0,2.5){1.5}{216.87}{323.13}
\psarc{o-o}(-3.333,0){2.667}{-36.87}{36.87}
\end{pspicture}
\end{center}

Below are some examples of asymptotically parallel lines in the upper half plane model:

\begin{center}
\begin{pspicture}(-3,-0.1)(3,3)
\psline[linestyle=dashed]{<->}(-3,0)(3,0)
\psline{o->}(-2,0)(-2,3)
\psarc{o-o}(0,0){2}{0}{180}
\end{pspicture}
\end{center}

\begin{center}
\begin{pspicture}(-3,-0.1)(3,3)
\psline[linestyle=dashed]{<->}(-3,0)(3,0)
\psarc{o-o}(0,0){2.5}{0}{180}
\psarc{o-o}(-0.9,0){1.6}{0}{180}
\end{pspicture}
\end{center}

\begin{center}
\begin{pspicture}(-3,-0.1)(3,3)
\psline[linestyle=dashed]{<->}(-3,0)(3,0)
\psline{o->}(-2.25,0)(-2.25,3)
\psline{o->}(2.25,0)(2.25,3)
\end{pspicture}
\end{center}

Below is an example of disjointly parallel lines in the Beltrami-Klein model:

\begin{center}
\begin{pspicture}(-2,-2)(2,2)
\pscircle[linestyle=dashed](0,0){2}
\psline{o-o}(-2,0)(1.414,1.414)
\psline{o-o}(-1.2,-1.6)(1.2,-1.6)
\end{pspicture}
\end{center}

Below is an example of disjointly parallel lines in the Poincar\'e disc model:

\begin{center}
\begin{pspicture}(-2,-2)(2,2)
\pscircle[linestyle=dashed](0,0){2}
\psarc{o-o}(-2,4.828){4.828}{270}{315}
\psarc{o-o}(0,-2.5){1.5}{36.87}{143.13}
\end{pspicture}
\end{center}

Below are some examples of disjointly parallel lines in the upper half plane model:

\begin{center}
\begin{pspicture}(-3,-0.1)(3,3)
\psline[linestyle=dashed]{<->}(-3,0)(3,0)
\psarc{o-o}(-1.3,0){1.2}{0}{180}
\psarc{o-o}(1.3,0){1.2}{0}{180}
\end{pspicture}
\end{center}

\begin{center}
\begin{pspicture}(-3,-0.1)(3,3)
\psline[linestyle=dashed]{<->}(-3,0)(3,0)
\psarc{o-o}(0,0){2.5}{0}{180}
\psarc{o-o}(-1,0){1}{0}{180}
\end{pspicture}
\end{center}

\begin{center}
\begin{pspicture}(-3,-0.1)(3,3)
\psline[linestyle=dashed]{<->}(-3,0)(3,0)
\psline{o->}(-2,0)(-2,3)
\psarc{o-o}(0.5,0){2}{0}{180}
\end{pspicture}
\end{center}
%%%%%
%%%%%
\end{document}
