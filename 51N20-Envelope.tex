\documentclass[12pt]{article}
\usepackage{pmmeta}
\pmcanonicalname{Envelope}
\pmcreated{2013-03-22 17:10:19}
\pmmodified{2013-03-22 17:10:19}
\pmowner{pahio}{2872}
\pmmodifier{pahio}{2872}
\pmtitle{envelope}
\pmrecord{23}{39484}
\pmprivacy{1}
\pmauthor{pahio}{2872}
\pmtype{Definition}
\pmcomment{trigger rebuild}
\pmclassification{msc}{51N20}
%\pmkeywords{family of curves}
\pmrelated{DistanceFromPointToALine}
\pmdefines{envelope}

\endmetadata

% this is the default PlanetMath preamble.  as your knowledge
% of TeX increases, you will probably want to edit this, but
% it should be fine as is for beginners.

% almost certainly you want these
\usepackage{amssymb}
\usepackage{amsmath}
\usepackage{amsfonts}

% used for TeXing text within eps files
%\usepackage{psfrag}
% need this for including graphics (\includegraphics)
%\usepackage{graphicx}
% for neatly defining theorems and propositions
 \usepackage{amsthm}
% making logically defined graphics
%%%\usepackage{xypic}

% there are many more packages, add them here as you need them
\usepackage{pstricks}
% define commands here

\theoremstyle{definition}
\newtheorem*{thmplain}{Theorem}

\begin{document}
Two plane curves are said to {\em touch each other} or {\em have a tangency} at a point if they have a common tangent line at that point.

The {\em envelope} of a family of plane curves is a curve which touches in each of its points one of the curves of the family.\\

For example, the envelope of the family\, $y = mx-\sqrt{1+m^2}$,\, with $m$ the parameter, may be justified geometrically.\, It is the \PMlinkname{open}{OpenSet} lower semicircle of the unit circle.\, Indeed, the distance of any line
$$mx-y-\sqrt{1+m^2} = 0$$
of the family from the center of the unit circle is
$$\frac{|m\cdot0-1\cdot0-\sqrt{1+m^2}|}{\sqrt{m^2+(-1)^2}} = 1,$$
whence the line is the tangent to the circle.

Below, the red curve is the lower semicircle of the unit circle, the black lines belong to the family\, $y=mx-\sqrt{1+m^2}$,\, and the equation of each line is given.

\begin{center}
\begin{pspicture}(-3,-3)(3,0)
\psarc[linecolor=red]{o-o}(0,0){2}{180}{360}
\psline{-}(-0.1716,-3)(3,0.1716)
\rput[l](2.9,-0.1716){$y=x+\sqrt{2}$}
\psline{-}(-3,-2)(3,-2)
\rput[a](-2.3,-2.3){$y=-1$}
\psline{-}(-2.3094,0)(-0.577,-3)
\rput[r](-2.3094,-0.1716){$y=-x\sqrt{3}-2$}
\rput[b](-0.577,-3){.}
\end{pspicture}
\end{center}


%%%%%
%%%%%
\end{document}
