\documentclass[12pt]{article}
\usepackage{pmmeta}
\pmcanonicalname{ProofOfRodriguesRotationFormula}
\pmcreated{2013-03-22 15:23:25}
\pmmodified{2013-03-22 15:23:25}
\pmowner{stevecheng}{10074}
\pmmodifier{stevecheng}{10074}
\pmtitle{proof of Rodrigues' rotation formula}
\pmrecord{14}{37221}
\pmprivacy{1}
\pmauthor{stevecheng}{10074}
\pmtype{Proof}
\pmcomment{trigger rebuild}
\pmclassification{msc}{51-00}
\pmclassification{msc}{15-00}
\pmrelated{DimensionOfTheSpecialOrthogonalGroup}

% this is the default PlanetMath preamble.  as your knowledge
% of TeX increases, you will probably want to edit this, but
% it should be fine as is for beginners.

% almost certainly you want these
\usepackage{amssymb}
\usepackage{amsmath}
\usepackage{amsfonts}

% used for TeXing text within eps files
%\usepackage{psfrag}
% need this for including graphics (\includegraphics)
\usepackage{graphicx}
% for neatly defining theorems and propositions
%\usepackage{amsthm}
% making logically defined graphics
%%%\usepackage{xypic}

% there are many more packages, add them here as you need them
\usepackage{enumerate}

% define commands here
\newcommand{\real}{\mathbb{R}}
\newcommand{\rat}{\mathbb{Q}}
\newcommand{\nat}{\mathbb{N}}

\providecommand{\abs}[1]{\lvert#1\rvert}
\providecommand{\absW}[1]{\left\lvert#1\right\rvert}
\providecommand{\absB}[1]{\Bigl\lvert#1\Bigr\rvert}
\providecommand{\norm}[1]{\lVert#1\rVert}
\providecommand{\normW}[1]{\left\lVert#1\right\rVert}
\providecommand{\normB}[1]{\Bigl\lVert#1\Bigr\rVert}
\providecommand{\defnterm}[1]{\emph{#1}}

\newcommand{\bx}{\mathbf{x}}
\newcommand{\by}{\mathbf{y}}
\newcommand{\bz}{\mathbf{z}}
\newcommand{\bv}{\mathbf{v}}

\providecommand{\transpose}[1]{{#1}^{\textrm{t}}}
\DeclareMathOperator{\trace}{tr}
\begin{document}
Let $[ \bx, \by, \bz]$ be a frame of right-handed orthonormal vectors in $\real^3$,
and let $\bv = a\bx + b\by + c\bz$ (with $a, b, c \in \real$) be any vector to be rotated on the $\bz$ axis,
by an angle $\theta$ counter-clockwise.

\begin{center}
\includegraphics{rodrigues.eps}
\end{center}

The image vector $\bv'$ is the vector $\bv$ with its component
in the $\bx,\by$ plane rotated, so we can write
\[
\bv' = a\bx' + b\by' + c\bz\,,\\
\]
where $\bx'$ and $\by'$ are the rotations by angle $\theta$ of the $\bx$ and $\by$ vectors in the $\bx,\by$ plane. By the rotation formula in two dimensions, we have
\begin{align*}
\bx' &= \cos \theta \, \bx + \sin \theta \, \by\,, \\
\by' &= -\sin \theta \, \bx + \cos \theta \, \by\,.
\end{align*}
So
\[
\bv' = \cos \theta (a\bx + b\by) + \sin \theta (a \by - b\bx) + c\bz\,.
\]
The vector $a \bx + b\by$ is the projection of $\bv$
onto the $\bx,\by$ plane, and $a \by - b\bx$ is its rotation by $90^\circ$.
So these two vectors form an orthogonal frame in the $\bx,\by$ plane,
although they are not necessarily unit vectors.
Alternate expressions for these vectors are easily derived --- especially with the help of the picture:
\begin{align*}
\bv - (\bv \cdot \bz) \bz &= \bv - c\bz = a\bx + b\by\,, \\
\bz \times \bv &= a (\bz\times \bx) + b (\bz \times \by) + c (\bz \times \bz) = a\by - b\bx \,.
\end{align*}
Substituting these into the expression for $\bv'$:
\[
\bv' = \cos \theta (\bv - (\bv \cdot \bz) \bz) + \sin\theta (\bz \times \bv) + c\bz\,,
\]
which could also have been derived directly 
if we had first considered the frame $[\bv - (\bv \cdot \bz), \bz \times \bv]$ instead of $[\bx, \by]$.

We attempt to simplify further:
\[
\bv' = \bv + \sin\theta (\bz \times \bv) + (\cos \theta - 1)(\bv - (\bv \cdot \bz)\bz )\,.
\]
Since $\bz \times \bv$ is linear in $\bv$, this transformation is represented by
a linear operator $A$. Under a right-handed orthonormal basis,
the matrix representation of $A$ is directly computed to be
\[
A\bv = \bz \times \bv
= \begin{bmatrix} 
0 & -z_3 & z_2 \\ 
z_3 & 0 & -z_1 \\
-z_2 & z_1 & 0
\end{bmatrix}
\,
\begin{bmatrix}
v_1 \\
v_2 \\
v_3
\end{bmatrix}
\,.
\]
We also have
\begin{align*}
-(\bv - (\bv \cdot \bz) \bz) &= -a\bx - b\by &
\text{(rotate $a\bx + b\by$ by $180^\circ$)} \\
&= \bz \times (a\by - b\bx) &
\text{(rotate $a\by - b\bx$ by $90^\circ$)} 
\\
&= \bz \times (\bz \times (a\bx + b\by +c\bz)) \\ 
&= A^2 \, \bv\,.
\end{align*}
So
\[
\bv' = I\bv + \sin \theta \, A\bv + (1- \cos\theta) A^2 \, \bv\,,
\]
proving Rodrigues' rotation formula.

\section*{Relation with the matrix exponential}
Here is a curious fact. Notice that the matrix\footnote{If we want to use coordinate-free \PMlinkescapetext{language}, then in this section, ``matrix'' should be replaced by ``linear operator'' and transposes should be replaced by the adjoint operation.} $A$ is skew-symmetric.
This is not a coincidence --- for any skew-symmetric matrix $B$, we have 
$\transpose{(e^B)} = e^{\transpose{B}} = e^{-B} = (e^B)^{-1}$, 
and $\det e^B = e^{\trace B} = e^0 = 1$,
so $e^B$ is always a rotation.  It is in fact the case that:
\begin{align*}
I + \sin \theta \, A + (1- \cos\theta) A^2 = e^{\theta A}
\end{align*}
for the matrix $A$ we had above! To prove this, observe that powers of $A$ cycle like so:
\begin{align*}
I, A, A^2, -A, -A^2, A, A^2, -A, -A^2, \dotsc
\end{align*}
Then 
\begin{align*}
\sin \theta \, A &= \sum_{k=0}^\infty \frac{(-1)^k \theta^{2k+1} A}{(2k+1)!} 
= \sum_{k=0}^\infty \frac{\theta^{2k+1} A^{2k+1}}{(2k+1)!} = \sum_{k \textrm{ odd}} \frac{(\theta A)^k}{k!}
\\
(1 - \cos \theta) \, A^2 &= \sum_{k=1}^\infty \frac{(-1)^{k-1} \theta^{2k} A^2}{(2k)!}
= \sum_{k=1}^\infty \frac{\theta^{2k} A^{2k}}{(2k)!} = \sum_{k\geq 2 \textrm{ even}} 
\frac{(\theta A)^k}{k!}\,.
\end{align*}
Adding $\sin \theta \, A$, $(1-\cos \theta) A^2$ and $I$ together, we obtain the power series
for $e^{\theta A}$.

\emph{Second proof:}
If we regard $\theta$ as time, and differentiate the equation 
$\bv' = a\bx' + b\by' + c\bz$ with respect to $\theta$, we obtain
$d\bv'/d\theta = a\by' - b\bx' = \bz \times \bv' = A \bv'$, whence the solution (to this linear ODE)
is $\bv' = e^{\theta A} \bv$. 

\emph{Remark:} The operator $e^{\theta A}$, as $\theta$ ranges over $\real$,
is a one-parameter subgroup of $\mathrm{SO}(3)$.
In higher dimensions $n$, every rotation in $\mathrm{SO}(n)$ is of the form
$e^A$ for a skew-symmetric $A$, and the second proof above
can be modified to prove this more general fact.
%%%%%
%%%%%
\end{document}
