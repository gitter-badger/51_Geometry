\documentclass[12pt]{article}
\usepackage{pmmeta}
\pmcanonicalname{ProofOfVanAubelsTheorem}
\pmcreated{2013-03-22 13:58:00}
\pmmodified{2013-03-22 13:58:00}
\pmowner{mathcam}{2727}
\pmmodifier{mathcam}{2727}
\pmtitle{proof of Van Aubel's theorem}
\pmrecord{8}{34736}
\pmprivacy{1}
\pmauthor{mathcam}{2727}
\pmtype{Proof}
\pmcomment{trigger rebuild}
\pmclassification{msc}{51N20}
\pmrelated{VanAubelTheorem}
\pmrelated{ProofOfVanAubelTheorem}

\endmetadata

% this is the default PlanetMath preamble.  as your knowledge
% of TeX increases, you will probably want to edit this, but
% it should be fine as is for beginners.

% almost certainly you want these
\usepackage{amssymb}
\usepackage{amsmath}
\usepackage{amsfonts}
\usepackage{amsthm}

% used for TeXing text within eps files
%\usepackage{psfrag}
% need this for including graphics (\includegraphics)
\usepackage{graphicx}
% for neatly defining theorems and propositions
%\usepackage{amsthm}
% making logically defined graphics
%%%\usepackage{xypic}

% there are many more packages, add them here as you need them

% define commands here

\newcommand{\mc}{\mathcal}
\newcommand{\mb}{\mathbb}
\newcommand{\mf}{\mathfrak}
\newcommand{\ol}{\overline}
\newcommand{\ra}{\rightarrow}
\newcommand{\la}{\leftarrow}
\newcommand{\La}{\Leftarrow}
\newcommand{\Ra}{\Rightarrow}
\newcommand{\nor}{\vartriangleleft}
\newcommand{\Gal}{\text{Gal}}
\newcommand{\GL}{\text{GL}}
\newcommand{\Z}{\mb{Z}}
\newcommand{\R}{\mb{R}}
\newcommand{\Q}{\mb{Q}}
\newcommand{\C}{\mb{C}}
\newcommand{\<}{\langle}
\renewcommand{\>}{\rangle}
\begin{document}
\PMlinkescapeword{bases}
\PMlinkescapeword{component}
\PMlinkescapeword{terms}
\PMlinkescapeword{complete}
\includegraphics{vanaubel}

As in the figure, let us denote by $u,v,w,x,y,z$ the areas of the
six component triangles.
Given any two triangles of the same height, their areas are in the
same proportion as their bases (Euclid VI.1). Therefore
$$\frac{y+z}{x}=\frac{u+v}{w}
\qquad\frac{w+x}{v}=\frac{y+z}{u}
\qquad\frac{u+v}{z}=\frac{w+x}{y}$$
and the conclusion we want is
$$\frac{y+z+u}{v+w+x}+\frac{z+u+v}{w+x+y}=\frac{y+z}{x}\;.$$

Clearing the denominators, the hypotheses are
\begin{eqnarray}
w(y+z)&=&x(u+v) \\
y(u+v)&=&z(w+x) \\
u(w+x)&=&v(y+z)
\end{eqnarray}
which imply
\begin{equation}
vxz=uwy
\end{equation}
and the conclusion says that
$$x(\underline{wy}+\underline{wz}+uw
+\underline{xy}+\underline{xz}+ux+\underline{y^2}+\underline{yz}+uy$$
$$+vz+uv+v^2+wz+uw+vw+xz+ux+vx)$$
equals
$$(y+z)(vw+vx+vy+w^2+wx+wy+\underline{wx}+\underline{x^2}+\underline{xy})$$
or equivalently (after cancelling the underlined terms)
$$x(uw+xz+ux+uy+vz+uv+v^2+wz+uw+vw+ux+vx)$$
equals
$$(y+z)(vw+vx+vy+w^2+wx+wy)=(y+z)(v+w)(w+x+y)\;.$$
i.e.
$$x(u+v)(v+w+x)+x(xz+ux+uy+vz+wz+uw)=$$
$$(y+z)w(v+w+x)+(y+z)(vx+vy+wy)$$
i.e. by (1)
$$x(xz+ux+uy+vz+wz+uw)=(y+z)(vx+vy+wy)$$
i.e. by (3)
$$x(xz+uy+vz+wz)=(y+z)(vy+wy)\;.$$
Using (4), we are down to
$$x^2z+xuy+uwy+xwz=(y+z)y(v+w)$$
i.e. by (3)
$$x^2z+vy(y+z)+xwz=(y+z)y(v+w)$$
i.e.
$$xz(x+w)=(y+z)yw\;.$$
But in view of (2), this is the same as (4), and
the proof is complete.

\textbf{Remarks: }
Ceva's theorem is an easy consequence of (4).
%%%%%
%%%%%
\end{document}
