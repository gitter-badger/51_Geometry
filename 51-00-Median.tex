\documentclass[12pt]{article}
\usepackage{pmmeta}
\pmcanonicalname{Median}
\pmcreated{2013-03-22 11:44:01}
\pmmodified{2013-03-22 11:44:01}
\pmowner{CWoo}{3771}
\pmmodifier{CWoo}{3771}
\pmtitle{median}
\pmrecord{18}{30142}
\pmprivacy{1}
\pmauthor{CWoo}{3771}
\pmtype{Definition}
\pmcomment{trigger rebuild}
\pmclassification{msc}{51-00}
\pmclassification{msc}{55-00}
\pmclassification{msc}{55-01}
%\pmkeywords{Triangle}
%\pmkeywords{Geometry}
\pmrelated{Triangle}
\pmrelated{ApolloniusTheorem}
\pmrelated{Orthocenter}
\pmrelated{CevasTheorem}
\pmrelated{Centroid}
\pmrelated{ProofOfApolloniusTheorem2}
\pmrelated{ParallelogramLaw}
\pmrelated{TrigonometricVersionOfCevasTheorem}
\pmrelated{ProofOfParallelogramLaw}
\pmrelated{HeightOfATriangle}
\pmrelated{Cevian}

\usepackage{amssymb}
\usepackage{amsmath}
\usepackage{amsfonts}
\usepackage{graphicx}
%%%%%%%\usepackage{xypic}
\begin{document}
The \emph{median} of a triangle is a line segment joining a vertex with the midpoint of the opposite side.

In the next figure, $AA'$ is a median. That is, $BA'=A'C$, or equivalently, $A'$ is the midpoint of $BC$.

\begin{center}
\includegraphics[scale=0.5]{median}
\end{center}

If the length of the three sides of the triangle are known, the length of the medians can be found by means of Apollonius theorem.
%%%%%
%%%%%
%%%%%
%%%%%
%%%%%
%%%%%
%%%%%
\end{document}
