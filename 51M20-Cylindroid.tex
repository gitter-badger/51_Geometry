\documentclass[12pt]{article}
\usepackage{pmmeta}
\pmcanonicalname{Cylindroid}
\pmcreated{2013-03-22 16:41:12}
\pmmodified{2013-03-22 16:41:12}
\pmowner{PrimeFan}{13766}
\pmmodifier{PrimeFan}{13766}
\pmtitle{cylindroid}
\pmrecord{7}{38897}
\pmprivacy{1}
\pmauthor{PrimeFan}{13766}
\pmtype{Definition}
\pmcomment{trigger rebuild}
\pmclassification{msc}{51M20}
\pmclassification{msc}{14J25}
\pmclassification{msc}{51M04}

\endmetadata

% this is the default PlanetMath preamble.  as your knowledge
% of TeX increases, you will probably want to edit this, but
% it should be fine as is for beginners.

% almost certainly you want these
\usepackage{amssymb}
\usepackage{amsmath}
\usepackage{amsfonts}

% used for TeXing text within eps files
%\usepackage{psfrag}
% need this for including graphics (\includegraphics)
%\usepackage{graphicx}
% for neatly defining theorems and propositions
%\usepackage{amsthm}
% making logically defined graphics
%%%\usepackage{xypic}

% there are many more packages, add them here as you need them

% define commands here

\begin{document}
At the most general level, a {\em cylindroid} is simply a cylinder that has been deformed in an intentional and well-defined way. The barrel die used in certain board role-playing games are sometimes described as being cylindroids in shape. 

Most mathematical dictionaries that define the term at all say nothing more than ``a cylinder with an elliptical cross-section.'' (see for example the {\it Harper-Collins Dictionary of Mathematics} and the {\it Oxford Concise Dictionary of Mathematics}.) The term is also used for certain conoids, such as Pl\"ucker's conoid.

The term is usually not included in general pocket dictionaries. The {\it Random House Unabridged Dictionary} defines cylindroid the noun as ``a solid having the form of a cylinder, esp[ecially] one with an elliptical, as opposed to a circular, cross section,'' and the adjective as ``resembling a cylinder.''

\begin{thebibliography}{1}
\bibitem{sr} S. P. Radzevich, ``A Possibility of Application of Pliicker's Conoid for Mathematical Modeling of Contact of Two Smooth Regular Surfaces in the First Order of Tangency'', {\it Mathematical and Computer Modelling} {\bf 42} (2005): 999 - 1022
\end{thebibliography}
%%%%%
%%%%%
\end{document}
