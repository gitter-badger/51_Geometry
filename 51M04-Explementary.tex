\documentclass[12pt]{article}
\usepackage{pmmeta}
\pmcanonicalname{Explementary}
\pmcreated{2013-03-22 17:34:35}
\pmmodified{2013-03-22 17:34:35}
\pmowner{pahio}{2872}
\pmmodifier{pahio}{2872}
\pmtitle{explementary}
\pmrecord{8}{39988}
\pmprivacy{1}
\pmauthor{pahio}{2872}
\pmtype{Definition}
\pmcomment{trigger rebuild}
\pmclassification{msc}{51M04}
\pmclassification{msc}{51F20}
\pmrelated{ComplementaryAngles}
\pmdefines{explementary angle}
\pmdefines{explementary arc}
\pmdefines{full angle}

% this is the default PlanetMath preamble.  as your knowledge
% of TeX increases, you will probably want to edit this, but
% it should be fine as is for beginners.

% almost certainly you want these
\usepackage{amssymb}
\usepackage{amsmath}
\usepackage{amsfonts}

% used for TeXing text within eps files
%\usepackage{psfrag}
% need this for including graphics (\includegraphics)
%\usepackage{graphicx}
% for neatly defining theorems and propositions
 \usepackage{amsthm}
% making logically defined graphics
%%%\usepackage{xypic}
\usepackage{pstricks}
\usepackage{pst-plot}

% there are many more packages, add them here as you need them

% define commands here

\theoremstyle{definition}
\newtheorem*{thmplain}{Theorem}

\begin{document}
The {\em explementary arc} of an arc $a$ of a circle is the arc forming together with $a$ the full circle.

Two angles are called {\em explementary angles} of each other, if their sum is the {\em full angle} $2\pi$, i.e. $360^\circ$.  In the below picture, the \PMlinkescapetext{interior angle} \,$\alpha = 60^\circ$\, of an equilateral triangle and its explementary angle\, $\beta = 300^\circ$\, (which is an \PMlinkescapetext{exterior angle} of the triangle) are seen.

\begin{center}
\begin{pspicture}(-2,-1)(2,3)
\pspolygon(-1.5,0)(1.5,0)(0,2.6)
\psarc(0,2.6){0.2}{-60}{240}
\rput[a](0,2.2){$\alpha$}
\rput[a](-0.2,2.9){$\beta$}
\rput(-2,-0.5){.}
\rput(2,3){.}
\end{pspicture}
\end{center}

%%%%%
%%%%%
\end{document}
