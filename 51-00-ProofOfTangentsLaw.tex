\documentclass[12pt]{article}
\usepackage{pmmeta}
\pmcanonicalname{ProofOfTangentsLaw}
\pmcreated{2013-03-22 13:11:04}
\pmmodified{2013-03-22 13:11:04}
\pmowner{CWoo}{3771}
\pmmodifier{CWoo}{3771}
\pmtitle{proof of tangents law}
\pmrecord{6}{33632}
\pmprivacy{1}
\pmauthor{CWoo}{3771}
\pmtype{Proof}
\pmcomment{trigger rebuild}
\pmclassification{msc}{51-00}

% this is the default PlanetMath preamble.  as your knowledge
% of TeX increases, you will probably want to edit this, but
% it should be fine as is for beginners.

% almost certainly you want these
\usepackage{amssymb}
\usepackage{amsmath}
\usepackage{amsfonts}

% used for TeXing text within eps files
%\usepackage{psfrag}
% need this for including graphics (\includegraphics)
\usepackage{graphicx}
% for neatly defining theorems and propositions
%\usepackage{amsthm}
% making logically defined graphics
%%%\usepackage{xypic}

% there are many more packages, add them here as you need them

% define commands here
\begin{document}
To prove that $$\frac{a-b}{a+b} = \frac{\tan(\frac{A-B}{2})}{\tan(\frac{A+B}{2})}$$
we start with the sines law, which says that $$\frac{a}{\sin(A)} = \frac{b}{\sin(B)}.$$
This implies that $$a \sin(B) = b \sin(A)$$
We can write $\sin(A)$ as $$\sin(A) = \sin(\frac{A+B}{2})\cos(\frac{A-B}{2}) + \cos(\frac{A+B}{2})\sin(\frac{A-B}{2}).$$
and $\sin(B)$ as $$\sin(B) = \sin(\frac{A+B}{2})\cos(\frac{A-B}{2}) - \cos(\frac{A+B}{2})\sin(\frac{A-B}{2}).$$
Therefore, we have 
$$a (\sin(\frac{A+B}{2})\cos(\frac{A-B}{2}) - \cos(\frac{A+B}{2})\sin(\frac{A-B}{2})) = b (\sin(\frac{A+B}{2})\cos(\frac{A-B}{2}) + \cos(\frac{A+B}{2})\sin(\frac{A-B}{2}))$$
Dividing both sides by $\cos(\frac{A-B}{2})\cos(\frac{A+B}{2}),$ we have,
$$a (\tan(\frac{A+B}{2}) - \tan(\frac{A-B}{2}) )  =  b (\tan(\frac{A+B}{2}) + \tan(\frac{A-B}{2}) )$$
This gives us
$$ \frac{a}{b} = \frac{\tan(\frac{A+B}{2}) + \tan(\frac{A-B}{2})}{\tan(\frac{A+B}{2}) - \tan(\frac{A-B}{2})}$$
Hence we find that
$$
\frac{a-b}{a+b} =  \frac{\displaystyle{\frac{a}{b}}-1}{\displaystyle{\frac{a}{b}}+1} =
\frac{\tan(\frac{A-B}{2})}{\tan(\frac{A+B}{2})}.
$$
%%%%%
%%%%%
\end{document}
