\documentclass[12pt]{article}
\usepackage{pmmeta}
\pmcanonicalname{BetweennessRelation}
\pmcreated{2013-03-22 17:18:44}
\pmmodified{2013-03-22 17:18:44}
\pmowner{Mathprof}{13753}
\pmmodifier{Mathprof}{13753}
\pmtitle{betweenness relation}
\pmrecord{6}{39661}
\pmprivacy{1}
\pmauthor{Mathprof}{13753}
\pmtype{Definition}
\pmcomment{trigger rebuild}
\pmclassification{msc}{51G05}
\pmsynonym{axioms of order}{BetweennessRelation}
\pmrelated{SomeTheoremsOnTheAxiomsOfOrder}

% this is the default PlanetMath preamble.  as your knowledge
% of TeX increases, you will probably want to edit this, but
% it should be fine as is for beginners.

% almost certainly you want these
\usepackage{amssymb}
\usepackage{amsmath}
\usepackage{amsfonts}

% used for TeXing text within eps files
%\usepackage{psfrag}
% need this for including graphics (\includegraphics)
%\usepackage{graphicx}
% for neatly defining theorems and propositions
%\usepackage{amsthm}
% making logically defined graphics
%%%\usepackage{xypic}

% there are many more packages, add them here as you need them

% define commands here

\begin{document}
\section{Definition}
 Let $A$ be a set.  A ternary relation $B$ on
$A$ is said to be a \emph{betweenness relation} if it has the following properties:
\begin{enumerate}
\item[O1] if $(a,b,c)\in B$, then $(c,b,a)\in B$; in other words, the set $$B(b)=
\lbrace (a,c)\mid (a,b,c)\in B\rbrace$$ is a \PMlinkname{symmetric relation}{Symmetric} for \emph{each}
$b$;  thus, from now on, we may say, without any ambiguity, that $b$
is \emph{between} $a$ and $c$ if $(a,b,c)\in B$;
\item[O2] if $(a,b,a)\in B$, then $a=b$;
\item[O3] for each $a,b\in A$, there is a $c\in A$ such that $(a,b,c)\in B$;
\item[O4]for each $a,b\in A$, there is a $c\in A$ such that $(a,c,b)\in B$;
\item[O5] if $(a,b,c)\in B$ and $(b,a,c)\in B$, then $a=b$;
\item[O6] if $(a,b,c)\in B$ and $(b,c,d)\in B$, then $(a,b,d)\in B$;
\item[O7] if $(a,b,d)\in B$ and $(b,c,d)\in B$, then $(a,b,c)\in B$.
\end{enumerate}
%%%%%
%%%%%
\end{document}
